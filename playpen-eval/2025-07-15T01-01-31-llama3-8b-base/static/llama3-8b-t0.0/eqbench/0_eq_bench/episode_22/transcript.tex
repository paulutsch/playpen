
\documentclass{article}
\usepackage{colortbl}
\usepackage{makecell}
\usepackage{multirow}
\usepackage{supertabular}

\begin{document}

\newcounter{utterance}

\twocolumn

{ \footnotesize  \setcounter{utterance}{1}
\setlength{\tabcolsep}{0pt}
\begin{supertabular}{c@{$\;$}|p{.15\linewidth}@{}p{.15\linewidth}p{.15\linewidth}p{.15\linewidth}p{.15\linewidth}p{.15\linewidth}}

    \# & $\;$A & \multicolumn{4}{c}{Game Master} & $\;\:$B\\
    \hline 

    \theutterance \stepcounter{utterance}  

    & & \multicolumn{4}{p{0.6\linewidth}}{\cellcolor[rgb]{0.9,0.9,0.9}{%
	\makecell[{{p{\linewidth}}}]{% 
	  \tt {\tiny [A$\langle$GM]}  
	 \\ \tt Your task is to predict the likely emotional responses of a character in this dialogue:\\ \tt \\ \tt Jacques: Alessandra, you've been under-seasoning the scallops all night. \\ \tt Alessandra: I've been following the recipe, chef.\\ \tt Jacques: (Laughs) The recipe? That's your problem, Alessandra. You're a slave to the recipe. You need to feel the food, taste it, not just blindly follow instructions.\\ \tt Alessandra: I understand that, chef.\\ \tt Jacques: Do you? Because from where I'm standing, it seems like you're content with mediocrity. \\ \tt Alessandra: That's not fair, Jacques.\\ \tt Jacques: Fair? This isn't about fairness, Alessandra. This is about excellence. \\ \tt Alessandra: I'm doing my best.\\ \tt Jacques: Your best isn't good enough. Not for this kitchen. Not for me.\\ \tt Alessandra: I'm not here to please you, Jacques.\\ \tt Jacques: No, you're here to learn. And part of learning is accepting criticism. \\ \tt Alessandra: There's a difference between criticism and belittlement, Jacques.\\ \tt Jacques: (Laughs) Belittlement? Is that what you think this is? \\ \tt Alessandra: What else would you call it?\\ \tt Jacques: I call it pushing you to be better. To be more than just a recipe follower.\\ \tt Alessandra: I don't need to be pushed, Jacques. I need to be respected.\\ \tt Jacques: (Throws a dish towel at Alessandra) Respect? You think you've earned respect? You're ungrateful. Uncommitted.\\ \tt Alessandra: (Catches the towel, her eyes welling up with tears) You're wrong, Jacques. I am committed. More than you'll ever know.\\ \tt [End dialogue]\\ \tt \\ \tt At the end of this dialogue, Alessandra would feel...\\ \tt Defiance\\ \tt Vindication\\ \tt Resignation\\ \tt Despair\\ \tt \\ \tt Give each of these possible emotions a score from 0-10 for the relative intensity that they are likely to be feeling each.\\ \tt \\ \tt You must output in the following format, including headings (of course, you should give your own scores), with no additional commentary:\\ \tt \\ \tt Defiance: <score>\\ \tt Vindication: <score>\\ \tt Resignation: <score>\\ \tt Despair: <score>\\ \tt \\ \tt \\ \tt [End of answer]\\ \tt \\ \tt Remember: zero is a valid score, meaning they are likely not feeling that emotion. You must score at least one emotion > 0.\\ \tt \\ \tt Your answer:\\ \tt  
	  } 
	   } 
	   } 
	 & \\ 
 

    \theutterance \stepcounter{utterance}  

    & \multicolumn{4}{p{0.6\linewidth}}{\cellcolor[rgb]{0.8,1,0.9}{%
	\makecell[{{p{\linewidth}}}]{% 
	  \tt {\tiny [A$\rangle$GM]}  
	 Defiance: 8\\ \tt Vindication: 2\\ \tt Resignation: 6\\ \tt Despair: 9 
	  } 
	   } 
	   } 
	 & & \\ 
 

    \theutterance \stepcounter{utterance}  

    & & & \multicolumn{2}{p{0.3\linewidth}}{\cellcolor[rgb]{0.95,0.95,0.95}{%
	\makecell[{{p{\linewidth}}}]{% 
	  \tt {\tiny [GM$|$GM]}  
	 Defiance: 8\\ \tt Vindication: 2\\ \tt Resignation: 6\\ \tt Despair: 9 
	  } 
	   } 
	   } 
	 & & \\ 
 

    \theutterance \stepcounter{utterance}  

    & & & \multicolumn{2}{p{0.3\linewidth}}{\cellcolor[rgb]{0.95,0.95,0.95}{%
	\makecell[{{p{\linewidth}}}]{% 
	  \tt {\tiny [GM$|$GM]}  
	 Defiance: 7\\ \tt Vindication: 0\\ \tt Resignation: 4\\ \tt Despair: 6 
	  } 
	   } 
	   } 
	 & & \\ 
 

    \theutterance \stepcounter{utterance}  

    & & & \multicolumn{2}{p{0.3\linewidth}}{\cellcolor[rgb]{0.95,0.95,0.95}{%
	\makecell[{{p{\linewidth}}}]{% 
	  \tt {\tiny [GM$|$GM]}  
	 game_result = LOSE 
	  } 
	   } 
	   } 
	 & & \\ 
 

\end{supertabular}
}

\end{document}
