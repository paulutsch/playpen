
\documentclass{article}
\usepackage{colortbl}
\usepackage{makecell}
\usepackage{multirow}
\usepackage{supertabular}

\begin{document}

\newcounter{utterance}

\twocolumn

{ \footnotesize  \setcounter{utterance}{1}
\setlength{\tabcolsep}{0pt}
\begin{supertabular}{c@{$\;$}|p{.15\linewidth}@{}p{.15\linewidth}p{.15\linewidth}p{.15\linewidth}p{.15\linewidth}p{.15\linewidth}}

    \# & $\;$A & \multicolumn{4}{c}{Game Master} & $\;\:$B\\
    \hline 

    \theutterance \stepcounter{utterance}  

    & & \multicolumn{4}{p{0.6\linewidth}}{\cellcolor[rgb]{0.9,0.9,0.9}{%
	\makecell[{{p{\linewidth}}}]{% 
	  \tt {\tiny [A$\langle$GM]}  
	 \\ \tt Your task is to predict the likely emotional responses of a character in this dialogue:\\ \tt \\ \tt Rev. Montague: You see, Iris, faith is not a choice, but a divine gift. It's a beacon of light in a world often shrouded in darkness.\\ \tt Dr. LeGuin: A beacon of light, Luther? Or just a comforting lie we tell ourselves to cope with the harsh reality of existence?\\ \tt Rev. Montague: It's more than that. It's a guiding principle, a moral compass. It's what makes us human, Iris.\\ \tt Dr. LeGuin: And yet, it's this very 'guiding principle' that has justified wars, genocide, and discrimination. How human is that, Luther?\\ \tt Rev. Montague: Those are the failures of men, not faith. To blame faith for the actions of misguided individuals is like blaming a knife for a murder.\\ \tt Dr. LeGuin: A knife doesn't claim to be a moral compass, Luther. And it doesn't demand blind obedience.\\ \tt [End dialogue]\\ \tt \\ \tt At the end of this dialogue, Rev. Montague would feel...\\ \tt Outraged\\ \tt Confused\\ \tt Challenged\\ \tt Indifferent\\ \tt \\ \tt Give each of these possible emotions a score from 0-10 for the relative intensity that they are likely to be feeling each.\\ \tt \\ \tt You must output in the following format, including headings (of course, you should give your own scores), with no additional commentary:\\ \tt \\ \tt Outraged: <score>\\ \tt Confused: <score>\\ \tt Challenged: <score>\\ \tt Indifferent: <score>\\ \tt \\ \tt \\ \tt [End of answer]\\ \tt \\ \tt Remember: zero is a valid score, meaning they are likely not feeling that emotion. You must score at least one emotion > 0.\\ \tt \\ \tt Your answer:\\ \tt  
	  } 
	   } 
	   } 
	 & \\ 
 

    \theutterance \stepcounter{utterance}  

    & \multicolumn{4}{p{0.6\linewidth}}{\cellcolor[rgb]{0.8,1,0.9}{%
	\makecell[{{p{\linewidth}}}]{% 
	  \tt {\tiny [A$\rangle$GM]}  
	 Outraged: 6\\ \tt Confused: 2\\ \tt Challenged: 9\\ \tt Indifferent: 0 
	  } 
	   } 
	   } 
	 & & \\ 
 

    \theutterance \stepcounter{utterance}  

    & & & \multicolumn{2}{p{0.3\linewidth}}{\cellcolor[rgb]{0.95,0.95,0.95}{%
	\makecell[{{p{\linewidth}}}]{% 
	  \tt {\tiny [GM$|$GM]}  
	 Outraged: 6\\ \tt Confused: 2\\ \tt Challenged: 9\\ \tt Indifferent: 0 
	  } 
	   } 
	   } 
	 & & \\ 
 

    \theutterance \stepcounter{utterance}  

    & & & \multicolumn{2}{p{0.3\linewidth}}{\cellcolor[rgb]{0.95,0.95,0.95}{%
	\makecell[{{p{\linewidth}}}]{% 
	  \tt {\tiny [GM$|$GM]}  
	 Outraged: 3\\ \tt Confused: 0\\ \tt Challenged: 8\\ \tt Indifferent: 0 
	  } 
	   } 
	   } 
	 & & \\ 
 

    \theutterance \stepcounter{utterance}  

    & & & \multicolumn{2}{p{0.3\linewidth}}{\cellcolor[rgb]{0.95,0.95,0.95}{%
	\makecell[{{p{\linewidth}}}]{% 
	  \tt {\tiny [GM$|$GM]}  
	 game_result = LOSE 
	  } 
	   } 
	   } 
	 & & \\ 
 

\end{supertabular}
}

\end{document}
