
\documentclass{article}
\usepackage{colortbl}
\usepackage{makecell}
\usepackage{multirow}
\usepackage{supertabular}

\begin{document}

\newcounter{utterance}

\twocolumn

{ \footnotesize  \setcounter{utterance}{1}
\setlength{\tabcolsep}{0pt}
\begin{supertabular}{c@{$\;$}|p{.15\linewidth}@{}p{.15\linewidth}p{.15\linewidth}p{.15\linewidth}p{.15\linewidth}p{.15\linewidth}}

    \# & $\;$A & \multicolumn{4}{c}{Game Master} & $\;\:$B\\
    \hline 

    \theutterance \stepcounter{utterance}  

    & & \multicolumn{4}{p{0.6\linewidth}}{\cellcolor[rgb]{0.9,0.9,0.9}{%
	\makecell[{{p{\linewidth}}}]{% 
	  \tt {\tiny [A$\langle$GM]}  
	 You are an expert in causal inference. The following question is not a typical commonsense query, but rather a meticulously designed question created by a professor specializing in causal inference, intended to assess the students' mastery of the course content.\\ \tt Imagine a self-contained, hypothetical world with only the following conditions, and without any unmentioned factors or causal relationships: Jyka has a direct effect on gyzp and hwax. Hwax has a direct effect on lirg. Gyzp has a direct effect on lirg. For those who are not jyka, the probability of lirg is 63%. For those who are jyka, the probability of lirg is 49%. For those who are jyka, would it be more likely to see lirg if the individual was not jyka?\\ \tt Start your answer with 'Yes' or 'No', followed by additional reasoning or evidence to support your explanation. 
	  } 
	   } 
	   } 
	 & \\ 
 

    \theutterance \stepcounter{utterance}  

    & \multicolumn{4}{p{0.6\linewidth}}{\cellcolor[rgb]{0.8,1,0.9}{%
	\makecell[{{p{\linewidth}}}]{% 
	  \tt {\tiny [A$\rangle$GM]}  
	 No.\\ \tt \\ \tt To answer this question, we need to apply the concept of counterfactuals and potential outcomes. We are comparing the probability of LIRG among Jyka individuals if they were not Jyka (i.e., the counterfactual probability) to the probability of LIRG among non-Jyka individuals.\\ \tt \\ \tt Let's denote the probability of LIRG among non-Jyka individuals as P(LIRG|~Jyka) = 0.63. We are given that the probability of LIRG among Jyka individuals is P(LIRG|Jyka) = 0.49.\\ \tt \\ \tt To determine the counterfactual probability of LIRG among Jyka individuals if they were not Jyka, we need to consider the causal relationships between Jyka, gyzp, hwax, and LIRG. Since Jyka has a direct effect on gyzp and hwax, and both gyzp and hwax have a direct effect on LIRG, we can use the backdoor criterion to identify the causal effect of Jyka on LIRG.\\ \tt \\ \tt However, the problem statement does not provide sufficient information to directly estimate the counterfactual probability. We can use the potential outcomes framework to reason about this problem. Let's denote the potential outcomes as Y(1) and Y(0), where Y(1) is the outcome (LIRG) if the individual is Jyka 
	  } 
	   } 
	   } 
	 & & \\ 
 

    \theutterance \stepcounter{utterance}  

    & & & \multicolumn{2}{p{0.3\linewidth}}{\cellcolor[rgb]{0.95,0.95,0.95}{%
	\makecell[{{p{\linewidth}}}]{% 
	  \tt {\tiny [GM$|$GM]}  
	 No 
	  } 
	   } 
	   } 
	 & & \\ 
 

    \theutterance \stepcounter{utterance}  

    & & & \multicolumn{2}{p{0.3\linewidth}}{\cellcolor[rgb]{0.95,0.95,0.95}{%
	\makecell[{{p{\linewidth}}}]{% 
	  \tt {\tiny [GM$|$GM]}  
	 yes 
	  } 
	   } 
	   } 
	 & & \\ 
 

    \theutterance \stepcounter{utterance}  

    & & & \multicolumn{2}{p{0.3\linewidth}}{\cellcolor[rgb]{0.95,0.95,0.95}{%
	\makecell[{{p{\linewidth}}}]{% 
	  \tt {\tiny [GM$|$GM]}  
	 game_result = LOSE 
	  } 
	   } 
	   } 
	 & & \\ 
 

\end{supertabular}
}

\end{document}
