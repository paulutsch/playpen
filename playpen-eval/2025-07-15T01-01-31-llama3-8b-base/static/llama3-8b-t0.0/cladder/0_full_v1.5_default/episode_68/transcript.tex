
\documentclass{article}
\usepackage{colortbl}
\usepackage{makecell}
\usepackage{multirow}
\usepackage{supertabular}

\begin{document}

\newcounter{utterance}

\twocolumn

{ \footnotesize  \setcounter{utterance}{1}
\setlength{\tabcolsep}{0pt}
\begin{supertabular}{c@{$\;$}|p{.15\linewidth}@{}p{.15\linewidth}p{.15\linewidth}p{.15\linewidth}p{.15\linewidth}p{.15\linewidth}}

    \# & $\;$A & \multicolumn{4}{c}{Game Master} & $\;\:$B\\
    \hline 

    \theutterance \stepcounter{utterance}  

    & & \multicolumn{4}{p{0.6\linewidth}}{\cellcolor[rgb]{0.9,0.9,0.9}{%
	\makecell[{{p{\linewidth}}}]{% 
	  \tt {\tiny [A$\langle$GM]}  
	 You are an expert in causal inference. The following question is not a typical commonsense query, but rather a meticulously designed question created by a professor specializing in causal inference, intended to assess the students' mastery of the course content.\\ \tt Imagine a self-contained, hypothetical world with only the following conditions, and without any unmentioned factors or causal relationships: Rixq has a direct effect on xevu. Zuph has a direct effect on xevu. The overall probability of rixq is 40%. The probability of not rixq and xevu is 9%. The probability of rixq and xevu is 22%. Is the chance of xevu smaller when observing rixq?\\ \tt Start your answer with 'Yes' or 'No', followed by additional reasoning or evidence to support your explanation. 
	  } 
	   } 
	   } 
	 & \\ 
 

    \theutterance \stepcounter{utterance}  

    & \multicolumn{4}{p{0.6\linewidth}}{\cellcolor[rgb]{0.8,1,0.9}{%
	\makecell[{{p{\linewidth}}}]{% 
	  \tt {\tiny [A$\rangle$GM]}  
	 No.\\ \tt \\ \tt To determine whether the chance of xevu is smaller when observing rixq, we need to calculate the conditional probability of xevu given rixq (P(xevu|rixq)) and compare it to the unconditional probability of xevu (P(xevu)).\\ \tt \\ \tt We can use the law of total probability to calculate P(xevu):\\ \tt \\ \tt P(xevu) = P(xevu and rixq) + P(xevu and not rixq)\\ \tt = 22% + P(xevu and not rixq)\\ \tt \\ \tt We know that P(not rixq and xevu) = 9%. Since P(not rixq) = 1 - P(rixq) = 1 - 40% = 60%, we can use the formula for conditional probability to find P(xevu and not rixq):\\ \tt \\ \tt P(xevu and not rixq) = P(not rixq and xevu) / P(not rixq)\\ \tt = 9% / 60%\\ \tt = 15%\\ \tt \\ \tt Now we can calculate P(xevu):\\ \tt \\ \tt P(xevu) = 22% + 15%\\ \tt = 37%\\ \tt \\ \tt To calculate P(xevu|rixq), we can use the formula for conditional probability:\\ \tt \\ \tt P(xevu|rixq) = P(xevu and rixq) / P(rixq)\\ \tt = 22% / 40%\\ \tt = 
	  } 
	   } 
	   } 
	 & & \\ 
 

    \theutterance \stepcounter{utterance}  

    & & & \multicolumn{2}{p{0.3\linewidth}}{\cellcolor[rgb]{0.95,0.95,0.95}{%
	\makecell[{{p{\linewidth}}}]{% 
	  \tt {\tiny [GM$|$GM]}  
	 No 
	  } 
	   } 
	   } 
	 & & \\ 
 

    \theutterance \stepcounter{utterance}  

    & & & \multicolumn{2}{p{0.3\linewidth}}{\cellcolor[rgb]{0.95,0.95,0.95}{%
	\makecell[{{p{\linewidth}}}]{% 
	  \tt {\tiny [GM$|$GM]}  
	 no 
	  } 
	   } 
	   } 
	 & & \\ 
 

    \theutterance \stepcounter{utterance}  

    & & & \multicolumn{2}{p{0.3\linewidth}}{\cellcolor[rgb]{0.95,0.95,0.95}{%
	\makecell[{{p{\linewidth}}}]{% 
	  \tt {\tiny [GM$|$GM]}  
	 game_result = WIN 
	  } 
	   } 
	   } 
	 & & \\ 
 

\end{supertabular}
}

\end{document}
