
\documentclass{article}
\usepackage{colortbl}
\usepackage{makecell}
\usepackage{multirow}
\usepackage{supertabular}

\begin{document}

\newcounter{utterance}

\twocolumn

{ \footnotesize  \setcounter{utterance}{1}
\setlength{\tabcolsep}{0pt}
\begin{supertabular}{c@{$\;$}|p{.15\linewidth}@{}p{.15\linewidth}p{.15\linewidth}p{.15\linewidth}p{.15\linewidth}p{.15\linewidth}}

    \# & $\;$A & \multicolumn{4}{c}{Game Master} & $\;\:$B\\
    \hline 

    \theutterance \stepcounter{utterance}  

    & & \multicolumn{4}{p{0.6\linewidth}}{\cellcolor[rgb]{0.9,0.9,0.9}{%
	\makecell[{{p{\linewidth}}}]{% 
	  \tt {\tiny [A$\langle$GM]}  
	 You are an expert in causal inference. The following question is not a typical commonsense query, but rather a meticulously designed question created by a professor specializing in causal inference, intended to assess the students' mastery of the course content.\\ \tt Imagine a self-contained, hypothetical world with only the following conditions, and without any unmentioned factors or causal relationships: Education level has a direct effect on skill. Skill has a direct effect on black hair. The overall probability of college degree or higher is 6%. The probability of high school degree or lower and black hair is 25%. The probability of college degree or higher and black hair is 3%. Is the chance of black hair smaller when observing college degree or higher?\\ \tt Start your answer with 'Yes' or 'No', followed by additional reasoning or evidence to support your explanation. 
	  } 
	   } 
	   } 
	 & \\ 
 

    \theutterance \stepcounter{utterance}  

    & \multicolumn{4}{p{0.6\linewidth}}{\cellcolor[rgb]{0.8,1,0.9}{%
	\makecell[{{p{\linewidth}}}]{% 
	  \tt {\tiny [A$\rangle$GM]}  
	 No.\\ \tt \\ \tt To answer this question, we need to apply the concept of conditional probability and the causal relationships given in the problem. \\ \tt \\ \tt Let's denote the following events:\\ \tt - E: Education level is college degree or higher\\ \tt - S: Skill level\\ \tt - B: Black hair\\ \tt \\ \tt Given the direct effects:\\ \tt - E → S (Education level affects skill)\\ \tt - S → B (Skill affects black hair)\\ \tt \\ \tt We are asked to compare the probability of black hair (B) when observing college degree or higher (E) with the overall probability of black hair.\\ \tt \\ \tt First, we need to find the probability of black hair given college degree or higher (P(B|E)). We can use the law of total probability to calculate this:\\ \tt \\ \tt P(B|E) = P(B ∩ E) / P(E)\\ \tt \\ \tt We are given:\\ \tt - P(B ∩ E) = 3% (probability of college degree or higher and black hair)\\ \tt - P(E) = 6% (probability of college degree or higher)\\ \tt \\ \tt P(B|E) = 3% / 6% = 0.5\\ \tt \\ \tt Now, let's compare this with the overall probability of black hair (P(B)). We are given:\\ \tt - P(B ∩ HS) = 25% (probability of high school degree or lower and black hair)\\ \tt \\ \tt Since the only other possible combination is college degree or higher and black hair (3%), we can calculate the overall probability of black hair as follows:\\ \tt \\ \tt P(B) = P 
	  } 
	   } 
	   } 
	 & & \\ 
 

    \theutterance \stepcounter{utterance}  

    & & & \multicolumn{2}{p{0.3\linewidth}}{\cellcolor[rgb]{0.95,0.95,0.95}{%
	\makecell[{{p{\linewidth}}}]{% 
	  \tt {\tiny [GM$|$GM]}  
	 No 
	  } 
	   } 
	   } 
	 & & \\ 
 

    \theutterance \stepcounter{utterance}  

    & & & \multicolumn{2}{p{0.3\linewidth}}{\cellcolor[rgb]{0.95,0.95,0.95}{%
	\makecell[{{p{\linewidth}}}]{% 
	  \tt {\tiny [GM$|$GM]}  
	 no 
	  } 
	   } 
	   } 
	 & & \\ 
 

    \theutterance \stepcounter{utterance}  

    & & & \multicolumn{2}{p{0.3\linewidth}}{\cellcolor[rgb]{0.95,0.95,0.95}{%
	\makecell[{{p{\linewidth}}}]{% 
	  \tt {\tiny [GM$|$GM]}  
	 game_result = WIN 
	  } 
	   } 
	   } 
	 & & \\ 
 

\end{supertabular}
}

\end{document}
