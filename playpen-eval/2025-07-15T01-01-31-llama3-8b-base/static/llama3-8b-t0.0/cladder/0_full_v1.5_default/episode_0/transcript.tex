
\documentclass{article}
\usepackage{colortbl}
\usepackage{makecell}
\usepackage{multirow}
\usepackage{supertabular}

\begin{document}

\newcounter{utterance}

\twocolumn

{ \footnotesize  \setcounter{utterance}{1}
\setlength{\tabcolsep}{0pt}
\begin{supertabular}{c@{$\;$}|p{.15\linewidth}@{}p{.15\linewidth}p{.15\linewidth}p{.15\linewidth}p{.15\linewidth}p{.15\linewidth}}

    \# & $\;$A & \multicolumn{4}{c}{Game Master} & $\;\:$B\\
    \hline 

    \theutterance \stepcounter{utterance}  

    & & \multicolumn{4}{p{0.6\linewidth}}{\cellcolor[rgb]{0.9,0.9,0.9}{%
	\makecell[{{p{\linewidth}}}]{% 
	  \tt {\tiny [A$\langle$GM]}  
	 You are an expert in causal inference. The following question is not a typical commonsense query, but rather a meticulously designed question created by a professor specializing in causal inference, intended to assess the students' mastery of the course content.\\ \tt Imagine a self-contained, hypothetical world with only the following conditions, and without any unmentioned factors or causal relationships: Husband has a direct effect on wife and alarm clock. Wife has a direct effect on alarm clock. For husbands that don't set the alarm and wives that don't set the alarm, the probability of ringing alarm is 11%. For husbands that don't set the alarm and wives that set the alarm, the probability of ringing alarm is 60%. For husbands that set the alarm and wives that don't set the alarm, the probability of ringing alarm is 46%. For husbands that set the alarm and wives that set the alarm, the probability of ringing alarm is 92%. For husbands that don't set the alarm, the probability of alarm set by wife is 61%. For husbands that set the alarm, the probability of alarm set by wife is 1%. Does husband positively affect alarm clock through wife?\\ \tt Start your answer with 'Yes' or 'No', followed by additional reasoning or evidence to support your explanation. 
	  } 
	   } 
	   } 
	 & \\ 
 

    \theutterance \stepcounter{utterance}  

    & \multicolumn{4}{p{0.6\linewidth}}{\cellcolor[rgb]{0.8,1,0.9}{%
	\makecell[{{p{\linewidth}}}]{% 
	  \tt {\tiny [A$\rangle$GM]}  
	 No, husband does not positively affect alarm clock through wife.\\ \tt \\ \tt To assess this, we can use the backdoor criterion, which is a method for identifying causal relationships in the presence of confounding variables. In this case, the wife is the potential confounding variable.\\ \tt \\ \tt We can calculate the probability of the alarm ringing given that the husband sets the alarm and the wife does not set the alarm (P(Ring|Husband sets, Wife does not set)), and compare it to the probability of the alarm ringing given that the husband does not set the alarm and the wife does not set the alarm (P(Ring|Husband does not set, Wife does not set)).\\ \tt \\ \tt P(Ring|Husband sets, Wife does not set) = 46% (given in the problem statement)\\ \tt \\ \tt P(Ring|Husband does not set, Wife does not set) = 11% (given in the problem statement)\\ \tt \\ \tt Since P(Ring|Husband sets, Wife does not set) > P(Ring|Husband does not set, Wife does not set), we might initially think that the husband has a positive effect on the alarm clock through the wife. However, this is not the case.\\ \tt \\ \tt To see why, let's calculate the probability of the wife setting the alarm given that the husband does not set the alarm (P(Wife sets|Husband does not set)) and compare it to the probability of the wife setting the alarm given that the husband sets the alarm (P(W 
	  } 
	   } 
	   } 
	 & & \\ 
 

    \theutterance \stepcounter{utterance}  

    & & & \multicolumn{2}{p{0.3\linewidth}}{\cellcolor[rgb]{0.95,0.95,0.95}{%
	\makecell[{{p{\linewidth}}}]{% 
	  \tt {\tiny [GM$|$GM]}  
	 No 
	  } 
	   } 
	   } 
	 & & \\ 
 

    \theutterance \stepcounter{utterance}  

    & & & \multicolumn{2}{p{0.3\linewidth}}{\cellcolor[rgb]{0.95,0.95,0.95}{%
	\makecell[{{p{\linewidth}}}]{% 
	  \tt {\tiny [GM$|$GM]}  
	 no 
	  } 
	   } 
	   } 
	 & & \\ 
 

    \theutterance \stepcounter{utterance}  

    & & & \multicolumn{2}{p{0.3\linewidth}}{\cellcolor[rgb]{0.95,0.95,0.95}{%
	\makecell[{{p{\linewidth}}}]{% 
	  \tt {\tiny [GM$|$GM]}  
	 game_result = WIN 
	  } 
	   } 
	   } 
	 & & \\ 
 

\end{supertabular}
}

\end{document}
