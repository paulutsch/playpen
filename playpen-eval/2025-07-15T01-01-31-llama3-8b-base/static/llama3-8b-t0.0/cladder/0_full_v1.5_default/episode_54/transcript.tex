
\documentclass{article}
\usepackage{colortbl}
\usepackage{makecell}
\usepackage{multirow}
\usepackage{supertabular}

\begin{document}

\newcounter{utterance}

\twocolumn

{ \footnotesize  \setcounter{utterance}{1}
\setlength{\tabcolsep}{0pt}
\begin{supertabular}{c@{$\;$}|p{.15\linewidth}@{}p{.15\linewidth}p{.15\linewidth}p{.15\linewidth}p{.15\linewidth}p{.15\linewidth}}

    \# & $\;$A & \multicolumn{4}{c}{Game Master} & $\;\:$B\\
    \hline 

    \theutterance \stepcounter{utterance}  

    & & \multicolumn{4}{p{0.6\linewidth}}{\cellcolor[rgb]{0.9,0.9,0.9}{%
	\makecell[{{p{\linewidth}}}]{% 
	  \tt {\tiny [A$\langle$GM]}  
	 You are an expert in causal inference. The following question is not a typical commonsense query, but rather a meticulously designed question created by a professor specializing in causal inference, intended to assess the students' mastery of the course content.\\ \tt Imagine a self-contained, hypothetical world with only the following conditions, and without any unmentioned factors or causal relationships: Ability to speak english has a direct effect on the forest. The smoker has a direct effect on the forest. The overall probability of speaking english is 93%. For people who do not speak english, the probability of the forest on fire is 79%. For people who speak english, the probability of the forest on fire is 65%. Is the forest on fire more likely than the forest not on fire overall?\\ \tt Start your answer with 'Yes' or 'No', followed by additional reasoning or evidence to support your explanation. 
	  } 
	   } 
	   } 
	 & \\ 
 

    \theutterance \stepcounter{utterance}  

    & \multicolumn{4}{p{0.6\linewidth}}{\cellcolor[rgb]{0.8,1,0.9}{%
	\makecell[{{p{\linewidth}}}]{% 
	  \tt {\tiny [A$\rangle$GM]}  
	 No.\\ \tt \\ \tt To answer this question, we need to apply the principles of causal inference, specifically the concept of confounding variables and the potential outcomes framework. \\ \tt \\ \tt Let's define the following variables:\\ \tt \\ \tt - E: Ability to speak English\\ \tt - F: Forest on fire\\ \tt - S: Smoker\\ \tt \\ \tt We are given the following information:\\ \tt \\ \tt - P(E) = 0.93 (probability of speaking English)\\ \tt - P(F|E') = 0.79 (probability of forest on fire given no English)\\ \tt - P(F|E) = 0.65 (probability of forest on fire given English)\\ \tt \\ \tt We are also told that there is a direct effect of speaking English on the forest and a direct effect of smoking on the forest. However, we are not given any information about the relationship between smoking and speaking English.\\ \tt \\ \tt To determine whether the forest is more likely to be on fire than not, we need to calculate the overall probability of the forest being on fire, P(F). We can do this using the law of total probability:\\ \tt \\ \tt P(F) = P(F|E)P(E) + P(F|E')P(E')\\ \tt \\ \tt First, we need to calculate P(E'):\\ \tt \\ \tt P(E') = 1 - P(E) = 1 - 0.93 = 0.07\\ \tt \\ \tt Now, we can plug in the values:\\ \tt \\ \tt P(F) = P(F|E)P(E) + P(F|E')P(E')\\ \tt = 0.65(0 
	  } 
	   } 
	   } 
	 & & \\ 
 

    \theutterance \stepcounter{utterance}  

    & & & \multicolumn{2}{p{0.3\linewidth}}{\cellcolor[rgb]{0.95,0.95,0.95}{%
	\makecell[{{p{\linewidth}}}]{% 
	  \tt {\tiny [GM$|$GM]}  
	 No 
	  } 
	   } 
	   } 
	 & & \\ 
 

    \theutterance \stepcounter{utterance}  

    & & & \multicolumn{2}{p{0.3\linewidth}}{\cellcolor[rgb]{0.95,0.95,0.95}{%
	\makecell[{{p{\linewidth}}}]{% 
	  \tt {\tiny [GM$|$GM]}  
	 yes 
	  } 
	   } 
	   } 
	 & & \\ 
 

    \theutterance \stepcounter{utterance}  

    & & & \multicolumn{2}{p{0.3\linewidth}}{\cellcolor[rgb]{0.95,0.95,0.95}{%
	\makecell[{{p{\linewidth}}}]{% 
	  \tt {\tiny [GM$|$GM]}  
	 game_result = LOSE 
	  } 
	   } 
	   } 
	 & & \\ 
 

\end{supertabular}
}

\end{document}
