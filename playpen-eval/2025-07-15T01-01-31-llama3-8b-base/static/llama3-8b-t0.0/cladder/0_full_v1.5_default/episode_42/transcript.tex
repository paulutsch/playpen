
\documentclass{article}
\usepackage{colortbl}
\usepackage{makecell}
\usepackage{multirow}
\usepackage{supertabular}

\begin{document}

\newcounter{utterance}

\twocolumn

{ \footnotesize  \setcounter{utterance}{1}
\setlength{\tabcolsep}{0pt}
\begin{supertabular}{c@{$\;$}|p{.15\linewidth}@{}p{.15\linewidth}p{.15\linewidth}p{.15\linewidth}p{.15\linewidth}p{.15\linewidth}}

    \# & $\;$A & \multicolumn{4}{c}{Game Master} & $\;\:$B\\
    \hline 

    \theutterance \stepcounter{utterance}  

    & & \multicolumn{4}{p{0.6\linewidth}}{\cellcolor[rgb]{0.9,0.9,0.9}{%
	\makecell[{{p{\linewidth}}}]{% 
	  \tt {\tiny [A$\langle$GM]}  
	 You are an expert in causal inference. The following question is not a typical commonsense query, but rather a meticulously designed question created by a professor specializing in causal inference, intended to assess the students' mastery of the course content.\\ \tt Imagine a self-contained, hypothetical world with only the following conditions, and without any unmentioned factors or causal relationships: Gender has a direct effect on department competitiveness and peanut allergy. Department competitiveness has a direct effect on peanut allergy. For individuals who are not male, the probability of being allergic to peanuts is 48%. For individuals who are male, the probability of being allergic to peanuts is 36%. For individuals who are male, would it be more likely to see being allergic to peanuts if the individual had not been male?\\ \tt Start your answer with 'Yes' or 'No', followed by additional reasoning or evidence to support your explanation. 
	  } 
	   } 
	   } 
	 & \\ 
 

    \theutterance \stepcounter{utterance}  

    & \multicolumn{4}{p{0.6\linewidth}}{\cellcolor[rgb]{0.8,1,0.9}{%
	\makecell[{{p{\linewidth}}}]{% 
	  \tt {\tiny [A$\rangle$GM]}  
	 No.\\ \tt \\ \tt To address this question, we need to apply the concept of counterfactuals and potential outcomes. The question essentially asks: "If an individual who is male were not male, would it be more likely for them to be allergic to peanuts?"\\ \tt \\ \tt Let's denote the potential outcome of being allergic to peanuts as Y, the gender as X, and the counterfactual gender as X'. We are interested in the probability of Y given X' (i.e., P(Y|X')).\\ \tt \\ \tt Given the information provided:\\ \tt \\ \tt - P(Y|X=Male) = 36%\\ \tt - P(Y|X=Not Male) = 48%\\ \tt \\ \tt We can calculate the probability of Y given X' (i.e., P(Y|X'=Not Male)) by considering the effect of the counterfactual gender on the probability of being allergic to peanuts. Since the direct effect of gender on peanut allergy is given, we can use the law of total probability to calculate the counterfactual probability.\\ \tt \\ \tt However, the given information does not provide a direct causal relationship between being male and the probability of being allergic to peanuts. Instead, it provides the probabilities of being allergic to peanuts given the gender. \\ \tt \\ \tt To answer the question, we need to consider the effect of department competitiveness on peanut allergy. Since department competitiveness has a direct effect on peanut allergy, and gender has a direct effect on department competitiveness, we can infer that the effect of gender on peanut allergy is mediated by department competitiveness.\\ \tt \\ \tt Given the information provided, 
	  } 
	   } 
	   } 
	 & & \\ 
 

    \theutterance \stepcounter{utterance}  

    & & & \multicolumn{2}{p{0.3\linewidth}}{\cellcolor[rgb]{0.95,0.95,0.95}{%
	\makecell[{{p{\linewidth}}}]{% 
	  \tt {\tiny [GM$|$GM]}  
	 No 
	  } 
	   } 
	   } 
	 & & \\ 
 

    \theutterance \stepcounter{utterance}  

    & & & \multicolumn{2}{p{0.3\linewidth}}{\cellcolor[rgb]{0.95,0.95,0.95}{%
	\makecell[{{p{\linewidth}}}]{% 
	  \tt {\tiny [GM$|$GM]}  
	 yes 
	  } 
	   } 
	   } 
	 & & \\ 
 

    \theutterance \stepcounter{utterance}  

    & & & \multicolumn{2}{p{0.3\linewidth}}{\cellcolor[rgb]{0.95,0.95,0.95}{%
	\makecell[{{p{\linewidth}}}]{% 
	  \tt {\tiny [GM$|$GM]}  
	 game_result = LOSE 
	  } 
	   } 
	   } 
	 & & \\ 
 

\end{supertabular}
}

\end{document}
