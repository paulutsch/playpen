
\documentclass{article}
\usepackage{colortbl}
\usepackage{makecell}
\usepackage{multirow}
\usepackage{supertabular}

\begin{document}

\newcounter{utterance}

\twocolumn

{ \footnotesize  \setcounter{utterance}{1}
\setlength{\tabcolsep}{0pt}
\begin{supertabular}{c@{$\;$}|p{.15\linewidth}@{}p{.15\linewidth}p{.15\linewidth}p{.15\linewidth}p{.15\linewidth}p{.15\linewidth}}

    \# & $\;$A & \multicolumn{4}{c}{Game Master} & $\;\:$B\\
    \hline 

    \theutterance \stepcounter{utterance}  

    & & \multicolumn{4}{p{0.6\linewidth}}{\cellcolor[rgb]{0.9,0.9,0.9}{%
	\makecell[{{p{\linewidth}}}]{% 
	  \tt {\tiny [A$\langle$GM]}  
	 You are an expert in causal inference. The following question is not a typical commonsense query, but rather a meticulously designed question created by a professor specializing in causal inference, intended to assess the students' mastery of the course content.\\ \tt Imagine a self-contained, hypothetical world with only the following conditions, and without any unmentioned factors or causal relationships: Hwax has a direct effect on kraz and pexu. Pexu has a direct effect on rukz. Kraz has a direct effect on rukz. The overall probability of pexu is 41%. For those who are not pexu, the probability of rukz is 66%. For those who are pexu, the probability of rukz is 22%. Is rukz more likely than not rukz overall?\\ \tt Start your answer with 'Yes' or 'No', followed by additional reasoning or evidence to support your explanation. 
	  } 
	   } 
	   } 
	 & \\ 
 

    \theutterance \stepcounter{utterance}  

    & \multicolumn{4}{p{0.6\linewidth}}{\cellcolor[rgb]{0.8,1,0.9}{%
	\makecell[{{p{\linewidth}}}]{% 
	  \tt {\tiny [A$\rangle$GM]}  
	 No, rukz is not more likely than not rukz overall.\\ \tt \\ \tt To determine the overall probability of rukz, we need to consider the probabilities of rukz given different values of pexu and the probability of pexu itself. \\ \tt \\ \tt Let's denote the probability of rukz given pexu as P(Rukz|Pexu) and the probability of rukz given not pexu as P(Rukz|~Pexu). We are given that P(Rukz|~Pexu) = 0.66 and P(Rukz|Pexu) = 0.22.\\ \tt \\ \tt We also know that P(Pexu) = 0.41. To find the probability of not pexu, we can subtract this value from 1: P(~Pexu) = 1 - P(Pexu) = 1 - 0.41 = 0.59.\\ \tt \\ \tt Now, we can use the law of total probability to find the overall probability of rukz:\\ \tt \\ \tt P(Rukz) = P(Rukz|Pexu) * P(Pexu) + P(Rukz|~Pexu) * P(~Pexu)\\ \tt = 0.22 * 0.41 + 0.66 * 0.59\\ \tt = 0.0902 + 0.3894\\ \tt = 0.4796\\ \tt \\ \tt Since 0.4796 is less than 
	  } 
	   } 
	   } 
	 & & \\ 
 

    \theutterance \stepcounter{utterance}  

    & & & \multicolumn{2}{p{0.3\linewidth}}{\cellcolor[rgb]{0.95,0.95,0.95}{%
	\makecell[{{p{\linewidth}}}]{% 
	  \tt {\tiny [GM$|$GM]}  
	 No 
	  } 
	   } 
	   } 
	 & & \\ 
 

    \theutterance \stepcounter{utterance}  

    & & & \multicolumn{2}{p{0.3\linewidth}}{\cellcolor[rgb]{0.95,0.95,0.95}{%
	\makecell[{{p{\linewidth}}}]{% 
	  \tt {\tiny [GM$|$GM]}  
	 no 
	  } 
	   } 
	   } 
	 & & \\ 
 

    \theutterance \stepcounter{utterance}  

    & & & \multicolumn{2}{p{0.3\linewidth}}{\cellcolor[rgb]{0.95,0.95,0.95}{%
	\makecell[{{p{\linewidth}}}]{% 
	  \tt {\tiny [GM$|$GM]}  
	 game_result = WIN 
	  } 
	   } 
	   } 
	 & & \\ 
 

\end{supertabular}
}

\end{document}
