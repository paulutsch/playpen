
\documentclass{article}
\usepackage{colortbl}
\usepackage{makecell}
\usepackage{multirow}
\usepackage{supertabular}

\begin{document}

\newcounter{utterance}

\twocolumn

{ \footnotesize  \setcounter{utterance}{1}
\setlength{\tabcolsep}{0pt}
\begin{supertabular}{c@{$\;$}|p{.15\linewidth}@{}p{.15\linewidth}p{.15\linewidth}p{.15\linewidth}p{.15\linewidth}p{.15\linewidth}}

    \# & $\;$A & \multicolumn{4}{c}{Game Master} & $\;\:$B\\
    \hline 

    \theutterance \stepcounter{utterance}  

    & & \multicolumn{4}{p{0.6\linewidth}}{\cellcolor[rgb]{0.9,0.9,0.9}{%
	\makecell[{{p{\linewidth}}}]{% 
	  \tt {\tiny [A$\langle$GM]}  
	 You are an expert in causal inference. The following question is not a typical commonsense query, but rather a meticulously designed question created by a professor specializing in causal inference, intended to assess the students' mastery of the course content.\\ \tt Imagine a self-contained, hypothetical world with only the following conditions, and without any unmentioned factors or causal relationships: Kidney stone size has a direct effect on treatment and lip thickness. Treatment has a direct effect on lip thickness. For patients who have small kidney stones and not receiving treatment, the probability of thick lips is 6%. For patients who have small kidney stones and receiving treatment, the probability of thick lips is 38%. For patients who have large kidney stones and not receiving treatment, the probability of thick lips is 63%. For patients who have large kidney stones and receiving treatment, the probability of thick lips is 95%. The overall probability of large kidney stone is 50%. For patients receiving treatment, would it be more likely to see thick lips if the patient had received no treatment?\\ \tt Start your answer with 'Yes' or 'No', followed by additional reasoning or evidence to support your explanation. 
	  } 
	   } 
	   } 
	 & \\ 
 

    \theutterance \stepcounter{utterance}  

    & \multicolumn{4}{p{0.6\linewidth}}{\cellcolor[rgb]{0.8,1,0.9}{%
	\makecell[{{p{\linewidth}}}]{% 
	  \tt {\tiny [A$\rangle$GM]}  
	 No.\\ \tt \\ \tt To answer this question, we need to apply the concept of counterfactuals and potential outcomes. We are given the following probabilities:\\ \tt \\ \tt - P(thick lips | small kidney stone, no treatment) = 0.06\\ \tt - P(thick lips | small kidney stone, treatment) = 0.38\\ \tt - P(thick lips | large kidney stone, no treatment) = 0.63\\ \tt - P(thick lips | large kidney stone, treatment) = 0.95\\ \tt - P(large kidney stone) = 0.5\\ \tt \\ \tt We want to find the probability of thick lips for a patient who received treatment, assuming they would have had no treatment otherwise. This is equivalent to finding the probability of thick lips in the counterfactual scenario where the patient received no treatment.\\ \tt \\ \tt Using the law of total probability, we can calculate the probability of thick lips for a patient who received treatment as follows:\\ \tt \\ \tt P(thick lips | treatment) = P(thick lips | treatment, small kidney stone)P(small kidney stone | treatment) + P(thick lips | treatment, large kidney stone)P(large kidney stone | treatment)\\ \tt \\ \tt We can calculate the probabilities of small and large kidney stones given treatment using Bayes' theorem:\\ \tt \\ \tt P(small kidney stone | treatment) = P(treatment | small kidney stone)P(small kidney stone) / P(treatment)\\ \tt = (P(treatment, small kidney stone) + P(treatment, large kidney stone 
	  } 
	   } 
	   } 
	 & & \\ 
 

    \theutterance \stepcounter{utterance}  

    & & & \multicolumn{2}{p{0.3\linewidth}}{\cellcolor[rgb]{0.95,0.95,0.95}{%
	\makecell[{{p{\linewidth}}}]{% 
	  \tt {\tiny [GM$|$GM]}  
	 No 
	  } 
	   } 
	   } 
	 & & \\ 
 

    \theutterance \stepcounter{utterance}  

    & & & \multicolumn{2}{p{0.3\linewidth}}{\cellcolor[rgb]{0.95,0.95,0.95}{%
	\makecell[{{p{\linewidth}}}]{% 
	  \tt {\tiny [GM$|$GM]}  
	 no 
	  } 
	   } 
	   } 
	 & & \\ 
 

    \theutterance \stepcounter{utterance}  

    & & & \multicolumn{2}{p{0.3\linewidth}}{\cellcolor[rgb]{0.95,0.95,0.95}{%
	\makecell[{{p{\linewidth}}}]{% 
	  \tt {\tiny [GM$|$GM]}  
	 game_result = WIN 
	  } 
	   } 
	   } 
	 & & \\ 
 

\end{supertabular}
}

\end{document}
