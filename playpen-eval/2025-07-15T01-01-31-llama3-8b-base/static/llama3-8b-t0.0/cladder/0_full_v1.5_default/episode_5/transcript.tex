
\documentclass{article}
\usepackage{colortbl}
\usepackage{makecell}
\usepackage{multirow}
\usepackage{supertabular}

\begin{document}

\newcounter{utterance}

\twocolumn

{ \footnotesize  \setcounter{utterance}{1}
\setlength{\tabcolsep}{0pt}
\begin{supertabular}{c@{$\;$}|p{.15\linewidth}@{}p{.15\linewidth}p{.15\linewidth}p{.15\linewidth}p{.15\linewidth}p{.15\linewidth}}

    \# & $\;$A & \multicolumn{4}{c}{Game Master} & $\;\:$B\\
    \hline 

    \theutterance \stepcounter{utterance}  

    & & \multicolumn{4}{p{0.6\linewidth}}{\cellcolor[rgb]{0.9,0.9,0.9}{%
	\makecell[{{p{\linewidth}}}]{% 
	  \tt {\tiny [A$\langle$GM]}  
	 You are an expert in causal inference. The following question is not a typical commonsense query, but rather a meticulously designed question created by a professor specializing in causal inference, intended to assess the students' mastery of the course content.\\ \tt Imagine a self-contained, hypothetical world with only the following conditions, and without any unmentioned factors or causal relationships: Vaccination status has a direct effect on vaccination reaction and getting smallpox. Getting smallpox has a direct effect on smallpox survival. Vaccination reaction has a direct effect on smallpox survival. Method 1: We look directly at how vaccination status correlates with smallpox survival in general. Method 2: We look at this correlation case by case according to vaccination reaction. To understand how vaccination status affects smallpox survival, is it more correct to use the Method 1 than Method 2?\\ \tt Start your answer with 'Yes' or 'No', followed by additional reasoning or evidence to support your explanation. 
	  } 
	   } 
	   } 
	 & \\ 
 

    \theutterance \stepcounter{utterance}  

    & \multicolumn{4}{p{0.6\linewidth}}{\cellcolor[rgb]{0.8,1,0.9}{%
	\makecell[{{p{\linewidth}}}]{% 
	  \tt {\tiny [A$\rangle$GM]}  
	 No, it is not more correct to use Method 1 than Method 2.\\ \tt \\ \tt In this hypothetical world, we have a complex causal relationship between vaccination status, vaccination reaction, getting smallpox, and smallpox survival. To accurately understand the effect of vaccination status on smallpox survival, we need to consider the indirect effect of vaccination status on smallpox survival through vaccination reaction.\\ \tt \\ \tt Method 1, which looks directly at the correlation between vaccination status and smallpox survival in general, may lead to biased estimates due to the presence of an unmeasured confounder (vaccination reaction). This is because individuals with a severe vaccination reaction may be more likely to avoid getting smallpox, which in turn affects their smallpox survival. By not accounting for vaccination reaction, Method 1 may attribute the effect of vaccination status on smallpox survival to the vaccination status itself, rather than the underlying causal pathway.\\ \tt \\ \tt In contrast, Method 2, which looks at the correlation case by case according to vaccination reaction, allows us to control for the confounding effect of vaccination reaction. By stratifying the analysis by vaccination reaction, we can estimate the effect of vaccination status on smallpox survival within each stratum, which provides a more accurate representation of the underlying causal relationship.\\ \tt \\ \tt Therefore, Method 2 is a more correct approach than Method 1, as it allows us to account for the indirect effect of vaccination status on smallpox survival through vaccination reaction, and provides a more 
	  } 
	   } 
	   } 
	 & & \\ 
 

    \theutterance \stepcounter{utterance}  

    & & & \multicolumn{2}{p{0.3\linewidth}}{\cellcolor[rgb]{0.95,0.95,0.95}{%
	\makecell[{{p{\linewidth}}}]{% 
	  \tt {\tiny [GM$|$GM]}  
	 No 
	  } 
	   } 
	   } 
	 & & \\ 
 

    \theutterance \stepcounter{utterance}  

    & & & \multicolumn{2}{p{0.3\linewidth}}{\cellcolor[rgb]{0.95,0.95,0.95}{%
	\makecell[{{p{\linewidth}}}]{% 
	  \tt {\tiny [GM$|$GM]}  
	 yes 
	  } 
	   } 
	   } 
	 & & \\ 
 

    \theutterance \stepcounter{utterance}  

    & & & \multicolumn{2}{p{0.3\linewidth}}{\cellcolor[rgb]{0.95,0.95,0.95}{%
	\makecell[{{p{\linewidth}}}]{% 
	  \tt {\tiny [GM$|$GM]}  
	 game_result = LOSE 
	  } 
	   } 
	   } 
	 & & \\ 
 

\end{supertabular}
}

\end{document}
