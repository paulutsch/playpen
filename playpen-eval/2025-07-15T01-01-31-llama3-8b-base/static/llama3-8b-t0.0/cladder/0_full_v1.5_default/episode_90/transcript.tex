
\documentclass{article}
\usepackage{colortbl}
\usepackage{makecell}
\usepackage{multirow}
\usepackage{supertabular}

\begin{document}

\newcounter{utterance}

\twocolumn

{ \footnotesize  \setcounter{utterance}{1}
\setlength{\tabcolsep}{0pt}
\begin{supertabular}{c@{$\;$}|p{.15\linewidth}@{}p{.15\linewidth}p{.15\linewidth}p{.15\linewidth}p{.15\linewidth}p{.15\linewidth}}

    \# & $\;$A & \multicolumn{4}{c}{Game Master} & $\;\:$B\\
    \hline 

    \theutterance \stepcounter{utterance}  

    & & \multicolumn{4}{p{0.6\linewidth}}{\cellcolor[rgb]{0.9,0.9,0.9}{%
	\makecell[{{p{\linewidth}}}]{% 
	  \tt {\tiny [A$\langle$GM]}  
	 You are an expert in causal inference. The following question is not a typical commonsense query, but rather a meticulously designed question created by a professor specializing in causal inference, intended to assess the students' mastery of the course content.\\ \tt Imagine a self-contained, hypothetical world with only the following conditions, and without any unmentioned factors or causal relationships: Pexu has a direct effect on rukz. Hwax has a direct effect on rukz. For those who are not pexu, the probability of rukz is 9%. For those who are pexu, the probability of rukz is 18%. For those who are pexu, would it be less likely to see rukz if the individual was not pexu?\\ \tt Start your answer with 'Yes' or 'No', followed by additional reasoning or evidence to support your explanation. 
	  } 
	   } 
	   } 
	 & \\ 
 

    \theutterance \stepcounter{utterance}  

    & \multicolumn{4}{p{0.6\linewidth}}{\cellcolor[rgb]{0.8,1,0.9}{%
	\makecell[{{p{\linewidth}}}]{% 
	  \tt {\tiny [A$\rangle$GM]}  
	 No.\\ \tt \\ \tt To answer this question, we need to consider the causal relationships and probabilities given. We know that Pexu has a direct effect on rukz, and for those who are pexu, the probability of rukz is 18%. This implies that Pexu is a cause of rukz.\\ \tt \\ \tt We are asked to compare the probability of rukz for those who are pexu with the probability of rukz for those who are not pexu, but hypothetically not pexu. This is a counterfactual question, which can be addressed using the potential outcomes framework.\\ \tt \\ \tt Let's denote the potential outcome of rukz for an individual who is pexu as Y(1), and the potential outcome of rukz for an individual who is not pexu as Y(0). We are given that P(Y(1)) = 0.18 and P(Y(0)) = 0.09.\\ \tt \\ \tt The question asks whether it would be less likely to see rukz if the individual was not pexu, given that they are pexu. This is equivalent to asking whether the probability of rukz for an individual who is pexu is less than the probability of rukz for an individual who is not pexu, but would not be pexu if they were not.\\ \tt \\ \tt Using the potential outcomes framework, we can express this as:\\ \tt \\ \tt P(Y(0) | Y(1)) < P(Y(1))\\ \tt \\ \tt However, we 
	  } 
	   } 
	   } 
	 & & \\ 
 

    \theutterance \stepcounter{utterance}  

    & & & \multicolumn{2}{p{0.3\linewidth}}{\cellcolor[rgb]{0.95,0.95,0.95}{%
	\makecell[{{p{\linewidth}}}]{% 
	  \tt {\tiny [GM$|$GM]}  
	 No 
	  } 
	   } 
	   } 
	 & & \\ 
 

    \theutterance \stepcounter{utterance}  

    & & & \multicolumn{2}{p{0.3\linewidth}}{\cellcolor[rgb]{0.95,0.95,0.95}{%
	\makecell[{{p{\linewidth}}}]{% 
	  \tt {\tiny [GM$|$GM]}  
	 yes 
	  } 
	   } 
	   } 
	 & & \\ 
 

    \theutterance \stepcounter{utterance}  

    & & & \multicolumn{2}{p{0.3\linewidth}}{\cellcolor[rgb]{0.95,0.95,0.95}{%
	\makecell[{{p{\linewidth}}}]{% 
	  \tt {\tiny [GM$|$GM]}  
	 game_result = LOSE 
	  } 
	   } 
	   } 
	 & & \\ 
 

\end{supertabular}
}

\end{document}
