
\documentclass{article}
\usepackage{colortbl}
\usepackage{makecell}
\usepackage{multirow}
\usepackage{supertabular}

\begin{document}

\newcounter{utterance}

\twocolumn

{ \footnotesize  \setcounter{utterance}{1}
\setlength{\tabcolsep}{0pt}
\begin{supertabular}{c@{$\;$}|p{.15\linewidth}@{}p{.15\linewidth}p{.15\linewidth}p{.15\linewidth}p{.15\linewidth}p{.15\linewidth}}

    \# & $\;$A & \multicolumn{4}{c}{Game Master} & $\;\:$B\\
    \hline 

    \theutterance \stepcounter{utterance}  

    & & \multicolumn{4}{p{0.6\linewidth}}{\cellcolor[rgb]{0.9,0.9,0.9}{%
	\makecell[{{p{\linewidth}}}]{% 
	  \tt {\tiny [A$\langle$GM]}  
	 You are an expert in causal inference. The following question is not a typical commonsense query, but rather a meticulously designed question created by a professor specializing in causal inference, intended to assess the students' mastery of the course content.\\ \tt Imagine a self-contained, hypothetical world with only the following conditions, and without any unmentioned factors or causal relationships: Having a brother has a direct effect on wife and alarm clock. Wife has a direct effect on alarm clock. The overall probability of having a brother is 1%. The probability of not having a brother and ringing alarm is 64%. The probability of having a brother and ringing alarm is 0%. Is the chance of ringing alarm larger when observing having a brother?\\ \tt Start your answer with 'Yes' or 'No', followed by additional reasoning or evidence to support your explanation. 
	  } 
	   } 
	   } 
	 & \\ 
 

    \theutterance \stepcounter{utterance}  

    & \multicolumn{4}{p{0.6\linewidth}}{\cellcolor[rgb]{0.8,1,0.9}{%
	\makecell[{{p{\linewidth}}}]{% 
	  \tt {\tiny [A$\rangle$GM]}  
	 No.\\ \tt \\ \tt To answer this question, we need to apply the principles of causal inference, specifically the concept of confounding variables and the causal graph. \\ \tt \\ \tt Let's construct the causal graph based on the given conditions:\\ \tt \\ \tt - Having a brother (B) → Wife (W)\\ \tt - Wife (W) → Alarm Clock (AC)\\ \tt \\ \tt We are interested in the effect of B on AC. To assess this, we need to consider the probability distributions of the variables involved.\\ \tt \\ \tt Given:\\ \tt - P(B) = 0.01 (probability of having a brother)\\ \tt - P(¬B ∧ AC) = 0.64 (probability of not having a brother and ringing alarm)\\ \tt - P(B ∧ AC) = 0.00 (probability of having a brother and ringing alarm)\\ \tt \\ \tt We can calculate the probability of ringing alarm (AC) given having a brother (B) using the law of total probability:\\ \tt \\ \tt P(AC|B) = P(B ∧ AC) / P(B)\\ \tt = 0.00 / 0.01\\ \tt = 0.00\\ \tt \\ \tt However, we also need to consider the probability of ringing alarm (AC) given not having a brother (¬B):\\ \tt \\ \tt P(AC|¬B) = P(¬B ∧ AC) / P(¬B)\\ \tt = 0.64 / (1 - 0.01)\\ \tt = 0.64 / 0.99\\ \tt = 0.6465\\ \tt \\ \tt Comparing the two probabilities, we see that P 
	  } 
	   } 
	   } 
	 & & \\ 
 

    \theutterance \stepcounter{utterance}  

    & & & \multicolumn{2}{p{0.3\linewidth}}{\cellcolor[rgb]{0.95,0.95,0.95}{%
	\makecell[{{p{\linewidth}}}]{% 
	  \tt {\tiny [GM$|$GM]}  
	 No 
	  } 
	   } 
	   } 
	 & & \\ 
 

    \theutterance \stepcounter{utterance}  

    & & & \multicolumn{2}{p{0.3\linewidth}}{\cellcolor[rgb]{0.95,0.95,0.95}{%
	\makecell[{{p{\linewidth}}}]{% 
	  \tt {\tiny [GM$|$GM]}  
	 no 
	  } 
	   } 
	   } 
	 & & \\ 
 

    \theutterance \stepcounter{utterance}  

    & & & \multicolumn{2}{p{0.3\linewidth}}{\cellcolor[rgb]{0.95,0.95,0.95}{%
	\makecell[{{p{\linewidth}}}]{% 
	  \tt {\tiny [GM$|$GM]}  
	 game_result = WIN 
	  } 
	   } 
	   } 
	 & & \\ 
 

\end{supertabular}
}

\end{document}
