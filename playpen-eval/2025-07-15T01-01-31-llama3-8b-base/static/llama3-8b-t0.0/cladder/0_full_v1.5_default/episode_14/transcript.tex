
\documentclass{article}
\usepackage{colortbl}
\usepackage{makecell}
\usepackage{multirow}
\usepackage{supertabular}

\begin{document}

\newcounter{utterance}

\twocolumn

{ \footnotesize  \setcounter{utterance}{1}
\setlength{\tabcolsep}{0pt}
\begin{supertabular}{c@{$\;$}|p{.15\linewidth}@{}p{.15\linewidth}p{.15\linewidth}p{.15\linewidth}p{.15\linewidth}p{.15\linewidth}}

    \# & $\;$A & \multicolumn{4}{c}{Game Master} & $\;\:$B\\
    \hline 

    \theutterance \stepcounter{utterance}  

    & & \multicolumn{4}{p{0.6\linewidth}}{\cellcolor[rgb]{0.9,0.9,0.9}{%
	\makecell[{{p{\linewidth}}}]{% 
	  \tt {\tiny [A$\langle$GM]}  
	 You are an expert in causal inference. The following question is not a typical commonsense query, but rather a meticulously designed question created by a professor specializing in causal inference, intended to assess the students' mastery of the course content.\\ \tt Imagine a self-contained, hypothetical world with only the following conditions, and without any unmentioned factors or causal relationships: Gender has a direct effect on smoking and lung cancer. Smoking has a direct effect on tar deposit. Tar deposit has a direct effect on lung cancer. Gender is unobserved. The overall probability of smoking is 12%. For nonsmokers, the probability of lung cancer is 58%. For smokers, the probability of lung cancer is 61%. Is lung cancer more likely than absence of lung cancer overall?\\ \tt Start your answer with 'Yes' or 'No', followed by additional reasoning or evidence to support your explanation. 
	  } 
	   } 
	   } 
	 & \\ 
 

    \theutterance \stepcounter{utterance}  

    & \multicolumn{4}{p{0.6\linewidth}}{\cellcolor[rgb]{0.8,1,0.9}{%
	\makecell[{{p{\linewidth}}}]{% 
	  \tt {\tiny [A$\rangle$GM]}  
	 No.\\ \tt \\ \tt To answer this question, we need to apply the principles of causal inference, specifically the concept of counterfactuals and potential outcomes. \\ \tt \\ \tt Let's denote the potential outcomes as follows: \\ \tt \\ \tt - Y = 1 if the individual has lung cancer, and 0 otherwise.\\ \tt - S = 1 if the individual smokes, and 0 otherwise.\\ \tt - G = 1 if the individual is male, and 0 if female (although G is unobserved).\\ \tt \\ \tt We are given the following information:\\ \tt \\ \tt - P(S) = 0.12 (probability of smoking)\\ \tt - P(Y|S=0) = 0.58 (probability of lung cancer among nonsmokers)\\ \tt - P(Y|S=1) = 0.61 (probability of lung cancer among smokers)\\ \tt \\ \tt We want to determine whether P(Y > 0.5) > 0.5, i.e., whether lung cancer is more likely than its absence overall.\\ \tt \\ \tt To do this, we can use the law of total probability to calculate the overall probability of lung cancer:\\ \tt \\ \tt P(Y > 0) = P(Y=1) = P(Y=1|S=1)P(S) + P(Y=1|S=0)P(S=0)\\ \tt = 0.61(0.12) + 0.58(0.88)\\ \tt = 0.0732 + 0.5104\\ \tt = 0.5836\\ \tt \\ \tt However, this is 
	  } 
	   } 
	   } 
	 & & \\ 
 

    \theutterance \stepcounter{utterance}  

    & & & \multicolumn{2}{p{0.3\linewidth}}{\cellcolor[rgb]{0.95,0.95,0.95}{%
	\makecell[{{p{\linewidth}}}]{% 
	  \tt {\tiny [GM$|$GM]}  
	 No 
	  } 
	   } 
	   } 
	 & & \\ 
 

    \theutterance \stepcounter{utterance}  

    & & & \multicolumn{2}{p{0.3\linewidth}}{\cellcolor[rgb]{0.95,0.95,0.95}{%
	\makecell[{{p{\linewidth}}}]{% 
	  \tt {\tiny [GM$|$GM]}  
	 yes 
	  } 
	   } 
	   } 
	 & & \\ 
 

    \theutterance \stepcounter{utterance}  

    & & & \multicolumn{2}{p{0.3\linewidth}}{\cellcolor[rgb]{0.95,0.95,0.95}{%
	\makecell[{{p{\linewidth}}}]{% 
	  \tt {\tiny [GM$|$GM]}  
	 game_result = LOSE 
	  } 
	   } 
	   } 
	 & & \\ 
 

\end{supertabular}
}

\end{document}
