
\documentclass{article}
\usepackage{colortbl}
\usepackage{makecell}
\usepackage{multirow}
\usepackage{supertabular}

\begin{document}

\newcounter{utterance}

\twocolumn

{ \footnotesize  \setcounter{utterance}{1}
\setlength{\tabcolsep}{0pt}
\begin{supertabular}{c@{$\;$}|p{.15\linewidth}@{}p{.15\linewidth}p{.15\linewidth}p{.15\linewidth}p{.15\linewidth}p{.15\linewidth}}

    \# & $\;$A & \multicolumn{4}{c}{Game Master} & $\;\:$B\\
    \hline 

    \theutterance \stepcounter{utterance}  

    & & \multicolumn{4}{p{0.6\linewidth}}{\cellcolor[rgb]{0.9,0.9,0.9}{%
	\makecell[{{p{\linewidth}}}]{% 
	  \tt {\tiny [A$\langle$GM]}  
	 You are an expert in causal inference. The following question is not a typical commonsense query, but rather a meticulously designed question created by a professor specializing in causal inference, intended to assess the students' mastery of the course content.\\ \tt Imagine a self-contained, hypothetical world with only the following conditions, and without any unmentioned factors or causal relationships: Xevo has a direct effect on gyzp. Tijv has a direct effect on gyzp. We know that xevo or tijv causes gyzp. We observed an individual is not tijv. Would an individual is gyzp if not xevo instead of xevo?\\ \tt Start your answer with 'Yes' or 'No', followed by additional reasoning or evidence to support your explanation. 
	  } 
	   } 
	   } 
	 & \\ 
 

    \theutterance \stepcounter{utterance}  

    & \multicolumn{4}{p{0.6\linewidth}}{\cellcolor[rgb]{0.8,1,0.9}{%
	\makecell[{{p{\linewidth}}}]{% 
	  \tt {\tiny [A$\rangle$GM]}  
	 No.\\ \tt \\ \tt To answer this question, we need to apply the principles of causal inference, specifically the concept of counterfactuals and the backdoor criterion. \\ \tt \\ \tt First, let's establish the causal graph:\\ \tt \\ \tt - Xevo → Gyzp\\ \tt - Tijv → Gyzp\\ \tt \\ \tt We know that Xevo or Tijv causes Gyzp, which implies that there is a common effect (Gyzp) caused by both Xevo and Tijv. This is a classic example of a common cause problem.\\ \tt \\ \tt Now, we are given that the individual is not Tijv. We want to know whether the individual would be Gyzp if not Xevo instead of Xevo.\\ \tt \\ \tt To answer this question, we need to consider the counterfactual: "What would happen if the individual was not Xevo?" \\ \tt \\ \tt Using the backdoor criterion, we can block the direct effect of Xevo on Gyzp by conditioning on Tijv. However, since we know that the individual is not Tijv, we cannot use Tijv as a backdoor variable.\\ \tt \\ \tt In this case, we can use the common effect principle, which states that if Xevo and Tijv are common causes of Gyzp, then the effect of Xevo on Gyzp is the same as the effect of Tijv on Gyzp.\\ \tt \\ \tt Since the individual is not Tijv, we can conclude that the individual would 
	  } 
	   } 
	   } 
	 & & \\ 
 

    \theutterance \stepcounter{utterance}  

    & & & \multicolumn{2}{p{0.3\linewidth}}{\cellcolor[rgb]{0.95,0.95,0.95}{%
	\makecell[{{p{\linewidth}}}]{% 
	  \tt {\tiny [GM$|$GM]}  
	 No 
	  } 
	   } 
	   } 
	 & & \\ 
 

    \theutterance \stepcounter{utterance}  

    & & & \multicolumn{2}{p{0.3\linewidth}}{\cellcolor[rgb]{0.95,0.95,0.95}{%
	\makecell[{{p{\linewidth}}}]{% 
	  \tt {\tiny [GM$|$GM]}  
	 no 
	  } 
	   } 
	   } 
	 & & \\ 
 

    \theutterance \stepcounter{utterance}  

    & & & \multicolumn{2}{p{0.3\linewidth}}{\cellcolor[rgb]{0.95,0.95,0.95}{%
	\makecell[{{p{\linewidth}}}]{% 
	  \tt {\tiny [GM$|$GM]}  
	 game_result = WIN 
	  } 
	   } 
	   } 
	 & & \\ 
 

\end{supertabular}
}

\end{document}
