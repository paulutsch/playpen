
\documentclass{article}
\usepackage{colortbl}
\usepackage{makecell}
\usepackage{multirow}
\usepackage{supertabular}

\begin{document}

\newcounter{utterance}

\twocolumn

{ \footnotesize  \setcounter{utterance}{1}
\setlength{\tabcolsep}{0pt}
\begin{supertabular}{c@{$\;$}|p{.15\linewidth}@{}p{.15\linewidth}p{.15\linewidth}p{.15\linewidth}p{.15\linewidth}p{.15\linewidth}}

    \# & $\;$A & \multicolumn{4}{c}{Game Master} & $\;\:$B\\
    \hline 

    \theutterance \stepcounter{utterance}  

    & & \multicolumn{4}{p{0.6\linewidth}}{\cellcolor[rgb]{0.9,0.9,0.9}{%
	\makecell[{{p{\linewidth}}}]{% 
	  \tt {\tiny [A$\langle$GM]}  
	 You are an expert in causal inference. The following question is not a typical commonsense query, but rather a meticulously designed question created by a professor specializing in causal inference, intended to assess the students' mastery of the course content.\\ \tt Imagine a self-contained, hypothetical world with only the following conditions, and without any unmentioned factors or causal relationships: Yupt has a direct effect on muvq. Zupj has a direct effect on muvq. Method 1: We look directly at how yupt correlates with muvq in general. Method 2: We look at this correlation case by case according to zupj. To understand how yupt affects muvq, is it more correct to use the Method 1 than Method 2?\\ \tt Start your answer with 'Yes' or 'No', followed by additional reasoning or evidence to support your explanation. 
	  } 
	   } 
	   } 
	 & \\ 
 

    \theutterance \stepcounter{utterance}  

    & \multicolumn{4}{p{0.6\linewidth}}{\cellcolor[rgb]{0.8,1,0.9}{%
	\makecell[{{p{\linewidth}}}]{% 
	  \tt {\tiny [A$\rangle$GM]}  
	 No, it is not more correct to use Method 1 than Method 2.\\ \tt \\ \tt In this hypothetical world, both Yupt and Zupj have direct effects on Muvq. This means that the relationship between Yupt and Muvq is confounded by Zupj, as Zupj's effect on Muvq is not accounted for in Method 1. \\ \tt \\ \tt When we look directly at the correlation between Yupt and Muvq in general (Method 1), we are essentially ignoring the effect of Zupj on Muvq. This can lead to biased estimates of the effect of Yupt on Muvq, as the correlation between Yupt and Muvq may be influenced by the shared effect of Zupj.\\ \tt \\ \tt On the other hand, Method 2, which looks at the correlation between Yupt and Muvq case by case according to Zupj, allows us to control for the effect of Zupj on Muvq. By stratifying the analysis by Zupj, we can estimate the effect of Yupt on Muvq within each stratum, which can provide a more accurate estimate of the causal relationship between Yupt and Muvq.\\ \tt \\ \tt Therefore, Method 2 is a more correct approach than Method 1, as it allows us to account for the confounding effect of Zupj on the relationship between Yupt and Muvq. 
	  } 
	   } 
	   } 
	 & & \\ 
 

    \theutterance \stepcounter{utterance}  

    & & & \multicolumn{2}{p{0.3\linewidth}}{\cellcolor[rgb]{0.95,0.95,0.95}{%
	\makecell[{{p{\linewidth}}}]{% 
	  \tt {\tiny [GM$|$GM]}  
	 No 
	  } 
	   } 
	   } 
	 & & \\ 
 

    \theutterance \stepcounter{utterance}  

    & & & \multicolumn{2}{p{0.3\linewidth}}{\cellcolor[rgb]{0.95,0.95,0.95}{%
	\makecell[{{p{\linewidth}}}]{% 
	  \tt {\tiny [GM$|$GM]}  
	 yes 
	  } 
	   } 
	   } 
	 & & \\ 
 

    \theutterance \stepcounter{utterance}  

    & & & \multicolumn{2}{p{0.3\linewidth}}{\cellcolor[rgb]{0.95,0.95,0.95}{%
	\makecell[{{p{\linewidth}}}]{% 
	  \tt {\tiny [GM$|$GM]}  
	 game_result = LOSE 
	  } 
	   } 
	   } 
	 & & \\ 
 

\end{supertabular}
}

\end{document}
