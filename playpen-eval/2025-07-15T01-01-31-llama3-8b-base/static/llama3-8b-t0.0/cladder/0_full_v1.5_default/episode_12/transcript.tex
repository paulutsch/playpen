
\documentclass{article}
\usepackage{colortbl}
\usepackage{makecell}
\usepackage{multirow}
\usepackage{supertabular}

\begin{document}

\newcounter{utterance}

\twocolumn

{ \footnotesize  \setcounter{utterance}{1}
\setlength{\tabcolsep}{0pt}
\begin{supertabular}{c@{$\;$}|p{.15\linewidth}@{}p{.15\linewidth}p{.15\linewidth}p{.15\linewidth}p{.15\linewidth}p{.15\linewidth}}

    \# & $\;$A & \multicolumn{4}{c}{Game Master} & $\;\:$B\\
    \hline 

    \theutterance \stepcounter{utterance}  

    & & \multicolumn{4}{p{0.6\linewidth}}{\cellcolor[rgb]{0.9,0.9,0.9}{%
	\makecell[{{p{\linewidth}}}]{% 
	  \tt {\tiny [A$\langle$GM]}  
	 You are an expert in causal inference. The following question is not a typical commonsense query, but rather a meticulously designed question created by a professor specializing in causal inference, intended to assess the students' mastery of the course content.\\ \tt Imagine a self-contained, hypothetical world with only the following conditions, and without any unmentioned factors or causal relationships: Eating citrus has a direct effect on vitmain C. Vitmain C has a direct effect on scurvy. Method 1: We look at how eating citrus correlates with scurvy case by case according to vitmain C. Method 2: We look directly at how eating citrus correlates with scurvy in general. To understand how eating citrus affects scurvy, is it more correct to use the Method 1 than Method 2?\\ \tt Start your answer with 'Yes' or 'No', followed by additional reasoning or evidence to support your explanation. 
	  } 
	   } 
	   } 
	 & \\ 
 

    \theutterance \stepcounter{utterance}  

    & \multicolumn{4}{p{0.6\linewidth}}{\cellcolor[rgb]{0.8,1,0.9}{%
	\makecell[{{p{\linewidth}}}]{% 
	  \tt {\tiny [A$\rangle$GM]}  
	 No.\\ \tt \\ \tt In this hypothetical world, we have a direct causal relationship between eating citrus and vitamin C, and between vitamin C and scurvy. This is a classic example of a mediator variable, where vitamin C mediates the effect of eating citrus on scurvy.\\ \tt \\ \tt Method 1, which controls for vitamin C, is an example of an instrumental variable (IV) approach. By conditioning on vitamin C, we are effectively using it as an instrument to identify the causal effect of eating citrus on scurvy. This approach is more correct because it allows us to isolate the direct effect of eating citrus on scurvy, while controlling for the mediator variable (vitamin C).\\ \tt \\ \tt Method 2, which looks directly at the correlation between eating citrus and scurvy, is a naive approach that does not account for the mediator variable. This approach will capture both the direct effect of eating citrus on scurvy and the indirect effect through vitamin C. As a result, the estimated effect will be biased and will not accurately represent the causal effect of eating citrus on scurvy.\\ \tt \\ \tt In general, when there is a mediator variable, using an IV approach like Method 1 is more correct than a naive approach like Method 2, as it allows us to identify the direct causal effect of the treatment (eating citrus) on the outcome (scurvy). 
	  } 
	   } 
	   } 
	 & & \\ 
 

    \theutterance \stepcounter{utterance}  

    & & & \multicolumn{2}{p{0.3\linewidth}}{\cellcolor[rgb]{0.95,0.95,0.95}{%
	\makecell[{{p{\linewidth}}}]{% 
	  \tt {\tiny [GM$|$GM]}  
	 No 
	  } 
	   } 
	   } 
	 & & \\ 
 

    \theutterance \stepcounter{utterance}  

    & & & \multicolumn{2}{p{0.3\linewidth}}{\cellcolor[rgb]{0.95,0.95,0.95}{%
	\makecell[{{p{\linewidth}}}]{% 
	  \tt {\tiny [GM$|$GM]}  
	 no 
	  } 
	   } 
	   } 
	 & & \\ 
 

    \theutterance \stepcounter{utterance}  

    & & & \multicolumn{2}{p{0.3\linewidth}}{\cellcolor[rgb]{0.95,0.95,0.95}{%
	\makecell[{{p{\linewidth}}}]{% 
	  \tt {\tiny [GM$|$GM]}  
	 game_result = WIN 
	  } 
	   } 
	   } 
	 & & \\ 
 

\end{supertabular}
}

\end{document}
