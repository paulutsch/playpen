
\documentclass{article}
\usepackage{colortbl}
\usepackage{makecell}
\usepackage{multirow}
\usepackage{supertabular}

\begin{document}

\newcounter{utterance}

\twocolumn

{ \footnotesize  \setcounter{utterance}{1}
\setlength{\tabcolsep}{0pt}
\begin{supertabular}{c@{$\;$}|p{.15\linewidth}@{}p{.15\linewidth}p{.15\linewidth}p{.15\linewidth}p{.15\linewidth}p{.15\linewidth}}

    \# & $\;$A & \multicolumn{4}{c}{Game Master} & $\;\:$B\\
    \hline 

    \theutterance \stepcounter{utterance}  

    & & \multicolumn{4}{p{0.6\linewidth}}{\cellcolor[rgb]{0.9,0.9,0.9}{%
	\makecell[{{p{\linewidth}}}]{% 
	  \tt {\tiny [A$\langle$GM]}  
	 You are an expert in causal inference. The following question is not a typical commonsense query, but rather a meticulously designed question created by a professor specializing in causal inference, intended to assess the students' mastery of the course content.\\ \tt Imagine a self-contained, hypothetical world with only the following conditions, and without any unmentioned factors or causal relationships: Husband has a direct effect on wife and alarm clock. Wife has a direct effect on alarm clock. The overall probability of alarm set by husband is 8%. For husbands that don't set the alarm, the probability of ringing alarm is 75%. For husbands that set the alarm, the probability of ringing alarm is 16%. Is ringing alarm more likely than silent alarm overall?\\ \tt Start your answer with 'Yes' or 'No', followed by additional reasoning or evidence to support your explanation. 
	  } 
	   } 
	   } 
	 & \\ 
 

    \theutterance \stepcounter{utterance}  

    & \multicolumn{4}{p{0.6\linewidth}}{\cellcolor[rgb]{0.8,1,0.9}{%
	\makecell[{{p{\linewidth}}}]{% 
	  \tt {\tiny [A$\rangle$GM]}  
	 No, ringing alarm is not more likely than silent alarm overall.\\ \tt \\ \tt To answer this question, we need to apply the principles of causal inference, specifically the concept of counterfactuals and potential outcomes. Let's break down the information given:\\ \tt \\ \tt 1. The probability of alarm set by husband is 8%.\\ \tt 2. For husbands that don't set the alarm, the probability of ringing alarm is 75%.\\ \tt 3. For husbands that set the alarm, the probability of ringing alarm is 16%.\\ \tt \\ \tt We want to compare the overall probability of ringing alarm (P(Ringing Alarm)) to the overall probability of silent alarm (P(Silent Alarm)).\\ \tt \\ \tt To calculate P(Ringing Alarm), we need to consider two scenarios:\\ \tt \\ \tt - Scenario 1: Husband sets the alarm. In this case, the probability of ringing alarm is 16% (P(Ringing Alarm | Set Alarm)).\\ \tt - Scenario 2: Husband doesn't set the alarm. In this case, the probability of ringing alarm is 75% (P(Ringing Alarm | Not Set Alarm)).\\ \tt \\ \tt Since the probability of alarm set by husband is 8%, we can calculate the weighted average of the two scenarios:\\ \tt \\ \tt P(Ringing Alarm) = P(Set Alarm) \* P(Ringing Alarm | Set Alarm) + P(Not Set Alarm) \* P(Ringing Alarm | Not Set Alarm)\\ \tt = 0.08 \* 0.16 + (1 - 0.08) \* 0.75\\ \tt = 
	  } 
	   } 
	   } 
	 & & \\ 
 

    \theutterance \stepcounter{utterance}  

    & & & \multicolumn{2}{p{0.3\linewidth}}{\cellcolor[rgb]{0.95,0.95,0.95}{%
	\makecell[{{p{\linewidth}}}]{% 
	  \tt {\tiny [GM$|$GM]}  
	 No 
	  } 
	   } 
	   } 
	 & & \\ 
 

    \theutterance \stepcounter{utterance}  

    & & & \multicolumn{2}{p{0.3\linewidth}}{\cellcolor[rgb]{0.95,0.95,0.95}{%
	\makecell[{{p{\linewidth}}}]{% 
	  \tt {\tiny [GM$|$GM]}  
	 yes 
	  } 
	   } 
	   } 
	 & & \\ 
 

    \theutterance \stepcounter{utterance}  

    & & & \multicolumn{2}{p{0.3\linewidth}}{\cellcolor[rgb]{0.95,0.95,0.95}{%
	\makecell[{{p{\linewidth}}}]{% 
	  \tt {\tiny [GM$|$GM]}  
	 game_result = LOSE 
	  } 
	   } 
	   } 
	 & & \\ 
 

\end{supertabular}
}

\end{document}
