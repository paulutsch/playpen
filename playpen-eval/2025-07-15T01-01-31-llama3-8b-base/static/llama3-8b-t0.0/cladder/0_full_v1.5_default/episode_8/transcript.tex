
\documentclass{article}
\usepackage{colortbl}
\usepackage{makecell}
\usepackage{multirow}
\usepackage{supertabular}

\begin{document}

\newcounter{utterance}

\twocolumn

{ \footnotesize  \setcounter{utterance}{1}
\setlength{\tabcolsep}{0pt}
\begin{supertabular}{c@{$\;$}|p{.15\linewidth}@{}p{.15\linewidth}p{.15\linewidth}p{.15\linewidth}p{.15\linewidth}p{.15\linewidth}}

    \# & $\;$A & \multicolumn{4}{c}{Game Master} & $\;\:$B\\
    \hline 

    \theutterance \stepcounter{utterance}  

    & & \multicolumn{4}{p{0.6\linewidth}}{\cellcolor[rgb]{0.9,0.9,0.9}{%
	\makecell[{{p{\linewidth}}}]{% 
	  \tt {\tiny [A$\langle$GM]}  
	 You are an expert in causal inference. The following question is not a typical commonsense query, but rather a meticulously designed question created by a professor specializing in causal inference, intended to assess the students' mastery of the course content.\\ \tt Imagine a self-contained, hypothetical world with only the following conditions, and without any unmentioned factors or causal relationships: Gender has a direct effect on department competitiveness and admission status. Department competitiveness has a direct effect on admission status. The overall probability of male gender is 54%. For individuals who are not male, the probability of admission acceptance is 49%. For individuals who are male, the probability of admission acceptance is 24%. Is admission acceptance less likely than admission rejection overall?\\ \tt Start your answer with 'Yes' or 'No', followed by additional reasoning or evidence to support your explanation. 
	  } 
	   } 
	   } 
	 & \\ 
 

    \theutterance \stepcounter{utterance}  

    & \multicolumn{4}{p{0.6\linewidth}}{\cellcolor[rgb]{0.8,1,0.9}{%
	\makecell[{{p{\linewidth}}}]{% 
	  \tt {\tiny [A$\rangle$GM]}  
	 No.\\ \tt \\ \tt To answer this question, we need to apply the principles of causal inference, specifically the concept of causal graphs and the do-calculus. \\ \tt \\ \tt Let's denote the variables as follows: \\ \tt \\ \tt - M: Male gender (binary variable, 1 if male, 0 if not)\\ \tt - D: Department competitiveness (binary variable, 1 if high, 0 if low)\\ \tt - A: Admission status (binary variable, 1 if accepted, 0 if rejected)\\ \tt \\ \tt Given the conditions, we have the following causal relationships:\\ \tt \\ \tt - M → D (Gender affects department competitiveness)\\ \tt - D → A (Department competitiveness affects admission status)\\ \tt \\ \tt We are also given the following probabilities:\\ \tt \\ \tt - P(M = 1) = 0.54 (probability of male gender)\\ \tt - P(A = 1 | M = 0) = 0.49 (probability of admission acceptance for non-male individuals)\\ \tt - P(A = 1 | M = 1) = 0.24 (probability of admission acceptance for male individuals)\\ \tt \\ \tt To determine whether admission acceptance is less likely than admission rejection overall, we need to calculate the overall probability of admission acceptance, P(A = 1). We can do this by using the law of total probability:\\ \tt \\ \tt P(A = 1) = P(A = 1 | M = 1) \* P(M = 1) + P(A = 1 | M = 0) \* P(M = 0)\\ \tt \\ \tt Substituting the 
	  } 
	   } 
	   } 
	 & & \\ 
 

    \theutterance \stepcounter{utterance}  

    & & & \multicolumn{2}{p{0.3\linewidth}}{\cellcolor[rgb]{0.95,0.95,0.95}{%
	\makecell[{{p{\linewidth}}}]{% 
	  \tt {\tiny [GM$|$GM]}  
	 No 
	  } 
	   } 
	   } 
	 & & \\ 
 

    \theutterance \stepcounter{utterance}  

    & & & \multicolumn{2}{p{0.3\linewidth}}{\cellcolor[rgb]{0.95,0.95,0.95}{%
	\makecell[{{p{\linewidth}}}]{% 
	  \tt {\tiny [GM$|$GM]}  
	 yes 
	  } 
	   } 
	   } 
	 & & \\ 
 

    \theutterance \stepcounter{utterance}  

    & & & \multicolumn{2}{p{0.3\linewidth}}{\cellcolor[rgb]{0.95,0.95,0.95}{%
	\makecell[{{p{\linewidth}}}]{% 
	  \tt {\tiny [GM$|$GM]}  
	 game_result = LOSE 
	  } 
	   } 
	   } 
	 & & \\ 
 

\end{supertabular}
}

\end{document}
