
\documentclass{article}
\usepackage{colortbl}
\usepackage{makecell}
\usepackage{multirow}
\usepackage{supertabular}

\begin{document}

\newcounter{utterance}

\twocolumn

{ \footnotesize  \setcounter{utterance}{1}
\setlength{\tabcolsep}{0pt}
\begin{supertabular}{c@{$\;$}|p{.15\linewidth}@{}p{.15\linewidth}p{.15\linewidth}p{.15\linewidth}p{.15\linewidth}p{.15\linewidth}}

    \# & $\;$A & \multicolumn{4}{c}{Game Master} & $\;\:$B\\
    \hline 

    \theutterance \stepcounter{utterance}  

    & & \multicolumn{4}{p{0.6\linewidth}}{\cellcolor[rgb]{0.9,0.9,0.9}{%
	\makecell[{{p{\linewidth}}}]{% 
	  \tt {\tiny [A$\langle$GM]}  
	 You are an expert in causal inference. The following question is not a typical commonsense query, but rather a meticulously designed question created by a professor specializing in causal inference, intended to assess the students' mastery of the course content.\\ \tt Imagine a self-contained, hypothetical world with only the following conditions, and without any unmentioned factors or causal relationships: CEO has a direct effect on director and manager. Manager has a direct effect on foot size. Director has a direct effect on foot size. For CEOs who fire employees and managers who don't sign termination letters, the probability of large feet is 62%. For CEOs who fire employees and managers who sign termination letters, the probability of large feet is 15%. For CEOs who fire employees and managers who don't sign termination letters, the probability of large feet is 83%. For CEOs who fire employees and managers who sign termination letters, the probability of large feet is 35%. The overall probability of CEO's decision to fire the employee is 94%. Will manager signing the termination letter increase the chance of large feet?\\ \tt Start your answer with 'Yes' or 'No', followed by additional reasoning or evidence to support your explanation. 
	  } 
	   } 
	   } 
	 & \\ 
 

    \theutterance \stepcounter{utterance}  

    & \multicolumn{4}{p{0.6\linewidth}}{\cellcolor[rgb]{0.8,1,0.9}{%
	\makecell[{{p{\linewidth}}}]{% 
	  \tt {\tiny [A$\rangle$GM]}  
	 No, manager signing the termination letter will not increase the chance of large feet.\\ \tt \\ \tt To analyze this problem, we can use the potential outcomes framework and the backdoor criterion from causal inference. \\ \tt \\ \tt Let's define the variables:\\ \tt - L: Large feet\\ \tt - F: CEO fires the employee\\ \tt - S: Manager signs the termination letter\\ \tt - D: Director\\ \tt \\ \tt We are interested in the effect of S on L. \\ \tt \\ \tt First, we can see that the probability of large feet (L) is different for different combinations of F and S. However, we need to consider the overall effect of S on L, not just the conditional probabilities.\\ \tt \\ \tt To do this, we can use the backdoor criterion, which states that if we can block all backdoor paths from S to L, then S is a causal effect of L.\\ \tt \\ \tt In this case, we have the following backdoor paths:\\ \tt - S -> F (direct effect)\\ \tt - F -> L (direct effect)\\ \tt - S -> D (no direct effect, but D has a direct effect on L)\\ \tt - D -> L (direct effect)\\ \tt \\ \tt However, we also have the following confounding variables:\\ \tt - F and S are correlated (CEOs who fire employees are more likely to have managers who sign termination letters)\\ \tt - F and L are correlated (CEOs who fire employees are more likely to have employees with large feet)\\ \tt \\ \tt To block these backdoor paths, we need to condition on the common cause of F and S, which is the CEO's 
	  } 
	   } 
	   } 
	 & & \\ 
 

    \theutterance \stepcounter{utterance}  

    & & & \multicolumn{2}{p{0.3\linewidth}}{\cellcolor[rgb]{0.95,0.95,0.95}{%
	\makecell[{{p{\linewidth}}}]{% 
	  \tt {\tiny [GM$|$GM]}  
	 No 
	  } 
	   } 
	   } 
	 & & \\ 
 

    \theutterance \stepcounter{utterance}  

    & & & \multicolumn{2}{p{0.3\linewidth}}{\cellcolor[rgb]{0.95,0.95,0.95}{%
	\makecell[{{p{\linewidth}}}]{% 
	  \tt {\tiny [GM$|$GM]}  
	 no 
	  } 
	   } 
	   } 
	 & & \\ 
 

    \theutterance \stepcounter{utterance}  

    & & & \multicolumn{2}{p{0.3\linewidth}}{\cellcolor[rgb]{0.95,0.95,0.95}{%
	\makecell[{{p{\linewidth}}}]{% 
	  \tt {\tiny [GM$|$GM]}  
	 game_result = WIN 
	  } 
	   } 
	   } 
	 & & \\ 
 

\end{supertabular}
}

\end{document}
