
\documentclass{article}
\usepackage{colortbl}
\usepackage{makecell}
\usepackage{multirow}
\usepackage{supertabular}

\begin{document}

\newcounter{utterance}

\twocolumn

{ \footnotesize  \setcounter{utterance}{1}
\setlength{\tabcolsep}{0pt}
\begin{supertabular}{c@{$\;$}|p{.15\linewidth}@{}p{.15\linewidth}p{.15\linewidth}p{.15\linewidth}p{.15\linewidth}p{.15\linewidth}}

    \# & $\;$A & \multicolumn{4}{c}{Game Master} & $\;\:$B\\
    \hline 

    \theutterance \stepcounter{utterance}  

    & & \multicolumn{4}{p{0.6\linewidth}}{\cellcolor[rgb]{0.9,0.9,0.9}{%
	\makecell[{{p{\linewidth}}}]{% 
	  \tt {\tiny [A$\langle$GM]}  
	 You are an expert in causal inference. The following question is not a typical commonsense query, but rather a meticulously designed question created by a professor specializing in causal inference, intended to assess the students' mastery of the course content.\\ \tt Imagine a self-contained, hypothetical world with only the following conditions, and without any unmentioned factors or causal relationships: Kidney stone size has a direct effect on treatment and recovery. Treatment has a direct effect on recovery. Method 1: We look directly at how treatment correlates with recovery in general. Method 2: We look at this correlation case by case according to kidney stone size. To understand how treatment affects recovery, is it more correct to use the Method 1 than Method 2?\\ \tt Start your answer with 'Yes' or 'No', followed by additional reasoning or evidence to support your explanation. 
	  } 
	   } 
	   } 
	 & \\ 
 

    \theutterance \stepcounter{utterance}  

    & \multicolumn{4}{p{0.6\linewidth}}{\cellcolor[rgb]{0.8,1,0.9}{%
	\makecell[{{p{\linewidth}}}]{% 
	  \tt {\tiny [A$\rangle$GM]}  
	 No, it is not more correct to use Method 1 than Method 2.\\ \tt \\ \tt In this hypothetical world, both methods are attempting to estimate the causal effect of treatment on recovery. However, Method 1, which looks directly at the correlation between treatment and recovery in general, is subject to confounding by kidney stone size. This is because kidney stone size has a direct effect on both treatment and recovery, which means that the observed correlation between treatment and recovery may be due to the underlying relationship between kidney stone size and recovery, rather than the causal effect of treatment on recovery.\\ \tt \\ \tt On the other hand, Method 2, which looks at the correlation between treatment and recovery case by case according to kidney stone size, is a form of stratification or matching, which can help to control for the confounding effect of kidney stone size. By stratifying the analysis by kidney stone size, we can estimate the causal effect of treatment on recovery within each stratum, which can provide a more accurate estimate of the causal effect.\\ \tt \\ \tt This is an example of the problem of Simpson's paradox, where the overall correlation between treatment and recovery masks the true causal effect of treatment on recovery within each stratum. By using Method 2, we can avoid this paradox and obtain a more accurate estimate of the causal effect of treatment on recovery.\\ \tt \\ \tt Therefore, in this hypothetical world, it is more correct to use Method 2 than Method 1 to understand how treatment affects recovery. 
	  } 
	   } 
	   } 
	 & & \\ 
 

    \theutterance \stepcounter{utterance}  

    & & & \multicolumn{2}{p{0.3\linewidth}}{\cellcolor[rgb]{0.95,0.95,0.95}{%
	\makecell[{{p{\linewidth}}}]{% 
	  \tt {\tiny [GM$|$GM]}  
	 No 
	  } 
	   } 
	   } 
	 & & \\ 
 

    \theutterance \stepcounter{utterance}  

    & & & \multicolumn{2}{p{0.3\linewidth}}{\cellcolor[rgb]{0.95,0.95,0.95}{%
	\makecell[{{p{\linewidth}}}]{% 
	  \tt {\tiny [GM$|$GM]}  
	 no 
	  } 
	   } 
	   } 
	 & & \\ 
 

    \theutterance \stepcounter{utterance}  

    & & & \multicolumn{2}{p{0.3\linewidth}}{\cellcolor[rgb]{0.95,0.95,0.95}{%
	\makecell[{{p{\linewidth}}}]{% 
	  \tt {\tiny [GM$|$GM]}  
	 game_result = WIN 
	  } 
	   } 
	   } 
	 & & \\ 
 

\end{supertabular}
}

\end{document}
