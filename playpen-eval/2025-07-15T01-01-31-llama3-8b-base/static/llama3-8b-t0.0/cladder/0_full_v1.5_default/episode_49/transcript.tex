
\documentclass{article}
\usepackage{colortbl}
\usepackage{makecell}
\usepackage{multirow}
\usepackage{supertabular}

\begin{document}

\newcounter{utterance}

\twocolumn

{ \footnotesize  \setcounter{utterance}{1}
\setlength{\tabcolsep}{0pt}
\begin{supertabular}{c@{$\;$}|p{.15\linewidth}@{}p{.15\linewidth}p{.15\linewidth}p{.15\linewidth}p{.15\linewidth}p{.15\linewidth}}

    \# & $\;$A & \multicolumn{4}{c}{Game Master} & $\;\:$B\\
    \hline 

    \theutterance \stepcounter{utterance}  

    & & \multicolumn{4}{p{0.6\linewidth}}{\cellcolor[rgb]{0.9,0.9,0.9}{%
	\makecell[{{p{\linewidth}}}]{% 
	  \tt {\tiny [A$\langle$GM]}  
	 You are an expert in causal inference. The following question is not a typical commonsense query, but rather a meticulously designed question created by a professor specializing in causal inference, intended to assess the students' mastery of the course content.\\ \tt Imagine a self-contained, hypothetical world with only the following conditions, and without any unmentioned factors or causal relationships: Poverty has a direct effect on liking spicy food and cholera. Water company has a direct effect on liking spicy food. Liking spicy food has a direct effect on cholera. Poverty is unobserved. The overall probability of liking spicy food is 41%. For people who do not like spicy food, the probability of cholera contraction is 59%. For people who like spicy food, the probability of cholera contraction is 33%. Is cholera contraction more likely than cholera prevention overall?\\ \tt Start your answer with 'Yes' or 'No', followed by additional reasoning or evidence to support your explanation. 
	  } 
	   } 
	   } 
	 & \\ 
 

    \theutterance \stepcounter{utterance}  

    & \multicolumn{4}{p{0.6\linewidth}}{\cellcolor[rgb]{0.8,1,0.9}{%
	\makecell[{{p{\linewidth}}}]{% 
	  \tt {\tiny [A$\rangle$GM]}  
	 No, cholera contraction is not more likely than cholera prevention overall.\\ \tt \\ \tt To address this question, we need to apply the principles of causal inference, specifically the concept of counterfactuals and potential outcomes. \\ \tt \\ \tt Let's denote the unobserved variable as P (poverty), the observed variable as S (liking spicy food), and the outcome variable as C (cholera contraction). We are given the following information:\\ \tt \\ \tt 1. P has a direct effect on S and C.\\ \tt 2. S has a direct effect on C.\\ \tt 3. The overall probability of S is 41%.\\ \tt 4. For people who do not like spicy food (S=0), the probability of C is 59%.\\ \tt 5. For people who like spicy food (S=1), the probability of C is 33%.\\ \tt \\ \tt We want to compare the overall probability of cholera contraction (C=1) to the overall probability of cholera prevention (C=0). To do this, we need to estimate the counterfactual probability of C given S=0 (i.e., the probability of cholera contraction if everyone did not like spicy food) and the counterfactual probability of C given S=1 (i.e., the probability of cholera contraction if everyone liked spicy food).\\ \tt \\ \tt Using the law of total probability, we can write the overall probability of cholera contraction as:\\ \tt \\ \tt P(C=1) = P(S=1) \* P(C=1|S=1) + 
	  } 
	   } 
	   } 
	 & & \\ 
 

    \theutterance \stepcounter{utterance}  

    & & & \multicolumn{2}{p{0.3\linewidth}}{\cellcolor[rgb]{0.95,0.95,0.95}{%
	\makecell[{{p{\linewidth}}}]{% 
	  \tt {\tiny [GM$|$GM]}  
	 No 
	  } 
	   } 
	   } 
	 & & \\ 
 

    \theutterance \stepcounter{utterance}  

    & & & \multicolumn{2}{p{0.3\linewidth}}{\cellcolor[rgb]{0.95,0.95,0.95}{%
	\makecell[{{p{\linewidth}}}]{% 
	  \tt {\tiny [GM$|$GM]}  
	 no 
	  } 
	   } 
	   } 
	 & & \\ 
 

    \theutterance \stepcounter{utterance}  

    & & & \multicolumn{2}{p{0.3\linewidth}}{\cellcolor[rgb]{0.95,0.95,0.95}{%
	\makecell[{{p{\linewidth}}}]{% 
	  \tt {\tiny [GM$|$GM]}  
	 game_result = WIN 
	  } 
	   } 
	   } 
	 & & \\ 
 

\end{supertabular}
}

\end{document}
