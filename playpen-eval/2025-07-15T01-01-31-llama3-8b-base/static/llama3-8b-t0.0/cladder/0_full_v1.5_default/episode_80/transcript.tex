
\documentclass{article}
\usepackage{colortbl}
\usepackage{makecell}
\usepackage{multirow}
\usepackage{supertabular}

\begin{document}

\newcounter{utterance}

\twocolumn

{ \footnotesize  \setcounter{utterance}{1}
\setlength{\tabcolsep}{0pt}
\begin{supertabular}{c@{$\;$}|p{.15\linewidth}@{}p{.15\linewidth}p{.15\linewidth}p{.15\linewidth}p{.15\linewidth}p{.15\linewidth}}

    \# & $\;$A & \multicolumn{4}{c}{Game Master} & $\;\:$B\\
    \hline 

    \theutterance \stepcounter{utterance}  

    & & \multicolumn{4}{p{0.6\linewidth}}{\cellcolor[rgb]{0.9,0.9,0.9}{%
	\makecell[{{p{\linewidth}}}]{% 
	  \tt {\tiny [A$\langle$GM]}  
	 You are an expert in causal inference. The following question is not a typical commonsense query, but rather a meticulously designed question created by a professor specializing in causal inference, intended to assess the students' mastery of the course content.\\ \tt Imagine a self-contained, hypothetical world with only the following conditions, and without any unmentioned factors or causal relationships: Zory has a direct effect on zuph and glimx. Zuph has a direct effect on jyka. Jyka has a direct effect on glimx. Zory is unobserved. The overall probability of zuph is 8%. The probability of not zuph and glimx is 58%. The probability of zuph and glimx is 1%. Is the chance of glimx smaller when observing zuph?\\ \tt Start your answer with 'Yes' or 'No', followed by additional reasoning or evidence to support your explanation. 
	  } 
	   } 
	   } 
	 & \\ 
 

    \theutterance \stepcounter{utterance}  

    & \multicolumn{4}{p{0.6\linewidth}}{\cellcolor[rgb]{0.8,1,0.9}{%
	\makecell[{{p{\linewidth}}}]{% 
	  \tt {\tiny [A$\rangle$GM]}  
	 No.\\ \tt \\ \tt To determine whether the chance of glimx is smaller when observing zuph, we need to consider the causal relationships and the given probabilities. \\ \tt \\ \tt First, let's establish the causal relationships:\\ \tt \\ \tt 1. Zory → zuph\\ \tt 2. Zory → glimx\\ \tt 3. zuph → jyka\\ \tt 4. jyka → glimx\\ \tt \\ \tt We are given the following probabilities:\\ \tt \\ \tt - P(zuph) = 8%\\ \tt - P(not zuph and glimx) = 58%\\ \tt - P(zuph and glimx) = 1%\\ \tt \\ \tt We want to determine if P(glimx|zuph) < P(glimx). To do this, we can use the law of total probability to calculate P(glimx) and P(glimx|zuph).\\ \tt \\ \tt P(glimx) = P(glimx|zuph) * P(zuph) + P(glimx|not zuph) * P(not zuph)\\ \tt \\ \tt We know P(zuph) = 8% and P(not zuph and glimx) = 58%. Since P(not zuph and glimx) includes the cases where glimx is true and zuph is false, we can infer that P(glimx|not zuph) is greater than 0. \\ \tt \\ \tt However, we cannot directly calculate P(glimx|zup 
	  } 
	   } 
	   } 
	 & & \\ 
 

    \theutterance \stepcounter{utterance}  

    & & & \multicolumn{2}{p{0.3\linewidth}}{\cellcolor[rgb]{0.95,0.95,0.95}{%
	\makecell[{{p{\linewidth}}}]{% 
	  \tt {\tiny [GM$|$GM]}  
	 No 
	  } 
	   } 
	   } 
	 & & \\ 
 

    \theutterance \stepcounter{utterance}  

    & & & \multicolumn{2}{p{0.3\linewidth}}{\cellcolor[rgb]{0.95,0.95,0.95}{%
	\makecell[{{p{\linewidth}}}]{% 
	  \tt {\tiny [GM$|$GM]}  
	 yes 
	  } 
	   } 
	   } 
	 & & \\ 
 

    \theutterance \stepcounter{utterance}  

    & & & \multicolumn{2}{p{0.3\linewidth}}{\cellcolor[rgb]{0.95,0.95,0.95}{%
	\makecell[{{p{\linewidth}}}]{% 
	  \tt {\tiny [GM$|$GM]}  
	 game_result = LOSE 
	  } 
	   } 
	   } 
	 & & \\ 
 

\end{supertabular}
}

\end{document}
