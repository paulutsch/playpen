
\documentclass{article}
\usepackage{colortbl}
\usepackage{makecell}
\usepackage{multirow}
\usepackage{supertabular}

\begin{document}

\newcounter{utterance}

\twocolumn

{ \footnotesize  \setcounter{utterance}{1}
\setlength{\tabcolsep}{0pt}
\begin{supertabular}{c@{$\;$}|p{.15\linewidth}@{}p{.15\linewidth}p{.15\linewidth}p{.15\linewidth}p{.15\linewidth}p{.15\linewidth}}

    \# & $\;$A & \multicolumn{4}{c}{Game Master} & $\;\:$B\\
    \hline 

    \theutterance \stepcounter{utterance}  

    & & \multicolumn{4}{p{0.6\linewidth}}{\cellcolor[rgb]{0.9,0.9,0.9}{%
	\makecell[{{p{\linewidth}}}]{% 
	  \tt {\tiny [A$\langle$GM]}  
	 You are an expert in causal inference. The following question is not a typical commonsense query, but rather a meticulously designed question created by a professor specializing in causal inference, intended to assess the students' mastery of the course content.\\ \tt Imagine a self-contained, hypothetical world with only the following conditions, and without any unmentioned factors or causal relationships: Vaccination status has a direct effect on vaccination reaction and getting smallpox. Getting smallpox has a direct effect on smallpox survival. Vaccination reaction has a direct effect on smallpox survival. For unvaccinated individuals, the probability of smallpox survival is 39%. For vaccinated individuals, the probability of smallpox survival is 46%. For vaccinated individuals, would it be more likely to see smallpox survival if the person had been vaccinated?\\ \tt Start your answer with 'Yes' or 'No', followed by additional reasoning or evidence to support your explanation. 
	  } 
	   } 
	   } 
	 & \\ 
 

    \theutterance \stepcounter{utterance}  

    & \multicolumn{4}{p{0.6\linewidth}}{\cellcolor[rgb]{0.8,1,0.9}{%
	\makecell[{{p{\linewidth}}}]{% 
	  \tt {\tiny [A$\rangle$GM]}  
	 No.\\ \tt \\ \tt Although vaccination status has a direct effect on smallpox survival, the effect of vaccination reaction on smallpox survival is also present. In this scenario, the effect of vaccination status on smallpox survival is mediated by the vaccination reaction. \\ \tt \\ \tt To assess the effect of vaccination status on smallpox survival, we need to consider the total effect, which includes both the direct effect and the indirect effect through vaccination reaction. \\ \tt \\ \tt Given that the probability of smallpox survival for unvaccinated individuals is 39% and for vaccinated individuals is 46%, it might seem that vaccination status has a positive effect on smallpox survival. However, we need to consider the effect of vaccination reaction on smallpox survival.\\ \tt \\ \tt Since vaccination reaction has a direct effect on smallpox survival, and vaccinated individuals are more likely to have a vaccination reaction, this could potentially offset the positive effect of vaccination status on smallpox survival. \\ \tt \\ \tt To determine the effect of vaccination status on smallpox survival, we would need to know the probability of smallpox survival for vaccinated individuals with a vaccination reaction and for vaccinated individuals without a vaccination reaction. \\ \tt \\ \tt Without this information, we cannot conclude that vaccination status has a positive effect on smallpox survival. Therefore, the answer is No. 
	  } 
	   } 
	   } 
	 & & \\ 
 

    \theutterance \stepcounter{utterance}  

    & & & \multicolumn{2}{p{0.3\linewidth}}{\cellcolor[rgb]{0.95,0.95,0.95}{%
	\makecell[{{p{\linewidth}}}]{% 
	  \tt {\tiny [GM$|$GM]}  
	 No 
	  } 
	   } 
	   } 
	 & & \\ 
 

    \theutterance \stepcounter{utterance}  

    & & & \multicolumn{2}{p{0.3\linewidth}}{\cellcolor[rgb]{0.95,0.95,0.95}{%
	\makecell[{{p{\linewidth}}}]{% 
	  \tt {\tiny [GM$|$GM]}  
	 no 
	  } 
	   } 
	   } 
	 & & \\ 
 

    \theutterance \stepcounter{utterance}  

    & & & \multicolumn{2}{p{0.3\linewidth}}{\cellcolor[rgb]{0.95,0.95,0.95}{%
	\makecell[{{p{\linewidth}}}]{% 
	  \tt {\tiny [GM$|$GM]}  
	 game_result = WIN 
	  } 
	   } 
	   } 
	 & & \\ 
 

\end{supertabular}
}

\end{document}
