
\documentclass{article}
\usepackage{colortbl}
\usepackage{makecell}
\usepackage{multirow}
\usepackage{supertabular}

\begin{document}

\newcounter{utterance}

\twocolumn

{ \footnotesize  \setcounter{utterance}{1}
\setlength{\tabcolsep}{0pt}
\begin{supertabular}{c@{$\;$}|p{.15\linewidth}@{}p{.15\linewidth}p{.15\linewidth}p{.15\linewidth}p{.15\linewidth}p{.15\linewidth}}

    \# & $\;$A & \multicolumn{4}{c}{Game Master} & $\;\:$B\\
    \hline 

    \theutterance \stepcounter{utterance}  

    & & \multicolumn{4}{p{0.6\linewidth}}{\cellcolor[rgb]{0.9,0.9,0.9}{%
	\makecell[{{p{\linewidth}}}]{% 
	  \tt {\tiny [A$\langle$GM]}  
	 You are an expert in causal inference. The following question is not a typical commonsense query, but rather a meticulously designed question created by a professor specializing in causal inference, intended to assess the students' mastery of the course content.\\ \tt Imagine a self-contained, hypothetical world with only the following conditions, and without any unmentioned factors or causal relationships: Appearance has a direct effect on fame. Lip thickness has a direct effect on fame. Method 1: We look at how appearance correlates with lip thickness case by case according to fame. Method 2: We look directly at how appearance correlates with lip thickness in general. To understand how appearance affects lip thickness, is it more correct to use the Method 1 than Method 2?\\ \tt Start your answer with 'Yes' or 'No', followed by additional reasoning or evidence to support your explanation. 
	  } 
	   } 
	   } 
	 & \\ 
 

    \theutterance \stepcounter{utterance}  

    & \multicolumn{4}{p{0.6\linewidth}}{\cellcolor[rgb]{0.8,1,0.9}{%
	\makecell[{{p{\linewidth}}}]{% 
	  \tt {\tiny [A$\rangle$GM]}  
	 No, it is not more correct to use Method 1 than Method 2 to understand how appearance affects lip thickness.\\ \tt \\ \tt In this hypothetical world, we are trying to estimate the causal effect of appearance on lip thickness. However, the problem statement also mentions that lip thickness has a direct effect on fame, and appearance has a direct effect on fame. This creates a problem of confounding by fame, where the correlation between appearance and lip thickness is influenced by their shared effect on fame.\\ \tt \\ \tt Method 1, which examines the correlation between appearance and lip thickness case by case according to fame, is essentially a form of stratification or matching. While this method can help to control for the confounding effect of fame, it is not a reliable method for estimating the causal effect of appearance on lip thickness. This is because the correlation between appearance and lip thickness within each stratum (i.e., within each level of fame) may still be influenced by the residual effect of fame.\\ \tt \\ \tt Method 2, which directly examines the correlation between appearance and lip thickness in general, is a more straightforward approach. However, in the presence of confounding by fame, this method will also be biased. The bias arises because the correlation between appearance and lip thickness is influenced by their shared effect on fame, rather than any direct causal relationship between appearance and lip thickness.\\ \tt \\ \tt To accurately estimate the causal effect of appearance on lip thickness, we would need to use a method that controls for the confounding effect of fame, such as instrumental variable analysis 
	  } 
	   } 
	   } 
	 & & \\ 
 

    \theutterance \stepcounter{utterance}  

    & & & \multicolumn{2}{p{0.3\linewidth}}{\cellcolor[rgb]{0.95,0.95,0.95}{%
	\makecell[{{p{\linewidth}}}]{% 
	  \tt {\tiny [GM$|$GM]}  
	 No 
	  } 
	   } 
	   } 
	 & & \\ 
 

    \theutterance \stepcounter{utterance}  

    & & & \multicolumn{2}{p{0.3\linewidth}}{\cellcolor[rgb]{0.95,0.95,0.95}{%
	\makecell[{{p{\linewidth}}}]{% 
	  \tt {\tiny [GM$|$GM]}  
	 no 
	  } 
	   } 
	   } 
	 & & \\ 
 

    \theutterance \stepcounter{utterance}  

    & & & \multicolumn{2}{p{0.3\linewidth}}{\cellcolor[rgb]{0.95,0.95,0.95}{%
	\makecell[{{p{\linewidth}}}]{% 
	  \tt {\tiny [GM$|$GM]}  
	 game_result = WIN 
	  } 
	   } 
	   } 
	 & & \\ 
 

\end{supertabular}
}

\end{document}
