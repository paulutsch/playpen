
\documentclass{article}
\usepackage{colortbl}
\usepackage{makecell}
\usepackage{multirow}
\usepackage{supertabular}

\begin{document}

\newcounter{utterance}

\twocolumn

{ \footnotesize  \setcounter{utterance}{1}
\setlength{\tabcolsep}{0pt}
\begin{supertabular}{c@{$\;$}|p{.15\linewidth}@{}p{.15\linewidth}p{.15\linewidth}p{.15\linewidth}p{.15\linewidth}p{.15\linewidth}}

    \# & $\;$A & \multicolumn{4}{c}{Game Master} & $\;\:$B\\
    \hline 

    \theutterance \stepcounter{utterance}  

    & & \multicolumn{4}{p{0.6\linewidth}}{\cellcolor[rgb]{0.9,0.9,0.9}{%
	\makecell[{{p{\linewidth}}}]{% 
	  \tt {\tiny [A$\langle$GM]}  
	 You are an expert in causal inference. The following question is not a typical commonsense query, but rather a meticulously designed question created by a professor specializing in causal inference, intended to assess the students' mastery of the course content.\\ \tt Imagine a self-contained, hypothetical world with only the following conditions, and without any unmentioned factors or causal relationships: Kidney stone size has a direct effect on treatment and lip thickness. Treatment has a direct effect on lip thickness. The overall probability of receives treatment is 48%. For patients not receiving treatment, the probability of thick lips is 64%. For patients receiving treatment, the probability of thick lips is 29%. Is thick lips more likely than thin lips overall?\\ \tt Start your answer with 'Yes' or 'No', followed by additional reasoning or evidence to support your explanation. 
	  } 
	   } 
	   } 
	 & \\ 
 

    \theutterance \stepcounter{utterance}  

    & \multicolumn{4}{p{0.6\linewidth}}{\cellcolor[rgb]{0.8,1,0.9}{%
	\makecell[{{p{\linewidth}}}]{% 
	  \tt {\tiny [A$\rangle$GM]}  
	 No, thick lips are not more likely than thin lips overall.\\ \tt \\ \tt To answer this question, we need to apply the principles of causal inference, specifically the concept of counterfactuals and potential outcomes. Let's break down the given information:\\ \tt \\ \tt 1. Kidney stone size has a direct effect on treatment and lip thickness.\\ \tt 2. Treatment has a direct effect on lip thickness.\\ \tt 3. The overall probability of receiving treatment is 48%.\\ \tt 4. For patients not receiving treatment, the probability of thick lips is 64%.\\ \tt 5. For patients receiving treatment, the probability of thick lips is 29%.\\ \tt \\ \tt We are interested in determining whether thick lips are more likely than thin lips overall. To do this, we need to consider the potential outcomes for patients who receive treatment and those who do not.\\ \tt \\ \tt Let's define two potential outcomes:\\ \tt \\ \tt - P(T): Probability of receiving treatment\\ \tt - P(L|T): Probability of thick lips given treatment\\ \tt - P(L|not T): Probability of thick lips given no treatment\\ \tt \\ \tt We are given the following values:\\ \tt \\ \tt - P(T) = 0.48\\ \tt - P(L|not T) = 0.64\\ \tt - P(L|T) = 0.29\\ \tt \\ \tt To determine the overall probability of thick lips, we need to consider the probability of receiving treatment and the probability of thick lips given treatment, as well as the probability of not receiving treatment and the probability of thick lips given no treatment.\\ \tt \\ \tt Using the law of total probability, we can calculate 
	  } 
	   } 
	   } 
	 & & \\ 
 

    \theutterance \stepcounter{utterance}  

    & & & \multicolumn{2}{p{0.3\linewidth}}{\cellcolor[rgb]{0.95,0.95,0.95}{%
	\makecell[{{p{\linewidth}}}]{% 
	  \tt {\tiny [GM$|$GM]}  
	 No 
	  } 
	   } 
	   } 
	 & & \\ 
 

    \theutterance \stepcounter{utterance}  

    & & & \multicolumn{2}{p{0.3\linewidth}}{\cellcolor[rgb]{0.95,0.95,0.95}{%
	\makecell[{{p{\linewidth}}}]{% 
	  \tt {\tiny [GM$|$GM]}  
	 no 
	  } 
	   } 
	   } 
	 & & \\ 
 

    \theutterance \stepcounter{utterance}  

    & & & \multicolumn{2}{p{0.3\linewidth}}{\cellcolor[rgb]{0.95,0.95,0.95}{%
	\makecell[{{p{\linewidth}}}]{% 
	  \tt {\tiny [GM$|$GM]}  
	 game_result = WIN 
	  } 
	   } 
	   } 
	 & & \\ 
 

\end{supertabular}
}

\end{document}
