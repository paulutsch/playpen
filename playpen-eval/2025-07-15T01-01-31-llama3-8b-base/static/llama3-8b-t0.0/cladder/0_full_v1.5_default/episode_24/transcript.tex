
\documentclass{article}
\usepackage{colortbl}
\usepackage{makecell}
\usepackage{multirow}
\usepackage{supertabular}

\begin{document}

\newcounter{utterance}

\twocolumn

{ \footnotesize  \setcounter{utterance}{1}
\setlength{\tabcolsep}{0pt}
\begin{supertabular}{c@{$\;$}|p{.15\linewidth}@{}p{.15\linewidth}p{.15\linewidth}p{.15\linewidth}p{.15\linewidth}p{.15\linewidth}}

    \# & $\;$A & \multicolumn{4}{c}{Game Master} & $\;\:$B\\
    \hline 

    \theutterance \stepcounter{utterance}  

    & & \multicolumn{4}{p{0.6\linewidth}}{\cellcolor[rgb]{0.9,0.9,0.9}{%
	\makecell[{{p{\linewidth}}}]{% 
	  \tt {\tiny [A$\langle$GM]}  
	 You are an expert in causal inference. The following question is not a typical commonsense query, but rather a meticulously designed question created by a professor specializing in causal inference, intended to assess the students' mastery of the course content.\\ \tt Imagine a self-contained, hypothetical world with only the following conditions, and without any unmentioned factors or causal relationships: Unobserved confounders has a direct effect on education level and salary. Proximity to a college has a direct effect on education level. Education level has a direct effect on salary. Unobserved confounders is unobserved. For people living far from a college, the probability of high salary is 39%. For people living close to a college, the probability of high salary is 49%. For people living far from a college, the probability of college degree or higher is 57%. For people living close to a college, the probability of college degree or higher is 30%. Will college degree or higher increase the chance of high salary?\\ \tt Start your answer with 'Yes' or 'No', followed by additional reasoning or evidence to support your explanation. 
	  } 
	   } 
	   } 
	 & \\ 
 

    \theutterance \stepcounter{utterance}  

    & \multicolumn{4}{p{0.6\linewidth}}{\cellcolor[rgb]{0.8,1,0.9}{%
	\makecell[{{p{\linewidth}}}]{% 
	  \tt {\tiny [A$\rangle$GM]}  
	 No\\ \tt \\ \tt To answer this question, we need to apply the principles of causal inference, specifically the concept of confounding variables and the potential outcomes framework. \\ \tt \\ \tt Let's denote the following variables:\\ \tt \\ \tt - U: Unobserved confounders\\ \tt - C: College degree or higher\\ \tt - S: High salary\\ \tt - P: Proximity to a college (far or close)\\ \tt \\ \tt Given the conditions, we have the following causal relationships:\\ \tt \\ \tt 1. U → C (Unobserved confounders affects education level)\\ \tt 2. U → S (Unobserved confounders affects salary)\\ \tt 3. C → S (Education level affects salary)\\ \tt 4. P → C (Proximity to a college affects education level)\\ \tt \\ \tt We are given the following probabilities:\\ \tt \\ \tt - P(S|P=far) = 0.39\\ \tt - P(S|P=close) = 0.49\\ \tt - P(C|P=far) = 0.57\\ \tt - P(C|P=close) = 0.30\\ \tt \\ \tt To determine whether a college degree or higher (C) increases the chance of a high salary (S), we need to compare the causal effect of C on S, while controlling for the confounding variable U.\\ \tt \\ \tt Using the potential outcomes framework, we can define the causal effect of C on S as:\\ \tt \\ \tt P(S|C=1, U=u) - P(S|C=0, U=u)\\ \tt \\ \tt where C=1 represents having a college degree or higher, and 
	  } 
	   } 
	   } 
	 & & \\ 
 

    \theutterance \stepcounter{utterance}  

    & & & \multicolumn{2}{p{0.3\linewidth}}{\cellcolor[rgb]{0.95,0.95,0.95}{%
	\makecell[{{p{\linewidth}}}]{% 
	  \tt {\tiny [GM$|$GM]}  
	 No 
	  } 
	   } 
	   } 
	 & & \\ 
 

    \theutterance \stepcounter{utterance}  

    & & & \multicolumn{2}{p{0.3\linewidth}}{\cellcolor[rgb]{0.95,0.95,0.95}{%
	\makecell[{{p{\linewidth}}}]{% 
	  \tt {\tiny [GM$|$GM]}  
	 no 
	  } 
	   } 
	   } 
	 & & \\ 
 

    \theutterance \stepcounter{utterance}  

    & & & \multicolumn{2}{p{0.3\linewidth}}{\cellcolor[rgb]{0.95,0.95,0.95}{%
	\makecell[{{p{\linewidth}}}]{% 
	  \tt {\tiny [GM$|$GM]}  
	 game_result = WIN 
	  } 
	   } 
	   } 
	 & & \\ 
 

\end{supertabular}
}

\end{document}
