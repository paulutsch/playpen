
\documentclass{article}
\usepackage{colortbl}
\usepackage{makecell}
\usepackage{multirow}
\usepackage{supertabular}

\begin{document}

\newcounter{utterance}

\twocolumn

{ \footnotesize  \setcounter{utterance}{1}
\setlength{\tabcolsep}{0pt}
\begin{supertabular}{c@{$\;$}|p{.15\linewidth}@{}p{.15\linewidth}p{.15\linewidth}p{.15\linewidth}p{.15\linewidth}p{.15\linewidth}}

    \# & $\;$A & \multicolumn{4}{c}{Game Master} & $\;\:$B\\
    \hline 

    \theutterance \stepcounter{utterance}  

    & & \multicolumn{4}{p{0.6\linewidth}}{\cellcolor[rgb]{0.9,0.9,0.9}{%
	\makecell[{{p{\linewidth}}}]{% 
	  \tt {\tiny [A$\langle$GM]}  
	 You are an expert in causal inference. The following question is not a typical commonsense query, but rather a meticulously designed question created by a professor specializing in causal inference, intended to assess the students' mastery of the course content.\\ \tt Imagine a self-contained, hypothetical world with only the following conditions, and without any unmentioned factors or causal relationships: CEO has a direct effect on director and manager. Manager has a direct effect on employee. Director has a direct effect on employee. For CEOs who fire employees and managers who don't sign termination letters, the probability of employee being fired is 24%. For CEOs who fire employees and managers who sign termination letters, the probability of employee being fired is 60%. For CEOs who fire employees and managers who don't sign termination letters, the probability of employee being fired is 39%. For CEOs who fire employees and managers who sign termination letters, the probability of employee being fired is 74%. The overall probability of CEO's decision to fire the employee is 18%. Will manager signing the termination letter increase the chance of employee being fired?\\ \tt Start your answer with 'Yes' or 'No', followed by additional reasoning or evidence to support your explanation. 
	  } 
	   } 
	   } 
	 & \\ 
 

    \theutterance \stepcounter{utterance}  

    & \multicolumn{4}{p{0.6\linewidth}}{\cellcolor[rgb]{0.8,1,0.9}{%
	\makecell[{{p{\linewidth}}}]{% 
	  \tt {\tiny [A$\rangle$GM]}  
	 No, manager signing the termination letter will not increase the chance of employee being fired.\\ \tt \\ \tt To analyze this problem, we can use the potential outcomes framework and the concept of causal mediation. We are interested in the effect of the manager's decision to sign the termination letter on the probability of the employee being fired, conditional on the CEO's decision to fire the employee.\\ \tt \\ \tt Let's define the following variables:\\ \tt \\ \tt - M: Manager signs the termination letter (M=1) or does not sign (M=0)\\ \tt - C: CEO decides to fire the employee (C=1) or does not fire (C=0)\\ \tt - E: Employee is fired (E=1) or not fired (E=0)\\ \tt \\ \tt We are given the following probabilities:\\ \tt \\ \tt - P(E=1|C=1, M=1) = 0.6 (probability of employee being fired given CEO fires and manager signs)\\ \tt - P(E=1|C=1, M=0) = 0.39 (probability of employee being fired given CEO fires and manager does not sign)\\ \tt - P(E=1|C=1) = 0.24 (overall probability of employee being fired given CEO fires)\\ \tt - P(C=1) = 0.18 (overall probability of CEO deciding to fire the employee)\\ \tt \\ \tt To determine the effect of the manager's decision on the probability of the employee being fired, we can use the formula for the causal mediation effect:\\ \tt \\ \tt P(E=1|C= 
	  } 
	   } 
	   } 
	 & & \\ 
 

    \theutterance \stepcounter{utterance}  

    & & & \multicolumn{2}{p{0.3\linewidth}}{\cellcolor[rgb]{0.95,0.95,0.95}{%
	\makecell[{{p{\linewidth}}}]{% 
	  \tt {\tiny [GM$|$GM]}  
	 No 
	  } 
	   } 
	   } 
	 & & \\ 
 

    \theutterance \stepcounter{utterance}  

    & & & \multicolumn{2}{p{0.3\linewidth}}{\cellcolor[rgb]{0.95,0.95,0.95}{%
	\makecell[{{p{\linewidth}}}]{% 
	  \tt {\tiny [GM$|$GM]}  
	 yes 
	  } 
	   } 
	   } 
	 & & \\ 
 

    \theutterance \stepcounter{utterance}  

    & & & \multicolumn{2}{p{0.3\linewidth}}{\cellcolor[rgb]{0.95,0.95,0.95}{%
	\makecell[{{p{\linewidth}}}]{% 
	  \tt {\tiny [GM$|$GM]}  
	 game_result = LOSE 
	  } 
	   } 
	   } 
	 & & \\ 
 

\end{supertabular}
}

\end{document}
