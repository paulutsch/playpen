
\documentclass{article}
\usepackage{colortbl}
\usepackage{makecell}
\usepackage{multirow}
\usepackage{supertabular}

\begin{document}

\newcounter{utterance}

\twocolumn

{ \footnotesize  \setcounter{utterance}{1}
\setlength{\tabcolsep}{0pt}
\begin{supertabular}{c@{$\;$}|p{.15\linewidth}@{}p{.15\linewidth}p{.15\linewidth}p{.15\linewidth}p{.15\linewidth}p{.15\linewidth}}

    \# & $\;$A & \multicolumn{4}{c}{Game Master} & $\;\:$B\\
    \hline 

    \theutterance \stepcounter{utterance}  

    & & \multicolumn{4}{p{0.6\linewidth}}{\cellcolor[rgb]{0.9,0.9,0.9}{%
	\makecell[{{p{\linewidth}}}]{% 
	  \tt {\tiny [A$\langle$GM]}  
	 You are an expert in causal inference. The following question is not a typical commonsense query, but rather a meticulously designed question created by a professor specializing in causal inference, intended to assess the students' mastery of the course content.\\ \tt Imagine a self-contained, hypothetical world with only the following conditions, and without any unmentioned factors or causal relationships: Unobserved confounders has a direct effect on having a sister and cholesterol level. Treatment assignment has a direct effect on having a sister. Having a sister has a direct effect on cholesterol level. Unobserved confounders is unobserved. The overall probability of having a sister is 20%. The probability of not having a sister and low cholesterol is 65%. The probability of having a sister and low cholesterol is 14%. Is the chance of low cholesterol larger when observing having a sister?\\ \tt Start your answer with 'Yes' or 'No', followed by additional reasoning or evidence to support your explanation. 
	  } 
	   } 
	   } 
	 & \\ 
 

    \theutterance \stepcounter{utterance}  

    & \multicolumn{4}{p{0.6\linewidth}}{\cellcolor[rgb]{0.8,1,0.9}{%
	\makecell[{{p{\linewidth}}}]{% 
	  \tt {\tiny [A$\rangle$GM]}  
	 No.\\ \tt \\ \tt To assess this question, we need to apply the principles of causal inference, specifically the concept of confounding and the causal diagram. \\ \tt \\ \tt Given the conditions:\\ \tt \\ \tt - Unobserved confounders has a direct effect on having a sister and cholesterol level.\\ \tt - Treatment assignment has a direct effect on having a sister.\\ \tt - Having a sister has a direct effect on cholesterol level.\\ \tt \\ \tt We can represent this causal diagram as follows:\\ \tt \\ \tt - U (unobserved confounders) -> S (having a sister)\\ \tt - U -> C (cholesterol level)\\ \tt - S -> C\\ \tt \\ \tt The question asks whether the chance of low cholesterol is larger when observing having a sister. To answer this, we need to compare the probability of low cholesterol among individuals with a sister (S=1) to the probability of low cholesterol among individuals without a sister (S=0).\\ \tt \\ \tt Using the given probabilities:\\ \tt \\ \tt - P(S=1) = 0.2\\ \tt - P(S=0) = 0.8\\ \tt - P(C=low, S=1) = 0.14\\ \tt - P(C=low, S=0) = 0.65\\ \tt \\ \tt We can calculate the probability of low cholesterol among individuals with a sister as follows:\\ \tt \\ \tt P(C=low | S=1) = P(C=low, S=1) / P(S=1) = 0.14 / 0.2 = 0.7\\ \tt \\ \tt We can calculate the probability of low cholesterol among 
	  } 
	   } 
	   } 
	 & & \\ 
 

    \theutterance \stepcounter{utterance}  

    & & & \multicolumn{2}{p{0.3\linewidth}}{\cellcolor[rgb]{0.95,0.95,0.95}{%
	\makecell[{{p{\linewidth}}}]{% 
	  \tt {\tiny [GM$|$GM]}  
	 No 
	  } 
	   } 
	   } 
	 & & \\ 
 

    \theutterance \stepcounter{utterance}  

    & & & \multicolumn{2}{p{0.3\linewidth}}{\cellcolor[rgb]{0.95,0.95,0.95}{%
	\makecell[{{p{\linewidth}}}]{% 
	  \tt {\tiny [GM$|$GM]}  
	 no 
	  } 
	   } 
	   } 
	 & & \\ 
 

    \theutterance \stepcounter{utterance}  

    & & & \multicolumn{2}{p{0.3\linewidth}}{\cellcolor[rgb]{0.95,0.95,0.95}{%
	\makecell[{{p{\linewidth}}}]{% 
	  \tt {\tiny [GM$|$GM]}  
	 game_result = WIN 
	  } 
	   } 
	   } 
	 & & \\ 
 

\end{supertabular}
}

\end{document}
