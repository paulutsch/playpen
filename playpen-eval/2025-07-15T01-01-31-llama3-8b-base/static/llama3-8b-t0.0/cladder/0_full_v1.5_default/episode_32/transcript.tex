
\documentclass{article}
\usepackage{colortbl}
\usepackage{makecell}
\usepackage{multirow}
\usepackage{supertabular}

\begin{document}

\newcounter{utterance}

\twocolumn

{ \footnotesize  \setcounter{utterance}{1}
\setlength{\tabcolsep}{0pt}
\begin{supertabular}{c@{$\;$}|p{.15\linewidth}@{}p{.15\linewidth}p{.15\linewidth}p{.15\linewidth}p{.15\linewidth}p{.15\linewidth}}

    \# & $\;$A & \multicolumn{4}{c}{Game Master} & $\;\:$B\\
    \hline 

    \theutterance \stepcounter{utterance}  

    & & \multicolumn{4}{p{0.6\linewidth}}{\cellcolor[rgb]{0.9,0.9,0.9}{%
	\makecell[{{p{\linewidth}}}]{% 
	  \tt {\tiny [A$\langle$GM]}  
	 You are an expert in causal inference. The following question is not a typical commonsense query, but rather a meticulously designed question created by a professor specializing in causal inference, intended to assess the students' mastery of the course content.\\ \tt Imagine a self-contained, hypothetical world with only the following conditions, and without any unmentioned factors or causal relationships: Parents' intelligence has a direct effect on parents' social status and child's intelligence. Other unobserved factors has a direct effect on parents' social status and child's intelligence. Parents' social status has a direct effect on child's intelligence. Other unobserved factors is unobserved. We know that intelligent parents and confounder active causes high parental social status. intelligent parents and confounder active and high parental social status causes intelligent child. We observed confounder active. Would the child is intelligent if intelligent parents instead of unintelligent parents?\\ \tt Start your answer with 'Yes' or 'No', followed by additional reasoning or evidence to support your explanation. 
	  } 
	   } 
	   } 
	 & \\ 
 

    \theutterance \stepcounter{utterance}  

    & \multicolumn{4}{p{0.6\linewidth}}{\cellcolor[rgb]{0.8,1,0.9}{%
	\makecell[{{p{\linewidth}}}]{% 
	  \tt {\tiny [A$\rangle$GM]}  
	 No.\\ \tt \\ \tt To answer this question, we need to apply the principles of causal inference, specifically the concept of confounding variables and the backdoor criterion. \\ \tt \\ \tt Let's denote the variables as follows: \\ \tt \\ \tt - P: Parents' intelligence\\ \tt - S: Parents' social status\\ \tt - C: Child's intelligence\\ \tt - U: Other unobserved factors (confounder)\\ \tt \\ \tt We are given the following causal relationships:\\ \tt \\ \tt 1. P → S (Parents' intelligence affects parents' social status)\\ \tt 2. P → C (Parents' intelligence affects child's intelligence)\\ \tt 3. S → C (Parents' social status affects child's intelligence)\\ \tt 4. U → S (Other unobserved factors affect parents' social status)\\ \tt 5. U → C (Other unobserved factors affect child's intelligence)\\ \tt 6. P, U → S (Intelligent parents and confounder active causes high parental social status)\\ \tt 7. P, U → C (Intelligent parents and confounder active and high parental social status causes intelligent child)\\ \tt \\ \tt We are also given that we observe confounder active (U = 1).\\ \tt \\ \tt To determine whether the child is intelligent if intelligent parents instead of unintelligent parents, we need to compare the potential outcomes under two scenarios: \\ \tt \\ \tt Scenario 1: Intelligent parents (P = 1)\\ \tt Scenario 2: Unintelligent parents (P = 0)\\ \tt \\ \tt We want to estimate the causal effect of P on C, which is the effect of having intelligent parents on the child's intelligence. 
	  } 
	   } 
	   } 
	 & & \\ 
 

    \theutterance \stepcounter{utterance}  

    & & & \multicolumn{2}{p{0.3\linewidth}}{\cellcolor[rgb]{0.95,0.95,0.95}{%
	\makecell[{{p{\linewidth}}}]{% 
	  \tt {\tiny [GM$|$GM]}  
	 No 
	  } 
	   } 
	   } 
	 & & \\ 
 

    \theutterance \stepcounter{utterance}  

    & & & \multicolumn{2}{p{0.3\linewidth}}{\cellcolor[rgb]{0.95,0.95,0.95}{%
	\makecell[{{p{\linewidth}}}]{% 
	  \tt {\tiny [GM$|$GM]}  
	 yes 
	  } 
	   } 
	   } 
	 & & \\ 
 

    \theutterance \stepcounter{utterance}  

    & & & \multicolumn{2}{p{0.3\linewidth}}{\cellcolor[rgb]{0.95,0.95,0.95}{%
	\makecell[{{p{\linewidth}}}]{% 
	  \tt {\tiny [GM$|$GM]}  
	 game_result = LOSE 
	  } 
	   } 
	   } 
	 & & \\ 
 

\end{supertabular}
}

\end{document}
