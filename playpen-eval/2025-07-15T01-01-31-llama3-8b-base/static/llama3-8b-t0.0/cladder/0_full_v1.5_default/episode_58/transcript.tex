
\documentclass{article}
\usepackage{colortbl}
\usepackage{makecell}
\usepackage{multirow}
\usepackage{supertabular}

\begin{document}

\newcounter{utterance}

\twocolumn

{ \footnotesize  \setcounter{utterance}{1}
\setlength{\tabcolsep}{0pt}
\begin{supertabular}{c@{$\;$}|p{.15\linewidth}@{}p{.15\linewidth}p{.15\linewidth}p{.15\linewidth}p{.15\linewidth}p{.15\linewidth}}

    \# & $\;$A & \multicolumn{4}{c}{Game Master} & $\;\:$B\\
    \hline 

    \theutterance \stepcounter{utterance}  

    & & \multicolumn{4}{p{0.6\linewidth}}{\cellcolor[rgb]{0.9,0.9,0.9}{%
	\makecell[{{p{\linewidth}}}]{% 
	  \tt {\tiny [A$\langle$GM]}  
	 You are an expert in causal inference. The following question is not a typical commonsense query, but rather a meticulously designed question created by a professor specializing in causal inference, intended to assess the students' mastery of the course content.\\ \tt Imagine a self-contained, hypothetical world with only the following conditions, and without any unmentioned factors or causal relationships: Vaccination status has a direct effect on vaccination reaction and getting smallpox. Getting smallpox has a direct effect on black hair. Vaccination reaction has a direct effect on black hair. For unvaccinated individuals, the probability of black hair is 27%. For vaccinated individuals, the probability of black hair is 37%. Will vaccination decrease the chance of black hair?\\ \tt Start your answer with 'Yes' or 'No', followed by additional reasoning or evidence to support your explanation. 
	  } 
	   } 
	   } 
	 & \\ 
 

    \theutterance \stepcounter{utterance}  

    & \multicolumn{4}{p{0.6\linewidth}}{\cellcolor[rgb]{0.8,1,0.9}{%
	\makecell[{{p{\linewidth}}}]{% 
	  \tt {\tiny [A$\rangle$GM]}  
	 No.\\ \tt \\ \tt To answer this question, we need to apply the principles of causal inference, specifically the concept of confounding variables and the problem of selection bias. \\ \tt \\ \tt In this hypothetical world, we have three variables: Vaccination status (V), Vaccination reaction (VR), Getting smallpox (S), and Black hair (BH). The causal relationships are as follows:\\ \tt \\ \tt 1. V → VR (Vaccination status has a direct effect on vaccination reaction)\\ \tt 2. V → S (Vaccination status has a direct effect on getting smallpox)\\ \tt 3. VR → BH (Vaccination reaction has a direct effect on black hair)\\ \tt 4. S → BH (Getting smallpox has a direct effect on black hair)\\ \tt \\ \tt We are given the following probabilities:\\ \tt \\ \tt - P(BH|V=0) = 0.27 (probability of black hair for unvaccinated individuals)\\ \tt - P(BH|V=1) = 0.37 (probability of black hair for vaccinated individuals)\\ \tt \\ \tt The question asks whether vaccination decreases the chance of black hair. To answer this, we need to consider the potential confounding effect of getting smallpox (S) on the relationship between vaccination status (V) and black hair (BH).\\ \tt \\ \tt Since getting smallpox (S) has a direct effect on black hair (BH) and vaccination status (V) has a direct effect on getting smallpox (S), we can infer that getting smallpox (S) is 
	  } 
	   } 
	   } 
	 & & \\ 
 

    \theutterance \stepcounter{utterance}  

    & & & \multicolumn{2}{p{0.3\linewidth}}{\cellcolor[rgb]{0.95,0.95,0.95}{%
	\makecell[{{p{\linewidth}}}]{% 
	  \tt {\tiny [GM$|$GM]}  
	 No 
	  } 
	   } 
	   } 
	 & & \\ 
 

    \theutterance \stepcounter{utterance}  

    & & & \multicolumn{2}{p{0.3\linewidth}}{\cellcolor[rgb]{0.95,0.95,0.95}{%
	\makecell[{{p{\linewidth}}}]{% 
	  \tt {\tiny [GM$|$GM]}  
	 no 
	  } 
	   } 
	   } 
	 & & \\ 
 

    \theutterance \stepcounter{utterance}  

    & & & \multicolumn{2}{p{0.3\linewidth}}{\cellcolor[rgb]{0.95,0.95,0.95}{%
	\makecell[{{p{\linewidth}}}]{% 
	  \tt {\tiny [GM$|$GM]}  
	 game_result = WIN 
	  } 
	   } 
	   } 
	 & & \\ 
 

\end{supertabular}
}

\end{document}
