
\documentclass{article}
\usepackage{colortbl}
\usepackage{makecell}
\usepackage{multirow}
\usepackage{supertabular}

\begin{document}

\newcounter{utterance}

\twocolumn

{ \footnotesize  \setcounter{utterance}{1}
\setlength{\tabcolsep}{0pt}
\begin{supertabular}{c@{$\;$}|p{.15\linewidth}@{}p{.15\linewidth}p{.15\linewidth}p{.15\linewidth}p{.15\linewidth}p{.15\linewidth}}

    \# & $\;$A & \multicolumn{4}{c}{Game Master} & $\;\:$B\\
    \hline 

    \theutterance \stepcounter{utterance}  

    & & \multicolumn{4}{p{0.6\linewidth}}{\cellcolor[rgb]{0.9,0.9,0.9}{%
	\makecell[{{p{\linewidth}}}]{% 
	  \tt {\tiny [A$\langle$GM]}  
	 You are an expert in causal inference. The following question is not a typical commonsense query, but rather a meticulously designed question created by a professor specializing in causal inference, intended to assess the students' mastery of the course content.\\ \tt Imagine a self-contained, hypothetical world with only the following conditions, and without any unmentioned factors or causal relationships: Wibl has a direct effect on zuph and uvzi. Zuph has a direct effect on uvzi. Method 1: We look at how zuph correlates with uvzi case by case according to wibl. Method 2: We look directly at how zuph correlates with uvzi in general. To understand how zuph affects uvzi, is it more correct to use the Method 1 than Method 2?\\ \tt Start your answer with 'Yes' or 'No', followed by additional reasoning or evidence to support your explanation. 
	  } 
	   } 
	   } 
	 & \\ 
 

    \theutterance \stepcounter{utterance}  

    & \multicolumn{4}{p{0.6\linewidth}}{\cellcolor[rgb]{0.8,1,0.9}{%
	\makecell[{{p{\linewidth}}}]{% 
	  \tt {\tiny [A$\rangle$GM]}  
	 No.\\ \tt \\ \tt In this hypothetical world, we have a situation where there are two paths of influence: Wibl → Zuph → Uvzi and Wibl → Uvzi. This is a classic example of a confounding variable problem, where Wibl is a common cause of both Zuph and Uvzi.\\ \tt \\ \tt Method 1, which looks at how Zuph correlates with Uvzi case by case according to Wibl, is essentially a form of stratification or matching. By conditioning on Wibl, we are controlling for the confounding effect of Wibl on the relationship between Zuph and Uvzi. This is a valid approach to identify the causal effect of Zuph on Uvzi, as it allows us to isolate the effect of Zuph while holding Wibl constant.\\ \tt \\ \tt On the other hand, Method 2, which looks directly at how Zuph correlates with Uvzi in general, is vulnerable to confounding by Wibl. Since Wibl is a common cause of both Zuph and Uvzi, any observed correlation between Zuph and Uvzi may be due to the effect of Wibl rather than the effect of Zuph on Uvzi. This would lead to an overestimation or underestimation of the causal effect of Zuph on Uvzi.\\ \tt \\ \tt Therefore, in this scenario, it is more correct to use Method 1 
	  } 
	   } 
	   } 
	 & & \\ 
 

    \theutterance \stepcounter{utterance}  

    & & & \multicolumn{2}{p{0.3\linewidth}}{\cellcolor[rgb]{0.95,0.95,0.95}{%
	\makecell[{{p{\linewidth}}}]{% 
	  \tt {\tiny [GM$|$GM]}  
	 No 
	  } 
	   } 
	   } 
	 & & \\ 
 

    \theutterance \stepcounter{utterance}  

    & & & \multicolumn{2}{p{0.3\linewidth}}{\cellcolor[rgb]{0.95,0.95,0.95}{%
	\makecell[{{p{\linewidth}}}]{% 
	  \tt {\tiny [GM$|$GM]}  
	 yes 
	  } 
	   } 
	   } 
	 & & \\ 
 

    \theutterance \stepcounter{utterance}  

    & & & \multicolumn{2}{p{0.3\linewidth}}{\cellcolor[rgb]{0.95,0.95,0.95}{%
	\makecell[{{p{\linewidth}}}]{% 
	  \tt {\tiny [GM$|$GM]}  
	 game_result = LOSE 
	  } 
	   } 
	   } 
	 & & \\ 
 

\end{supertabular}
}

\end{document}
