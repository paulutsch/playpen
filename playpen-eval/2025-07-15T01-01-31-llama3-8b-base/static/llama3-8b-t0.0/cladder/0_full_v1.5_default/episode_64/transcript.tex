
\documentclass{article}
\usepackage{colortbl}
\usepackage{makecell}
\usepackage{multirow}
\usepackage{supertabular}

\begin{document}

\newcounter{utterance}

\twocolumn

{ \footnotesize  \setcounter{utterance}{1}
\setlength{\tabcolsep}{0pt}
\begin{supertabular}{c@{$\;$}|p{.15\linewidth}@{}p{.15\linewidth}p{.15\linewidth}p{.15\linewidth}p{.15\linewidth}p{.15\linewidth}}

    \# & $\;$A & \multicolumn{4}{c}{Game Master} & $\;\:$B\\
    \hline 

    \theutterance \stepcounter{utterance}  

    & & \multicolumn{4}{p{0.6\linewidth}}{\cellcolor[rgb]{0.9,0.9,0.9}{%
	\makecell[{{p{\linewidth}}}]{% 
	  \tt {\tiny [A$\langle$GM]}  
	 You are an expert in causal inference. The following question is not a typical commonsense query, but rather a meticulously designed question created by a professor specializing in causal inference, intended to assess the students' mastery of the course content.\\ \tt Imagine a self-contained, hypothetical world with only the following conditions, and without any unmentioned factors or causal relationships: Zuph has a direct effect on rixq and xevu. Rixq has a direct effect on xevu. For those who are not zuph and are not rixq, the probability of xevu is 15%. For those who are not zuph and are rixq, the probability of xevu is 50%. For those who are zuph and are not rixq, the probability of xevu is 32%. For those who are zuph and are rixq, the probability of xevu is 85%. The overall probability of zuph is 84%. Will rixq increase the chance of xevu?\\ \tt Start your answer with 'Yes' or 'No', followed by additional reasoning or evidence to support your explanation. 
	  } 
	   } 
	   } 
	 & \\ 
 

    \theutterance \stepcounter{utterance}  

    & \multicolumn{4}{p{0.6\linewidth}}{\cellcolor[rgb]{0.8,1,0.9}{%
	\makecell[{{p{\linewidth}}}]{% 
	  \tt {\tiny [A$\rangle$GM]}  
	 No, rixq does not increase the chance of xevu.\\ \tt \\ \tt To assess this, we can use the concept of causal mediation analysis. We are given the following probabilities:\\ \tt \\ \tt - P(xevu | not zuph, not rixq) = 15%\\ \tt - P(xevu | not zuph, rixq) = 50%\\ \tt - P(xevu | zuph, not rixq) = 32%\\ \tt - P(xevu | zuph, rixq) = 85%\\ \tt - P(zuph) = 84%\\ \tt \\ \tt We can calculate the overall probability of xevu as follows:\\ \tt \\ \tt P(xevu) = P(xevu | not zuph, not rixq) * P(not zuph, not rixq) + P(xevu | not zuph, rixq) * P(not zuph, rixq) + P(xevu | zuph, not rixq) * P(zuph, not rixq) + P(xevu | zuph, rixq) * P(zuph, rixq)\\ \tt \\ \tt First, we need to calculate the joint probabilities of zuph and rixq. We can do this by using the law of total probability:\\ \tt \\ \tt P(zuph, rixq) = P(zuph | rixq) * P(rixq)\\ \tt Since 
	  } 
	   } 
	   } 
	 & & \\ 
 

    \theutterance \stepcounter{utterance}  

    & & & \multicolumn{2}{p{0.3\linewidth}}{\cellcolor[rgb]{0.95,0.95,0.95}{%
	\makecell[{{p{\linewidth}}}]{% 
	  \tt {\tiny [GM$|$GM]}  
	 No 
	  } 
	   } 
	   } 
	 & & \\ 
 

    \theutterance \stepcounter{utterance}  

    & & & \multicolumn{2}{p{0.3\linewidth}}{\cellcolor[rgb]{0.95,0.95,0.95}{%
	\makecell[{{p{\linewidth}}}]{% 
	  \tt {\tiny [GM$|$GM]}  
	 yes 
	  } 
	   } 
	   } 
	 & & \\ 
 

    \theutterance \stepcounter{utterance}  

    & & & \multicolumn{2}{p{0.3\linewidth}}{\cellcolor[rgb]{0.95,0.95,0.95}{%
	\makecell[{{p{\linewidth}}}]{% 
	  \tt {\tiny [GM$|$GM]}  
	 game_result = LOSE 
	  } 
	   } 
	   } 
	 & & \\ 
 

\end{supertabular}
}

\end{document}
