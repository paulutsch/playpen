
\documentclass{article}
\usepackage{colortbl}
\usepackage{makecell}
\usepackage{multirow}
\usepackage{supertabular}

\begin{document}

\newcounter{utterance}

\twocolumn

{ \footnotesize  \setcounter{utterance}{1}
\setlength{\tabcolsep}{0pt}
\begin{supertabular}{c@{$\;$}|p{.15\linewidth}@{}p{.15\linewidth}p{.15\linewidth}p{.15\linewidth}p{.15\linewidth}p{.15\linewidth}}

    \# & $\;$A & \multicolumn{4}{c}{Game Master} & $\;\:$B\\
    \hline 

    \theutterance \stepcounter{utterance}  

    & & \multicolumn{4}{p{0.6\linewidth}}{\cellcolor[rgb]{0.9,0.9,0.9}{%
	\makecell[{{p{\linewidth}}}]{% 
	  \tt {\tiny [A$\langle$GM]}  
	 You are an expert in causal inference. The following question is not a typical commonsense query, but rather a meticulously designed question created by a professor specializing in causal inference, intended to assess the students' mastery of the course content.\\ \tt Imagine a self-contained, hypothetical world with only the following conditions, and without any unmentioned factors or causal relationships: Gender has a direct effect on department competitiveness and admission status. Residency status has a direct effect on department competitiveness and admission status. Department competitiveness has a direct effect on admission status. Residency status is unobserved. The overall probability of male gender is 13%. For individuals who are not male, the probability of admission acceptance is 61%. For individuals who are male, the probability of admission acceptance is 65%. Is admission acceptance less likely than admission rejection overall?\\ \tt Start your answer with 'Yes' or 'No', followed by additional reasoning or evidence to support your explanation. 
	  } 
	   } 
	   } 
	 & \\ 
 

    \theutterance \stepcounter{utterance}  

    & \multicolumn{4}{p{0.6\linewidth}}{\cellcolor[rgb]{0.8,1,0.9}{%
	\makecell[{{p{\linewidth}}}]{% 
	  \tt {\tiny [A$\rangle$GM]}  
	 No.\\ \tt \\ \tt To address this question, we need to apply the principles of causal inference, specifically the concept of confounding variables and the potential outcomes framework.\\ \tt \\ \tt Let's denote the potential outcomes as follows:\\ \tt \\ \tt - Y(1) = Admission acceptance if the individual is male\\ \tt - Y(0) = Admission acceptance if the individual is not male\\ \tt - Y(1|1) = Admission acceptance if the individual is male and has a certain level of department competitiveness\\ \tt - Y(1|0) = Admission acceptance if the individual is not male and has a certain level of department competitiveness\\ \tt - Y(0|1) = Admission rejection if the individual is male and has a certain level of department competitiveness\\ \tt - Y(0|0) = Admission rejection if the individual is not male and has a certain level of department competitiveness\\ \tt \\ \tt We are given the following information:\\ \tt \\ \tt - P(Male) = 0.13\\ \tt - P(Admission|Not Male) = 0.61\\ \tt - P(Admission|Male) = 0.65\\ \tt \\ \tt We are also given that Residency status is unobserved and has a direct effect on department competitiveness and admission status. This implies that Residency status is a confounding variable.\\ \tt \\ \tt To determine whether admission acceptance is less likely than admission rejection overall, we need to consider the potential outcomes and the confounding variable.\\ \tt \\ \tt Since Residency status is unobserved, we cannot directly estimate the potential outcomes. However, we can use the given information 
	  } 
	   } 
	   } 
	 & & \\ 
 

    \theutterance \stepcounter{utterance}  

    & & & \multicolumn{2}{p{0.3\linewidth}}{\cellcolor[rgb]{0.95,0.95,0.95}{%
	\makecell[{{p{\linewidth}}}]{% 
	  \tt {\tiny [GM$|$GM]}  
	 No 
	  } 
	   } 
	   } 
	 & & \\ 
 

    \theutterance \stepcounter{utterance}  

    & & & \multicolumn{2}{p{0.3\linewidth}}{\cellcolor[rgb]{0.95,0.95,0.95}{%
	\makecell[{{p{\linewidth}}}]{% 
	  \tt {\tiny [GM$|$GM]}  
	 no 
	  } 
	   } 
	   } 
	 & & \\ 
 

    \theutterance \stepcounter{utterance}  

    & & & \multicolumn{2}{p{0.3\linewidth}}{\cellcolor[rgb]{0.95,0.95,0.95}{%
	\makecell[{{p{\linewidth}}}]{% 
	  \tt {\tiny [GM$|$GM]}  
	 game_result = WIN 
	  } 
	   } 
	   } 
	 & & \\ 
 

\end{supertabular}
}

\end{document}
