
\documentclass{article}
\usepackage{colortbl}
\usepackage{makecell}
\usepackage{multirow}
\usepackage{supertabular}

\begin{document}

\newcounter{utterance}

\twocolumn

{ \footnotesize  \setcounter{utterance}{1}
\setlength{\tabcolsep}{0pt}
\begin{supertabular}{c@{$\;$}|p{.15\linewidth}@{}p{.15\linewidth}p{.15\linewidth}p{.15\linewidth}p{.15\linewidth}p{.15\linewidth}}

    \# & $\;$A & \multicolumn{4}{c}{Game Master} & $\;\:$B\\
    \hline 

    \theutterance \stepcounter{utterance}  

    & & \multicolumn{4}{p{0.6\linewidth}}{\cellcolor[rgb]{0.9,0.9,0.9}{%
	\makecell[{{p{\linewidth}}}]{% 
	  \tt {\tiny [A$\langle$GM]}  
	 You are an expert in causal inference. The following question is not a typical commonsense query, but rather a meticulously designed question created by a professor specializing in causal inference, intended to assess the students' mastery of the course content.\\ \tt Imagine a self-contained, hypothetical world with only the following conditions, and without any unmentioned factors or causal relationships: My decision has a direct effect on penguin mood and curly hair. Penguin mood has a direct effect on curly hair. For those who choose to take the stairs, the probability of curly hair is 70%. For those who choose to take the elevator, the probability of curly hair is 17%. For those who choose to take the elevator, would it be less likely to see curly hair if I had taken the stairs?\\ \tt Start your answer with 'Yes' or 'No', followed by additional reasoning or evidence to support your explanation. 
	  } 
	   } 
	   } 
	 & \\ 
 

    \theutterance \stepcounter{utterance}  

    & \multicolumn{4}{p{0.6\linewidth}}{\cellcolor[rgb]{0.8,1,0.9}{%
	\makecell[{{p{\linewidth}}}]{% 
	  \tt {\tiny [A$\rangle$GM]}  
	 No.\\ \tt \\ \tt To address this question, we need to apply the principles of causal inference, specifically the concept of counterfactuals. We are asked to compare the probability of curly hair among those who choose to take the elevator, under two different scenarios: (1) the actual scenario where I take the elevator, and (2) a counterfactual scenario where I take the stairs.\\ \tt \\ \tt Let's denote the following:\\ \tt \\ \tt - P(C|E) = 0.17 (probability of curly hair given that I take the elevator)\\ \tt - P(C|S) = 0.70 (probability of curly hair given that I take the stairs)\\ \tt \\ \tt We want to know whether it would be less likely to see curly hair if I had taken the stairs, among those who actually take the elevator. This is equivalent to asking: P(C|E, S) ≤ P(C|E)?\\ \tt \\ \tt To answer this question, we need to consider the causal relationships between the variables. We know that my decision has a direct effect on penguin mood, and penguin mood has a direct effect on curly hair. However, we do not have any information about the causal relationship between my decision and curly hair.\\ \tt \\ \tt Using the backdoor criterion, we can identify a path from my decision to curly hair through penguin mood. However, this path is not blocked by any confounders, and we do not have any information about the causal effect of my decision on curly hair.\\ \tt \\ \tt To address this question, we can use the 
	  } 
	   } 
	   } 
	 & & \\ 
 

    \theutterance \stepcounter{utterance}  

    & & & \multicolumn{2}{p{0.3\linewidth}}{\cellcolor[rgb]{0.95,0.95,0.95}{%
	\makecell[{{p{\linewidth}}}]{% 
	  \tt {\tiny [GM$|$GM]}  
	 No 
	  } 
	   } 
	   } 
	 & & \\ 
 

    \theutterance \stepcounter{utterance}  

    & & & \multicolumn{2}{p{0.3\linewidth}}{\cellcolor[rgb]{0.95,0.95,0.95}{%
	\makecell[{{p{\linewidth}}}]{% 
	  \tt {\tiny [GM$|$GM]}  
	 no 
	  } 
	   } 
	   } 
	 & & \\ 
 

    \theutterance \stepcounter{utterance}  

    & & & \multicolumn{2}{p{0.3\linewidth}}{\cellcolor[rgb]{0.95,0.95,0.95}{%
	\makecell[{{p{\linewidth}}}]{% 
	  \tt {\tiny [GM$|$GM]}  
	 game_result = WIN 
	  } 
	   } 
	   } 
	 & & \\ 
 

\end{supertabular}
}

\end{document}
