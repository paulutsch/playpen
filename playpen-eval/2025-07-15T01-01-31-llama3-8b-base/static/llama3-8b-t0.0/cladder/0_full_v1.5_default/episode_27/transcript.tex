
\documentclass{article}
\usepackage{colortbl}
\usepackage{makecell}
\usepackage{multirow}
\usepackage{supertabular}

\begin{document}

\newcounter{utterance}

\twocolumn

{ \footnotesize  \setcounter{utterance}{1}
\setlength{\tabcolsep}{0pt}
\begin{supertabular}{c@{$\;$}|p{.15\linewidth}@{}p{.15\linewidth}p{.15\linewidth}p{.15\linewidth}p{.15\linewidth}p{.15\linewidth}}

    \# & $\;$A & \multicolumn{4}{c}{Game Master} & $\;\:$B\\
    \hline 

    \theutterance \stepcounter{utterance}  

    & & \multicolumn{4}{p{0.6\linewidth}}{\cellcolor[rgb]{0.9,0.9,0.9}{%
	\makecell[{{p{\linewidth}}}]{% 
	  \tt {\tiny [A$\langle$GM]}  
	 You are an expert in causal inference. The following question is not a typical commonsense query, but rather a meticulously designed question created by a professor specializing in causal inference, intended to assess the students' mastery of the course content.\\ \tt Imagine a self-contained, hypothetical world with only the following conditions, and without any unmentioned factors or causal relationships: Parents' intelligence has a direct effect on parents' social status and child's intelligence. Other unobserved factors has a direct effect on parents' social status and child's intelligence. Parents' social status has a direct effect on child's intelligence. Other unobserved factors is unobserved. The overall probability of intelligent parents is 15%. The probability of unintelligent parents and intelligent child is 54%. The probability of intelligent parents and intelligent child is 4%. Is the chance of intelligent child smaller when observing intelligent parents?\\ \tt Start your answer with 'Yes' or 'No', followed by additional reasoning or evidence to support your explanation. 
	  } 
	   } 
	   } 
	 & \\ 
 

    \theutterance \stepcounter{utterance}  

    & \multicolumn{4}{p{0.6\linewidth}}{\cellcolor[rgb]{0.8,1,0.9}{%
	\makecell[{{p{\linewidth}}}]{% 
	  \tt {\tiny [A$\rangle$GM]}  
	 No.\\ \tt \\ \tt To address this question, we need to apply the principles of causal inference, specifically the concept of confounding variables and the problem of selection bias. \\ \tt \\ \tt Let's denote the variables as follows: \\ \tt \\ \tt - P: Parents' intelligence (binary: intelligent or unintelligent)\\ \tt - S: Parents' social status (binary: high or low)\\ \tt - C: Child's intelligence (binary: intelligent or unintelligent)\\ \tt \\ \tt Given the conditions, we have the following causal relationships:\\ \tt \\ \tt 1. P → S (Parents' intelligence affects parents' social status)\\ \tt 2. P → C (Parents' intelligence affects child's intelligence)\\ \tt 3. S → C (Parents' social status affects child's intelligence)\\ \tt 4. Other unobserved factors → S and C (Other unobserved factors affect both parents' social status and child's intelligence)\\ \tt \\ \tt We are given the following probabilities:\\ \tt \\ \tt - P(P) = 0.15 (probability of intelligent parents)\\ \tt - P(C|P=0, C=1) = 0.54 (probability of intelligent child given unintelligent parents)\\ \tt - P(P=1, C=1) = 0.04 (probability of intelligent parents and intelligent child)\\ \tt \\ \tt To determine whether the chance of an intelligent child is smaller when observing intelligent parents, we need to calculate the probability of an intelligent child given intelligent parents, P(C|P=1). \\ \tt \\ \tt Using the law of total probability, we can express P(C|P=1) as:\\ \tt \\ \tt P(C|P 
	  } 
	   } 
	   } 
	 & & \\ 
 

    \theutterance \stepcounter{utterance}  

    & & & \multicolumn{2}{p{0.3\linewidth}}{\cellcolor[rgb]{0.95,0.95,0.95}{%
	\makecell[{{p{\linewidth}}}]{% 
	  \tt {\tiny [GM$|$GM]}  
	 No 
	  } 
	   } 
	   } 
	 & & \\ 
 

    \theutterance \stepcounter{utterance}  

    & & & \multicolumn{2}{p{0.3\linewidth}}{\cellcolor[rgb]{0.95,0.95,0.95}{%
	\makecell[{{p{\linewidth}}}]{% 
	  \tt {\tiny [GM$|$GM]}  
	 yes 
	  } 
	   } 
	   } 
	 & & \\ 
 

    \theutterance \stepcounter{utterance}  

    & & & \multicolumn{2}{p{0.3\linewidth}}{\cellcolor[rgb]{0.95,0.95,0.95}{%
	\makecell[{{p{\linewidth}}}]{% 
	  \tt {\tiny [GM$|$GM]}  
	 game_result = LOSE 
	  } 
	   } 
	   } 
	 & & \\ 
 

\end{supertabular}
}

\end{document}
