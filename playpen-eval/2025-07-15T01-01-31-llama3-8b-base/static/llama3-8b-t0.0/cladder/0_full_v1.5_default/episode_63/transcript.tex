
\documentclass{article}
\usepackage{colortbl}
\usepackage{makecell}
\usepackage{multirow}
\usepackage{supertabular}

\begin{document}

\newcounter{utterance}

\twocolumn

{ \footnotesize  \setcounter{utterance}{1}
\setlength{\tabcolsep}{0pt}
\begin{supertabular}{c@{$\;$}|p{.15\linewidth}@{}p{.15\linewidth}p{.15\linewidth}p{.15\linewidth}p{.15\linewidth}p{.15\linewidth}}

    \# & $\;$A & \multicolumn{4}{c}{Game Master} & $\;\:$B\\
    \hline 

    \theutterance \stepcounter{utterance}  

    & & \multicolumn{4}{p{0.6\linewidth}}{\cellcolor[rgb]{0.9,0.9,0.9}{%
	\makecell[{{p{\linewidth}}}]{% 
	  \tt {\tiny [A$\langle$GM]}  
	 You are an expert in causal inference. The following question is not a typical commonsense query, but rather a meticulously designed question created by a professor specializing in causal inference, intended to assess the students' mastery of the course content.\\ \tt Imagine a self-contained, hypothetical world with only the following conditions, and without any unmentioned factors or causal relationships: Gender has a direct effect on treatment and peanut allergy. Treatment has a direct effect on peanut allergy. For patients who are not male and not receiving treatment, the probability of being allergic to peanuts is 84%. For patients who are not male and receiving treatment, the probability of being allergic to peanuts is 71%. For patients who are male and not receiving treatment, the probability of being allergic to peanuts is 21%. For patients who are male and receiving treatment, the probability of being allergic to peanuts is 3%. The overall probability of male gender is 54%. For patients receiving treatment, would it be more likely to see being allergic to peanuts if the patient had received no treatment?\\ \tt Start your answer with 'Yes' or 'No', followed by additional reasoning or evidence to support your explanation. 
	  } 
	   } 
	   } 
	 & \\ 
 

    \theutterance \stepcounter{utterance}  

    & \multicolumn{4}{p{0.6\linewidth}}{\cellcolor[rgb]{0.8,1,0.9}{%
	\makecell[{{p{\linewidth}}}]{% 
	  \tt {\tiny [A$\rangle$GM]}  
	 No.\\ \tt \\ \tt To answer this question, we need to apply the concept of counterfactuals and potential outcomes. We are interested in the probability of being allergic to peanuts if a patient had received no treatment, given that they are receiving treatment.\\ \tt \\ \tt Let's denote the potential outcome of being allergic to peanuts if a patient had received no treatment as Y(0), and the potential outcome of being allergic to peanuts if a patient had received treatment as Y(1). We are interested in the probability of Y(0) given that the patient is receiving treatment, P(Y(0) | Y(1) = 1).\\ \tt \\ \tt We can use the backdoor criterion to identify the necessary conditions for identifying this causal effect. The backdoor criterion states that we need to control for all variables that are direct causes of the treatment and the outcome.\\ \tt \\ \tt In this case, the only direct cause of the treatment is gender. Therefore, we need to control for gender to identify the causal effect of treatment on the outcome.\\ \tt \\ \tt We can use the following formula to estimate the probability of being allergic to peanuts if a patient had received no treatment, given that they are receiving treatment:\\ \tt \\ \tt P(Y(0) | Y(1) = 1) = P(Y(0) | not male) \* P(not male | Y(1) = 1) + P(Y(0) | male) \* P(male | Y(1) = 1)\\ \tt \\ \tt We know that P(Y(0) | not 
	  } 
	   } 
	   } 
	 & & \\ 
 

    \theutterance \stepcounter{utterance}  

    & & & \multicolumn{2}{p{0.3\linewidth}}{\cellcolor[rgb]{0.95,0.95,0.95}{%
	\makecell[{{p{\linewidth}}}]{% 
	  \tt {\tiny [GM$|$GM]}  
	 No 
	  } 
	   } 
	   } 
	 & & \\ 
 

    \theutterance \stepcounter{utterance}  

    & & & \multicolumn{2}{p{0.3\linewidth}}{\cellcolor[rgb]{0.95,0.95,0.95}{%
	\makecell[{{p{\linewidth}}}]{% 
	  \tt {\tiny [GM$|$GM]}  
	 yes 
	  } 
	   } 
	   } 
	 & & \\ 
 

    \theutterance \stepcounter{utterance}  

    & & & \multicolumn{2}{p{0.3\linewidth}}{\cellcolor[rgb]{0.95,0.95,0.95}{%
	\makecell[{{p{\linewidth}}}]{% 
	  \tt {\tiny [GM$|$GM]}  
	 game_result = LOSE 
	  } 
	   } 
	   } 
	 & & \\ 
 

\end{supertabular}
}

\end{document}
