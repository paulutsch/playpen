
\documentclass{article}
\usepackage{colortbl}
\usepackage{makecell}
\usepackage{multirow}
\usepackage{supertabular}

\begin{document}

\newcounter{utterance}

\twocolumn

{ \footnotesize  \setcounter{utterance}{1}
\setlength{\tabcolsep}{0pt}
\begin{supertabular}{c@{$\;$}|p{.15\linewidth}@{}p{.15\linewidth}p{.15\linewidth}p{.15\linewidth}p{.15\linewidth}p{.15\linewidth}}

    \# & $\;$A & \multicolumn{4}{c}{Game Master} & $\;\:$B\\
    \hline 

    \theutterance \stepcounter{utterance}  

    & & \multicolumn{4}{p{0.6\linewidth}}{\cellcolor[rgb]{0.9,0.9,0.9}{%
	\makecell[{{p{\linewidth}}}]{% 
	  \tt {\tiny [A$\langle$GM]}  
	 You are an expert in causal inference. The following question is not a typical commonsense query, but rather a meticulously designed question created by a professor specializing in causal inference, intended to assess the students' mastery of the course content.\\ \tt Imagine a self-contained, hypothetical world with only the following conditions, and without any unmentioned factors or causal relationships: Jyka has a direct effect on zory and zuph. Zuph has a direct effect on glimx. Zory has a direct effect on glimx. The overall probability of zuph is 34%. The probability of not zuph and glimx is 64%. The probability of zuph and glimx is 28%. Is the chance of glimx smaller when observing zuph?\\ \tt Start your answer with 'Yes' or 'No', followed by additional reasoning or evidence to support your explanation. 
	  } 
	   } 
	   } 
	 & \\ 
 

    \theutterance \stepcounter{utterance}  

    & \multicolumn{4}{p{0.6\linewidth}}{\cellcolor[rgb]{0.8,1,0.9}{%
	\makecell[{{p{\linewidth}}}]{% 
	  \tt {\tiny [A$\rangle$GM]}  
	 No.\\ \tt \\ \tt To determine whether the chance of glimx is smaller when observing zuph, we need to assess the conditional probability of glimx given zuph (P(glimx|zuph)) and compare it to the unconditional probability of glimx (P(glimx)).\\ \tt \\ \tt We are given the following information:\\ \tt \\ \tt 1. P(zuph) = 34%\\ \tt 2. P(not zuph and glimx) = 64%\\ \tt 3. P(zuph and glimx) = 28%\\ \tt \\ \tt From the given information, we can calculate the unconditional probability of glimx as follows:\\ \tt \\ \tt P(glimx) = P(zuph and glimx) + P(not zuph and glimx)\\ \tt = 28% + 64%\\ \tt = 92%\\ \tt \\ \tt Now, we need to calculate the conditional probability of glimx given zuph. We can use the definition of conditional probability:\\ \tt \\ \tt P(glimx|zuph) = P(zuph and glimx) / P(zuph)\\ \tt = 28% / 34%\\ \tt = 82.35%\\ \tt \\ \tt Comparing the conditional probability of glimx given zuph (82.35%) to the unconditional probability of glimx (92%), we can see that the chance of glimx is actually larger when observing zuph. Therefore, the answer is No. 
	  } 
	   } 
	   } 
	 & & \\ 
 

    \theutterance \stepcounter{utterance}  

    & & & \multicolumn{2}{p{0.3\linewidth}}{\cellcolor[rgb]{0.95,0.95,0.95}{%
	\makecell[{{p{\linewidth}}}]{% 
	  \tt {\tiny [GM$|$GM]}  
	 No 
	  } 
	   } 
	   } 
	 & & \\ 
 

    \theutterance \stepcounter{utterance}  

    & & & \multicolumn{2}{p{0.3\linewidth}}{\cellcolor[rgb]{0.95,0.95,0.95}{%
	\makecell[{{p{\linewidth}}}]{% 
	  \tt {\tiny [GM$|$GM]}  
	 yes 
	  } 
	   } 
	   } 
	 & & \\ 
 

    \theutterance \stepcounter{utterance}  

    & & & \multicolumn{2}{p{0.3\linewidth}}{\cellcolor[rgb]{0.95,0.95,0.95}{%
	\makecell[{{p{\linewidth}}}]{% 
	  \tt {\tiny [GM$|$GM]}  
	 game_result = LOSE 
	  } 
	   } 
	   } 
	 & & \\ 
 

\end{supertabular}
}

\end{document}
