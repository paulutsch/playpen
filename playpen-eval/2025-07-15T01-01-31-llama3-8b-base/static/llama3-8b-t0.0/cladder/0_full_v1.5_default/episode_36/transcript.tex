
\documentclass{article}
\usepackage{colortbl}
\usepackage{makecell}
\usepackage{multirow}
\usepackage{supertabular}

\begin{document}

\newcounter{utterance}

\twocolumn

{ \footnotesize  \setcounter{utterance}{1}
\setlength{\tabcolsep}{0pt}
\begin{supertabular}{c@{$\;$}|p{.15\linewidth}@{}p{.15\linewidth}p{.15\linewidth}p{.15\linewidth}p{.15\linewidth}p{.15\linewidth}}

    \# & $\;$A & \multicolumn{4}{c}{Game Master} & $\;\:$B\\
    \hline 

    \theutterance \stepcounter{utterance}  

    & & \multicolumn{4}{p{0.6\linewidth}}{\cellcolor[rgb]{0.9,0.9,0.9}{%
	\makecell[{{p{\linewidth}}}]{% 
	  \tt {\tiny [A$\langle$GM]}  
	 You are an expert in causal inference. The following question is not a typical commonsense query, but rather a meticulously designed question created by a professor specializing in causal inference, intended to assess the students' mastery of the course content.\\ \tt Imagine a self-contained, hypothetical world with only the following conditions, and without any unmentioned factors or causal relationships: Full moon has a direct effect on sprinkler and weather. Weather has a direct effect on ground. Sprinkler has a direct effect on ground. The overall probability of full moon is 16%. For situations where there is no full moon, the probability of wet ground is 56%. For situations where there is a full moon, the probability of wet ground is 54%. Is wet ground more likely than dry ground overall?\\ \tt Start your answer with 'Yes' or 'No', followed by additional reasoning or evidence to support your explanation. 
	  } 
	   } 
	   } 
	 & \\ 
 

    \theutterance \stepcounter{utterance}  

    & \multicolumn{4}{p{0.6\linewidth}}{\cellcolor[rgb]{0.8,1,0.9}{%
	\makecell[{{p{\linewidth}}}]{% 
	  \tt {\tiny [A$\rangle$GM]}  
	 No, wet ground is not more likely than dry ground overall.\\ \tt \\ \tt To answer this question, we need to apply the principles of causal inference, specifically the concept of counterfactuals and potential outcomes. \\ \tt \\ \tt First, let's define the variables and their relationships:\\ \tt \\ \tt - Full Moon (F): a binary variable indicating the presence or absence of a full moon\\ \tt - Sprinkler (S): a binary variable indicating the presence or absence of a sprinkler\\ \tt - Weather (W): a binary variable indicating the presence or absence of wet weather\\ \tt - Ground (G): a binary variable indicating the presence or absence of wet ground\\ \tt \\ \tt Given the conditions, we have the following causal relationships:\\ \tt \\ \tt - F → W (Full moon affects weather)\\ \tt - F → S (Full moon affects sprinkler)\\ \tt - W → G (Weather affects ground)\\ \tt - S → G (Sprinkler affects ground)\\ \tt \\ \tt We are given the following probabilities:\\ \tt \\ \tt - P(F) = 0.16 (probability of full moon)\\ \tt - P(G|F) = 0.54 (probability of wet ground given full moon)\\ \tt - P(G|~F) = 0.56 (probability of wet ground given no full moon)\\ \tt \\ \tt To determine whether wet ground is more likely than dry ground overall, we need to calculate the overall probability of wet ground, P(G). We can do this by considering the potential outcomes under different scenarios.\\ \tt \\ \tt Let's calculate the probability of wet ground under two scenarios:\\ \tt \\ \tt Scenario 1: No full moon 
	  } 
	   } 
	   } 
	 & & \\ 
 

    \theutterance \stepcounter{utterance}  

    & & & \multicolumn{2}{p{0.3\linewidth}}{\cellcolor[rgb]{0.95,0.95,0.95}{%
	\makecell[{{p{\linewidth}}}]{% 
	  \tt {\tiny [GM$|$GM]}  
	 No 
	  } 
	   } 
	   } 
	 & & \\ 
 

    \theutterance \stepcounter{utterance}  

    & & & \multicolumn{2}{p{0.3\linewidth}}{\cellcolor[rgb]{0.95,0.95,0.95}{%
	\makecell[{{p{\linewidth}}}]{% 
	  \tt {\tiny [GM$|$GM]}  
	 yes 
	  } 
	   } 
	   } 
	 & & \\ 
 

    \theutterance \stepcounter{utterance}  

    & & & \multicolumn{2}{p{0.3\linewidth}}{\cellcolor[rgb]{0.95,0.95,0.95}{%
	\makecell[{{p{\linewidth}}}]{% 
	  \tt {\tiny [GM$|$GM]}  
	 game_result = LOSE 
	  } 
	   } 
	   } 
	 & & \\ 
 

\end{supertabular}
}

\end{document}
