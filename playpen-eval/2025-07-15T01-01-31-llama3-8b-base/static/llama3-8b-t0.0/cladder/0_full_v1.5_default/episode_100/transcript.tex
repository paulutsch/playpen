
\documentclass{article}
\usepackage{colortbl}
\usepackage{makecell}
\usepackage{multirow}
\usepackage{supertabular}

\begin{document}

\newcounter{utterance}

\twocolumn

{ \footnotesize  \setcounter{utterance}{1}
\setlength{\tabcolsep}{0pt}
\begin{supertabular}{c@{$\;$}|p{.15\linewidth}@{}p{.15\linewidth}p{.15\linewidth}p{.15\linewidth}p{.15\linewidth}p{.15\linewidth}}

    \# & $\;$A & \multicolumn{4}{c}{Game Master} & $\;\:$B\\
    \hline 

    \theutterance \stepcounter{utterance}  

    & & \multicolumn{4}{p{0.6\linewidth}}{\cellcolor[rgb]{0.9,0.9,0.9}{%
	\makecell[{{p{\linewidth}}}]{% 
	  \tt {\tiny [A$\langle$GM]}  
	 You are an expert in causal inference. The following question is not a typical commonsense query, but rather a meticulously designed question created by a professor specializing in causal inference, intended to assess the students' mastery of the course content.\\ \tt Imagine a self-contained, hypothetical world with only the following conditions, and without any unmentioned factors or causal relationships: Jyka has a direct effect on zory and zuph. Zuph has a direct effect on glimx. Zory has a direct effect on glimx. We know that jyka causes not zuph and zory. zuph and zory causes glimx. We observed an individual is jyka. Would an individual is glimx if zuph instead of not zuph?\\ \tt Start your answer with 'Yes' or 'No', followed by additional reasoning or evidence to support your explanation. 
	  } 
	   } 
	   } 
	 & \\ 
 

    \theutterance \stepcounter{utterance}  

    & \multicolumn{4}{p{0.6\linewidth}}{\cellcolor[rgb]{0.8,1,0.9}{%
	\makecell[{{p{\linewidth}}}]{% 
	  \tt {\tiny [A$\rangle$GM]}  
	 No.\\ \tt \\ \tt The given conditions imply a causal graph with the following structure:\\ \tt \\ \tt - Jyka -> Zory\\ \tt - Jyka -> Zuph\\ \tt - Zuph -> Glimx\\ \tt - Zory -> Glimx\\ \tt \\ \tt We are given that Jyka causes both Zuph and Zory, and both Zuph and Zory cause Glimx. We are asked to consider a counterfactual scenario where Zuph is instead not Zuph (i.e., Zuph is absent).\\ \tt \\ \tt To determine the effect of this counterfactual scenario on Glimx, we need to consider the causal relationships between the variables. Since Zuph and Zory are both causes of Glimx, and Jyka causes both Zuph and Zory, we can use the concept of causal mediation to analyze the effect of Zuph on Glimx.\\ \tt \\ \tt In this case, the causal effect of Jyka on Glimx is mediated by both Zuph and Zory. If Zuph is absent, the causal effect of Jyka on Glimx would still be mediated by Zory, as well as any other direct effects of Jyka on Glimx (if any).\\ \tt \\ \tt However, the absence of Zuph would not affect the causal effect of Jyka on Glimx, because the causal relationship between Jyka and Glimx is not entirely mediated by Zuph. In other words 
	  } 
	   } 
	   } 
	 & & \\ 
 

    \theutterance \stepcounter{utterance}  

    & & & \multicolumn{2}{p{0.3\linewidth}}{\cellcolor[rgb]{0.95,0.95,0.95}{%
	\makecell[{{p{\linewidth}}}]{% 
	  \tt {\tiny [GM$|$GM]}  
	 No 
	  } 
	   } 
	   } 
	 & & \\ 
 

    \theutterance \stepcounter{utterance}  

    & & & \multicolumn{2}{p{0.3\linewidth}}{\cellcolor[rgb]{0.95,0.95,0.95}{%
	\makecell[{{p{\linewidth}}}]{% 
	  \tt {\tiny [GM$|$GM]}  
	 no 
	  } 
	   } 
	   } 
	 & & \\ 
 

    \theutterance \stepcounter{utterance}  

    & & & \multicolumn{2}{p{0.3\linewidth}}{\cellcolor[rgb]{0.95,0.95,0.95}{%
	\makecell[{{p{\linewidth}}}]{% 
	  \tt {\tiny [GM$|$GM]}  
	 game_result = WIN 
	  } 
	   } 
	   } 
	 & & \\ 
 

\end{supertabular}
}

\end{document}
