
\documentclass{article}
\usepackage{colortbl}
\usepackage{makecell}
\usepackage{multirow}
\usepackage{supertabular}

\begin{document}

\newcounter{utterance}

\twocolumn

{ \footnotesize  \setcounter{utterance}{1}
\setlength{\tabcolsep}{0pt}
\begin{supertabular}{c@{$\;$}|p{.15\linewidth}@{}p{.15\linewidth}p{.15\linewidth}p{.15\linewidth}p{.15\linewidth}p{.15\linewidth}}

    \# & $\;$A & \multicolumn{4}{c}{Game Master} & $\;\:$B\\
    \hline 

    \theutterance \stepcounter{utterance}  

    & & \multicolumn{4}{p{0.6\linewidth}}{\cellcolor[rgb]{0.9,0.9,0.9}{%
	\makecell[{{p{\linewidth}}}]{% 
	  \tt {\tiny [A$\langle$GM]}  
	 You are an expert in causal inference. The following question is not a typical commonsense query, but rather a meticulously designed question created by a professor specializing in causal inference, intended to assess the students' mastery of the course content.\\ \tt Imagine a self-contained, hypothetical world with only the following conditions, and without any unmentioned factors or causal relationships: Tanning salon treatment has a direct effect on black hair. Going to the beach has a direct effect on black hair. The overall probability of tanning salon treatment is 99%. For people not using tanning salon treatments, the probability of black hair is 66%. For people who went to tanning salons, the probability of black hair is 24%. Is black hair less likely than blond hair overall?\\ \tt Start your answer with 'Yes' or 'No', followed by additional reasoning or evidence to support your explanation. 
	  } 
	   } 
	   } 
	 & \\ 
 

    \theutterance \stepcounter{utterance}  

    & \multicolumn{4}{p{0.6\linewidth}}{\cellcolor[rgb]{0.8,1,0.9}{%
	\makecell[{{p{\linewidth}}}]{% 
	  \tt {\tiny [A$\rangle$GM]}  
	 No, black hair is not less likely than blond hair overall.\\ \tt \\ \tt To answer this question, we need to apply the principles of causal inference, specifically the concept of confounding variables and the potential outcomes framework.\\ \tt \\ \tt Let's define the variables:\\ \tt \\ \tt - T: Tanning salon treatment\\ \tt - B: Black hair\\ \tt - W: Went to the beach\\ \tt \\ \tt We are given the following information:\\ \tt \\ \tt 1. P(T) = 0.99 (the probability of tanning salon treatment is 99%)\\ \tt 2. P(B|not T) = 0.66 (for people not using tanning salon treatments, the probability of black hair is 66%)\\ \tt 3. P(B|T) = 0.24 (for people who went to tanning salons, the probability of black hair is 24%)\\ \tt \\ \tt We are also given that Tanning salon treatment has a direct effect on black hair and going to the beach has a direct effect on black hair.\\ \tt \\ \tt To determine the overall probability of black hair, we need to consider the potential outcomes framework. Let's assume that the probability of going to the beach is independent of tanning salon treatment (this is a reasonable assumption given the information provided).\\ \tt \\ \tt We can calculate the probability of black hair as follows:\\ \tt \\ \tt P(B) = P(B|T) \* P(T) + P(B|not T) \* P(not T)\\ \tt \\ \tt Substituting the given values, we get:\\ \tt \\ \tt P(B) = 0.24 \* 0.99 
	  } 
	   } 
	   } 
	 & & \\ 
 

    \theutterance \stepcounter{utterance}  

    & & & \multicolumn{2}{p{0.3\linewidth}}{\cellcolor[rgb]{0.95,0.95,0.95}{%
	\makecell[{{p{\linewidth}}}]{% 
	  \tt {\tiny [GM$|$GM]}  
	 No 
	  } 
	   } 
	   } 
	 & & \\ 
 

    \theutterance \stepcounter{utterance}  

    & & & \multicolumn{2}{p{0.3\linewidth}}{\cellcolor[rgb]{0.95,0.95,0.95}{%
	\makecell[{{p{\linewidth}}}]{% 
	  \tt {\tiny [GM$|$GM]}  
	 yes 
	  } 
	   } 
	   } 
	 & & \\ 
 

    \theutterance \stepcounter{utterance}  

    & & & \multicolumn{2}{p{0.3\linewidth}}{\cellcolor[rgb]{0.95,0.95,0.95}{%
	\makecell[{{p{\linewidth}}}]{% 
	  \tt {\tiny [GM$|$GM]}  
	 game_result = LOSE 
	  } 
	   } 
	   } 
	 & & \\ 
 

\end{supertabular}
}

\end{document}
