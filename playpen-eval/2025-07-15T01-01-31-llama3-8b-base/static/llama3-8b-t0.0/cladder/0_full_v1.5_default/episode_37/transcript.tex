
\documentclass{article}
\usepackage{colortbl}
\usepackage{makecell}
\usepackage{multirow}
\usepackage{supertabular}

\begin{document}

\newcounter{utterance}

\twocolumn

{ \footnotesize  \setcounter{utterance}{1}
\setlength{\tabcolsep}{0pt}
\begin{supertabular}{c@{$\;$}|p{.15\linewidth}@{}p{.15\linewidth}p{.15\linewidth}p{.15\linewidth}p{.15\linewidth}p{.15\linewidth}}

    \# & $\;$A & \multicolumn{4}{c}{Game Master} & $\;\:$B\\
    \hline 

    \theutterance \stepcounter{utterance}  

    & & \multicolumn{4}{p{0.6\linewidth}}{\cellcolor[rgb]{0.9,0.9,0.9}{%
	\makecell[{{p{\linewidth}}}]{% 
	  \tt {\tiny [A$\langle$GM]}  
	 You are an expert in causal inference. The following question is not a typical commonsense query, but rather a meticulously designed question created by a professor specializing in causal inference, intended to assess the students' mastery of the course content.\\ \tt Imagine a self-contained, hypothetical world with only the following conditions, and without any unmentioned factors or causal relationships: Full moon has a direct effect on sprinkler and weather. Weather has a direct effect on ground. Sprinkler has a direct effect on ground. The overall probability of full moon is 31%. The probability of no full moon and wet ground is 33%. The probability of full moon and wet ground is 19%. Is the chance of wet ground smaller when observing full moon?\\ \tt Start your answer with 'Yes' or 'No', followed by additional reasoning or evidence to support your explanation. 
	  } 
	   } 
	   } 
	 & \\ 
 

    \theutterance \stepcounter{utterance}  

    & \multicolumn{4}{p{0.6\linewidth}}{\cellcolor[rgb]{0.8,1,0.9}{%
	\makecell[{{p{\linewidth}}}]{% 
	  \tt {\tiny [A$\rangle$GM]}  
	 No.\\ \tt \\ \tt To answer this question, we need to apply the principles of causal inference, specifically the concept of conditional probability and the backdoor criterion. \\ \tt \\ \tt First, let's define the variables and their relationships:\\ \tt \\ \tt - F: Full moon (probability = 0.31)\\ \tt - S: Sprinkler (we don't have direct information about S, but we know its effect on G)\\ \tt - W: Weather (we don't have direct information about W, but we know its effect on G)\\ \tt - G: Ground (wet or dry)\\ \tt \\ \tt We are given the following probabilities:\\ \tt \\ \tt - P(F) = 0.31\\ \tt - P(¬F ∩ G = wet) = 0.33\\ \tt - P(F ∩ G = wet) = 0.19\\ \tt \\ \tt We want to determine if the chance of wet ground (G = wet) is smaller when observing full moon (F). To do this, we need to compare the probability of G = wet given F (P(G = wet | F)) to the probability of G = wet given ¬F (P(G = wet | ¬F)).\\ \tt \\ \tt Using the law of total probability, we can express P(G = wet) as:\\ \tt \\ \tt P(G = wet) = P(G = wet | F)P(F) + P(G = wet | ¬F)P(¬F)\\ \tt \\ \tt We know that P(F) = 0.31 and P(¬F) = 1 - P(F) = 
	  } 
	   } 
	   } 
	 & & \\ 
 

    \theutterance \stepcounter{utterance}  

    & & & \multicolumn{2}{p{0.3\linewidth}}{\cellcolor[rgb]{0.95,0.95,0.95}{%
	\makecell[{{p{\linewidth}}}]{% 
	  \tt {\tiny [GM$|$GM]}  
	 No 
	  } 
	   } 
	   } 
	 & & \\ 
 

    \theutterance \stepcounter{utterance}  

    & & & \multicolumn{2}{p{0.3\linewidth}}{\cellcolor[rgb]{0.95,0.95,0.95}{%
	\makecell[{{p{\linewidth}}}]{% 
	  \tt {\tiny [GM$|$GM]}  
	 no 
	  } 
	   } 
	   } 
	 & & \\ 
 

    \theutterance \stepcounter{utterance}  

    & & & \multicolumn{2}{p{0.3\linewidth}}{\cellcolor[rgb]{0.95,0.95,0.95}{%
	\makecell[{{p{\linewidth}}}]{% 
	  \tt {\tiny [GM$|$GM]}  
	 game_result = WIN 
	  } 
	   } 
	   } 
	 & & \\ 
 

\end{supertabular}
}

\end{document}
