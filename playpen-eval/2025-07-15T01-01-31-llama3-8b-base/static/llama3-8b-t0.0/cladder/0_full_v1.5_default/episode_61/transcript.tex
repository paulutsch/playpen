
\documentclass{article}
\usepackage{colortbl}
\usepackage{makecell}
\usepackage{multirow}
\usepackage{supertabular}

\begin{document}

\newcounter{utterance}

\twocolumn

{ \footnotesize  \setcounter{utterance}{1}
\setlength{\tabcolsep}{0pt}
\begin{supertabular}{c@{$\;$}|p{.15\linewidth}@{}p{.15\linewidth}p{.15\linewidth}p{.15\linewidth}p{.15\linewidth}p{.15\linewidth}}

    \# & $\;$A & \multicolumn{4}{c}{Game Master} & $\;\:$B\\
    \hline 

    \theutterance \stepcounter{utterance}  

    & & \multicolumn{4}{p{0.6\linewidth}}{\cellcolor[rgb]{0.9,0.9,0.9}{%
	\makecell[{{p{\linewidth}}}]{% 
	  \tt {\tiny [A$\langle$GM]}  
	 You are an expert in causal inference. The following question is not a typical commonsense query, but rather a meticulously designed question created by a professor specializing in causal inference, intended to assess the students' mastery of the course content.\\ \tt Imagine a self-contained, hypothetical world with only the following conditions, and without any unmentioned factors or causal relationships: Encouragement level has a direct effect on studying habit and brown eyes. Studying habit has a direct effect on brown eyes. The overall probability of encouragement is 91%. The probability of discouragement and brown eyes is 7%. The probability of encouragement and brown eyes is 58%. Is the chance of brown eyes larger when observing encouragement?\\ \tt Start your answer with 'Yes' or 'No', followed by additional reasoning or evidence to support your explanation. 
	  } 
	   } 
	   } 
	 & \\ 
 

    \theutterance \stepcounter{utterance}  

    & \multicolumn{4}{p{0.6\linewidth}}{\cellcolor[rgb]{0.8,1,0.9}{%
	\makecell[{{p{\linewidth}}}]{% 
	  \tt {\tiny [A$\rangle$GM]}  
	 No.\\ \tt \\ \tt To determine whether the chance of brown eyes is larger when observing encouragement, we need to calculate the probability of brown eyes given encouragement (P(Brown|Encouragement)) and compare it to the overall probability of brown eyes (P(Brown)).\\ \tt \\ \tt We can use the law of total probability to calculate P(Brown|Encouragement):\\ \tt \\ \tt P(Brown|Encouragement) = P(Encouragement and Brown) / P(Encouragement)\\ \tt = 58% / 91%\\ \tt = 0.637\\ \tt \\ \tt We can also calculate the overall probability of brown eyes (P(Brown)) using the law of total probability:\\ \tt \\ \tt P(Brown) = P(Encouragement and Brown) + P(Discouragement and Brown)\\ \tt = 58% + 7%\\ \tt = 65%\\ \tt \\ \tt Comparing the two probabilities, we see that P(Brown) (65%) is greater than P(Brown|Encouragement) (63.7%). This suggests that the chance of brown eyes is not larger when observing encouragement.\\ \tt \\ \tt However, we can further investigate this by considering the causal relationships between the variables. The given conditions state that encouragement level has a direct effect on studying habit and brown eyes, and studying habit has a direct effect on brown eyes. This suggests that the relationship between encouragement and brown eyes may be mediated by studying habit.\\ \tt \\ \tt To determine whether the chance of brown eyes is larger when observing encouragement, we need to consider the potential confounding effect of studying habit. 
	  } 
	   } 
	   } 
	 & & \\ 
 

    \theutterance \stepcounter{utterance}  

    & & & \multicolumn{2}{p{0.3\linewidth}}{\cellcolor[rgb]{0.95,0.95,0.95}{%
	\makecell[{{p{\linewidth}}}]{% 
	  \tt {\tiny [GM$|$GM]}  
	 No 
	  } 
	   } 
	   } 
	 & & \\ 
 

    \theutterance \stepcounter{utterance}  

    & & & \multicolumn{2}{p{0.3\linewidth}}{\cellcolor[rgb]{0.95,0.95,0.95}{%
	\makecell[{{p{\linewidth}}}]{% 
	  \tt {\tiny [GM$|$GM]}  
	 no 
	  } 
	   } 
	   } 
	 & & \\ 
 

    \theutterance \stepcounter{utterance}  

    & & & \multicolumn{2}{p{0.3\linewidth}}{\cellcolor[rgb]{0.95,0.95,0.95}{%
	\makecell[{{p{\linewidth}}}]{% 
	  \tt {\tiny [GM$|$GM]}  
	 game_result = WIN 
	  } 
	   } 
	   } 
	 & & \\ 
 

\end{supertabular}
}

\end{document}
