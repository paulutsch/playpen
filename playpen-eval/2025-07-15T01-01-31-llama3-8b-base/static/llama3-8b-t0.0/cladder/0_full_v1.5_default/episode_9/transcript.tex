
\documentclass{article}
\usepackage{colortbl}
\usepackage{makecell}
\usepackage{multirow}
\usepackage{supertabular}

\begin{document}

\newcounter{utterance}

\twocolumn

{ \footnotesize  \setcounter{utterance}{1}
\setlength{\tabcolsep}{0pt}
\begin{supertabular}{c@{$\;$}|p{.15\linewidth}@{}p{.15\linewidth}p{.15\linewidth}p{.15\linewidth}p{.15\linewidth}p{.15\linewidth}}

    \# & $\;$A & \multicolumn{4}{c}{Game Master} & $\;\:$B\\
    \hline 

    \theutterance \stepcounter{utterance}  

    & & \multicolumn{4}{p{0.6\linewidth}}{\cellcolor[rgb]{0.9,0.9,0.9}{%
	\makecell[{{p{\linewidth}}}]{% 
	  \tt {\tiny [A$\langle$GM]}  
	 You are an expert in causal inference. The following question is not a typical commonsense query, but rather a meticulously designed question created by a professor specializing in causal inference, intended to assess the students' mastery of the course content.\\ \tt Imagine a self-contained, hypothetical world with only the following conditions, and without any unmentioned factors or causal relationships: Personality has a direct effect on relationship status. Appearance has a direct effect on relationship status. The overall probability of kindness is 85%. For people with mean personalities and single people, the probability of attractive appearance is 38%. For people with mean personalities and in a relationship, the probability of attractive appearance is 71%. For people with kind personalities and single people, the probability of attractive appearance is 32%. For people with kind personalities and in a relationship, the probability of attractive appearance is 57%. If we look at people in a relationship, does the chance of attractive appearance decrease when kindness?\\ \tt Start your answer with 'Yes' or 'No', followed by additional reasoning or evidence to support your explanation. 
	  } 
	   } 
	   } 
	 & \\ 
 

    \theutterance \stepcounter{utterance}  

    & \multicolumn{4}{p{0.6\linewidth}}{\cellcolor[rgb]{0.8,1,0.9}{%
	\makecell[{{p{\linewidth}}}]{% 
	  \tt {\tiny [A$\rangle$GM]}  
	 No, the chance of attractive appearance does not decrease when kindness, when looking at people in a relationship.\\ \tt \\ \tt To analyze this, we need to use the given information to estimate the probability of attractive appearance for people in a relationship with kind and mean personalities. \\ \tt \\ \tt First, we know that the probability of attractive appearance for people with mean personalities and in a relationship is 71%. \\ \tt \\ \tt Next, we need to compare this to the probability of attractive appearance for people with kind personalities and in a relationship. However, we cannot directly compare these probabilities because they are conditional on being in a relationship. \\ \tt \\ \tt To make a fair comparison, we need to consider the probability of being in a relationship for people with kind and mean personalities. However, we do not have direct information about this. \\ \tt \\ \tt Instead, we can use the given information to estimate the probability of being in a relationship for people with kind and mean personalities. We know that the overall probability of kindness is 85%, and the probability of attractive appearance for people with mean personalities and in a relationship is 71%. \\ \tt \\ \tt We can also use the information that the probability of attractive appearance for people with mean personalities and single people is 38%. \\ \tt \\ \tt Using these values, we can estimate the probability of being in a relationship for people with mean personalities as follows:\\ \tt \\ \tt P(in a relationship | mean personality) = P(attractive appearance | mean personality, in a relationship) / (P(attractive appearance | mean personality, in a relationship) + P(attractive appearance | mean 
	  } 
	   } 
	   } 
	 & & \\ 
 

    \theutterance \stepcounter{utterance}  

    & & & \multicolumn{2}{p{0.3\linewidth}}{\cellcolor[rgb]{0.95,0.95,0.95}{%
	\makecell[{{p{\linewidth}}}]{% 
	  \tt {\tiny [GM$|$GM]}  
	 No 
	  } 
	   } 
	   } 
	 & & \\ 
 

    \theutterance \stepcounter{utterance}  

    & & & \multicolumn{2}{p{0.3\linewidth}}{\cellcolor[rgb]{0.95,0.95,0.95}{%
	\makecell[{{p{\linewidth}}}]{% 
	  \tt {\tiny [GM$|$GM]}  
	 yes 
	  } 
	   } 
	   } 
	 & & \\ 
 

    \theutterance \stepcounter{utterance}  

    & & & \multicolumn{2}{p{0.3\linewidth}}{\cellcolor[rgb]{0.95,0.95,0.95}{%
	\makecell[{{p{\linewidth}}}]{% 
	  \tt {\tiny [GM$|$GM]}  
	 game_result = LOSE 
	  } 
	   } 
	   } 
	 & & \\ 
 

\end{supertabular}
}

\end{document}
