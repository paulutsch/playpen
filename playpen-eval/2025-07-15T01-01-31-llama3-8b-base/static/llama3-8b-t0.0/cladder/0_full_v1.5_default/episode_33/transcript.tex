
\documentclass{article}
\usepackage{colortbl}
\usepackage{makecell}
\usepackage{multirow}
\usepackage{supertabular}

\begin{document}

\newcounter{utterance}

\twocolumn

{ \footnotesize  \setcounter{utterance}{1}
\setlength{\tabcolsep}{0pt}
\begin{supertabular}{c@{$\;$}|p{.15\linewidth}@{}p{.15\linewidth}p{.15\linewidth}p{.15\linewidth}p{.15\linewidth}p{.15\linewidth}}

    \# & $\;$A & \multicolumn{4}{c}{Game Master} & $\;\:$B\\
    \hline 

    \theutterance \stepcounter{utterance}  

    & & \multicolumn{4}{p{0.6\linewidth}}{\cellcolor[rgb]{0.9,0.9,0.9}{%
	\makecell[{{p{\linewidth}}}]{% 
	  \tt {\tiny [A$\langle$GM]}  
	 You are an expert in causal inference. The following question is not a typical commonsense query, but rather a meticulously designed question created by a professor specializing in causal inference, intended to assess the students' mastery of the course content.\\ \tt Imagine a self-contained, hypothetical world with only the following conditions, and without any unmentioned factors or causal relationships: Poverty has a direct effect on water quality and cholera. Water company has a direct effect on water quality. Water quality has a direct effect on cholera. Poverty is unobserved. We know that high poverty or global water company causes clean water. high poverty or clean water causes cholera contraction. We observed the person is served by a global water company and the region has low poverty. Would the person gets cholera if polluted water instead of clean water?\\ \tt Start your answer with 'Yes' or 'No', followed by additional reasoning or evidence to support your explanation. 
	  } 
	   } 
	   } 
	 & \\ 
 

    \theutterance \stepcounter{utterance}  

    & \multicolumn{4}{p{0.6\linewidth}}{\cellcolor[rgb]{0.8,1,0.9}{%
	\makecell[{{p{\linewidth}}}]{% 
	  \tt {\tiny [A$\rangle$GM]}  
	 No.\\ \tt \\ \tt To answer this question, we need to apply the principles of causal inference, specifically the concept of counterfactuals and the backdoor criterion. \\ \tt \\ \tt First, let's establish the causal graph:\\ \tt \\ \tt - Poverty → Water Quality\\ \tt - Water Company → Water Quality\\ \tt - Water Quality → Cholera\\ \tt \\ \tt We are given the following information:\\ \tt \\ \tt - High Poverty or Global Water Company → Clean Water\\ \tt - High Poverty or Clean Water → Cholera\\ \tt \\ \tt We observe that the person is served by a global water company and the region has low poverty. This implies that the person has access to clean water.\\ \tt \\ \tt Now, we want to know whether the person would get cholera if they were exposed to polluted water instead of clean water. To answer this, we need to consider the counterfactual: "What would happen if the person were exposed to polluted water, given that they are served by a global water company and the region has low poverty?"\\ \tt \\ \tt Using the backdoor criterion, we can identify the following path:\\ \tt \\ \tt - Global Water Company → Clean Water → Cholera\\ \tt \\ \tt This path is blocked by the observed variables (Global Water Company and Low Poverty), so we cannot use it to estimate the effect of polluted water on cholera.\\ \tt \\ \tt However, we can use the following path:\\ \tt \\ \tt - Low Poverty → Clean Water → Cholera\\ \tt \\ \tt This path is also blocked by the observed variables, but we can use the fact that High Poverty or Clean Water → Cholera to estimate the effect of polluted water on 
	  } 
	   } 
	   } 
	 & & \\ 
 

    \theutterance \stepcounter{utterance}  

    & & & \multicolumn{2}{p{0.3\linewidth}}{\cellcolor[rgb]{0.95,0.95,0.95}{%
	\makecell[{{p{\linewidth}}}]{% 
	  \tt {\tiny [GM$|$GM]}  
	 No 
	  } 
	   } 
	   } 
	 & & \\ 
 

    \theutterance \stepcounter{utterance}  

    & & & \multicolumn{2}{p{0.3\linewidth}}{\cellcolor[rgb]{0.95,0.95,0.95}{%
	\makecell[{{p{\linewidth}}}]{% 
	  \tt {\tiny [GM$|$GM]}  
	 no 
	  } 
	   } 
	   } 
	 & & \\ 
 

    \theutterance \stepcounter{utterance}  

    & & & \multicolumn{2}{p{0.3\linewidth}}{\cellcolor[rgb]{0.95,0.95,0.95}{%
	\makecell[{{p{\linewidth}}}]{% 
	  \tt {\tiny [GM$|$GM]}  
	 game_result = WIN 
	  } 
	   } 
	   } 
	 & & \\ 
 

\end{supertabular}
}

\end{document}
