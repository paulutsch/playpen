
\documentclass{article}
\usepackage{colortbl}
\usepackage{makecell}
\usepackage{multirow}
\usepackage{supertabular}

\begin{document}

\newcounter{utterance}

\twocolumn

{ \footnotesize  \setcounter{utterance}{1}
\setlength{\tabcolsep}{0pt}
\begin{supertabular}{c@{$\;$}|p{.15\linewidth}@{}p{.15\linewidth}p{.15\linewidth}p{.15\linewidth}p{.15\linewidth}p{.15\linewidth}}

    \# & $\;$A & \multicolumn{4}{c}{Game Master} & $\;\:$B\\
    \hline 

    \theutterance \stepcounter{utterance}  

    & & \multicolumn{4}{p{0.6\linewidth}}{\cellcolor[rgb]{0.9,0.9,0.9}{%
	\makecell[{{p{\linewidth}}}]{% 
	  \tt {\tiny [A$\langle$GM]}  
	 You are an expert in causal inference. The following question is not a typical commonsense query, but rather a meticulously designed question created by a professor specializing in causal inference, intended to assess the students' mastery of the course content.\\ \tt Imagine a self-contained, hypothetical world with only the following conditions, and without any unmentioned factors or causal relationships: Gender has a direct effect on department competitiveness and admission status. Department competitiveness has a direct effect on admission status. For individuals who are not male and applicants to a non-competitive department, the probability of admission acceptance is 58%. For individuals who are not male and applicants to a competitive department, the probability of admission acceptance is 25%. For individuals who are male and applicants to a non-competitive department, the probability of admission acceptance is 59%. For individuals who are male and applicants to a competitive department, the probability of admission acceptance is 28%. For individuals who are not male, the probability of competitive department is 29%. For individuals who are male, the probability of competitive department is 77%. Does gender positively affect admission status through department competitiveness?\\ \tt Start your answer with 'Yes' or 'No', followed by additional reasoning or evidence to support your explanation. 
	  } 
	   } 
	   } 
	 & \\ 
 

    \theutterance \stepcounter{utterance}  

    & \multicolumn{4}{p{0.6\linewidth}}{\cellcolor[rgb]{0.8,1,0.9}{%
	\makecell[{{p{\linewidth}}}]{% 
	  \tt {\tiny [A$\rangle$GM]}  
	 No\\ \tt \\ \tt To determine whether gender positively affects admission status through department competitiveness, we need to assess the causal relationship between gender and admission status, mediated by department competitiveness. \\ \tt \\ \tt First, we need to establish the causal relationship between gender and department competitiveness. The given information states that for individuals who are not male, the probability of competitive department is 29%, and for individuals who are male, the probability of competitive department is 77%. This indicates a positive association between being male and being in a competitive department.\\ \tt \\ \tt Next, we need to establish the causal relationship between department competitiveness and admission status. The given information states that for individuals who are not male and applicants to a non-competitive department, the probability of admission acceptance is 58%, and for individuals who are not male and applicants to a competitive department, the probability of admission acceptance is 25%. This indicates a negative association between being in a competitive department and admission status for non-male individuals.\\ \tt \\ \tt Now, we need to assess the causal relationship between gender and admission status, mediated by department competitiveness. To do this, we can use the counterfactual framework and the concept of mediation analysis. \\ \tt \\ \tt Let's define the following variables:\\ \tt \\ \tt - G: Gender (1 = male, 0 = not male)\\ \tt - D: Department competitiveness (1 = competitive, 0 = non-competitive)\\ \tt - A: Admission status (1 = accepted, 0 = not accepted)\\ \tt \\ \tt We can estimate the following causal effects:\\ \tt \\ \tt - P(A|G=1, D 
	  } 
	   } 
	   } 
	 & & \\ 
 

    \theutterance \stepcounter{utterance}  

    & & & \multicolumn{2}{p{0.3\linewidth}}{\cellcolor[rgb]{0.95,0.95,0.95}{%
	\makecell[{{p{\linewidth}}}]{% 
	  \tt {\tiny [GM$|$GM]}  
	 No 
	  } 
	   } 
	   } 
	 & & \\ 
 

    \theutterance \stepcounter{utterance}  

    & & & \multicolumn{2}{p{0.3\linewidth}}{\cellcolor[rgb]{0.95,0.95,0.95}{%
	\makecell[{{p{\linewidth}}}]{% 
	  \tt {\tiny [GM$|$GM]}  
	 no 
	  } 
	   } 
	   } 
	 & & \\ 
 

    \theutterance \stepcounter{utterance}  

    & & & \multicolumn{2}{p{0.3\linewidth}}{\cellcolor[rgb]{0.95,0.95,0.95}{%
	\makecell[{{p{\linewidth}}}]{% 
	  \tt {\tiny [GM$|$GM]}  
	 game_result = WIN 
	  } 
	   } 
	   } 
	 & & \\ 
 

\end{supertabular}
}

\end{document}
