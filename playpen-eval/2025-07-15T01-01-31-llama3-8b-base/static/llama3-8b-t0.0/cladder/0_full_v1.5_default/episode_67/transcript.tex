
\documentclass{article}
\usepackage{colortbl}
\usepackage{makecell}
\usepackage{multirow}
\usepackage{supertabular}

\begin{document}

\newcounter{utterance}

\twocolumn

{ \footnotesize  \setcounter{utterance}{1}
\setlength{\tabcolsep}{0pt}
\begin{supertabular}{c@{$\;$}|p{.15\linewidth}@{}p{.15\linewidth}p{.15\linewidth}p{.15\linewidth}p{.15\linewidth}p{.15\linewidth}}

    \# & $\;$A & \multicolumn{4}{c}{Game Master} & $\;\:$B\\
    \hline 

    \theutterance \stepcounter{utterance}  

    & & \multicolumn{4}{p{0.6\linewidth}}{\cellcolor[rgb]{0.9,0.9,0.9}{%
	\makecell[{{p{\linewidth}}}]{% 
	  \tt {\tiny [A$\langle$GM]}  
	 You are an expert in causal inference. The following question is not a typical commonsense query, but rather a meticulously designed question created by a professor specializing in causal inference, intended to assess the students' mastery of the course content.\\ \tt Imagine a self-contained, hypothetical world with only the following conditions, and without any unmentioned factors or causal relationships: Rixq has a direct effect on zuph and xevu. Swoy has a direct effect on zuph and xevu. Zuph has a direct effect on xevu. Swoy is unobserved. The overall probability of rixq is 70%. The probability of not rixq and xevu is 6%. The probability of rixq and xevu is 10%. Is the chance of xevu smaller when observing rixq?\\ \tt Start your answer with 'Yes' or 'No', followed by additional reasoning or evidence to support your explanation. 
	  } 
	   } 
	   } 
	 & \\ 
 

    \theutterance \stepcounter{utterance}  

    & \multicolumn{4}{p{0.6\linewidth}}{\cellcolor[rgb]{0.8,1,0.9}{%
	\makecell[{{p{\linewidth}}}]{% 
	  \tt {\tiny [A$\rangle$GM]}  
	 No.\\ \tt \\ \tt To determine whether the chance of xevu is smaller when observing rixq, we need to consider the causal relationships and the given probabilities. \\ \tt \\ \tt First, let's establish the causal relationships:\\ \tt \\ \tt 1. Rixq → zuph\\ \tt 2. Rixq → xevu\\ \tt 3. Swoy → zuph\\ \tt 4. Swoy → xevu\\ \tt 5. zuph → xevu\\ \tt \\ \tt We are given the following probabilities:\\ \tt \\ \tt 1. P(Rixq) = 0.7\\ \tt 2. P(¬Rixq ∧ Xevu) = 0.06\\ \tt 3. P(Rixq ∧ Xevu) = 0.1\\ \tt \\ \tt We want to determine whether P(Xevu | Rixq) < P(Xevu | ¬Rixq). \\ \tt \\ \tt To do this, we can use the law of total probability to calculate P(Xevu | Rixq) and P(Xevu | ¬Rixq).\\ \tt \\ \tt First, let's calculate P(Xevu | Rixq):\\ \tt \\ \tt P(Xevu | Rixq) = P(Xevu ∧ Rixq) / P(Rixq)\\ \tt = 0.1 / 0.7\\ \tt = 0.1429\\ \tt \\ \tt Next, let's calculate P(Xevu | ¬Rixq):\\ \tt \\ \tt P(¬Rixq) = 1 - P(Rixq) = 1 - 
	  } 
	   } 
	   } 
	 & & \\ 
 

    \theutterance \stepcounter{utterance}  

    & & & \multicolumn{2}{p{0.3\linewidth}}{\cellcolor[rgb]{0.95,0.95,0.95}{%
	\makecell[{{p{\linewidth}}}]{% 
	  \tt {\tiny [GM$|$GM]}  
	 No 
	  } 
	   } 
	   } 
	 & & \\ 
 

    \theutterance \stepcounter{utterance}  

    & & & \multicolumn{2}{p{0.3\linewidth}}{\cellcolor[rgb]{0.95,0.95,0.95}{%
	\makecell[{{p{\linewidth}}}]{% 
	  \tt {\tiny [GM$|$GM]}  
	 yes 
	  } 
	   } 
	   } 
	 & & \\ 
 

    \theutterance \stepcounter{utterance}  

    & & & \multicolumn{2}{p{0.3\linewidth}}{\cellcolor[rgb]{0.95,0.95,0.95}{%
	\makecell[{{p{\linewidth}}}]{% 
	  \tt {\tiny [GM$|$GM]}  
	 game_result = LOSE 
	  } 
	   } 
	   } 
	 & & \\ 
 

\end{supertabular}
}

\end{document}
