
\documentclass{article}
\usepackage{colortbl}
\usepackage{makecell}
\usepackage{multirow}
\usepackage{supertabular}

\begin{document}

\newcounter{utterance}

\twocolumn

{ \footnotesize  \setcounter{utterance}{1}
\setlength{\tabcolsep}{0pt}
\begin{supertabular}{c@{$\;$}|p{.15\linewidth}@{}p{.15\linewidth}p{.15\linewidth}p{.15\linewidth}p{.15\linewidth}p{.15\linewidth}}

    \# & $\;$A & \multicolumn{4}{c}{Game Master} & $\;\:$B\\
    \hline 

    \theutterance \stepcounter{utterance}  

    & & \multicolumn{4}{p{0.6\linewidth}}{\cellcolor[rgb]{0.9,0.9,0.9}{%
	\makecell[{{p{\linewidth}}}]{% 
	  \tt {\tiny [A$\langle$GM]}  
	 You are an expert in causal inference. The following question is not a typical commonsense query, but rather a meticulously designed question created by a professor specializing in causal inference, intended to assess the students' mastery of the course content.\\ \tt Imagine a self-contained, hypothetical world with only the following conditions, and without any unmentioned factors or causal relationships: Education level has a direct effect on skill. Skill has a direct effect on salary. Method 1: We look at how education level correlates with salary case by case according to skill. Method 2: We look directly at how education level correlates with salary in general. To understand how education level affects salary, is it more correct to use the Method 1 than Method 2?\\ \tt Start your answer with 'Yes' or 'No', followed by additional reasoning or evidence to support your explanation. 
	  } 
	   } 
	   } 
	 & \\ 
 

    \theutterance \stepcounter{utterance}  

    & \multicolumn{4}{p{0.6\linewidth}}{\cellcolor[rgb]{0.8,1,0.9}{%
	\makecell[{{p{\linewidth}}}]{% 
	  \tt {\tiny [A$\rangle$GM]}  
	 No, it is not more correct to use Method 1 than Method 2.\\ \tt \\ \tt In this hypothetical world, we have a direct causal relationship between education level and skill, and a direct causal relationship between skill and salary. This is a classic example of a causal chain or a mediator variable, where education level affects salary indirectly through its effect on skill.\\ \tt \\ \tt Method 1, which controls for skill, is an attempt to identify the causal effect of education level on salary by adjusting for the mediator variable. However, this approach is not necessary in this case because the causal relationship between education level and salary is already direct and unconfounded.\\ \tt \\ \tt Method 2, which looks directly at the correlation between education level and salary, is a more straightforward and unbiased approach to estimating the causal effect of education level on salary. By not controlling for the mediator variable, we are not introducing any additional assumptions or biases into the analysis.\\ \tt \\ \tt In fact, controlling for the mediator variable (skill) in Method 1 can lead to biased estimates of the causal effect of education level on salary, a phenomenon known as the "mediation bias" or "confounding by the mediator." This is because the mediator variable (skill) is not only a causal intermediate between education level and salary, but also a confounder that can affect the relationship between education level and salary.\\ \tt \\ \tt Therefore, in this hypothetical world, Method 2 is a more correct and unbiased approach to estimating the causal effect of education level on salary. 
	  } 
	   } 
	   } 
	 & & \\ 
 

    \theutterance \stepcounter{utterance}  

    & & & \multicolumn{2}{p{0.3\linewidth}}{\cellcolor[rgb]{0.95,0.95,0.95}{%
	\makecell[{{p{\linewidth}}}]{% 
	  \tt {\tiny [GM$|$GM]}  
	 No 
	  } 
	   } 
	   } 
	 & & \\ 
 

    \theutterance \stepcounter{utterance}  

    & & & \multicolumn{2}{p{0.3\linewidth}}{\cellcolor[rgb]{0.95,0.95,0.95}{%
	\makecell[{{p{\linewidth}}}]{% 
	  \tt {\tiny [GM$|$GM]}  
	 no 
	  } 
	   } 
	   } 
	 & & \\ 
 

    \theutterance \stepcounter{utterance}  

    & & & \multicolumn{2}{p{0.3\linewidth}}{\cellcolor[rgb]{0.95,0.95,0.95}{%
	\makecell[{{p{\linewidth}}}]{% 
	  \tt {\tiny [GM$|$GM]}  
	 game_result = WIN 
	  } 
	   } 
	   } 
	 & & \\ 
 

\end{supertabular}
}

\end{document}
