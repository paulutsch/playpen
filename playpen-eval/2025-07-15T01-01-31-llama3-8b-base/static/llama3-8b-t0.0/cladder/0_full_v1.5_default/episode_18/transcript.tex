
\documentclass{article}
\usepackage{colortbl}
\usepackage{makecell}
\usepackage{multirow}
\usepackage{supertabular}

\begin{document}

\newcounter{utterance}

\twocolumn

{ \footnotesize  \setcounter{utterance}{1}
\setlength{\tabcolsep}{0pt}
\begin{supertabular}{c@{$\;$}|p{.15\linewidth}@{}p{.15\linewidth}p{.15\linewidth}p{.15\linewidth}p{.15\linewidth}p{.15\linewidth}}

    \# & $\;$A & \multicolumn{4}{c}{Game Master} & $\;\:$B\\
    \hline 

    \theutterance \stepcounter{utterance}  

    & & \multicolumn{4}{p{0.6\linewidth}}{\cellcolor[rgb]{0.9,0.9,0.9}{%
	\makecell[{{p{\linewidth}}}]{% 
	  \tt {\tiny [A$\langle$GM]}  
	 You are an expert in causal inference. The following question is not a typical commonsense query, but rather a meticulously designed question created by a professor specializing in causal inference, intended to assess the students' mastery of the course content.\\ \tt Imagine a self-contained, hypothetical world with only the following conditions, and without any unmentioned factors or causal relationships: CEO has a direct effect on director and manager. Manager has a direct effect on employee. Director has a direct effect on employee. The overall probability of manager signing the termination letter is 35%. For managers who don't sign termination letters, the probability of employee being fired is 76%. For managers who sign termination letters, the probability of employee being fired is 42%. Is employee being fired more likely than employee not being fired overall?\\ \tt Start your answer with 'Yes' or 'No', followed by additional reasoning or evidence to support your explanation. 
	  } 
	   } 
	   } 
	 & \\ 
 

    \theutterance \stepcounter{utterance}  

    & \multicolumn{4}{p{0.6\linewidth}}{\cellcolor[rgb]{0.8,1,0.9}{%
	\makecell[{{p{\linewidth}}}]{% 
	  \tt {\tiny [A$\rangle$GM]}  
	 No.\\ \tt \\ \tt To answer this question, we need to consider the causal relationships and probabilities provided. Let's break it down step by step:\\ \tt \\ \tt 1. The probability of a manager signing the termination letter is 35%.\\ \tt 2. For managers who don't sign termination letters, the probability of employee being fired is 76%.\\ \tt 3. For managers who sign termination letters, the probability of employee being fired is 42%.\\ \tt \\ \tt We can calculate the overall probability of employee being fired by considering two scenarios:\\ \tt \\ \tt - Scenario 1: Manager signs the termination letter (35% probability)\\ \tt   - In this case, the probability of employee being fired is 42%.\\ \tt - Scenario 2: Manager doesn't sign the termination letter (65% probability, since 100% - 35% = 65%)\\ \tt   - In this case, the probability of employee being fired is 76%.\\ \tt \\ \tt To find the overall probability of employee being fired, we need to calculate the weighted average of these two scenarios:\\ \tt \\ \tt Overall probability of employee being fired = (Probability of manager signing the termination letter * Probability of employee being fired in this scenario) + (Probability of manager not signing the termination letter * Probability of employee being fired in this scenario)\\ \tt = (0.35 * 0.42) + (0.65 * 0.76)\\ \tt = 0.147 + 0.494\\ \tt = 0.641\\ \tt \\ \tt Now, let's compare this overall probability of employee being fired (0.641) with the probability of 
	  } 
	   } 
	   } 
	 & & \\ 
 

    \theutterance \stepcounter{utterance}  

    & & & \multicolumn{2}{p{0.3\linewidth}}{\cellcolor[rgb]{0.95,0.95,0.95}{%
	\makecell[{{p{\linewidth}}}]{% 
	  \tt {\tiny [GM$|$GM]}  
	 No 
	  } 
	   } 
	   } 
	 & & \\ 
 

    \theutterance \stepcounter{utterance}  

    & & & \multicolumn{2}{p{0.3\linewidth}}{\cellcolor[rgb]{0.95,0.95,0.95}{%
	\makecell[{{p{\linewidth}}}]{% 
	  \tt {\tiny [GM$|$GM]}  
	 yes 
	  } 
	   } 
	   } 
	 & & \\ 
 

    \theutterance \stepcounter{utterance}  

    & & & \multicolumn{2}{p{0.3\linewidth}}{\cellcolor[rgb]{0.95,0.95,0.95}{%
	\makecell[{{p{\linewidth}}}]{% 
	  \tt {\tiny [GM$|$GM]}  
	 game_result = LOSE 
	  } 
	   } 
	   } 
	 & & \\ 
 

\end{supertabular}
}

\end{document}
