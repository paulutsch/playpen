
\documentclass{article}
\usepackage{colortbl}
\usepackage{makecell}
\usepackage{multirow}
\usepackage{supertabular}

\begin{document}

\newcounter{utterance}

\twocolumn

{ \footnotesize  \setcounter{utterance}{1}
\setlength{\tabcolsep}{0pt}
\begin{supertabular}{c@{$\;$}|p{.15\linewidth}@{}p{.15\linewidth}p{.15\linewidth}p{.15\linewidth}p{.15\linewidth}p{.15\linewidth}}

    \# & $\;$A & \multicolumn{4}{c}{Game Master} & $\;\:$B\\
    \hline 

    \theutterance \stepcounter{utterance}  

    & & \multicolumn{4}{p{0.6\linewidth}}{\cellcolor[rgb]{0.9,0.9,0.9}{%
	\makecell[{{p{\linewidth}}}]{% 
	  \tt {\tiny [A$\langle$GM]}  
	 You are an expert in causal inference. The following question is not a typical commonsense query, but rather a meticulously designed question created by a professor specializing in causal inference, intended to assess the students' mastery of the course content.\\ \tt Imagine a self-contained, hypothetical world with only the following conditions, and without any unmentioned factors or causal relationships: Smoking has a direct effect on effort and college admission. Effort has a direct effect on college admission. For nonsmokers, the probability of college admission is 76%. For smokers, the probability of college admission is 22%. For smokers, would it be less likely to see college admission if the student had been a nonsmoker?\\ \tt Start your answer with 'Yes' or 'No', followed by additional reasoning or evidence to support your explanation. 
	  } 
	   } 
	   } 
	 & \\ 
 

    \theutterance \stepcounter{utterance}  

    & \multicolumn{4}{p{0.6\linewidth}}{\cellcolor[rgb]{0.8,1,0.9}{%
	\makecell[{{p{\linewidth}}}]{% 
	  \tt {\tiny [A$\rangle$GM]}  
	 No.\\ \tt \\ \tt To address this question, we need to consider the causal relationships and their implications. Let's denote the variables as follows: \\ \tt \\ \tt - S: Smoking status (1 for smokers, 0 for nonsmokers)\\ \tt - E: Effort (1 for high effort, 0 for low effort)\\ \tt - A: College admission (1 for admitted, 0 for not admitted)\\ \tt \\ \tt Given the conditions, we have the following causal relationships:\\ \tt \\ \tt 1. S → E (Smoking affects effort)\\ \tt 2. E → A (Effort affects college admission)\\ \tt \\ \tt We are also given the following probabilities:\\ \tt \\ \tt - P(A = 1 | S = 0) = 0.76 (Probability of college admission for nonsmokers)\\ \tt - P(A = 1 | S = 1) = 0.22 (Probability of college admission for smokers)\\ \tt \\ \tt The question asks whether it would be less likely to see college admission if the student had been a nonsmoker, given that they are a smoker. This is equivalent to asking whether the probability of college admission would decrease if the student's smoking status were to change from 1 to 0, while keeping all other variables constant.\\ \tt \\ \tt To answer this question, we need to consider the potential outcomes (PO) framework, which is a fundamental concept in causal inference. The PO framework allows us to analyze the causal effects by comparing the potential outcomes under different treatment assignments.\\ \tt \\ \tt Let's denote the potential outcomes as follows:\\ \tt \\ \tt - A0: Potential outcome if S 
	  } 
	   } 
	   } 
	 & & \\ 
 

    \theutterance \stepcounter{utterance}  

    & & & \multicolumn{2}{p{0.3\linewidth}}{\cellcolor[rgb]{0.95,0.95,0.95}{%
	\makecell[{{p{\linewidth}}}]{% 
	  \tt {\tiny [GM$|$GM]}  
	 No 
	  } 
	   } 
	   } 
	 & & \\ 
 

    \theutterance \stepcounter{utterance}  

    & & & \multicolumn{2}{p{0.3\linewidth}}{\cellcolor[rgb]{0.95,0.95,0.95}{%
	\makecell[{{p{\linewidth}}}]{% 
	  \tt {\tiny [GM$|$GM]}  
	 no 
	  } 
	   } 
	   } 
	 & & \\ 
 

    \theutterance \stepcounter{utterance}  

    & & & \multicolumn{2}{p{0.3\linewidth}}{\cellcolor[rgb]{0.95,0.95,0.95}{%
	\makecell[{{p{\linewidth}}}]{% 
	  \tt {\tiny [GM$|$GM]}  
	 game_result = WIN 
	  } 
	   } 
	   } 
	 & & \\ 
 

\end{supertabular}
}

\end{document}
