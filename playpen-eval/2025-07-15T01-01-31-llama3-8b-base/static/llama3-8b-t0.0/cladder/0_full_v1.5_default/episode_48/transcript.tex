
\documentclass{article}
\usepackage{colortbl}
\usepackage{makecell}
\usepackage{multirow}
\usepackage{supertabular}

\begin{document}

\newcounter{utterance}

\twocolumn

{ \footnotesize  \setcounter{utterance}{1}
\setlength{\tabcolsep}{0pt}
\begin{supertabular}{c@{$\;$}|p{.15\linewidth}@{}p{.15\linewidth}p{.15\linewidth}p{.15\linewidth}p{.15\linewidth}p{.15\linewidth}}

    \# & $\;$A & \multicolumn{4}{c}{Game Master} & $\;\:$B\\
    \hline 

    \theutterance \stepcounter{utterance}  

    & & \multicolumn{4}{p{0.6\linewidth}}{\cellcolor[rgb]{0.9,0.9,0.9}{%
	\makecell[{{p{\linewidth}}}]{% 
	  \tt {\tiny [A$\langle$GM]}  
	 You are an expert in causal inference. The following question is not a typical commonsense query, but rather a meticulously designed question created by a professor specializing in causal inference, intended to assess the students' mastery of the course content.\\ \tt Imagine a self-contained, hypothetical world with only the following conditions, and without any unmentioned factors or causal relationships: Listening to jazz has a direct effect on tar deposit. Tar deposit has a direct effect on lung cancer. For people who do not listen to jazz, the probability of lung cancer is 61%. For people who listen to jazz, the probability of lung cancer is 79%. For people who listen to jazz, would it be less likely to see lung cancer if the person did not listen to jazz?\\ \tt Start your answer with 'Yes' or 'No', followed by additional reasoning or evidence to support your explanation. 
	  } 
	   } 
	   } 
	 & \\ 
 

    \theutterance \stepcounter{utterance}  

    & \multicolumn{4}{p{0.6\linewidth}}{\cellcolor[rgb]{0.8,1,0.9}{%
	\makecell[{{p{\linewidth}}}]{% 
	  \tt {\tiny [A$\rangle$GM]}  
	 No.\\ \tt \\ \tt To answer this question, we need to apply the concept of counterfactuals and potential outcomes. The question asks whether, for a person who listens to jazz, it would be less likely to see lung cancer if they did not listen to jazz. This is equivalent to comparing the probability of lung cancer in the actual world (where the person listens to jazz) to the probability of lung cancer in a counterfactual world (where the person does not listen to jazz).\\ \tt \\ \tt Let's denote the probability of lung cancer as P(LC), the probability of listening to jazz as P(J), and the probability of not listening to jazz as P(¬J). We are given that P(LC|¬J) = 0.61 and P(LC|J) = 0.79.\\ \tt \\ \tt To answer the question, we need to calculate the probability of lung cancer in the counterfactual world, P(LC|J, ¬J). This is a classic example of a "front-door" adjustment, which involves using the direct effect of the exposure (listening to jazz) on the outcome (lung cancer) to estimate the counterfactual probability.\\ \tt \\ \tt We can use the following formula to estimate P(LC|J, ¬J):\\ \tt \\ \tt P(LC|J, ¬J) = P(LC|¬J) + [P(LC|J) - P(LC|¬J)] \* P(J|¬J) / P(J)\\ \tt \\ \tt where P(J| 
	  } 
	   } 
	   } 
	 & & \\ 
 

    \theutterance \stepcounter{utterance}  

    & & & \multicolumn{2}{p{0.3\linewidth}}{\cellcolor[rgb]{0.95,0.95,0.95}{%
	\makecell[{{p{\linewidth}}}]{% 
	  \tt {\tiny [GM$|$GM]}  
	 No 
	  } 
	   } 
	   } 
	 & & \\ 
 

    \theutterance \stepcounter{utterance}  

    & & & \multicolumn{2}{p{0.3\linewidth}}{\cellcolor[rgb]{0.95,0.95,0.95}{%
	\makecell[{{p{\linewidth}}}]{% 
	  \tt {\tiny [GM$|$GM]}  
	 yes 
	  } 
	   } 
	   } 
	 & & \\ 
 

    \theutterance \stepcounter{utterance}  

    & & & \multicolumn{2}{p{0.3\linewidth}}{\cellcolor[rgb]{0.95,0.95,0.95}{%
	\makecell[{{p{\linewidth}}}]{% 
	  \tt {\tiny [GM$|$GM]}  
	 game_result = LOSE 
	  } 
	   } 
	   } 
	 & & \\ 
 

\end{supertabular}
}

\end{document}
