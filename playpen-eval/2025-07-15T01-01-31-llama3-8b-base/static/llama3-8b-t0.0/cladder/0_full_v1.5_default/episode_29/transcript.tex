
\documentclass{article}
\usepackage{colortbl}
\usepackage{makecell}
\usepackage{multirow}
\usepackage{supertabular}

\begin{document}

\newcounter{utterance}

\twocolumn

{ \footnotesize  \setcounter{utterance}{1}
\setlength{\tabcolsep}{0pt}
\begin{supertabular}{c@{$\;$}|p{.15\linewidth}@{}p{.15\linewidth}p{.15\linewidth}p{.15\linewidth}p{.15\linewidth}p{.15\linewidth}}

    \# & $\;$A & \multicolumn{4}{c}{Game Master} & $\;\:$B\\
    \hline 

    \theutterance \stepcounter{utterance}  

    & & \multicolumn{4}{p{0.6\linewidth}}{\cellcolor[rgb]{0.9,0.9,0.9}{%
	\makecell[{{p{\linewidth}}}]{% 
	  \tt {\tiny [A$\langle$GM]}  
	 You are an expert in causal inference. The following question is not a typical commonsense query, but rather a meticulously designed question created by a professor specializing in causal inference, intended to assess the students' mastery of the course content.\\ \tt Imagine a self-contained, hypothetical world with only the following conditions, and without any unmentioned factors or causal relationships: Eating citrus has a direct effect on vitmain C. Vitmain C has a direct effect on scurvy. For patients not consuming citrus, the probability of scurvy is 73%. For patients consuming citrus, the probability of scurvy is 54%. Does eating citrus positively affect scurvy through vitmain C?\\ \tt Start your answer with 'Yes' or 'No', followed by additional reasoning or evidence to support your explanation. 
	  } 
	   } 
	   } 
	 & \\ 
 

    \theutterance \stepcounter{utterance}  

    & \multicolumn{4}{p{0.6\linewidth}}{\cellcolor[rgb]{0.8,1,0.9}{%
	\makecell[{{p{\linewidth}}}]{% 
	  \tt {\tiny [A$\rangle$GM]}  
	 No.\\ \tt \\ \tt To assess whether eating citrus positively affects scurvy through vitamin C, we need to consider the potential causal relationships and their directions. \\ \tt \\ \tt First, let's establish the given causal relationships: \\ \tt \\ \tt 1. Eating citrus → Vitamin C (direct effect)\\ \tt 2. Vitamin C → Scurvy (direct effect)\\ \tt \\ \tt We are also given the probabilities of scurvy for patients not consuming citrus (73%) and those consuming citrus (54%). \\ \tt \\ \tt To determine whether eating citrus positively affects scurvy through vitamin C, we need to consider the potential causal relationship between eating citrus and scurvy. \\ \tt \\ \tt One way to approach this is to use the backdoor criterion, which states that a path from the cause (eating citrus) to the effect (scurvy) is unconfounded if all backdoor paths (paths from the cause to the effect through intermediate variables) are blocked. \\ \tt \\ \tt In this case, the backdoor path from eating citrus to scurvy through vitamin C is: \\ \tt \\ \tt Eating citrus → Vitamin C → Scurvy\\ \tt \\ \tt However, we also need to consider the potential confounding variable: the probability of scurvy for patients not consuming citrus (73%). \\ \tt \\ \tt If we assume that the probability of scurvy for patients not consuming citrus is the counterfactual probability of scurvy for patients consuming citrus (i.e., the probability of scurvy if they had not eaten citrus), then we can use the backdoor criterion to assess the causal relationship between eating citrus 
	  } 
	   } 
	   } 
	 & & \\ 
 

    \theutterance \stepcounter{utterance}  

    & & & \multicolumn{2}{p{0.3\linewidth}}{\cellcolor[rgb]{0.95,0.95,0.95}{%
	\makecell[{{p{\linewidth}}}]{% 
	  \tt {\tiny [GM$|$GM]}  
	 No 
	  } 
	   } 
	   } 
	 & & \\ 
 

    \theutterance \stepcounter{utterance}  

    & & & \multicolumn{2}{p{0.3\linewidth}}{\cellcolor[rgb]{0.95,0.95,0.95}{%
	\makecell[{{p{\linewidth}}}]{% 
	  \tt {\tiny [GM$|$GM]}  
	 no 
	  } 
	   } 
	   } 
	 & & \\ 
 

    \theutterance \stepcounter{utterance}  

    & & & \multicolumn{2}{p{0.3\linewidth}}{\cellcolor[rgb]{0.95,0.95,0.95}{%
	\makecell[{{p{\linewidth}}}]{% 
	  \tt {\tiny [GM$|$GM]}  
	 game_result = WIN 
	  } 
	   } 
	   } 
	 & & \\ 
 

\end{supertabular}
}

\end{document}
