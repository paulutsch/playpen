
\documentclass{article}
\usepackage{colortbl}
\usepackage{makecell}
\usepackage{multirow}
\usepackage{supertabular}

\begin{document}

\newcounter{utterance}

\twocolumn

{ \footnotesize  \setcounter{utterance}{1}
\setlength{\tabcolsep}{0pt}
\begin{supertabular}{c@{$\;$}|p{.15\linewidth}@{}p{.15\linewidth}p{.15\linewidth}p{.15\linewidth}p{.15\linewidth}p{.15\linewidth}}

    \# & $\;$A & \multicolumn{4}{c}{Game Master} & $\;\:$B\\
    \hline 

    \theutterance \stepcounter{utterance}  

    & & \multicolumn{4}{p{0.6\linewidth}}{\cellcolor[rgb]{0.9,0.9,0.9}{%
	\makecell[{{p{\linewidth}}}]{% 
	  \tt {\tiny [A$\langle$GM]}  
	 You are an expert in causal inference. The following question is not a typical commonsense query, but rather a meticulously designed question created by a professor specializing in causal inference, intended to assess the students' mastery of the course content.\\ \tt Imagine a self-contained, hypothetical world with only the following conditions, and without any unmentioned factors or causal relationships: Gwet has a direct effect on yomx and xevu. Yomx has a direct effect on xevu. The overall probability of yomx is 75%. The probability of not yomx and xevu is 22%. The probability of yomx and xevu is 56%. Is the chance of xevu smaller when observing yomx?\\ \tt Start your answer with 'Yes' or 'No', followed by additional reasoning or evidence to support your explanation. 
	  } 
	   } 
	   } 
	 & \\ 
 

    \theutterance \stepcounter{utterance}  

    & \multicolumn{4}{p{0.6\linewidth}}{\cellcolor[rgb]{0.8,1,0.9}{%
	\makecell[{{p{\linewidth}}}]{% 
	  \tt {\tiny [A$\rangle$GM]}  
	 No.\\ \tt \\ \tt To answer this question, we need to apply the principles of causal inference, specifically the concept of conditional probability and the causal relationships between the variables.\\ \tt \\ \tt Let's denote the probability of xevu as P(xevu) and the probability of xevu given yomx as P(xevu|yomx). We are asked to compare P(xevu) with P(xevu|yomx).\\ \tt \\ \tt We are given the following information:\\ \tt \\ \tt 1. P(yomx) = 0.75\\ \tt 2. P(not yomx and xevu) = 0.22\\ \tt 3. P(yomx and xevu) = 0.56\\ \tt \\ \tt We can use the law of total probability to find P(xevu):\\ \tt \\ \tt P(xevu) = P(xevu and yomx) + P(xevu and not yomx)\\ \tt = P(yomx and xevu) + P(not yomx and xevu)\\ \tt = 0.56 + 0.22\\ \tt = 0.78\\ \tt \\ \tt Now, we need to find P(xevu|yomx). We can use the definition of conditional probability:\\ \tt \\ \tt P(xevu|yomx) = P(xevu and yomx) / P(yomx)\\ \tt = 0.56 / 0.75\\ \tt = 0.7467\\ \tt \\ \tt Comparing P(xevu) and P(xevu 
	  } 
	   } 
	   } 
	 & & \\ 
 

    \theutterance \stepcounter{utterance}  

    & & & \multicolumn{2}{p{0.3\linewidth}}{\cellcolor[rgb]{0.95,0.95,0.95}{%
	\makecell[{{p{\linewidth}}}]{% 
	  \tt {\tiny [GM$|$GM]}  
	 No 
	  } 
	   } 
	   } 
	 & & \\ 
 

    \theutterance \stepcounter{utterance}  

    & & & \multicolumn{2}{p{0.3\linewidth}}{\cellcolor[rgb]{0.95,0.95,0.95}{%
	\makecell[{{p{\linewidth}}}]{% 
	  \tt {\tiny [GM$|$GM]}  
	 yes 
	  } 
	   } 
	   } 
	 & & \\ 
 

    \theutterance \stepcounter{utterance}  

    & & & \multicolumn{2}{p{0.3\linewidth}}{\cellcolor[rgb]{0.95,0.95,0.95}{%
	\makecell[{{p{\linewidth}}}]{% 
	  \tt {\tiny [GM$|$GM]}  
	 game_result = LOSE 
	  } 
	   } 
	   } 
	 & & \\ 
 

\end{supertabular}
}

\end{document}
