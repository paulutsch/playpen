
\documentclass{article}
\usepackage{colortbl}
\usepackage{makecell}
\usepackage{multirow}
\usepackage{supertabular}

\begin{document}

\newcounter{utterance}

\twocolumn

{ \footnotesize  \setcounter{utterance}{1}
\setlength{\tabcolsep}{0pt}
\begin{supertabular}{c@{$\;$}|p{.15\linewidth}@{}p{.15\linewidth}p{.15\linewidth}p{.15\linewidth}p{.15\linewidth}p{.15\linewidth}}

    \# & $\;$A & \multicolumn{4}{c}{Game Master} & $\;\:$B\\
    \hline 

    \theutterance \stepcounter{utterance}  

    & & \multicolumn{4}{p{0.6\linewidth}}{\cellcolor[rgb]{0.9,0.9,0.9}{%
	\makecell[{{p{\linewidth}}}]{% 
	  \tt {\tiny [A$\langle$GM]}  
	 You are an expert in causal inference. The following question is not a typical commonsense query, but rather a meticulously designed question created by a professor specializing in causal inference, intended to assess the students' mastery of the course content.\\ \tt Imagine a self-contained, hypothetical world with only the following conditions, and without any unmentioned factors or causal relationships: Rixq has a direct effect on zuph and xevu. Swoy has a direct effect on zuph and xevu. Zuph has a direct effect on xevu. Swoy is unobserved. The overall probability of rixq is 3%. The probability of not rixq and xevu is 78%. The probability of rixq and xevu is 2%. Is the chance of xevu larger when observing rixq?\\ \tt Start your answer with 'Yes' or 'No', followed by additional reasoning or evidence to support your explanation. 
	  } 
	   } 
	   } 
	 & \\ 
 

    \theutterance \stepcounter{utterance}  

    & \multicolumn{4}{p{0.6\linewidth}}{\cellcolor[rgb]{0.8,1,0.9}{%
	\makecell[{{p{\linewidth}}}]{% 
	  \tt {\tiny [A$\rangle$GM]}  
	 No.\\ \tt \\ \tt To determine the effect of observing Rixq on the probability of Xevu, we need to consider the causal relationships and the given probabilities. \\ \tt \\ \tt First, let's establish the causal diagram:\\ \tt \\ \tt - Rixq → Zuph\\ \tt - Rixq → Xevu\\ \tt - Swoy → Zuph\\ \tt - Swoy → Xevu\\ \tt - Zuph → Xevu\\ \tt \\ \tt Since Swoy is unobserved, we cannot directly estimate its effect on Xevu. However, we can use the given probabilities to infer the relationship between Rixq and Xevu.\\ \tt \\ \tt Let's denote the probability of Xevu as P(Xevu) and the probability of Rixq as P(Rixq) = 0.03.\\ \tt \\ \tt We are given that P(¬Rixq, Xevu) = 0.78 and P(Rixq, Xevu) = 0.02.\\ \tt \\ \tt Using the law of total probability, we can express P(Xevu) as:\\ \tt \\ \tt P(Xevu) = P(Xevu | Rixq) * P(Rixq) + P(Xevu | ¬Rixq) * P(¬Rixq)\\ \tt \\ \tt We know that P(Rixq) = 0.03 and P(¬Rixq) = 1 - P(Rixq) = 0.97.\\ \tt \\ \tt We can rewrite P(Xevu) as: 
	  } 
	   } 
	   } 
	 & & \\ 
 

    \theutterance \stepcounter{utterance}  

    & & & \multicolumn{2}{p{0.3\linewidth}}{\cellcolor[rgb]{0.95,0.95,0.95}{%
	\makecell[{{p{\linewidth}}}]{% 
	  \tt {\tiny [GM$|$GM]}  
	 No 
	  } 
	   } 
	   } 
	 & & \\ 
 

    \theutterance \stepcounter{utterance}  

    & & & \multicolumn{2}{p{0.3\linewidth}}{\cellcolor[rgb]{0.95,0.95,0.95}{%
	\makecell[{{p{\linewidth}}}]{% 
	  \tt {\tiny [GM$|$GM]}  
	 no 
	  } 
	   } 
	   } 
	 & & \\ 
 

    \theutterance \stepcounter{utterance}  

    & & & \multicolumn{2}{p{0.3\linewidth}}{\cellcolor[rgb]{0.95,0.95,0.95}{%
	\makecell[{{p{\linewidth}}}]{% 
	  \tt {\tiny [GM$|$GM]}  
	 game_result = WIN 
	  } 
	   } 
	   } 
	 & & \\ 
 

\end{supertabular}
}

\end{document}
