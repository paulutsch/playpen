
\documentclass{article}
\usepackage{colortbl}
\usepackage{makecell}
\usepackage{multirow}
\usepackage{supertabular}

\begin{document}

\newcounter{utterance}

\twocolumn

{ \footnotesize  \setcounter{utterance}{1}
\setlength{\tabcolsep}{0pt}
\begin{supertabular}{c@{$\;$}|p{.15\linewidth}@{}p{.15\linewidth}p{.15\linewidth}p{.15\linewidth}p{.15\linewidth}p{.15\linewidth}}

    \# & $\;$A & \multicolumn{4}{c}{Game Master} & $\;\:$B\\
    \hline 

    \theutterance \stepcounter{utterance}  

    & & \multicolumn{4}{p{0.6\linewidth}}{\cellcolor[rgb]{0.9,0.9,0.9}{%
	\makecell[{{p{\linewidth}}}]{% 
	  \tt {\tiny [A$\langle$GM]}  
	 The following are multiple choice questions (with answers) about physics. Answer only with "The answer is (X)" where X is the correct letter choice.\\ \tt \\ \tt Question:\\ \tt Say the pupil of your eye has a diameter of 5 mm and you have a telescope with an aperture of 50 cm. How much more light can the telescope gather than your eye?\\ \tt Options:\\ \tt A. 1000 times more\\ \tt B. 50 times more\\ \tt C. 5000 times more\\ \tt D. 500 times more\\ \tt E. 10000 times more\\ \tt F. 20000 times more\\ \tt G. 2000 times more\\ \tt H. 100 times more\\ \tt I. 10 times more\\ \tt J. N/A\\ \tt The answer is (E)\\ \tt \\ \tt Question:\\ \tt Where do most short-period comets come from and how do we know?\\ \tt Options:\\ \tt A. The Kuiper belt; short period comets tend to be in the plane of the solar system just like the Kuiper belt.\\ \tt B. The asteroid belt; short period comets tend to come from random directions indicating a spherical distribution of comets called the asteroid belt.\\ \tt C. The asteroid belt; short period comets tend to be in the plane of the solar system just like the asteroid belt.\\ \tt D. The Oort cloud; short period comets have orbital periods similar to asteroids like Vesta and are found in the plane of the solar system just like the Oort cloud.\\ \tt E. The Oort Cloud; short period comets tend to come from random directions indicating a spherical distribution of comets called the Oort Cloud.\\ \tt F. The Oort cloud; short period comets tend to be in the plane of the solar system just like the Oort cloud.\\ \tt G. The asteroid belt; short period comets have orbital periods similar to asteroids like Vesta and are found in the plane of the solar system just like the asteroid belt.\\ \tt H. N/A\\ \tt I. N/A\\ \tt J. N/A\\ \tt The answer is (A)\\ \tt \\ \tt Question:\\ \tt A refracting telescope consists of two converging lenses separated by 100 cm. The eye-piece lens has a focal length of 20 cm. The angular magnification of the telescope is\\ \tt Options:\\ \tt A. 10\\ \tt B. 40\\ \tt C. 6\\ \tt D. 25\\ \tt E. 15\\ \tt F. 50\\ \tt G. 30\\ \tt H. 4\\ \tt I. 5\\ \tt J. 20\\ \tt The answer is (H)\\ \tt \\ \tt Question:\\ \tt Colors in a soap bubble result from light\\ \tt Options:\\ \tt A. dispersion\\ \tt B. deflection\\ \tt C. refraction\\ \tt D. reflection\\ \tt E. interference\\ \tt F. converted to a different frequency\\ \tt G. polarization\\ \tt H. absorption\\ \tt I. diffraction\\ \tt J. transmission\\ \tt The answer is (E)\\ \tt \\ \tt Question:\\ \tt A microwave oven is connected to an outlet, 120 V, and draws a current of 2 amps. At what rate is energy being used by the microwave oven?\\ \tt Options:\\ \tt A. 240 W\\ \tt B. 120 W\\ \tt C. 10 W\\ \tt D. 480 W\\ \tt E. 360 W\\ \tt F. 200 W\\ \tt G. 30 W\\ \tt H. 150 W\\ \tt I. 60 W\\ \tt J. 300 W\\ \tt The answer is (A)\\ \tt \\ \tt Question:\\ \tt The planet Pluto (radius 1180 km) is populated by three species of purple caterpillar. Studies have established the following facts: 1. A line of 5 mauve caterpillars is as long as a line of 7 violet caterpillars. 2. A line of 3 lavender caterpillars and 1 mauve caterpillar is as long as a line of 8 violet caterpillars. 3. A line of 5 lavender caterpillars, 5 mauve caterpillars and 2 violet caterpillars is 1 m long in total. 4.  A lavender caterpillar takes 10 s to crawl the length of a violet caterpillar. 5. Violet and mauve caterpillars both crawl twice as fast as lavender caterpillars. How many years would it take a mauve caterpillar to crawl around the equator of Pluto?\\ \tt Options:\\ \tt A. 30.0\\ \tt B. 15.0\\ \tt C. 25.0\\ \tt D. 28.0\\ \tt E. 18.0\\ \tt F. 20.0\\ \tt G. 40.0\\ \tt H. 12.0\\ \tt I. 35.0\\ \tt J. 23.0\\ \tt  
	  } 
	   } 
	   } 
	 & \\ 
 

    \theutterance \stepcounter{utterance}  

    & \multicolumn{4}{p{0.6\linewidth}}{\cellcolor[rgb]{0.8,1,0.9}{%
	\makecell[{{p{\linewidth}}}]{% 
	  \tt {\tiny [A$\rangle$GM]}  
	 The answer is (D). 
	  } 
	   } 
	   } 
	 & & \\ 
 

    \theutterance \stepcounter{utterance}  

    & & & \multicolumn{2}{p{0.3\linewidth}}{\cellcolor[rgb]{0.95,0.95,0.95}{%
	\makecell[{{p{\linewidth}}}]{% 
	  \tt {\tiny [GM$|$GM]}  
	 D 
	  } 
	   } 
	   } 
	 & & \\ 
 

    \theutterance \stepcounter{utterance}  

    & & & \multicolumn{2}{p{0.3\linewidth}}{\cellcolor[rgb]{0.95,0.95,0.95}{%
	\makecell[{{p{\linewidth}}}]{% 
	  \tt {\tiny [GM$|$GM]}  
	 J 
	  } 
	   } 
	   } 
	 & & \\ 
 

    \theutterance \stepcounter{utterance}  

    & & & \multicolumn{2}{p{0.3\linewidth}}{\cellcolor[rgb]{0.95,0.95,0.95}{%
	\makecell[{{p{\linewidth}}}]{% 
	  \tt {\tiny [GM$|$GM]}  
	 game_result = LOSE 
	  } 
	   } 
	   } 
	 & & \\ 
 

\end{supertabular}
}

\end{document}
