
\documentclass{article}
\usepackage{colortbl}
\usepackage{makecell}
\usepackage{multirow}
\usepackage{supertabular}

\begin{document}

\newcounter{utterance}

\twocolumn

{ \footnotesize  \setcounter{utterance}{1}
\setlength{\tabcolsep}{0pt}
\begin{supertabular}{c@{$\;$}|p{.15\linewidth}@{}p{.15\linewidth}p{.15\linewidth}p{.15\linewidth}p{.15\linewidth}p{.15\linewidth}}

    \# & $\;$A & \multicolumn{4}{c}{Game Master} & $\;\:$B\\
    \hline 

    \theutterance \stepcounter{utterance}  

    & & \multicolumn{4}{p{0.6\linewidth}}{\cellcolor[rgb]{0.9,0.9,0.9}{%
	\makecell[{{p{\linewidth}}}]{% 
	  \tt {\tiny [A$\langle$GM]}  
	 The following are multiple choice questions (with answers) about business. Answer only with "The answer is (X)" where X is the correct letter choice.\\ \tt \\ \tt Question:\\ \tt In contrast to _______, _______ aim to reward favourable behaviour by companies. The success of such campaigns have been heightened through the use of ___________, which allow campaigns to facilitate the company in achieving _________ .\\ \tt Options:\\ \tt A. Boycotts, Buyalls, Blockchain technology, Increased Sales\\ \tt B. Buycotts, Boycotts, Digital technology, Decreased Sales\\ \tt C. Boycotts, Buycotts, Digital technology, Decreased Sales\\ \tt D. Buycotts, Boycotts, Blockchain technology, Charitable donations\\ \tt E. Boycotts, Buyalls, Blockchain technology, Charitable donations\\ \tt F. Boycotts, Buycotts, Digital technology, Increased Sales\\ \tt G. Buycotts, Boycotts, Digital technology, Increased Sales\\ \tt H. Boycotts, Buycotts, Physical technology, Increased Sales\\ \tt I. Buycotts, Buyalls, Blockchain technology, Charitable donations\\ \tt J. Boycotts, Buycotts, Blockchain technology, Decreased Sales\\ \tt The answer is (F)\\ \tt \\ \tt Question:\\ \tt _______ is the direct attempt to formally or informally manage ethical issues or problems, through specific policies, practices and programmes.\\ \tt Options:\\ \tt A. Operational management\\ \tt B. Corporate governance\\ \tt C. Environmental management\\ \tt D. Business ethics management\\ \tt E. Sustainability\\ \tt F. Stakeholder management\\ \tt G. Social marketing\\ \tt H. Human resource management\\ \tt I. N/A\\ \tt J. N/A\\ \tt The answer is (D)\\ \tt \\ \tt Question:\\ \tt How can organisational structures that are characterised by democratic and inclusive styles of management be described?\\ \tt Options:\\ \tt A. Flat\\ \tt B. Bureaucratic\\ \tt C. Autocratic\\ \tt D. Hierarchical\\ \tt E. Functional\\ \tt F. Decentralized\\ \tt G. Matrix\\ \tt H. Network\\ \tt I. Divisional\\ \tt J. Centralized\\ \tt The answer is (A)\\ \tt \\ \tt Question:\\ \tt Although the content and quality can be as controlled as direct mail, response rates of this medium are lower because of the lack of a personal address mechanism. This media format is known as:\\ \tt Options:\\ \tt A. Online banners.\\ \tt B. Television advertising.\\ \tt C. Email marketing.\\ \tt D. Care lines.\\ \tt E. Direct mail.\\ \tt F. Inserts.\\ \tt G. Door to door.\\ \tt H. Radio advertising.\\ \tt I. Billboards.\\ \tt J. Social media advertising.\\ \tt The answer is (G)\\ \tt \\ \tt Question:\\ \tt In an organization, the group of people tasked with buying decisions is referred to as the _______________.\\ \tt Options:\\ \tt A. Procurement centre.\\ \tt B. Chief executive unit.\\ \tt C. Resources allocation group.\\ \tt D. Marketing department.\\ \tt E. Purchasing department.\\ \tt F. Supply chain management team.\\ \tt G. Outsourcing unit.\\ \tt H. Decision-making unit.\\ \tt I. Operations unit.\\ \tt J. Financial management team.\\ \tt The answer is (H)\\ \tt \\ \tt Question:\\ \tt The current price of gold is $412 per ounce. The storage cost is $2 per ounce per year, payable quaterly in advance. Assuming a constant intrest rate of 9% compounded quarterly, what is the theoretial forward price of gold for delivery in 9 months?\\ \tt Options:\\ \tt A. 448.25\\ \tt B. 465.75\\ \tt C. 435.67\\ \tt D. 442.02\\ \tt E. 455.90\\ \tt F. 460.30\\ \tt G. 450.50\\ \tt H. 438.88\\ \tt I. 427.15\\ \tt J. 470.10\\ \tt  
	  } 
	   } 
	   } 
	 & \\ 
 

    \theutterance \stepcounter{utterance}  

    & \multicolumn{4}{p{0.6\linewidth}}{\cellcolor[rgb]{0.8,1,0.9}{%
	\makecell[{{p{\linewidth}}}]{% 
	  \tt {\tiny [A$\rangle$GM]}  
	 To solve the last question, we need to calculate the theoretical forward price of gold for delivery in 9 months.\\ \tt \\ \tt First, we need to calculate the number of periods (n) and the interest rate per period (i). \\ \tt \\ \tt n = 9 months / 3 months (quarterly) = 3 periods\\ \tt i = 9%/4 (quarterly) = 2.25%/period = 0.0225\\ \tt \\ \tt The storage cost per period is $2/4 = $0.50 (quarterly in advance)\\ \tt \\ \tt The current price of gold is $412 per ounce.\\ \tt \\ \tt The storage cost for 3 periods is $0.50 * 3 = $1.50\\ \tt \\ \tt The interest rate for 3 periods is (1 + 0.0225)^3 - 1 = 0.0681 or 6.81%\\ \tt \\ \tt The theoretical forward price of gold for delivery in 9 months is:\\ \tt F = P + (C * n) + (P * i * n)\\ \tt = 412 + 1.50 + (412 * 0.0681 * 3)\\ \tt = 412 + 1.50 + 82.51\\ \tt = 496.01\\ \tt \\ \tt However, this is not among the options. We need to calculate the price using the given options as possible answers.\\ \tt \\ \tt The correct answer is not among the options. However, we can calculate the price using the given options as possible answers.\\ \tt \\ \tt The closest answer is A. 
	  } 
	   } 
	   } 
	 & & \\ 
 

    \theutterance \stepcounter{utterance}  

    & & & \multicolumn{2}{p{0.3\linewidth}}{\cellcolor[rgb]{0.95,0.95,0.95}{%
	\makecell[{{p{\linewidth}}}]{% 
	  \tt {\tiny [GM$|$GM]}  
	 A 
	  } 
	   } 
	   } 
	 & & \\ 
 

    \theutterance \stepcounter{utterance}  

    & & & \multicolumn{2}{p{0.3\linewidth}}{\cellcolor[rgb]{0.95,0.95,0.95}{%
	\makecell[{{p{\linewidth}}}]{% 
	  \tt {\tiny [GM$|$GM]}  
	 D 
	  } 
	   } 
	   } 
	 & & \\ 
 

    \theutterance \stepcounter{utterance}  

    & & & \multicolumn{2}{p{0.3\linewidth}}{\cellcolor[rgb]{0.95,0.95,0.95}{%
	\makecell[{{p{\linewidth}}}]{% 
	  \tt {\tiny [GM$|$GM]}  
	 game_result = LOSE 
	  } 
	   } 
	   } 
	 & & \\ 
 

\end{supertabular}
}

\end{document}
