
\documentclass{article}
\usepackage{colortbl}
\usepackage{makecell}
\usepackage{multirow}
\usepackage{supertabular}

\begin{document}

\newcounter{utterance}

\twocolumn

{ \footnotesize  \setcounter{utterance}{1}
\setlength{\tabcolsep}{0pt}
\begin{supertabular}{c@{$\;$}|p{.15\linewidth}@{}p{.15\linewidth}p{.15\linewidth}p{.15\linewidth}p{.15\linewidth}p{.15\linewidth}}

    \# & $\;$A & \multicolumn{4}{c}{Game Master} & $\;\:$B\\
    \hline 

    \theutterance \stepcounter{utterance}  

    & & \multicolumn{4}{p{0.6\linewidth}}{\cellcolor[rgb]{0.9,0.9,0.9}{%
	\makecell[{{p{\linewidth}}}]{% 
	  \tt {\tiny [A$\langle$GM]}  
	 Write an essay about how aluminium cans are used in food storage. Don’t forget to include the keywords waste, material and meal. Have more than 30 sentences in your response. 
	  } 
	   } 
	   } 
	 & \\ 
 

    \theutterance \stepcounter{utterance}  

    & \multicolumn{4}{p{0.6\linewidth}}{\cellcolor[rgb]{0.8,1,0.9}{%
	\makecell[{{p{\linewidth}}}]{% 
	  \tt {\tiny [A$\rangle$GM]}  
	 The use of aluminium cans in food storage has become a ubiquitous practice in the modern world. These lightweight, corrosion-resistant containers have revolutionized the way we store and consume food, making it easier to preserve and transport a wide variety of products. From beverages to soups, and from fruits to meats, aluminium cans have become an essential component of the food industry.\\ \tt \\ \tt One of the primary advantages of aluminium cans is their ability to prevent spoilage and extend the shelf life of food products. The airtight seal of the can prevents oxygen, moisture, and other contaminants from entering the container, thereby preserving the freshness and quality of the food. This is particularly important for perishable items such as fruits, vegetables, and dairy products, which are highly susceptible to spoilage.\\ \tt \\ \tt Aluminium cans are also incredibly versatile, making them suitable for a wide range of food products. From soft drinks and juices to soups and stews, and from canned meats to fruits and vegetables, aluminium cans can be used to store almost any type of food. The cans are also available in various sizes, from small cans for individual servings to large cans for bulk storage.\\ \tt \\ \tt In addition to their versatility, aluminium cans are also highly durable and resistant to corrosion. The metal is non-reactive, meaning that it does not transfer flavors or odors to the food, and it is also resistant to rust and corrosion, making it an ideal material for storing food in harsh environments.\\ \tt \\ \tt The use of aluminium cans has also had a significant impact on the reduction of food waste. By preserving food and preventing spoilage, aluminium cans help to reduce the amount of food that is discarded and wasted. This is particularly important in today's world, where food waste is a significant problem. According to the United Nations Food and Agriculture Organization (FAO), one-third of all food produced globally is lost or wasted, which is equivalent to about 1.3 billion tonnes of food per year.\\ \tt \\ \tt Furthermore, aluminium cans are also an environmentally friendly option for food storage. The cans are made from a recyclable material, and they can be recycled multiple times without losing their quality. In fact, aluminium is one of the most recyclable materials in the world, with over 75% of all aluminium ever produced still in use today.\\ \tt \\ \tt The production of aluminium cans also has a lower environmental impact compared to other materials. The energy required to produce aluminium cans is significantly lower than that required to produce glass or plastic containers, and the cans also require less water and land to produce.\\ \tt \\ \tt In addition to their environmental benefits, aluminium cans are also a cost-effective option for food storage. The cans are relatively inexpensive to produce and transport, making them an attractive option for food manufacturers and retailers. The cans are also easy to stack and store, making them a convenient option for food storage and transportation.\\ \tt \\ \tt The use of aluminium cans has also had a significant impact on the way we consume food. With the rise of convenience foods and ready-to-eat meals, aluminium cans have become an essential component of the food industry. From canned soups to ready-to-eat meals, aluminium cans have made it easier to prepare and consume food on the go.\\ \tt \\ \tt In conclusion, aluminium cans have revolutionized the way we store and consume food. Their ability to prevent spoilage, extend shelf life, and reduce food waste has made them an essential component of the food industry. The cans are also environmentally friendly, cost-effective, and versatile, making them an attractive option for food manufacturers and retailers. As the world continues to grapple with the challenges of food waste and sustainability, the use of aluminium cans is likely to remain an important part of the solution.\\ \tt \\ \tt In the context of meal preparation, aluminium cans are often used to store and transport ingredients. For example, canned beans, tomatoes, and vegetables are commonly used in soups, stews, and curries. The cans are also used to store and transport ready-to-eat meals, such as canned soups and stews.\\ \tt \\ \tt The use of aluminium cans has also had a significant impact on the way we prepare meals. With the rise of convenience foods and ready-to-eat meals, aluminium cans have made it easier to prepare and consume food on the go. For example, canned soups and stews can be easily heated and served, making them a convenient option for busy households.\\ \tt \\ \tt In terms of material, aluminium cans are made from a lightweight and corrosion-resistant metal. The cans are made from a combination of aluminium and other materials, such as steel and plastic, which provide additional strength and durability. The cans are also coated with a thin layer of protective material to prevent corrosion and extend the lifespan of the can.\\ \tt \\ \tt The production of aluminium cans also has a significant impact on the environment. The energy required to produce aluminium cans is significantly lower than that required to produce glass or plastic containers, and the cans also require less water and land to produce. However, the production of aluminium cans also requires the use of non-renewable energy sources, such as coal and natural gas, which can contribute to greenhouse gas emissions.\\ \tt \\ \tt In terms of waste, aluminium cans are highly recyclable and can be recycled multiple times without losing their quality. In fact, aluminium is one of the most recyclable materials in the world, with over 75% of all aluminium ever produced still in use today. The recycling of aluminium cans also helps to conserve natural resources and reduce the amount of waste sent to landfills.\\ \tt \\ \tt Overall, the use of aluminium cans in food storage has had a significant impact on the way we store and consume food. The cans are versatile, durable, and environmentally friendly, making them an attractive option for food manufacturers and retailers. As the world continues to grapple with the challenges of food waste and sustainability, the use of aluminium cans is likely to remain an important part of the solution. 
	  } 
	   } 
	   } 
	 & & \\ 
 

    \theutterance \stepcounter{utterance}  

    & & & \multicolumn{2}{p{0.3\linewidth}}{\cellcolor[rgb]{0.95,0.95,0.95}{%
	\makecell[{{p{\linewidth}}}]{% 
	  \tt {\tiny [GM$|$GM]}  
	 {'keywords:existence': {'keywords': ['waste', 'material', 'meal']}, 'length_constraints:number_sentences': {'relation': 'at least', 'num_sentences': 31}} 
	  } 
	   } 
	   } 
	 & & \\ 
 

    \theutterance \stepcounter{utterance}  

    & & & \multicolumn{2}{p{0.3\linewidth}}{\cellcolor[rgb]{0.95,0.95,0.95}{%
	\makecell[{{p{\linewidth}}}]{% 
	  \tt {\tiny [GM$|$GM]}  
	 game_result = WIN 
	  } 
	   } 
	   } 
	 & & \\ 
 

\end{supertabular}
}

\end{document}
