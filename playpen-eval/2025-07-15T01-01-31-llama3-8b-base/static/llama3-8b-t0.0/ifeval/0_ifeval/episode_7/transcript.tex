
\documentclass{article}
\usepackage{colortbl}
\usepackage{makecell}
\usepackage{multirow}
\usepackage{supertabular}

\begin{document}

\newcounter{utterance}

\twocolumn

{ \footnotesize  \setcounter{utterance}{1}
\setlength{\tabcolsep}{0pt}
\begin{supertabular}{c@{$\;$}|p{.15\linewidth}@{}p{.15\linewidth}p{.15\linewidth}p{.15\linewidth}p{.15\linewidth}p{.15\linewidth}}

    \# & $\;$A & \multicolumn{4}{c}{Game Master} & $\;\:$B\\
    \hline 

    \theutterance \stepcounter{utterance}  

    & & \multicolumn{4}{p{0.6\linewidth}}{\cellcolor[rgb]{0.9,0.9,0.9}{%
	\makecell[{{p{\linewidth}}}]{% 
	  \tt {\tiny [A$\langle$GM]}  
	 Take the text below as a starting point, and make it a complete article: "You may have to meet with a helper to work out a parenting plan. The first would be to estimate how many time you have everyday for parenting, and is that enough...."\\ \tt Avoid using the following keywords: sleep, cook, feed\\ \tt Mention the keyword 'schedule' for more than 5 times. 
	  } 
	   } 
	   } 
	 & \\ 
 

    \theutterance \stepcounter{utterance}  

    & \multicolumn{4}{p{0.6\linewidth}}{\cellcolor[rgb]{0.8,1,0.9}{%
	\makecell[{{p{\linewidth}}}]{% 
	  \tt {\tiny [A$\rangle$GM]}  
	 Creating a Parenting Plan: Understanding Your Daily Schedule\\ \tt \\ \tt When parents decide to separate or divorce, one of the most critical aspects to consider is creating a parenting plan. This plan outlines how you and your co-parent will share responsibilities and make decisions about your child's care. Meeting with a helper, such as a mediator or counselor, can be beneficial in working out a plan that suits your family's unique needs.\\ \tt \\ \tt The first step in creating a parenting plan is to estimate how many hours you have available each day for parenting. This involves taking a close look at your daily schedule and determining if it's sufficient to meet your child's needs. Consider your work schedule, personal commitments, and other responsibilities to get a clear picture of your available time.\\ \tt \\ \tt To create a realistic schedule, it's essential to be honest about your daily routine. Ask yourself questions like: What time do I wake up and go to bed? How many hours do I spend at work or school? Are there any regular appointments or activities that I need to attend? By understanding your daily schedule, you can determine how much time you have available for parenting and whether it's enough to meet your child's needs.\\ \tt \\ \tt Once you have a clear understanding of your daily schedule, you can start thinking about how to allocate your time. This may involve creating a schedule that outlines specific times for parenting responsibilities, such as spending time with your child, attending school events, or handling homework. A well-planned schedule can help ensure that your child receives the care and attention they need, even on busy days.\\ \tt \\ \tt When creating a parenting plan, it's also essential to consider your child's schedule. This may involve coordinating with your co-parent to ensure that your child's needs are met, even when you're not together. For example, you may need to schedule regular check-ins to discuss your child's progress, share information about their daily activities, or make decisions about their care.\\ \tt \\ \tt A parenting plan should be flexible and adaptable to changing circumstances. Life is unpredictable, and schedules can change unexpectedly. By having a flexible schedule in place, you can adjust to these changes and ensure that your child's needs are still met.\\ \tt \\ \tt In addition to creating a daily schedule, it's also essential to consider your long-term schedule. This may involve thinking about your child's future needs, such as education, extracurricular activities, or career goals. By having a clear understanding of your long-term schedule, you can make informed decisions about your child's care and ensure that they receive the support they need to thrive.\\ \tt \\ \tt Ultimately, creating a parenting plan requires careful consideration of your daily and long-term schedule. By working with a helper and being honest about your available time, you can create a plan that meets your child's needs and provides a stable and loving environment for them to grow and develop. 
	  } 
	   } 
	   } 
	 & & \\ 
 

    \theutterance \stepcounter{utterance}  

    & & & \multicolumn{2}{p{0.3\linewidth}}{\cellcolor[rgb]{0.95,0.95,0.95}{%
	\makecell[{{p{\linewidth}}}]{% 
	  \tt {\tiny [GM$|$GM]}  
	 {'keywords:forbidden_words': {'forbidden_words': ['sleep', 'cook', 'feed']}, 'keywords:frequency': {'relation': 'at least', 'keyword': 'schedule', 'frequency': 6}} 
	  } 
	   } 
	   } 
	 & & \\ 
 

    \theutterance \stepcounter{utterance}  

    & & & \multicolumn{2}{p{0.3\linewidth}}{\cellcolor[rgb]{0.95,0.95,0.95}{%
	\makecell[{{p{\linewidth}}}]{% 
	  \tt {\tiny [GM$|$GM]}  
	 game_result = WIN 
	  } 
	   } 
	   } 
	 & & \\ 
 

\end{supertabular}
}

\end{document}
