
\documentclass{article}
\usepackage{colortbl}
\usepackage{makecell}
\usepackage{multirow}
\usepackage{supertabular}

\begin{document}

\newcounter{utterance}

\twocolumn

{ \footnotesize  \setcounter{utterance}{1}
\setlength{\tabcolsep}{0pt}
\begin{supertabular}{c@{$\;$}|p{.15\linewidth}@{}p{.15\linewidth}p{.15\linewidth}p{.15\linewidth}p{.15\linewidth}p{.15\linewidth}}

    \# & $\;$A & \multicolumn{4}{c}{Game Master} & $\;\:$B\\
    \hline 

    \theutterance \stepcounter{utterance}  

    & & \multicolumn{4}{p{0.6\linewidth}}{\cellcolor[rgb]{0.9,0.9,0.9}{%
	\makecell[{{p{\linewidth}}}]{% 
	  \tt {\tiny [A$\langle$GM]}  
	 Write a casual blog post about how the outer solar system is different from the inner solar system, and what that means for the possibility of life. Wrap your entire response with double quotation marks. Your response should contain 17 or more sentences. 
	  } 
	   } 
	   } 
	 & \\ 
 

    \theutterance \stepcounter{utterance}  

    & \multicolumn{4}{p{0.6\linewidth}}{\cellcolor[rgb]{0.8,1,0.9}{%
	\makecell[{{p{\linewidth}}}]{% 
	  \tt {\tiny [A$\rangle$GM]}  
	 "Exploring the vastness of our solar system, we often find ourselves comparing the inner and outer regions. While the inner solar system is home to the rocky planets like Earth, Mercury, and Mars, the outer solar system is a realm of gas giants and icy worlds. The most notable difference between the two is the temperature and composition of the planets.\\ \tt \\ \tt In the inner solar system, the temperatures are relatively warm, with the average temperature on Mercury reaching up to 800 degrees Fahrenheit during the day. This is due to the close proximity to the sun, which provides a significant amount of solar radiation. In contrast, the outer solar system is much colder, with temperatures ranging from -200 to -400 degrees Fahrenheit on some of the gas giants like Neptune and Uranus.\\ \tt \\ \tt Another key difference is the composition of the planets. The inner solar system is dominated by rocky planets, which are primarily made up of silicate rocks and metals. These planets have solid surfaces and are capable of supporting liquid water, a crucial ingredient for life. On the other hand, the outer solar system is home to gas giants like Jupiter and Saturn, which are primarily composed of hydrogen and helium gases. These planets have no solid surfaces and are not capable of supporting liquid water.\\ \tt \\ \tt The possibility of life in the outer solar system is a topic of ongoing debate among scientists. While the temperatures and composition of the gas giants make it unlikely for life to exist on their surfaces, there are some interesting possibilities for life in their moons and rings. For example, the moon of Enceladus, which orbits Saturn, has a subsurface ocean that is thought to be in contact with rock, making it a potential candidate for supporting life.\\ \tt \\ \tt Similarly, the moon of Europa, which orbits Jupiter, has a subsurface ocean that is thought to be in contact with rock. This moon is considered one of the most promising places in the solar system to search for life. The presence of liquid water and a possible energy source in the form of tidal heating make Europa an attractive candidate for supporting life.\\ \tt \\ \tt In conclusion, the outer solar system is a vastly different place from the inner solar system, with colder temperatures and a composition dominated by gas and ice. While the possibility of life in the outer solar system is still a topic of debate, there are some interesting possibilities for life in the moons and rings of the gas giants. As we continue to explore our solar system, we may uncover even more surprises and possibilities for life beyond Earth." 
	  } 
	   } 
	   } 
	 & & \\ 
 

    \theutterance \stepcounter{utterance}  

    & & & \multicolumn{2}{p{0.3\linewidth}}{\cellcolor[rgb]{0.95,0.95,0.95}{%
	\makecell[{{p{\linewidth}}}]{% 
	  \tt {\tiny [GM$|$GM]}  
	 {'startend:quotation': {}, 'length_constraints:number_sentences': {'relation': 'at least', 'num_sentences': 17}} 
	  } 
	   } 
	   } 
	 & & \\ 
 

    \theutterance \stepcounter{utterance}  

    & & & \multicolumn{2}{p{0.3\linewidth}}{\cellcolor[rgb]{0.95,0.95,0.95}{%
	\makecell[{{p{\linewidth}}}]{% 
	  \tt {\tiny [GM$|$GM]}  
	 game_result = WIN 
	  } 
	   } 
	   } 
	 & & \\ 
 

\end{supertabular}
}

\end{document}
