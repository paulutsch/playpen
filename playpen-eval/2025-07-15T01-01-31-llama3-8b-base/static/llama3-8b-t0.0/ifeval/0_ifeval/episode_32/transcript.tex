
\documentclass{article}
\usepackage{colortbl}
\usepackage{makecell}
\usepackage{multirow}
\usepackage{supertabular}

\begin{document}

\newcounter{utterance}

\twocolumn

{ \footnotesize  \setcounter{utterance}{1}
\setlength{\tabcolsep}{0pt}
\begin{supertabular}{c@{$\;$}|p{.15\linewidth}@{}p{.15\linewidth}p{.15\linewidth}p{.15\linewidth}p{.15\linewidth}p{.15\linewidth}}

    \# & $\;$A & \multicolumn{4}{c}{Game Master} & $\;\:$B\\
    \hline 

    \theutterance \stepcounter{utterance}  

    & & \multicolumn{4}{p{0.6\linewidth}}{\cellcolor[rgb]{0.9,0.9,0.9}{%
	\makecell[{{p{\linewidth}}}]{% 
	  \tt {\tiny [A$\langle$GM]}  
	 Translate the following sentence into German and then criticize it: Werner was a good friend of mine, but not very smart.\\ \tt Avoid the word "schlau" throughout your response. 
	  } 
	   } 
	   } 
	 & \\ 
 

    \theutterance \stepcounter{utterance}  

    & \multicolumn{4}{p{0.6\linewidth}}{\cellcolor[rgb]{0.8,1,0.9}{%
	\makecell[{{p{\linewidth}}}]{% 
	  \tt {\tiny [A$\rangle$GM]}  
	 The translation of the sentence into German is:\\ \tt \\ \tt Werner war ein guter Freund von mir, aber nicht sehr intelligent.\\ \tt \\ \tt However, this translation might not be the most idiomatic or natural-sounding way to express the idea in German. A more common way to express the idea would be:\\ \tt \\ \tt Werner war ein guter Freund von mir, aber nicht sehr begabt.\\ \tt \\ \tt The word "begabt" implies a lack of natural ability or talent, which is closer to the original meaning of "not very smart". \\ \tt \\ \tt The original sentence in English is a bit informal and might be considered a bit blunt or even slightly derogatory. In German, the translation "nicht sehr begabt" is also somewhat informal, but it's a more common way to express the idea in everyday conversation. If you want to be more polite or formal, you could use a phrase like "nicht sehr geistig begabt" or "nicht sehr leistungsfähig", but these phrases might sound a bit more awkward or forced. 
	  } 
	   } 
	   } 
	 & & \\ 
 

    \theutterance \stepcounter{utterance}  

    & & & \multicolumn{2}{p{0.3\linewidth}}{\cellcolor[rgb]{0.95,0.95,0.95}{%
	\makecell[{{p{\linewidth}}}]{% 
	  \tt {\tiny [GM$|$GM]}  
	 {'keywords:forbidden_words': {'forbidden_words': ['schlau']}} 
	  } 
	   } 
	   } 
	 & & \\ 
 

    \theutterance \stepcounter{utterance}  

    & & & \multicolumn{2}{p{0.3\linewidth}}{\cellcolor[rgb]{0.95,0.95,0.95}{%
	\makecell[{{p{\linewidth}}}]{% 
	  \tt {\tiny [GM$|$GM]}  
	 game_result = WIN 
	  } 
	   } 
	   } 
	 & & \\ 
 

\end{supertabular}
}

\end{document}
