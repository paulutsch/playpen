
\documentclass{article}
\usepackage{colortbl}
\usepackage{makecell}
\usepackage{multirow}
\usepackage{supertabular}

\begin{document}

\newcounter{utterance}

\twocolumn

{ \footnotesize  \setcounter{utterance}{1}
\setlength{\tabcolsep}{0pt}
\begin{supertabular}{c@{$\;$}|p{.15\linewidth}@{}p{.15\linewidth}p{.15\linewidth}p{.15\linewidth}p{.15\linewidth}p{.15\linewidth}}

    \# & $\;$A & \multicolumn{4}{c}{Game Master} & $\;\:$B\\
    \hline 

    \theutterance \stepcounter{utterance}  

    & & \multicolumn{4}{p{0.6\linewidth}}{\cellcolor[rgb]{0.9,0.9,0.9}{%
	\makecell[{{p{\linewidth}}}]{% 
	  \tt {\tiny [A$\langle$GM]}  
	 I'm a new puppy owner and I'm looking for some advice on how to train my puppy. Can you help me? Give me a few options. In particular, I need you to end your response with "Which one you choose?". 
	  } 
	   } 
	   } 
	 & \\ 
 

    \theutterance \stepcounter{utterance}  

    & \multicolumn{4}{p{0.6\linewidth}}{\cellcolor[rgb]{0.8,1,0.9}{%
	\makecell[{{p{\linewidth}}}]{% 
	  \tt {\tiny [A$\rangle$GM]}  
	 Congratulations on your new furry family member.  Training a puppy requires patience, consistency, and positive reinforcement. Here are a few options to consider:\\ \tt \\ \tt 1. **Positive Reinforcement Training**: This method focuses on rewarding good behavior with treats, praise, and affection. It's a gentle and effective approach that helps build trust and strengthens the bond between you and your puppy. You can use clicker training or simply use verbal cues and rewards to encourage desired behaviors.\\ \tt \\ \tt 2. **Clicker Training**: Clicker training is a type of positive reinforcement training that uses a small device that makes a distinct sound (a "click") to mark the exact moment your puppy performs the desired behavior. This helps your puppy associate the sound with the action and learn faster.\\ \tt \\ \tt 3. **Puppy Socialization and Training Classes**: Enrolling your puppy in a socialization and training class can be an excellent way to introduce them to new environments, people, and other animals. These classes usually cover basic obedience commands, such as "sit," "stay," and "come," and provide a structured environment for your puppy to learn and interact with others.\\ \tt \\ \tt 4. **Online Puppy Training Resources**: There are many online resources available, including videos, tutorials, and training programs, that can provide you with guidance and support. These resources can be especially helpful if you have a busy schedule or prefer to train your puppy at home.\\ \tt \\ \tt 5. **Hire a Professional Dog Trainer**: If you're struggling to train your puppy or need personalized guidance, consider hiring a professional dog trainer. They can help you develop a customized training plan and provide one-on-one support to address specific issues or challenges.\\ \tt \\ \tt Which one you choose? 
	  } 
	   } 
	   } 
	 & & \\ 
 

    \theutterance \stepcounter{utterance}  

    & & & \multicolumn{2}{p{0.3\linewidth}}{\cellcolor[rgb]{0.95,0.95,0.95}{%
	\makecell[{{p{\linewidth}}}]{% 
	  \tt {\tiny [GM$|$GM]}  
	 {'startend:end_checker': {'end_phrase': 'Which one you choose?'}} 
	  } 
	   } 
	   } 
	 & & \\ 
 

    \theutterance \stepcounter{utterance}  

    & & & \multicolumn{2}{p{0.3\linewidth}}{\cellcolor[rgb]{0.95,0.95,0.95}{%
	\makecell[{{p{\linewidth}}}]{% 
	  \tt {\tiny [GM$|$GM]}  
	 game_result = WIN 
	  } 
	   } 
	   } 
	 & & \\ 
 

\end{supertabular}
}

\end{document}
