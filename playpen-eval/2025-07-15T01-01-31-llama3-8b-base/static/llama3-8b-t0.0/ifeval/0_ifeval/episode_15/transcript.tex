
\documentclass{article}
\usepackage{colortbl}
\usepackage{makecell}
\usepackage{multirow}
\usepackage{supertabular}

\begin{document}

\newcounter{utterance}

\twocolumn

{ \footnotesize  \setcounter{utterance}{1}
\setlength{\tabcolsep}{0pt}
\begin{supertabular}{c@{$\;$}|p{.15\linewidth}@{}p{.15\linewidth}p{.15\linewidth}p{.15\linewidth}p{.15\linewidth}p{.15\linewidth}}

    \# & $\;$A & \multicolumn{4}{c}{Game Master} & $\;\:$B\\
    \hline 

    \theutterance \stepcounter{utterance}  

    & & \multicolumn{4}{p{0.6\linewidth}}{\cellcolor[rgb]{0.9,0.9,0.9}{%
	\makecell[{{p{\linewidth}}}]{% 
	  \tt {\tiny [A$\langle$GM]}  
	 Write an interesting and funny article about the biology of a banana peel. In your response, the word disappointed should appear at least 2 times, and at least six section should be highlighted with markdown,  i.e *banana peel*. 
	  } 
	   } 
	   } 
	 & \\ 
 

    \theutterance \stepcounter{utterance}  

    & \multicolumn{4}{p{0.6\linewidth}}{\cellcolor[rgb]{0.8,1,0.9}{%
	\makecell[{{p{\linewidth}}}]{% 
	  \tt {\tiny [A$\rangle$GM]}  
	 **The Peel of Despair: Unpeeling the Secrets of the Banana**\\ \tt \\ \tt As we sit here, munching on our favorite yellow snack, have you ever stopped to think about the humble banana peel? It's the unsung hero of the fruit world, often discarded and forgotten in the trash. But fear not, dear reader, for today we're going to delve into the fascinating world of *banana peel* biology and uncover its secrets.\\ \tt \\ \tt ### **The Anatomy of a *Banana Peel***\\ \tt \\ \tt \\ \tt A *banana peel* is made up of several layers, including the epidermis, the mesocarp, and the endocarp. The epidermis is the outermost layer, responsible for protecting the delicate fruit beneath. The mesocarp is the middle layer, composed of long, thin cells that provide structure and support. And the endocarp is the innermost layer, a thin, papery membrane that separates the fruit from the stem.\\ \tt \\ \tt But what's really interesting is that the *banana peel* is not just a passive wrapper – it's actually a dynamic, living tissue that responds to its environment. For example, when a *banana peel* is exposed to light, it can change color and become more susceptible to disease. It's like it's trying to tell us something, but we're too busy eating the fruit to listen.\\ \tt \\ \tt ### **The Life Cycle of a *Banana Peel***\\ \tt \\ \tt \\ \tt A *banana peel* has a relatively short lifespan, typically lasting only a few days after the fruit is ripe. But during its brief existence, it undergoes a remarkable transformation. At first, it's a bright yellow color, but as it ages, it turns brown and becomes increasingly brittle. It's like it's going through a midlife crisis, trying to hold on to its youth but ultimately succumbing to the inevitable.\\ \tt \\ \tt But don't be disappointed – the *banana peel* may be short-lived, but it plays a crucial role in the fruit's development. It helps to regulate the fruit's water content, prevent disease, and even attract beneficial insects. So the next time you're tempted to toss that *banana peel* in the trash, remember its importance in the grand scheme of things.\\ \tt \\ \tt ### **The Science of *Banana Peel* Slipping**\\ \tt \\ \tt \\ \tt We've all been there – you're walking down the street, enjoying a snack, when suddenly your *banana peel* slips out of your hand and lands on the ground with a splat. It's a classic comedy moment, but have you ever wondered why it happens? The answer lies in the unique properties of the *banana peel* itself.\\ \tt \\ \tt You see, the *banana peel* is made up of a type of cellulose that's particularly slippery. When it's exposed to moisture, it becomes even more slippery, making it prone to slipping out of your hand. It's like it's trying to escape, but ultimately ends up on the ground, where it's trampled by unsuspecting pedestrians.\\ \tt \\ \tt ### **The Cultural Significance of *Banana Peels***\\ \tt \\ \tt \\ \tt *Banana peels* have been a part of human culture for centuries, featuring in everything from cartoons to comedy sketches. Who can forget the classic "slip on a *banana peel*" gag, popularized by the likes of Charlie Chaplin and the Three Stooges? It's a timeless joke that never gets old, and it's all thanks to the humble *banana peel*.\\ \tt \\ \tt But *banana peels* have also been used in more serious contexts, such as in traditional medicine and even as a natural dye. They're a versatile ingredient that's been used in countless ways, and it's a testament to the ingenuity of our ancestors that they were able to find so many uses for something that's often discarded.\\ \tt \\ \tt ### **The Future of *Banana Peels***\\ \tt \\ \tt \\ \tt As we look to the future, it's clear that *banana peels* will continue to play an important role in our lives. With the rise of sustainable living and eco-friendly practices, *banana peels* are being repurposed in all sorts of creative ways. From biodegradable packaging to natural fertilizers, the possibilities are endless.\\ \tt \\ \tt So the next time you're tempted to toss that *banana peel* in the trash, remember its potential. It may be disappointing to think that it's just a disposable wrapper, but trust us – it's so much more than that.\\ \tt \\ \tt ### **Conclusion: The *Banana Peel* Revolution**\\ \tt \\ \tt \\ \tt In conclusion, the *banana peel* is more than just a piece of fruit – it's a living, breathing entity that's full of surprises. From its dynamic anatomy to its cultural significance, the *banana peel* is a true marvel of nature. So the next time you're eating a banana, take a moment to appreciate the humble *banana peel* – it's more than just a wrapper, it's a work of art. 
	  } 
	   } 
	   } 
	 & & \\ 
 

    \theutterance \stepcounter{utterance}  

    & & & \multicolumn{2}{p{0.3\linewidth}}{\cellcolor[rgb]{0.95,0.95,0.95}{%
	\makecell[{{p{\linewidth}}}]{% 
	  \tt {\tiny [GM$|$GM]}  
	 {'keywords:frequency': {'relation': 'at least', 'keyword': 'disappointed', 'frequency': 2}, 'detectable_format:number_highlighted_sections': {'num_highlights': 6}} 
	  } 
	   } 
	   } 
	 & & \\ 
 

    \theutterance \stepcounter{utterance}  

    & & & \multicolumn{2}{p{0.3\linewidth}}{\cellcolor[rgb]{0.95,0.95,0.95}{%
	\makecell[{{p{\linewidth}}}]{% 
	  \tt {\tiny [GM$|$GM]}  
	 game_result = LOSE 
	  } 
	   } 
	   } 
	 & & \\ 
 

\end{supertabular}
}

\end{document}
