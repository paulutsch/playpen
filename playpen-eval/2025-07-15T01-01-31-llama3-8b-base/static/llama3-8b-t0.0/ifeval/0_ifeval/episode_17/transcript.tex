
\documentclass{article}
\usepackage{colortbl}
\usepackage{makecell}
\usepackage{multirow}
\usepackage{supertabular}

\begin{document}

\newcounter{utterance}

\twocolumn

{ \footnotesize  \setcounter{utterance}{1}
\setlength{\tabcolsep}{0pt}
\begin{supertabular}{c@{$\;$}|p{.15\linewidth}@{}p{.15\linewidth}p{.15\linewidth}p{.15\linewidth}p{.15\linewidth}p{.15\linewidth}}

    \# & $\;$A & \multicolumn{4}{c}{Game Master} & $\;\:$B\\
    \hline 

    \theutterance \stepcounter{utterance}  

    & & \multicolumn{4}{p{0.6\linewidth}}{\cellcolor[rgb]{0.9,0.9,0.9}{%
	\makecell[{{p{\linewidth}}}]{% 
	  \tt {\tiny [A$\langle$GM]}  
	 The Jimenez family has 20 members. Each member has at least one pet. If there are 32 pets in total, what is the maximum number of members that can possibly have 3 pets? Solve this step by step, and wrap your entire response (reasoning process and the final answer) with double quotation marks. 
	  } 
	   } 
	   } 
	 & \\ 
 

    \theutterance \stepcounter{utterance}  

    & \multicolumn{4}{p{0.6\linewidth}}{\cellcolor[rgb]{0.8,1,0.9}{%
	\makecell[{{p{\linewidth}}}]{% 
	  \tt {\tiny [A$\rangle$GM]}  
	 "Let's break down the problem step by step:\\ \tt \\ \tt 1. The total number of members in the Jimenez family is 20.\\ \tt 2. Each member has at least one pet, so the minimum number of pets is 20.\\ \tt 3. However, we know that there are 32 pets in total, which is more than the minimum.\\ \tt 4. To find the maximum number of members that can have 3 pets, we need to consider the remaining pets after we assign 1 pet to each of the 20 members.\\ \tt 5. Since each member has at least one pet, we can assign 1 pet to each of the 20 members, leaving us with 32 - 20 = 12 pets.\\ \tt 6. We want to maximize the number of members with 3 pets, so we'll assign 3 pets to as many members as possible.\\ \tt 7. Let's say x members have 3 pets. Then, the number of pets assigned to these x members is 3x.\\ \tt 8. The remaining (20 - x) members will have 1 pet each, so the number of pets assigned to them is 20 - x.\\ \tt 9. Since we have 12 pets left, we can set up the equation: 3x + (20 - x) = 12.\\ \tt 10. Simplifying the equation, we get: 3x + 20 - x = 12.\\ \tt 11. Combining like terms, we get: 2x + 20 = 12.\\ \tt 12. Subtracting 20 from both sides, we get: 2x = -8.\\ \tt 13. Dividing both sides by 2, we get: x = -4.\\ \tt 14. However, x cannot be negative, since we're counting the number of members.\\ \tt 15. This means that our initial assumption that we can assign 3 pets to x members is incorrect.\\ \tt 16. Let's try a different approach. We'll assign 3 pets to as many members as possible, and then assign 1 pet to the remaining members.\\ \tt 17. Let's say x members have 3 pets, and y members have 1 pet. Then, the total number of pets is 3x + y.\\ \tt 18. We know that 3x + y = 32.\\ \tt 19. We also know that x + y = 20, since there are 20 members in total.\\ \tt 20. We can solve this system of equations by substitution or elimination.\\ \tt 21. Let's use substitution. We can solve the second equation for y: y = 20 - x.\\ \tt 22. Substituting this expression for y into the first equation, we get: 3x + (20 - x) = 32.\\ \tt 23. Simplifying the equation, we get: 3x + 20 - x = 32.\\ \tt 24. Combining like terms, we get: 2x + 20 = 32.\\ \tt 25. Subtracting 20 from both sides, we get: 2x = 12.\\ \tt 26. Dividing both sides by 2, we get: x = 6.\\ \tt 27. Now that we have the value of x, we can find the value of y: y = 20 - x = 20 - 6 = 14.\\ \tt 28. This means that 6 members have 3 pets, and 14 members have 1 pet.\\ \tt 29. Therefore, the maximum number of members that can possibly have 3 pets is 6."\\ \tt \\ \tt "6" 
	  } 
	   } 
	   } 
	 & & \\ 
 

    \theutterance \stepcounter{utterance}  

    & & & \multicolumn{2}{p{0.3\linewidth}}{\cellcolor[rgb]{0.95,0.95,0.95}{%
	\makecell[{{p{\linewidth}}}]{% 
	  \tt {\tiny [GM$|$GM]}  
	 {'startend:quotation': {}} 
	  } 
	   } 
	   } 
	 & & \\ 
 

    \theutterance \stepcounter{utterance}  

    & & & \multicolumn{2}{p{0.3\linewidth}}{\cellcolor[rgb]{0.95,0.95,0.95}{%
	\makecell[{{p{\linewidth}}}]{% 
	  \tt {\tiny [GM$|$GM]}  
	 game_result = WIN 
	  } 
	   } 
	   } 
	 & & \\ 
 

\end{supertabular}
}

\end{document}
