
\documentclass{article}
\usepackage{colortbl}
\usepackage{makecell}
\usepackage{multirow}
\usepackage{supertabular}

\begin{document}

\newcounter{utterance}

\twocolumn

{ \footnotesize  \setcounter{utterance}{1}
\setlength{\tabcolsep}{0pt}
\begin{supertabular}{c@{$\;$}|p{.15\linewidth}@{}p{.15\linewidth}p{.15\linewidth}p{.15\linewidth}p{.15\linewidth}p{.15\linewidth}}

    \# & $\;$A & \multicolumn{4}{c}{Game Master} & $\;\:$B\\
    \hline 

    \theutterance \stepcounter{utterance}  

    & & \multicolumn{4}{p{0.6\linewidth}}{\cellcolor[rgb]{0.9,0.9,0.9}{%
	\makecell[{{p{\linewidth}}}]{% 
	  \tt {\tiny [A$\langle$GM]}  
	 You are an expert in causal inference. The following question is not a typical commonsense query, but rather a meticulously designed question created by a professor specializing in causal inference, intended to assess the students' mastery of the course content.\\ \tt Imagine a self-contained, hypothetical world with only the following conditions, and without any unmentioned factors or causal relationships: Hwax has a direct effect on kraz and pexu. Pexu has a direct effect on rukz. Kraz has a direct effect on rukz. The overall probability of pexu is 41%. For those who are not pexu, the probability of rukz is 66%. For those who are pexu, the probability of rukz is 22%. Is rukz more likely than not rukz overall?\\ \tt Start your answer with 'Yes' or 'No', followed by additional reasoning or evidence to support your explanation. 
	  } 
	   } 
	   } 
	 & \\ 
 

    \theutterance \stepcounter{utterance}  

    & \multicolumn{4}{p{0.6\linewidth}}{\cellcolor[rgb]{0.8,1,0.9}{%
	\makecell[{{p{\linewidth}}}]{% 
	  \tt {\tiny [A$\rangle$GM]}  
	 Yes. To determine if rukz is more likely than not rukz overall, we need to calculate the overall probability of rukz. We are given the following information:\\ \tt \\ \tt 1. The probability of pexu is 41%.\\ \tt 2. For those who are not pexu, the probability of rukz is 66%.\\ \tt 3. For those who are pexu, the probability of rukz is 22%.\\ \tt \\ \tt Let's denote the probability of rukz as P(rukz). We can calculate P(rukz) using the law of total probability:\\ \tt \\ \tt P(rukz) = P(rukz | not pexu) \* P(not pexu) + P(rukz | pexu) \* P(pexu)\\ \tt \\ \tt We know that P(pexu) = 0.41. We need to find P(not pexu).\\ \tt \\ \tt P(not pexu) = 1 - P(pexu) = 1 - 0.41 = 0.59\\ \tt \\ \tt Now, we can substitute the values into the equation:\\ \tt \\ \tt P(rukz) = 0.66 \* 0.59 + 0.22 \* 0.41\\ \tt = 0.3884 + 0.0902\\ \tt = 0.4786\\ \tt \\ \tt Since 0.4786 is greater than 0.5, we can conclude that rukz is more likely than not rukz overall. 
	  } 
	   } 
	   } 
	 & & \\ 
 

    \theutterance \stepcounter{utterance}  

    & & & \multicolumn{2}{p{0.3\linewidth}}{\cellcolor[rgb]{0.95,0.95,0.95}{%
	\makecell[{{p{\linewidth}}}]{% 
	  \tt {\tiny [GM$|$GM]}  
	 Yes 
	  } 
	   } 
	   } 
	 & & \\ 
 

    \theutterance \stepcounter{utterance}  

    & & & \multicolumn{2}{p{0.3\linewidth}}{\cellcolor[rgb]{0.95,0.95,0.95}{%
	\makecell[{{p{\linewidth}}}]{% 
	  \tt {\tiny [GM$|$GM]}  
	 no 
	  } 
	   } 
	   } 
	 & & \\ 
 

    \theutterance \stepcounter{utterance}  

    & & & \multicolumn{2}{p{0.3\linewidth}}{\cellcolor[rgb]{0.95,0.95,0.95}{%
	\makecell[{{p{\linewidth}}}]{% 
	  \tt {\tiny [GM$|$GM]}  
	 game_result = LOSE 
	  } 
	   } 
	   } 
	 & & \\ 
 

\end{supertabular}
}

\end{document}
