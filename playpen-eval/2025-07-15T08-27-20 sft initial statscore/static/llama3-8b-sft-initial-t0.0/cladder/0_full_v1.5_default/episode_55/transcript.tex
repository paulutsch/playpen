
\documentclass{article}
\usepackage{colortbl}
\usepackage{makecell}
\usepackage{multirow}
\usepackage{supertabular}

\begin{document}

\newcounter{utterance}

\twocolumn

{ \footnotesize  \setcounter{utterance}{1}
\setlength{\tabcolsep}{0pt}
\begin{supertabular}{c@{$\;$}|p{.15\linewidth}@{}p{.15\linewidth}p{.15\linewidth}p{.15\linewidth}p{.15\linewidth}p{.15\linewidth}}

    \# & $\;$A & \multicolumn{4}{c}{Game Master} & $\;\:$B\\
    \hline 

    \theutterance \stepcounter{utterance}  

    & & \multicolumn{4}{p{0.6\linewidth}}{\cellcolor[rgb]{0.9,0.9,0.9}{%
	\makecell[{{p{\linewidth}}}]{% 
	  \tt {\tiny [A$\langle$GM]}  
	 You are an expert in causal inference. The following question is not a typical commonsense query, but rather a meticulously designed question created by a professor specializing in causal inference, intended to assess the students' mastery of the course content.\\ \tt Imagine a self-contained, hypothetical world with only the following conditions, and without any unmentioned factors or causal relationships: Alice waking up has a direct effect on rainfall. Traffic has a direct effect on rainfall. The overall probability of waking up late is 9%. The probability of waking up on time and high rainfall is 53%. The probability of waking up late and high rainfall is 8%. Is the chance of high rainfall smaller when observing waking up late?\\ \tt Start your answer with 'Yes' or 'No', followed by additional reasoning or evidence to support your explanation. 
	  } 
	   } 
	   } 
	 & \\ 
 

    \theutterance \stepcounter{utterance}  

    & \multicolumn{4}{p{0.6\linewidth}}{\cellcolor[rgb]{0.8,1,0.9}{%
	\makecell[{{p{\linewidth}}}]{% 
	  \tt {\tiny [A$\rangle$GM]}  
	 Yes. The probability of high rainfall given waking up late can be calculated using Bayes' theorem. Let's denote the event of waking up late as \(L\), the event of high rainfall as \(H\), and the event of waking up on time as \(T\). We are given the following probabilities:\\ \tt \\ \tt 1. \(P(L) = 0.09\)\\ \tt 2. \(P(H \cap T) = 0.53\)\\ \tt 3. \(P(H \cap L) = 0.08\)\\ \tt \\ \tt We need to find \(P(H \mid L)\).\\ \tt \\ \tt Using Bayes' theorem:\\ \tt \\ \tt \[P(H \mid L) = \frac{P(L \mid H) \cdot P(H)}{P(L)}\]\\ \tt \\ \tt We need to calculate \(P(L \mid H)\). We know that:\\ \tt \\ \tt \[P(L \mid H) = \frac{P(L \cap H)}{P(H)}\]\\ \tt \\ \tt Given \(P(H \cap L) = 0.08\) and \(P(H) = P(H \cap T) + P(H \cap L)\), we can find \(P(H)\):\\ \tt \\ \tt \[P(H) = 0.53 + 0.08 = 0.61\]\\ \tt \\ \tt Thus,\\ \tt \\ \tt \[P(L \mid H) = \frac{0.08}{0.61}\]\\ \tt \\ \tt Now, we can calculate \(P(H \mid L)\):\\ \tt \\ \tt \[P(H \mid L) = 
	  } 
	   } 
	   } 
	 & & \\ 
 

    \theutterance \stepcounter{utterance}  

    & & & \multicolumn{2}{p{0.3\linewidth}}{\cellcolor[rgb]{0.95,0.95,0.95}{%
	\makecell[{{p{\linewidth}}}]{% 
	  \tt {\tiny [GM$|$GM]}  
	 Yes 
	  } 
	   } 
	   } 
	 & & \\ 
 

    \theutterance \stepcounter{utterance}  

    & & & \multicolumn{2}{p{0.3\linewidth}}{\cellcolor[rgb]{0.95,0.95,0.95}{%
	\makecell[{{p{\linewidth}}}]{% 
	  \tt {\tiny [GM$|$GM]}  
	 no 
	  } 
	   } 
	   } 
	 & & \\ 
 

    \theutterance \stepcounter{utterance}  

    & & & \multicolumn{2}{p{0.3\linewidth}}{\cellcolor[rgb]{0.95,0.95,0.95}{%
	\makecell[{{p{\linewidth}}}]{% 
	  \tt {\tiny [GM$|$GM]}  
	 game_result = LOSE 
	  } 
	   } 
	   } 
	 & & \\ 
 

\end{supertabular}
}

\end{document}
