
\documentclass{article}
\usepackage{colortbl}
\usepackage{makecell}
\usepackage{multirow}
\usepackage{supertabular}

\begin{document}

\newcounter{utterance}

\twocolumn

{ \footnotesize  \setcounter{utterance}{1}
\setlength{\tabcolsep}{0pt}
\begin{supertabular}{c@{$\;$}|p{.15\linewidth}@{}p{.15\linewidth}p{.15\linewidth}p{.15\linewidth}p{.15\linewidth}p{.15\linewidth}}

    \# & $\;$A & \multicolumn{4}{c}{Game Master} & $\;\:$B\\
    \hline 

    \theutterance \stepcounter{utterance}  

    & & \multicolumn{4}{p{0.6\linewidth}}{\cellcolor[rgb]{0.9,0.9,0.9}{%
	\makecell[{{p{\linewidth}}}]{% 
	  \tt {\tiny [A$\langle$GM]}  
	 You are an expert in causal inference. The following question is not a typical commonsense query, but rather a meticulously designed question created by a professor specializing in causal inference, intended to assess the students' mastery of the course content.\\ \tt Imagine a self-contained, hypothetical world with only the following conditions, and without any unmentioned factors or causal relationships: Xevo has a direct effect on tijw and tijv. Tijv has a direct effect on gyzp. Tijw has a direct effect on gyzp. The overall probability of xevo is 48%. For those who are not xevo, the probability of gyzp is 39%. For those who are xevo, the probability of gyzp is 78%. Is gyzp more likely than not gyzp overall?\\ \tt Start your answer with 'Yes' or 'No', followed by additional reasoning or evidence to support your explanation. 
	  } 
	   } 
	   } 
	 & \\ 
 

    \theutterance \stepcounter{utterance}  

    & \multicolumn{4}{p{0.6\linewidth}}{\cellcolor[rgb]{0.8,1,0.9}{%
	\makecell[{{p{\linewidth}}}]{% 
	  \tt {\tiny [A$\rangle$GM]}  
	 Yes. The probability of gyzp overall can be calculated using the law of total probability. Let's denote the probability of gyzp as P(gyzp) and the probability of xevo as P(xevo). We are given that P(xevo) = 0.48. For those who are not xevo, the probability of gyzp is 0.39, so the probability of gyzp given not xevo is P(gyzp | not xevo) = 0.39. For those who are xevo, the probability of gyzp is 0.78, so the probability of gyzp given xevo is P(gyzp | xevo) = 0.78.\\ \tt \\ \tt Using the law of total probability, we can calculate the overall probability of gyzp as follows:\\ \tt \\ \tt P(gyzp) = P(gyzp | xevo) \* P(xevo) + P(gyzp | not xevo) \* P(not xevo)\\ \tt \\ \tt Since P(not xevo) = 1 - P(xevo) = 1 - 0.48 = 0.52, we have:\\ \tt \\ \tt P(gyzp) = 0.78 \* 0.48 + 0.39 \* 0.52\\ \tt = 0.3744 + 0.2032\\ \tt = 0.5776\\ \tt \\ \tt Since 0.5776 > 0. 
	  } 
	   } 
	   } 
	 & & \\ 
 

    \theutterance \stepcounter{utterance}  

    & & & \multicolumn{2}{p{0.3\linewidth}}{\cellcolor[rgb]{0.95,0.95,0.95}{%
	\makecell[{{p{\linewidth}}}]{% 
	  \tt {\tiny [GM$|$GM]}  
	 Yes 
	  } 
	   } 
	   } 
	 & & \\ 
 

    \theutterance \stepcounter{utterance}  

    & & & \multicolumn{2}{p{0.3\linewidth}}{\cellcolor[rgb]{0.95,0.95,0.95}{%
	\makecell[{{p{\linewidth}}}]{% 
	  \tt {\tiny [GM$|$GM]}  
	 yes 
	  } 
	   } 
	   } 
	 & & \\ 
 

    \theutterance \stepcounter{utterance}  

    & & & \multicolumn{2}{p{0.3\linewidth}}{\cellcolor[rgb]{0.95,0.95,0.95}{%
	\makecell[{{p{\linewidth}}}]{% 
	  \tt {\tiny [GM$|$GM]}  
	 game_result = WIN 
	  } 
	   } 
	   } 
	 & & \\ 
 

\end{supertabular}
}

\end{document}
