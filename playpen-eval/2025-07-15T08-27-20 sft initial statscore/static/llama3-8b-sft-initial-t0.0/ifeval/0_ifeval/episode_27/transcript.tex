
\documentclass{article}
\usepackage{colortbl}
\usepackage{makecell}
\usepackage{multirow}
\usepackage{supertabular}

\begin{document}

\newcounter{utterance}

\twocolumn

{ \footnotesize  \setcounter{utterance}{1}
\setlength{\tabcolsep}{0pt}
\begin{supertabular}{c@{$\;$}|p{.15\linewidth}@{}p{.15\linewidth}p{.15\linewidth}p{.15\linewidth}p{.15\linewidth}p{.15\linewidth}}

    \# & $\;$A & \multicolumn{4}{c}{Game Master} & $\;\:$B\\
    \hline 

    \theutterance \stepcounter{utterance}  

    & & \multicolumn{4}{p{0.6\linewidth}}{\cellcolor[rgb]{0.9,0.9,0.9}{%
	\makecell[{{p{\linewidth}}}]{% 
	  \tt {\tiny [A$\langle$GM]}  
	 Write a casual blog post about how the outer solar system is different from the inner solar system, and what that means for the possibility of life. Wrap your entire response with double quotation marks. Your response should contain 17 or more sentences. 
	  } 
	   } 
	   } 
	 & \\ 
 

    \theutterance \stepcounter{utterance}  

    & \multicolumn{4}{p{0.6\linewidth}}{\cellcolor[rgb]{0.8,1,0.9}{%
	\makecell[{{p{\linewidth}}}]{% 
	  \tt {\tiny [A$\rangle$GM]}  
	 "Exploring the vastness of our solar system, we often find ourselves fascinated by the stark contrasts between the inner and outer regions. The inner solar system, comprising the four rocky planets (Mercury, Venus, Earth, and Mars), is characterized by its proximity to the sun, resulting in a relatively stable and warm environment. In contrast, the outer solar system, consisting of the gas giants (Jupiter, Saturn, Uranus, and Neptune) and their moons, is a realm of icy cold temperatures and immense distances from the sun.\\ \tt \\ \tt The primary difference between the inner and outer solar system lies in their composition and the processes that shape them. The inner planets are primarily composed of rock and metal, with Earth being the only known planet to support life. The outer planets, on the other hand, are gas giants, primarily composed of hydrogen and helium gases. These planets are thought to have formed from the solar nebula, a cloud of gas and dust that surrounded the young sun.\\ \tt \\ \tt The outer solar system is also characterized by its extreme conditions, with temperatures plummeting to -200°C (-330°F) in the outer reaches of the solar system. This makes it a challenging environment for life as we know it. However, the discovery of exoplanets and the study of the outer solar system's moons have revealed potential for life in these distant regions.\\ \tt \\ \tt For instance, the moon of Jupiter, Europa, is believed to have a liquid water ocean beneath its icy surface, making it a prime candidate for supporting life. Similarly, the moon of Saturn, Enceladus, has geysers of water vapor and organic compounds, suggesting a possible habitable environment. These discoveries have sparked interest in the possibility of life existing beyond Earth, particularly in the outer solar system.\\ \tt \\ \tt The study of the outer solar system also provides insights into the formation and evolution of our solar system. The gas giants are thought to have played a crucial role in shaping the orbits of the inner planets, and their moons offer a window into the early solar system's conditions.\\ \tt \\ \tt In conclusion, the outer solar system is a realm of contrasts, with its icy cold temperatures and distant locations making it a challenging environment for life. However, the discovery of potential habitable environments on the moons of the gas giants has opened up new avenues for research and exploration. As we continue to explore the outer solar system, we may uncover more secrets about the origins of life in our solar system and the possibility of life existing elsewhere in the universe." 
	  } 
	   } 
	   } 
	 & & \\ 
 

    \theutterance \stepcounter{utterance}  

    & & & \multicolumn{2}{p{0.3\linewidth}}{\cellcolor[rgb]{0.95,0.95,0.95}{%
	\makecell[{{p{\linewidth}}}]{% 
	  \tt {\tiny [GM$|$GM]}  
	 {'startend:quotation': {}, 'length_constraints:number_sentences': {'relation': 'at least', 'num_sentences': 17}} 
	  } 
	   } 
	   } 
	 & & \\ 
 

    \theutterance \stepcounter{utterance}  

    & & & \multicolumn{2}{p{0.3\linewidth}}{\cellcolor[rgb]{0.95,0.95,0.95}{%
	\makecell[{{p{\linewidth}}}]{% 
	  \tt {\tiny [GM$|$GM]}  
	 game_result = WIN 
	  } 
	   } 
	   } 
	 & & \\ 
 

\end{supertabular}
}

\end{document}
