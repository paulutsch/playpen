
\documentclass{article}
\usepackage{colortbl}
\usepackage{makecell}
\usepackage{multirow}
\usepackage{supertabular}

\begin{document}

\newcounter{utterance}

\twocolumn

{ \footnotesize  \setcounter{utterance}{1}
\setlength{\tabcolsep}{0pt}
\begin{supertabular}{c@{$\;$}|p{.15\linewidth}@{}p{.15\linewidth}p{.15\linewidth}p{.15\linewidth}p{.15\linewidth}p{.15\linewidth}}

    \# & $\;$A & \multicolumn{4}{c}{Game Master} & $\;\:$B\\
    \hline 

    \theutterance \stepcounter{utterance}  

    & & \multicolumn{4}{p{0.6\linewidth}}{\cellcolor[rgb]{0.9,0.9,0.9}{%
	\makecell[{{p{\linewidth}}}]{% 
	  \tt {\tiny [A$\langle$GM]}  
	 Write an interesting and funny article about the biology of a banana peel. In your response, the word disappointed should appear at least 2 times, and at least six section should be highlighted with markdown,  i.e *banana peel*. 
	  } 
	   } 
	   } 
	 & \\ 
 

    \theutterance \stepcounter{utterance}  

    & \multicolumn{4}{p{0.6\linewidth}}{\cellcolor[rgb]{0.8,1,0.9}{%
	\makecell[{{p{\linewidth}}}]{% 
	  \tt {\tiny [A$\rangle$GM]}  
	 **The Fascinating Biology of a Banana Peel: A Fruitful Exploration**\\ \tt \\ \tt As we peel back the layers of a banana, we often overlook the humble banana peel, dismissing it as nothing more than a discarded wrapper. But, dear reader, do not be disappointed! For beneath its unassuming exterior lies a world of fascinating biology, a tale of cellular structure, chemical composition, and the intricate dance of life.\\ \tt \\ \tt **The Anatomy of a Banana Peel**\\ \tt \\ \tt A banana peel is, in fact, the outermost layer of the banana fruit, a thin, edible skin that protects the delicate flesh within. Composed primarily of cellulose, hemicellulose, and pectin, the peel is a complex matrix of fibers and polysaccharides that provide structural support and maintain the fruit's shape. But, as we delve deeper, we find that the peel is more than just a passive barrier – it's a dynamic, living entity that plays a crucial role in the banana's growth and development.\\ \tt \\ \tt **The Chemistry of the Peel**\\ \tt \\ \tt The banana peel is rich in nutrients, including potassium, magnesium, and vitamins C and B6. These compounds are essential for the fruit's growth and ripening process, and they also contribute to the peel's unique texture and flavor. But, as the peel ages, its chemical composition changes, leading to the characteristic yellowing and softening that we associate with overripe bananas. Don't be disappointed – this process is a natural part of the fruit's life cycle!\\ \tt \\ \tt **The Microbiology of the Peel**\\ \tt \\ \tt The banana peel is home to a diverse array of microorganisms, including bacteria, fungi, and yeast. These microbes play a vital role in the fruit's decomposition process, breaking down the peel's complex molecules and releasing nutrients back into the environment. But, as the peel dries out, the microbial community shifts, leading to the formation of a tough, leathery texture that's perfect for composting. Don't be disappointed – this process is a testament to the peel's incredible adaptability!\\ \tt \\ \tt **The Ecology of the Peel**\\ \tt \\ \tt The banana peel is a vital component of the ecosystem, providing food and shelter for a wide range of animals, from insects to birds. The peel's high water content and nutrient-rich composition make it an attractive snack for many species, while its fibrous texture provides a comfortable nesting site for others. But, as the peel dries out, it becomes a valuable resource for gardeners and composters, adding valuable nutrients to the soil. Don't be disappointed – this process is a reminder of the peel's importance in the grand scheme of things!\\ \tt \\ \tt **The Cultural Significance of the Peel**\\ \tt \\ \tt In many cultures, the banana peel is considered a sacred or symbolic object, representing abundance, fertility, and new life. In some traditions, the peel is used in rituals and ceremonies, while in others, it's simply discarded as a waste product. But, as we reflect on the peel's biology and ecology, we're reminded of the importance of respecting and appreciating the natural world. Don't be disappointed – this process is a chance to reconnect with the earth and its many wonders!\\ \tt \\ \tt **Conclusion**\\ \tt \\ \tt The banana peel may seem like a humble, disposable thing, but it's so much more than that. It's a complex, dynamic system that's full of life, chemistry, and ecology. So, the next time you peel a banana, don't be disappointed – take a moment to appreciate the incredible biology that's hidden beneath the surface. Who knows? You might just discover a whole new world of wonder and awe! 
	  } 
	   } 
	   } 
	 & & \\ 
 

    \theutterance \stepcounter{utterance}  

    & & & \multicolumn{2}{p{0.3\linewidth}}{\cellcolor[rgb]{0.95,0.95,0.95}{%
	\makecell[{{p{\linewidth}}}]{% 
	  \tt {\tiny [GM$|$GM]}  
	 {'keywords:frequency': {'relation': 'at least', 'keyword': 'disappointed', 'frequency': 2}, 'detectable_format:number_highlighted_sections': {'num_highlights': 6}} 
	  } 
	   } 
	   } 
	 & & \\ 
 

    \theutterance \stepcounter{utterance}  

    & & & \multicolumn{2}{p{0.3\linewidth}}{\cellcolor[rgb]{0.95,0.95,0.95}{%
	\makecell[{{p{\linewidth}}}]{% 
	  \tt {\tiny [GM$|$GM]}  
	 game_result = WIN 
	  } 
	   } 
	   } 
	 & & \\ 
 

\end{supertabular}
}

\end{document}
