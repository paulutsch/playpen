
\documentclass{article}
\usepackage{colortbl}
\usepackage{makecell}
\usepackage{multirow}
\usepackage{supertabular}

\begin{document}

\newcounter{utterance}

\twocolumn

{ \footnotesize  \setcounter{utterance}{1}
\setlength{\tabcolsep}{0pt}
\begin{supertabular}{c@{$\;$}|p{.15\linewidth}@{}p{.15\linewidth}p{.15\linewidth}p{.15\linewidth}p{.15\linewidth}p{.15\linewidth}}

    \# & $\;$A & \multicolumn{4}{c}{Game Master} & $\;\:$B\\
    \hline 

    \theutterance \stepcounter{utterance}  

    & & \multicolumn{4}{p{0.6\linewidth}}{\cellcolor[rgb]{0.9,0.9,0.9}{%
	\makecell[{{p{\linewidth}}}]{% 
	  \tt {\tiny [A$\langle$GM]}  
	 The Jimenez family has 20 members. Each member has at least one pet. If there are 32 pets in total, what is the maximum number of members that can possibly have 3 pets? Solve this step by step, and wrap your entire response (reasoning process and the final answer) with double quotation marks. 
	  } 
	   } 
	   } 
	 & \\ 
 

    \theutterance \stepcounter{utterance}  

    & \multicolumn{4}{p{0.6\linewidth}}{\cellcolor[rgb]{0.8,1,0.9}{%
	\makecell[{{p{\linewidth}}}]{% 
	  \tt {\tiny [A$\rangle$GM]}  
	 "Let's denote the number of members with 3 pets as \(x\). Each of these members has 3 pets, so the total number of pets from these members is \(3x\). The remaining members have at least 1 pet, and at most 2 pets. Let's denote the number of members with 1 pet as \(y\) and the number of members with 2 pets as \(z\). We have the following constraints:\\ \tt \\ \tt 1. The total number of members is 20, so \(x + y + z = 20\).\\ \tt 2. The total number of pets is 32, so \(3x + y + 2z = 32\).\\ \tt \\ \tt We want to maximize \(x\), the number of members with 3 pets. To do this, we need to minimize \(y\) and \(z\). Since each member has at least one pet, \(y \geq 0\) and \(z \geq 0\).\\ \tt \\ \tt Let's solve the system of equations:\\ \tt \\ \tt \[\\ \tt \begin{align*}\\ \tt x + y + z &= 20 \\\\ \tt 3x + y + 2z &= 32\\ \tt \end{align*}\\ \tt \]\\ \tt \\ \tt Subtracting the first equation from the second, we get:\\ \tt \\ \tt \[\\ \tt 2x + z = 12\\ \tt \]\\ \tt \\ \tt Since \(z \geq 0\), we have:\\ \tt \\ \tt \[\\ \tt 2x \leq 12 \Rightarrow x \leq 6\\ \tt \]\\ \tt \\ \tt Now, let's substitute \(x = 6\) into the first equation:\\ \tt \\ \tt \[\\ \tt 6 + y + z = 20 \Rightarrow y + z = 14\\ \tt \]\\ \tt \\ \tt Since \(y \geq 0\) and \(z \geq 0\), we have:\\ \tt \\ \tt \[\\ \tt y + z = 14 \Rightarrow y = 14 - z\\ \tt \]\\ \tt \\ \tt Substituting \(y = 14 - z\) into the second equation, we get:\\ \tt \\ \tt \[\\ \tt 3(6) + (14 - z) + 2z = 32 \Rightarrow 18 + 14 - z + 2z = 32 \Rightarrow z = 0\\ \tt \]\\ \tt \\ \tt Now, we have \(z = 0\), so \(y = 14\). Substituting \(z = 0\) into the first equation, we get:\\ \tt \\ \tt \[\\ \tt 6 + y + 0 = 20 \Rightarrow y = 14\\ \tt \]\\ \tt \\ \tt So, we have \(x = 6\), \(y = 14\), and \(z = 0\).\\ \tt \\ \tt The maximum number of members that can possibly have 3 pets is \(x = \boxed{6}\)." 
	  } 
	   } 
	   } 
	 & & \\ 
 

    \theutterance \stepcounter{utterance}  

    & & & \multicolumn{2}{p{0.3\linewidth}}{\cellcolor[rgb]{0.95,0.95,0.95}{%
	\makecell[{{p{\linewidth}}}]{% 
	  \tt {\tiny [GM$|$GM]}  
	 {'startend:quotation': {}} 
	  } 
	   } 
	   } 
	 & & \\ 
 

    \theutterance \stepcounter{utterance}  

    & & & \multicolumn{2}{p{0.3\linewidth}}{\cellcolor[rgb]{0.95,0.95,0.95}{%
	\makecell[{{p{\linewidth}}}]{% 
	  \tt {\tiny [GM$|$GM]}  
	 game_result = WIN 
	  } 
	   } 
	   } 
	 & & \\ 
 

\end{supertabular}
}

\end{document}
