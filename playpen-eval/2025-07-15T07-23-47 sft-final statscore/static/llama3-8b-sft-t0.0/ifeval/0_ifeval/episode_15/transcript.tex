
\documentclass{article}
\usepackage{colortbl}
\usepackage{makecell}
\usepackage{multirow}
\usepackage{supertabular}

\begin{document}

\newcounter{utterance}

\twocolumn

{ \footnotesize  \setcounter{utterance}{1}
\setlength{\tabcolsep}{0pt}
\begin{supertabular}{c@{$\;$}|p{.15\linewidth}@{}p{.15\linewidth}p{.15\linewidth}p{.15\linewidth}p{.15\linewidth}p{.15\linewidth}}

    \# & $\;$A & \multicolumn{4}{c}{Game Master} & $\;\:$B\\
    \hline 

    \theutterance \stepcounter{utterance}  

    & & \multicolumn{4}{p{0.6\linewidth}}{\cellcolor[rgb]{0.9,0.9,0.9}{%
	\makecell[{{p{\linewidth}}}]{% 
	  \tt {\tiny [A$\langle$GM]}  
	 Write an interesting and funny article about the biology of a banana peel. In your response, the word disappointed should appear at least 2 times, and at least six section should be highlighted with markdown,  i.e *banana peel*. 
	  } 
	   } 
	   } 
	 & \\ 
 

    \theutterance \stepcounter{utterance}  

    & \multicolumn{4}{p{0.6\linewidth}}{\cellcolor[rgb]{0.8,1,0.9}{%
	\makecell[{{p{\linewidth}}}]{% 
	  \tt {\tiny [A$\rangle$GM]}  
	 **The Fascinating Biology of a Banana Peel: A Fruitful Exploration**\\ \tt \\ \tt As we peel back the layers of a banana, we often overlook the fascinating biology that lies beneath. The banana peel, a seemingly mundane part of the fruit, holds secrets that are both surprising and amusing. In this article, we'll delve into the world of banana peels and uncover the reasons why they're more than just a discarded wrapper.\\ \tt \\ \tt **The Structure of a Banana Peel**\\ \tt \\ \tt A banana peel is made up of several layers, each with its own unique characteristics. The outermost layer is the epidermis, a thin, waxy coating that protects the fruit from the environment. Beneath this lies the mesocarp, a thick, fibrous layer that provides support and structure to the fruit. The endocarp, the innermost layer, is a thin, membranous tissue that surrounds the fruit's pulp.\\ \tt \\ \tt **The Life Cycle of a Banana Peel**\\ \tt \\ \tt A banana peel's life cycle is a short but eventful one. From the moment it's formed, the peel begins to undergo a series of changes that ultimately lead to its demise. As the fruit ripens, the peel's cells begin to break down, releasing enzymes that soften the fruit's texture. This process, known as senescence, is a natural part of the fruit's life cycle.\\ \tt \\ \tt **The Disappointed Consumer**\\ \tt \\ \tt For many of us, the banana peel is a source of disappointment. We peel back the layers, expecting a juicy, sweet treat, only to find a mushy, brown mess. But what if we told you that the banana peel is more than just a discarded wrapper? In fact, it's a treasure trove of nutrients and antioxidants.\\ \tt \\ \tt **The Nutritional Value of Banana Peels**\\ \tt \\ \tt Banana peels are rich in vitamins, minerals, and antioxidants. They contain high levels of potassium, magnesium, and iron, making them a valuable addition to any diet. In fact, some studies have shown that banana peels may even have anti-inflammatory properties, making them a potential treatment for a range of health conditions.\\ \tt \\ \tt **The Disappointed Environmentalist**\\ \tt \\ \tt For those of us who care about the environment, the banana peel can be a source of disappointment. We toss them in the trash, unaware of the impact they have on our planet. But what if we told you that banana peels can be composted, reducing waste and creating a nutrient-rich soil amendment?\\ \tt \\ \tt **The Future of Banana Peels**\\ \tt \\ \tt As we continue to explore the biology of banana peels, we may uncover new uses for this often-overlooked part of the fruit. From biodegradable packaging to natural dyes, the possibilities are endless. So next time you peel back the layers of a banana, remember that there's more to the peel than meets the eye.\\ \tt \\ \tt **Conclusion**\\ \tt \\ \tt The banana peel may seem like a mundane part of the fruit, but it holds secrets that are both surprising and amusing. From its structure to its life cycle, the banana peel is a fascinating subject that's waiting to be explored. So next time you're disappointed by a mushy banana, remember that there's more to the peel than meets the eye. 
	  } 
	   } 
	   } 
	 & & \\ 
 

    \theutterance \stepcounter{utterance}  

    & & & \multicolumn{2}{p{0.3\linewidth}}{\cellcolor[rgb]{0.95,0.95,0.95}{%
	\makecell[{{p{\linewidth}}}]{% 
	  \tt {\tiny [GM$|$GM]}  
	 {'keywords:frequency': {'relation': 'at least', 'keyword': 'disappointed', 'frequency': 2}, 'detectable_format:number_highlighted_sections': {'num_highlights': 6}} 
	  } 
	   } 
	   } 
	 & & \\ 
 

    \theutterance \stepcounter{utterance}  

    & & & \multicolumn{2}{p{0.3\linewidth}}{\cellcolor[rgb]{0.95,0.95,0.95}{%
	\makecell[{{p{\linewidth}}}]{% 
	  \tt {\tiny [GM$|$GM]}  
	 game_result = WIN 
	  } 
	   } 
	   } 
	 & & \\ 
 

\end{supertabular}
}

\end{document}
