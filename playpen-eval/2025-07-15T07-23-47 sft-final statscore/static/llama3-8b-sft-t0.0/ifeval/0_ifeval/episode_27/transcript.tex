
\documentclass{article}
\usepackage{colortbl}
\usepackage{makecell}
\usepackage{multirow}
\usepackage{supertabular}

\begin{document}

\newcounter{utterance}

\twocolumn

{ \footnotesize  \setcounter{utterance}{1}
\setlength{\tabcolsep}{0pt}
\begin{supertabular}{c@{$\;$}|p{.15\linewidth}@{}p{.15\linewidth}p{.15\linewidth}p{.15\linewidth}p{.15\linewidth}p{.15\linewidth}}

    \# & $\;$A & \multicolumn{4}{c}{Game Master} & $\;\:$B\\
    \hline 

    \theutterance \stepcounter{utterance}  

    & & \multicolumn{4}{p{0.6\linewidth}}{\cellcolor[rgb]{0.9,0.9,0.9}{%
	\makecell[{{p{\linewidth}}}]{% 
	  \tt {\tiny [A$\langle$GM]}  
	 Write a casual blog post about how the outer solar system is different from the inner solar system, and what that means for the possibility of life. Wrap your entire response with double quotation marks. Your response should contain 17 or more sentences. 
	  } 
	   } 
	   } 
	 & \\ 
 

    \theutterance \stepcounter{utterance}  

    & \multicolumn{4}{p{0.6\linewidth}}{\cellcolor[rgb]{0.8,1,0.9}{%
	\makecell[{{p{\linewidth}}}]{% 
	  \tt {\tiny [A$\rangle$GM]}  
	 "Exploring the vastness of our solar system, we often find ourselves fascinated by the stark contrasts between its inner and outer regions. The inner solar system, comprising the four rocky planets - Mercury, Venus, Earth, and Mars - is a realm of intense heat, volcanic activity, and a relatively stable environment. In stark contrast, the outer solar system, home to the gas giants Jupiter, Saturn, Uranus, and Neptune, is a realm of icy cold, gaseous giants, and a dynamic, turbulent environment.\\ \tt \\ \tt The inner solar system is characterized by its proximity to the sun, which results in a significant amount of solar energy being absorbed by the planets. This energy is responsible for the high temperatures and volcanic activity observed on these planets. The rocky planets are also subject to a relatively stable gravitational pull, which has allowed them to maintain a stable climate over billions of years.\\ \tt \\ \tt On the other hand, the outer solar system is a realm of extreme cold, with temperatures plummeting to as low as -200 degrees Celsius in the outer reaches of the solar system. The gas giants are massive, gaseous worlds that are primarily composed of hydrogen and helium. Their immense size and mass result in a strong gravitational pull, which has led to the formation of extensive ring systems and numerous moons.\\ \tt \\ \tt The outer solar system is also characterized by its dynamic, turbulent environment. The gas giants are constantly interacting with their surroundings, with Jupiter's strong magnetic field and Saturn's ring system being prime examples. These interactions have resulted in the formation of complex systems, such as Jupiter's Great Red Spot, a persistent anticyclonic storm that has been raging for centuries.\\ \tt \\ \tt The possibility of life in the outer solar system is a topic of ongoing research and debate. While the conditions on the gas giants are inhospitable to life as we know it, there are some intriguing possibilities. For example, the moons of Jupiter and Saturn have been found to have subsurface oceans, which could potentially support life. These moons, such as Europa and Enceladus, have been found to have conditions similar to those of Earth, with liquid water and a stable energy source.\\ \tt \\ \tt In conclusion, the outer solar system is a realm of extreme cold, gaseous giants, and a dynamic, turbulent environment. While the conditions on the gas giants are inhospitable to life as we know it, there are some intriguing possibilities for life on the moons of Jupiter and Saturn. As we continue to explore the outer solar system, we may uncover new and exciting discoveries that challenge our understanding of the possibility of life beyond Earth." 
	  } 
	   } 
	   } 
	 & & \\ 
 

    \theutterance \stepcounter{utterance}  

    & & & \multicolumn{2}{p{0.3\linewidth}}{\cellcolor[rgb]{0.95,0.95,0.95}{%
	\makecell[{{p{\linewidth}}}]{% 
	  \tt {\tiny [GM$|$GM]}  
	 {'startend:quotation': {}, 'length_constraints:number_sentences': {'relation': 'at least', 'num_sentences': 17}} 
	  } 
	   } 
	   } 
	 & & \\ 
 

    \theutterance \stepcounter{utterance}  

    & & & \multicolumn{2}{p{0.3\linewidth}}{\cellcolor[rgb]{0.95,0.95,0.95}{%
	\makecell[{{p{\linewidth}}}]{% 
	  \tt {\tiny [GM$|$GM]}  
	 game_result = WIN 
	  } 
	   } 
	   } 
	 & & \\ 
 

\end{supertabular}
}

\end{document}
