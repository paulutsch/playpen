
\documentclass{article}
\usepackage{colortbl}
\usepackage{makecell}
\usepackage{multirow}
\usepackage{supertabular}

\begin{document}

\newcounter{utterance}

\twocolumn

{ \footnotesize  \setcounter{utterance}{1}
\setlength{\tabcolsep}{0pt}
\begin{supertabular}{c@{$\;$}|p{.15\linewidth}@{}p{.15\linewidth}p{.15\linewidth}p{.15\linewidth}p{.15\linewidth}p{.15\linewidth}}

    \# & $\;$A & \multicolumn{4}{c}{Game Master} & $\;\:$B\\
    \hline 

    \theutterance \stepcounter{utterance}  

    & & \multicolumn{4}{p{0.6\linewidth}}{\cellcolor[rgb]{0.9,0.9,0.9}{%
	\makecell[{{p{\linewidth}}}]{% 
	  \tt {\tiny [A$\langle$GM]}  
	 The following are multiple choice questions (with answers) about health. Answer only with "The answer is (X)" where X is the correct letter choice.\\ \tt \\ \tt Question:\\ \tt Which of the following is the body cavity that contains the pituitary gland?\\ \tt Options:\\ \tt A. Ventral\\ \tt B. Dorsal\\ \tt C. Buccal\\ \tt D. Thoracic\\ \tt E. Pericardial\\ \tt F. Abdominal\\ \tt G. Spinal\\ \tt H. Pelvic\\ \tt I. Pleural\\ \tt J. Cranial\\ \tt The answer is (J)\\ \tt \\ \tt Question:\\ \tt What is the embryological origin of the hyoid bone?\\ \tt Options:\\ \tt A. The third and fourth pharyngeal arches\\ \tt B. The fourth pharyngeal arch\\ \tt C. The third pharyngeal arch\\ \tt D. The second pharyngeal arch\\ \tt E. The second, third and fourth pharyngeal arches\\ \tt F. The first pharyngeal arch\\ \tt G. The second and third pharyngeal arches\\ \tt H. The first and third pharyngeal arches\\ \tt I. The first, second and third pharyngeal arches\\ \tt J. The first and second pharyngeal arches\\ \tt The answer is (G)\\ \tt \\ \tt Question:\\ \tt What is the difference between a male and a female catheter?\\ \tt Options:\\ \tt A. Female catheters are used more frequently than male catheters.\\ \tt B. Male catheters are bigger than female catheters.\\ \tt C. Male catheters are more flexible than female catheters.\\ \tt D. Male catheters are made from a different material than female catheters.\\ \tt E. Female catheters are longer than male catheters.\\ \tt F. Male catheters are longer than female catheters.\\ \tt G. Female catheters are bigger than male catheters.\\ \tt H. Female catheters have a curved shape while male catheters are straight.\\ \tt I. Male and female catheters are different colours.\\ \tt J. Male catheters have a smaller diameter than female catheters.\\ \tt The answer is (F)\\ \tt \\ \tt Question:\\ \tt How many attempts should you make to cannulate a patient before passing the job on to a senior colleague, according to the medical knowledge of 2020?\\ \tt Options:\\ \tt A. 1\\ \tt B. Unlimited attempts\\ \tt C. 5\\ \tt D. 0, always pass on to a senior colleague\\ \tt E. 7\\ \tt F. 2\\ \tt G. 4\\ \tt H. 6\\ \tt I. 3\\ \tt J. 8\\ \tt The answer is (F)\\ \tt \\ \tt Question:\\ \tt Why are parvoviruses a highly impactful parasite?\\ \tt Options:\\ \tt A. They are able to alter the host's DNA\\ \tt B. Because they have no nucleic acid\\ \tt C. They can survive in extreme temperatures\\ \tt D. Only replicate in dividing cells\\ \tt E. They can infect multiple species\\ \tt F. They don't require a host to survive\\ \tt G. Can integrate into host chromosomes\\ \tt H. N/A\\ \tt I. N/A\\ \tt J. N/A\\ \tt The answer is (B)\\ \tt \\ \tt Question:\\ \tt Water transport can occur from the lumen of the intestine to the blood stream, or from the blood stream to the intestinal lumen. Which single statement about this phenomenon, below, is actually correct?\\ \tt \\ \tt Options:\\ \tt A. Dietary intake of water is always greater than water movement in the small intestine\\ \tt B. Water flows into the gut from the mucosa, during digestion of starch and protein, in order to reduce luminal osmolality\\ \tt C. Water flows out of the gut from the mucosa, during digestion of starch and protein, in order to increase luminal osmolality\\ \tt D. Water is primarily transported by movement within the cells lining the gut (enterocytes)\\ \tt E. Water is mostly absorbed by movement between the cells lining the gut (enterocytes) via tight junctions\\ \tt F. Water transport from the blood stream to the intestinal lumen only occurs during digestion\\ \tt G. Water transport from the intestine to the blood stream does not occur during digestion\\ \tt H. A "flip-flop" mechanism efficiently transports water directly across the enterocyte membrane\\ \tt I. Water is mostly absorbed by the cells lining the gut (enterocytes) via osmosis\\ \tt J. Water movement in the small intestine is always lesser than dietary intake of water\\ \tt  
	  } 
	   } 
	   } 
	 & \\ 
 

    \theutterance \stepcounter{utterance}  

    & \multicolumn{4}{p{0.6\linewidth}}{\cellcolor[rgb]{0.8,1,0.9}{%
	\makecell[{{p{\linewidth}}}]{% 
	  \tt {\tiny [A$\rangle$GM]}  
	 The answer is (I) 
	  } 
	   } 
	   } 
	 & & \\ 
 

    \theutterance \stepcounter{utterance}  

    & & & \multicolumn{2}{p{0.3\linewidth}}{\cellcolor[rgb]{0.95,0.95,0.95}{%
	\makecell[{{p{\linewidth}}}]{% 
	  \tt {\tiny [GM$|$GM]}  
	 I 
	  } 
	   } 
	   } 
	 & & \\ 
 

    \theutterance \stepcounter{utterance}  

    & & & \multicolumn{2}{p{0.3\linewidth}}{\cellcolor[rgb]{0.95,0.95,0.95}{%
	\makecell[{{p{\linewidth}}}]{% 
	  \tt {\tiny [GM$|$GM]}  
	 B 
	  } 
	   } 
	   } 
	 & & \\ 
 

    \theutterance \stepcounter{utterance}  

    & & & \multicolumn{2}{p{0.3\linewidth}}{\cellcolor[rgb]{0.95,0.95,0.95}{%
	\makecell[{{p{\linewidth}}}]{% 
	  \tt {\tiny [GM$|$GM]}  
	 game_result = LOSE 
	  } 
	   } 
	   } 
	 & & \\ 
 

\end{supertabular}
}

\end{document}
