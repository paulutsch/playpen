
\documentclass{article}
\usepackage{colortbl}
\usepackage{makecell}
\usepackage{multirow}
\usepackage{supertabular}

\begin{document}

\newcounter{utterance}

\twocolumn

{ \footnotesize  \setcounter{utterance}{1}
\setlength{\tabcolsep}{0pt}
\begin{supertabular}{c@{$\;$}|p{.15\linewidth}@{}p{.15\linewidth}p{.15\linewidth}p{.15\linewidth}p{.15\linewidth}p{.15\linewidth}}

    \# & $\;$A & \multicolumn{4}{c}{Game Master} & $\;\:$B\\
    \hline 

    \theutterance \stepcounter{utterance}  

    & & \multicolumn{4}{p{0.6\linewidth}}{\cellcolor[rgb]{0.9,0.9,0.9}{%
	\makecell[{{p{\linewidth}}}]{% 
	  \tt {\tiny [A$\langle$GM]}  
	 \\ \tt Your task is to predict the likely emotional responses of a character in this dialogue:\\ \tt \\ \tt Rev. Montague: You see, Iris, faith is not a choice, but a divine gift. It's a beacon of light in a world often shrouded in darkness.\\ \tt Dr. LeGuin: A beacon of light, Luther? Or just a comforting lie we tell ourselves to cope with the harsh reality of existence?\\ \tt Rev. Montague: It's more than that. It's a guiding principle, a moral compass. It's what makes us human, Iris.\\ \tt Dr. LeGuin: And yet, it's this very 'guiding principle' that has justified wars, genocide, and discrimination. How human is that, Luther?\\ \tt Rev. Montague: Those are the failures of men, not faith. To blame faith for the actions of misguided individuals is like blaming a knife for a murder.\\ \tt Dr. LeGuin: A knife doesn't claim to be a moral compass, Luther. And it doesn't demand blind obedience.\\ \tt Rev. Montague: Faith doesn't demand blindness, Iris. It demands trust, surrender. It's a relationship, not a dictatorship.\\ \tt Dr. LeGuin: A relationship with a non-existent entity. How convenient. \\ \tt Rev. Montague: Just because you can't see something, doesn't mean it doesn't exist. Can you see your thoughts, Iris?\\ \tt Dr. LeGuin: That's a false equivalence, Luther. Thoughts can be measured, analyzed. Your God, on the other hand...\\ \tt Rev. Montague: Is beyond human comprehension. That's the point, Iris. We're not meant to understand everything.\\ \tt Dr. LeGuin: And there lies the crux of our disagreement, Luther. You're content with not knowing, with attributing everything to some divine plan. I, on the other hand, am not. \\ \tt Rev. Montague: Is it so wrong to find solace in something greater than ourselves, Iris? Is it so wrong to believe in hope, in love, in redemption?\\ \tt Dr. LeGuin: It's not wrong, Luther. It's just... futile. \\ \tt Rev. Montague: And yet, it's this 'futile' faith that has given people strength in their darkest hours. Can your logic do that, Iris?\\ \tt Dr. LeGuin: My logic doesn't offer false hope, Luther. It offers clarity, understanding. It offers truth.\\ \tt Rev. Montague: And what is truth, Iris? Your truth? My truth? Or some universal truth that we're all blindly searching for?\\ \tt Dr. LeGuin: There's no 'your truth' or 'my truth', Luther. There's just the truth. And it's not found in ancient texts or divine revelations. It's found in evidence, in reason.\\ \tt [End dialogue]\\ \tt \\ \tt At the end of this dialogue, At this point, Rev. Montague would feel...\\ \tt Dismissed\\ \tt Intrigued\\ \tt Resentful\\ \tt Relieved\\ \tt \\ \tt Give each of these possible emotions a score from 0-10 for the relative intensity that they are likely to be feeling each.\\ \tt \\ \tt You must output in the following format, including headings (of course, you should give your own scores), with no additional commentary:\\ \tt \\ \tt Dismissed: <score>\\ \tt Intrigued: <score>\\ \tt Resentful: <score>\\ \tt Relieved: <score>\\ \tt \\ \tt \\ \tt [End of answer]\\ \tt \\ \tt Remember: zero is a valid score, meaning they are likely not feeling that emotion. You must score at least one emotion > 0.\\ \tt \\ \tt Your answer:\\ \tt  
	  } 
	   } 
	   } 
	 & \\ 
 

    \theutterance \stepcounter{utterance}  

    & \multicolumn{4}{p{0.6\linewidth}}{\cellcolor[rgb]{0.8,1,0.9}{%
	\makecell[{{p{\linewidth}}}]{% 
	  \tt {\tiny [A$\rangle$GM]}  
	 Dismissed: 8\\ \tt Intrigued: 2\\ \tt Resentful: 6\\ \tt Relieved: 0 
	  } 
	   } 
	   } 
	 & & \\ 
 

    \theutterance \stepcounter{utterance}  

    & & & \multicolumn{2}{p{0.3\linewidth}}{\cellcolor[rgb]{0.95,0.95,0.95}{%
	\makecell[{{p{\linewidth}}}]{% 
	  \tt {\tiny [GM$|$GM]}  
	 Dismissed: 8\\ \tt Intrigued: 2\\ \tt Resentful: 6\\ \tt Relieved: 0 
	  } 
	   } 
	   } 
	 & & \\ 
 

    \theutterance \stepcounter{utterance}  

    & & & \multicolumn{2}{p{0.3\linewidth}}{\cellcolor[rgb]{0.95,0.95,0.95}{%
	\makecell[{{p{\linewidth}}}]{% 
	  \tt {\tiny [GM$|$GM]}  
	 Dismissed: 7\\ \tt Intrigued: 0\\ \tt Resentful: 5\\ \tt Relieved: 0 
	  } 
	   } 
	   } 
	 & & \\ 
 

    \theutterance \stepcounter{utterance}  

    & & & \multicolumn{2}{p{0.3\linewidth}}{\cellcolor[rgb]{0.95,0.95,0.95}{%
	\makecell[{{p{\linewidth}}}]{% 
	  \tt {\tiny [GM$|$GM]}  
	 game_result = LOSE 
	  } 
	   } 
	   } 
	 & & \\ 
 

\end{supertabular}
}

\end{document}
