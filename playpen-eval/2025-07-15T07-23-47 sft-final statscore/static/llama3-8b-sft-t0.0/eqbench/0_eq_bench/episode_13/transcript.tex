
\documentclass{article}
\usepackage{colortbl}
\usepackage{makecell}
\usepackage{multirow}
\usepackage{supertabular}

\begin{document}

\newcounter{utterance}

\twocolumn

{ \footnotesize  \setcounter{utterance}{1}
\setlength{\tabcolsep}{0pt}
\begin{supertabular}{c@{$\;$}|p{.15\linewidth}@{}p{.15\linewidth}p{.15\linewidth}p{.15\linewidth}p{.15\linewidth}p{.15\linewidth}}

    \# & $\;$A & \multicolumn{4}{c}{Game Master} & $\;\:$B\\
    \hline 

    \theutterance \stepcounter{utterance}  

    & & \multicolumn{4}{p{0.6\linewidth}}{\cellcolor[rgb]{0.9,0.9,0.9}{%
	\makecell[{{p{\linewidth}}}]{% 
	  \tt {\tiny [A$\langle$GM]}  
	 \\ \tt Your task is to predict the likely emotional responses of a character in this dialogue:\\ \tt \\ \tt Sean: Emma, I need to talk to you about something.\\ \tt Emma: Of course, Sean. What's on your mind?\\ \tt Sean: It's...it's about us. I mean, you and me.\\ \tt Emma: Us? What do you mean?\\ \tt Sean: I mean, I love you, Emma. I'm in love with you.\\ \tt Emma: Sean...I...\\ \tt Sean: Don't say anything yet. Just hear me out. I've been feeling this way for a while now. I can't ignore it anymore.\\ \tt Emma: Sean, I'm married. You know that.\\ \tt Sean: I do know that, Emma. But I can't help how I feel.\\ \tt Emma: This is...this is a lot to take in.\\ \tt Sean: I didn't mean to overwhelm you. I just...I needed to tell you.\\ \tt Emma: Sean, I care about you, but not in the way you want me to.\\ \tt [End dialogue]\\ \tt \\ \tt At the end of this dialogue, Sean would feel...\\ \tt Relief\\ \tt Rejection\\ \tt Indignation\\ \tt Frustration\\ \tt \\ \tt Give each of these possible emotions a score from 0-10 for the relative intensity that they are likely to be feeling each.\\ \tt \\ \tt You must output in the following format, including headings (of course, you should give your own scores), with no additional commentary:\\ \tt \\ \tt Relief: <score>\\ \tt Rejection: <score>\\ \tt Indignation: <score>\\ \tt Frustration: <score>\\ \tt \\ \tt \\ \tt [End of answer]\\ \tt \\ \tt Remember: zero is a valid score, meaning they are likely not feeling that emotion. You must score at least one emotion > 0.\\ \tt \\ \tt Your answer:\\ \tt  
	  } 
	   } 
	   } 
	 & \\ 
 

    \theutterance \stepcounter{utterance}  

    & \multicolumn{4}{p{0.6\linewidth}}{\cellcolor[rgb]{0.8,1,0.9}{%
	\makecell[{{p{\linewidth}}}]{% 
	  \tt {\tiny [A$\rangle$GM]}  
	 Relief: 0\\ \tt Rejection: 10\\ \tt Indignation: 0\\ \tt Frustration: 10 
	  } 
	   } 
	   } 
	 & & \\ 
 

    \theutterance \stepcounter{utterance}  

    & & & \multicolumn{2}{p{0.3\linewidth}}{\cellcolor[rgb]{0.95,0.95,0.95}{%
	\makecell[{{p{\linewidth}}}]{% 
	  \tt {\tiny [GM$|$GM]}  
	 Relief: 0\\ \tt Rejection: 10\\ \tt Indignation: 0\\ \tt Frustration: 10 
	  } 
	   } 
	   } 
	 & & \\ 
 

    \theutterance \stepcounter{utterance}  

    & & & \multicolumn{2}{p{0.3\linewidth}}{\cellcolor[rgb]{0.95,0.95,0.95}{%
	\makecell[{{p{\linewidth}}}]{% 
	  \tt {\tiny [GM$|$GM]}  
	 Relief: 0\\ \tt Rejection: 8\\ \tt Indignation: 1\\ \tt Frustration: 6 
	  } 
	   } 
	   } 
	 & & \\ 
 

    \theutterance \stepcounter{utterance}  

    & & & \multicolumn{2}{p{0.3\linewidth}}{\cellcolor[rgb]{0.95,0.95,0.95}{%
	\makecell[{{p{\linewidth}}}]{% 
	  \tt {\tiny [GM$|$GM]}  
	 game_result = LOSE 
	  } 
	   } 
	   } 
	 & & \\ 
 

\end{supertabular}
}

\end{document}
