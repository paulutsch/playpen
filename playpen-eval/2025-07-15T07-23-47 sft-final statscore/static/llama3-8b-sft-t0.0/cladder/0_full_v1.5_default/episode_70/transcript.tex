
\documentclass{article}
\usepackage{colortbl}
\usepackage{makecell}
\usepackage{multirow}
\usepackage{supertabular}

\begin{document}

\newcounter{utterance}

\twocolumn

{ \footnotesize  \setcounter{utterance}{1}
\setlength{\tabcolsep}{0pt}
\begin{supertabular}{c@{$\;$}|p{.15\linewidth}@{}p{.15\linewidth}p{.15\linewidth}p{.15\linewidth}p{.15\linewidth}p{.15\linewidth}}

    \# & $\;$A & \multicolumn{4}{c}{Game Master} & $\;\:$B\\
    \hline 

    \theutterance \stepcounter{utterance}  

    & & \multicolumn{4}{p{0.6\linewidth}}{\cellcolor[rgb]{0.9,0.9,0.9}{%
	\makecell[{{p{\linewidth}}}]{% 
	  \tt {\tiny [A$\langle$GM]}  
	 You are an expert in causal inference. The following question is not a typical commonsense query, but rather a meticulously designed question created by a professor specializing in causal inference, intended to assess the students' mastery of the course content.\\ \tt Imagine a self-contained, hypothetical world with only the following conditions, and without any unmentioned factors or causal relationships: Xevo has a direct effect on tijw and tijv. Tijv has a direct effect on gyzp. Tijw has a direct effect on gyzp. The overall probability of xevo is 48%. For those who are not xevo, the probability of gyzp is 39%. For those who are xevo, the probability of gyzp is 78%. Is gyzp more likely than not gyzp overall?\\ \tt Start your answer with 'Yes' or 'No', followed by additional reasoning or evidence to support your explanation. 
	  } 
	   } 
	   } 
	 & \\ 
 

    \theutterance \stepcounter{utterance}  

    & \multicolumn{4}{p{0.6\linewidth}}{\cellcolor[rgb]{0.8,1,0.9}{%
	\makecell[{{p{\linewidth}}}]{% 
	  \tt {\tiny [A$\rangle$GM]}  
	 To solve this problem, we need to determine the overall probability of gyzp in the hypothetical world. We are given the following information:\\ \tt \\ \tt 1. The probability of Xevo (xevo) is 48%.\\ \tt 2. For those who are not Xevo (not xevo), the probability of gyzp is 39%.\\ \tt 3. For those who are Xevo (xevo), the probability of gyzp is 78%.\\ \tt \\ \tt We need to calculate the overall probability of gyzp, which can be done by considering the probabilities of gyzp for both Xevo and not Xevo cases and then combining them based on the probability of Xevo.\\ \tt \\ \tt Let's denote the probability of gyzp as P(gyzp). We have two scenarios:\\ \tt \\ \tt - **Scenario 1:** Xevo (xevo)\\ \tt   - Probability of Xevo: 48% or 0.48\\ \tt   - Probability of gyzp given Xevo: 78% or 0.78\\ \tt \\ \tt - **Scenario 2:** Not Xevo (not xevo)\\ \tt   - Probability of not Xevo: 52% or 0.52 (since the probability of Xevo is 48%)\\ \tt   - Probability of gyzp given not Xevo: 39% or 0.39\\ \tt \\ \tt The overall probability of gyzp (P(gyzp)) can be calculated using the law of total probability: 
	  } 
	   } 
	   } 
	 & & \\ 
 

    \theutterance \stepcounter{utterance}  

    & & & \multicolumn{2}{p{0.3\linewidth}}{\cellcolor[rgb]{0.95,0.95,0.95}{%
	\makecell[{{p{\linewidth}}}]{% 
	  \tt {\tiny [GM$|$GM]}  
	 ParseError: The response doesn't start with one of the required word ['yes', 'no'], but To 
	  } 
	   } 
	   } 
	 & & \\ 
 

    \theutterance \stepcounter{utterance}  

    & & & \multicolumn{2}{p{0.3\linewidth}}{\cellcolor[rgb]{0.95,0.95,0.95}{%
	\makecell[{{p{\linewidth}}}]{% 
	  \tt {\tiny [GM$|$GM]}  
	 game_result = ABORT 
	  } 
	   } 
	   } 
	 & & \\ 
 

\end{supertabular}
}

\end{document}
