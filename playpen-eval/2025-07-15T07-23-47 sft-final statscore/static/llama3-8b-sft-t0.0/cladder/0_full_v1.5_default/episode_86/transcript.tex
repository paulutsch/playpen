
\documentclass{article}
\usepackage{colortbl}
\usepackage{makecell}
\usepackage{multirow}
\usepackage{supertabular}

\begin{document}

\newcounter{utterance}

\twocolumn

{ \footnotesize  \setcounter{utterance}{1}
\setlength{\tabcolsep}{0pt}
\begin{supertabular}{c@{$\;$}|p{.15\linewidth}@{}p{.15\linewidth}p{.15\linewidth}p{.15\linewidth}p{.15\linewidth}p{.15\linewidth}}

    \# & $\;$A & \multicolumn{4}{c}{Game Master} & $\;\:$B\\
    \hline 

    \theutterance \stepcounter{utterance}  

    & & \multicolumn{4}{p{0.6\linewidth}}{\cellcolor[rgb]{0.9,0.9,0.9}{%
	\makecell[{{p{\linewidth}}}]{% 
	  \tt {\tiny [A$\langle$GM]}  
	 You are an expert in causal inference. The following question is not a typical commonsense query, but rather a meticulously designed question created by a professor specializing in causal inference, intended to assess the students' mastery of the course content.\\ \tt Imagine a self-contained, hypothetical world with only the following conditions, and without any unmentioned factors or causal relationships: Jyka has a direct effect on hwax. Hwax has a direct effect on lirg. The overall probability of jyka is 14%. The probability of not jyka and lirg is 72%. The probability of jyka and lirg is 12%. Is the chance of lirg smaller when observing jyka?\\ \tt Start your answer with 'Yes' or 'No', followed by additional reasoning or evidence to support your explanation. 
	  } 
	   } 
	   } 
	 & \\ 
 

    \theutterance \stepcounter{utterance}  

    & \multicolumn{4}{p{0.6\linewidth}}{\cellcolor[rgb]{0.8,1,0.9}{%
	\makecell[{{p{\linewidth}}}]{% 
	  \tt {\tiny [A$\rangle$GM]}  
	 To solve this problem, we need to determine the probability of lirg given jyka and compare it to the overall probability of lirg. We are given the following information:\\ \tt \\ \tt 1. The probability of jyka (P(jyka)) is 14%.\\ \tt 2. The probability of not jyka and lirg (P(not jyka and lirg)) is 72%.\\ \tt 3. The probability of jyka and lirg (P(jyka and lirg)) is 12%.\\ \tt \\ \tt We need to find the probability of lirg given jyka, denoted as P(lirg|jyka).\\ \tt \\ \tt Using the law of total probability, we can express the probability of lirg as follows:\\ \tt \\ \tt P(lirg) = P(jyka and lirg) + P(not jyka and lirg)\\ \tt \\ \tt We are given P(jyka and lirg) = 12% and P(not jyka and lirg) = 72%. Therefore:\\ \tt \\ \tt P(lirg) = 12% + 72% = 84%\\ \tt \\ \tt Now, we need to find P(lirg|jyka). We can use Bayes' theorem:\\ \tt \\ \tt P(lirg|jyka) = P(jyka and lirg) / P(jyka)\\ \tt \\ \tt We are given P(jyka) = 14% and P(jyka and lir 
	  } 
	   } 
	   } 
	 & & \\ 
 

    \theutterance \stepcounter{utterance}  

    & & & \multicolumn{2}{p{0.3\linewidth}}{\cellcolor[rgb]{0.95,0.95,0.95}{%
	\makecell[{{p{\linewidth}}}]{% 
	  \tt {\tiny [GM$|$GM]}  
	 ParseError: The response doesn't start with one of the required word ['yes', 'no'], but To 
	  } 
	   } 
	   } 
	 & & \\ 
 

    \theutterance \stepcounter{utterance}  

    & & & \multicolumn{2}{p{0.3\linewidth}}{\cellcolor[rgb]{0.95,0.95,0.95}{%
	\makecell[{{p{\linewidth}}}]{% 
	  \tt {\tiny [GM$|$GM]}  
	 game_result = ABORT 
	  } 
	   } 
	   } 
	 & & \\ 
 

\end{supertabular}
}

\end{document}
