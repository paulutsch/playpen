
\documentclass{article}
\usepackage{colortbl}
\usepackage{makecell}
\usepackage{multirow}
\usepackage{supertabular}

\begin{document}

\newcounter{utterance}

\twocolumn

{ \footnotesize  \setcounter{utterance}{1}
\setlength{\tabcolsep}{0pt}
\begin{supertabular}{c@{$\;$}|p{.15\linewidth}@{}p{.15\linewidth}p{.15\linewidth}p{.15\linewidth}p{.15\linewidth}p{.15\linewidth}}

    \# & $\;$A & \multicolumn{4}{c}{Game Master} & $\;\:$B\\
    \hline 

    \theutterance \stepcounter{utterance}  

    & & \multicolumn{4}{p{0.6\linewidth}}{\cellcolor[rgb]{0.9,0.9,0.9}{%
	\makecell[{{p{\linewidth}}}]{% 
	  \tt {\tiny [A$\langle$GM]}  
	 You are an expert in causal inference. The following question is not a typical commonsense query, but rather a meticulously designed question created by a professor specializing in causal inference, intended to assess the students' mastery of the course content.\\ \tt Imagine a self-contained, hypothetical world with only the following conditions, and without any unmentioned factors or causal relationships: Gender has a direct effect on department competitiveness and admission status. Department competitiveness has a direct effect on admission status. For individuals who are not male and applicants to a non-competitive department, the probability of admission acceptance is 58%. For individuals who are not male and applicants to a competitive department, the probability of admission acceptance is 25%. For individuals who are male and applicants to a non-competitive department, the probability of admission acceptance is 59%. For individuals who are male and applicants to a competitive department, the probability of admission acceptance is 28%. For individuals who are not male, the probability of competitive department is 29%. For individuals who are male, the probability of competitive department is 77%. Does gender positively affect admission status through department competitiveness?\\ \tt Start your answer with 'Yes' or 'No', followed by additional reasoning or evidence to support your explanation. 
	  } 
	   } 
	   } 
	 & \\ 
 

    \theutterance \stepcounter{utterance}  

    & \multicolumn{4}{p{0.6\linewidth}}{\cellcolor[rgb]{0.8,1,0.9}{%
	\makecell[{{p{\linewidth}}}]{% 
	  \tt {\tiny [A$\rangle$GM]}  
	 Yes. The effect of gender on admission status through department competitiveness can be determined by analyzing the given probabilities. \\ \tt \\ \tt 1. **Effect of Gender on Department Competitiveness:**\\ \tt    - For individuals who are not male, the probability of being in a competitive department is 29%.\\ \tt    - For individuals who are male, the probability of being in a competitive department is 77%.\\ \tt \\ \tt 2. **Effect of Department Competitiveness on Admission Status:**\\ \tt    - For individuals in a non-competitive department, the probability of admission acceptance is 58% (regardless of gender).\\ \tt    - For individuals in a competitive department, the probability of admission acceptance is 25% (regardless of gender).\\ \tt \\ \tt 3. **Effect of Gender on Admission Status through Department Competitiveness:**\\ \tt    - **For individuals who are not male:**\\ \tt      - Probability of being in a non-competitive department: 1 - 0.29 = 0.71\\ \tt      - Probability of being in a competitive department: 0.29\\ \tt      - Probability of admission acceptance in a non-competitive department: 0.58\\ \tt      - Probability of admission acceptance in a competitive department: 0.25\\ \tt    - **For individuals who are male:**\\ \tt      - Probability of being in a non-competitive department: 1 - 0.77 = 0.23\\ \tt      - Probability of being in a competitive department: 0.77\\ \tt      - Probability of admission acceptance in a non 
	  } 
	   } 
	   } 
	 & & \\ 
 

    \theutterance \stepcounter{utterance}  

    & & & \multicolumn{2}{p{0.3\linewidth}}{\cellcolor[rgb]{0.95,0.95,0.95}{%
	\makecell[{{p{\linewidth}}}]{% 
	  \tt {\tiny [GM$|$GM]}  
	 Yes 
	  } 
	   } 
	   } 
	 & & \\ 
 

    \theutterance \stepcounter{utterance}  

    & & & \multicolumn{2}{p{0.3\linewidth}}{\cellcolor[rgb]{0.95,0.95,0.95}{%
	\makecell[{{p{\linewidth}}}]{% 
	  \tt {\tiny [GM$|$GM]}  
	 no 
	  } 
	   } 
	   } 
	 & & \\ 
 

    \theutterance \stepcounter{utterance}  

    & & & \multicolumn{2}{p{0.3\linewidth}}{\cellcolor[rgb]{0.95,0.95,0.95}{%
	\makecell[{{p{\linewidth}}}]{% 
	  \tt {\tiny [GM$|$GM]}  
	 game_result = LOSE 
	  } 
	   } 
	   } 
	 & & \\ 
 

\end{supertabular}
}

\end{document}
