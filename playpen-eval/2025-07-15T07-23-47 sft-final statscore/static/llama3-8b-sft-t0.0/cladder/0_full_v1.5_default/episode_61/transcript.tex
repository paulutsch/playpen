
\documentclass{article}
\usepackage{colortbl}
\usepackage{makecell}
\usepackage{multirow}
\usepackage{supertabular}

\begin{document}

\newcounter{utterance}

\twocolumn

{ \footnotesize  \setcounter{utterance}{1}
\setlength{\tabcolsep}{0pt}
\begin{supertabular}{c@{$\;$}|p{.15\linewidth}@{}p{.15\linewidth}p{.15\linewidth}p{.15\linewidth}p{.15\linewidth}p{.15\linewidth}}

    \# & $\;$A & \multicolumn{4}{c}{Game Master} & $\;\:$B\\
    \hline 

    \theutterance \stepcounter{utterance}  

    & & \multicolumn{4}{p{0.6\linewidth}}{\cellcolor[rgb]{0.9,0.9,0.9}{%
	\makecell[{{p{\linewidth}}}]{% 
	  \tt {\tiny [A$\langle$GM]}  
	 You are an expert in causal inference. The following question is not a typical commonsense query, but rather a meticulously designed question created by a professor specializing in causal inference, intended to assess the students' mastery of the course content.\\ \tt Imagine a self-contained, hypothetical world with only the following conditions, and without any unmentioned factors or causal relationships: Encouragement level has a direct effect on studying habit and brown eyes. Studying habit has a direct effect on brown eyes. The overall probability of encouragement is 91%. The probability of discouragement and brown eyes is 7%. The probability of encouragement and brown eyes is 58%. Is the chance of brown eyes larger when observing encouragement?\\ \tt Start your answer with 'Yes' or 'No', followed by additional reasoning or evidence to support your explanation. 
	  } 
	   } 
	   } 
	 & \\ 
 

    \theutterance \stepcounter{utterance}  

    & \multicolumn{4}{p{0.6\linewidth}}{\cellcolor[rgb]{0.8,1,0.9}{%
	\makecell[{{p{\linewidth}}}]{% 
	  \tt {\tiny [A$\rangle$GM]}  
	 To solve this problem, we need to determine the probability of brown eyes given encouragement. We can use Bayes' theorem for this purpose. Let's denote the following events:\\ \tt - \(E\) as the event of encouragement,\\ \tt - \(D\) as the event of discouragement,\\ \tt - \(B\) as the event of brown eyes.\\ \tt \\ \tt We are given the following probabilities:\\ \tt - \(P(E) = 0.91\),\\ \tt - \(P(D \cap B) = 0.07\),\\ \tt - \(P(E \cap B) = 0.58\).\\ \tt \\ \tt We need to find \(P(B|E)\), the probability of brown eyes given encouragement.\\ \tt \\ \tt Using Bayes' theorem, we have:\\ \tt \[P(B|E) = \frac{P(E \cap B)}{P(E)}\]\\ \tt \\ \tt Substituting the given values:\\ \tt \[P(B|E) = \frac{0.58}{0.91}\]\\ \tt \\ \tt Calculating the value:\\ \tt \[P(B|E) \approx 0.637\]\\ \tt \\ \tt Since \(0.637 > 0.5\), the chance of brown eyes is larger when observing encouragement.\\ \tt \\ \tt Thus, the answer is:\\ \tt Yes 
	  } 
	   } 
	   } 
	 & & \\ 
 

    \theutterance \stepcounter{utterance}  

    & & & \multicolumn{2}{p{0.3\linewidth}}{\cellcolor[rgb]{0.95,0.95,0.95}{%
	\makecell[{{p{\linewidth}}}]{% 
	  \tt {\tiny [GM$|$GM]}  
	 ParseError: The response doesn't start with one of the required word ['yes', 'no'], but To 
	  } 
	   } 
	   } 
	 & & \\ 
 

    \theutterance \stepcounter{utterance}  

    & & & \multicolumn{2}{p{0.3\linewidth}}{\cellcolor[rgb]{0.95,0.95,0.95}{%
	\makecell[{{p{\linewidth}}}]{% 
	  \tt {\tiny [GM$|$GM]}  
	 game_result = ABORT 
	  } 
	   } 
	   } 
	 & & \\ 
 

\end{supertabular}
}

\end{document}
