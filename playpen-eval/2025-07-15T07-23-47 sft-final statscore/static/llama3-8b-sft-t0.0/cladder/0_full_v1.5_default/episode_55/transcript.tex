
\documentclass{article}
\usepackage{colortbl}
\usepackage{makecell}
\usepackage{multirow}
\usepackage{supertabular}

\begin{document}

\newcounter{utterance}

\twocolumn

{ \footnotesize  \setcounter{utterance}{1}
\setlength{\tabcolsep}{0pt}
\begin{supertabular}{c@{$\;$}|p{.15\linewidth}@{}p{.15\linewidth}p{.15\linewidth}p{.15\linewidth}p{.15\linewidth}p{.15\linewidth}}

    \# & $\;$A & \multicolumn{4}{c}{Game Master} & $\;\:$B\\
    \hline 

    \theutterance \stepcounter{utterance}  

    & & \multicolumn{4}{p{0.6\linewidth}}{\cellcolor[rgb]{0.9,0.9,0.9}{%
	\makecell[{{p{\linewidth}}}]{% 
	  \tt {\tiny [A$\langle$GM]}  
	 You are an expert in causal inference. The following question is not a typical commonsense query, but rather a meticulously designed question created by a professor specializing in causal inference, intended to assess the students' mastery of the course content.\\ \tt Imagine a self-contained, hypothetical world with only the following conditions, and without any unmentioned factors or causal relationships: Alice waking up has a direct effect on rainfall. Traffic has a direct effect on rainfall. The overall probability of waking up late is 9%. The probability of waking up on time and high rainfall is 53%. The probability of waking up late and high rainfall is 8%. Is the chance of high rainfall smaller when observing waking up late?\\ \tt Start your answer with 'Yes' or 'No', followed by additional reasoning or evidence to support your explanation. 
	  } 
	   } 
	   } 
	 & \\ 
 

    \theutterance \stepcounter{utterance}  

    & \multicolumn{4}{p{0.6\linewidth}}{\cellcolor[rgb]{0.8,1,0.9}{%
	\makecell[{{p{\linewidth}}}]{% 
	  \tt {\tiny [A$\rangle$GM]}  
	 To solve this problem, we need to determine if the probability of high rainfall is smaller when Alice wakes up late compared to when she wakes up on time. We are given the following probabilities:\\ \tt \\ \tt 1. The overall probability of waking up late is 9%.\\ \tt 2. The probability of waking up on time and high rainfall is 53%.\\ \tt 3. The probability of waking up late and high rainfall is 8%.\\ \tt \\ \tt Let's denote:\\ \tt - \( P(L) \) as the probability of waking up late,\\ \tt - \( P(H|L) \) as the probability of high rainfall given that Alice wakes up late,\\ \tt - \( P(H|T) \) as the probability of high rainfall given that Alice wakes up on time.\\ \tt \\ \tt We are asked to determine if \( P(H|L) < P(H|T) \).\\ \tt \\ \tt First, we need to find the probability of waking up on time, which is \( P(T) = 1 - P(L) = 1 - 0.09 = 0.91 \).\\ \tt \\ \tt Next, we can use the law of total probability to find the overall probability of high rainfall:\\ \tt \\ \tt \[ P(H) = P(H|L) \times P(L) + P(H|T) \times P(T) \]\\ \tt \\ \tt We are given that the probability of waking up on time and high rainfall is 53%, so \( P(H|T) \times P(T) = 0.53 \). Therefore:\\ \tt \\ \tt \[ P(H|T) 
	  } 
	   } 
	   } 
	 & & \\ 
 

    \theutterance \stepcounter{utterance}  

    & & & \multicolumn{2}{p{0.3\linewidth}}{\cellcolor[rgb]{0.95,0.95,0.95}{%
	\makecell[{{p{\linewidth}}}]{% 
	  \tt {\tiny [GM$|$GM]}  
	 ParseError: The response doesn't start with one of the required word ['yes', 'no'], but To 
	  } 
	   } 
	   } 
	 & & \\ 
 

    \theutterance \stepcounter{utterance}  

    & & & \multicolumn{2}{p{0.3\linewidth}}{\cellcolor[rgb]{0.95,0.95,0.95}{%
	\makecell[{{p{\linewidth}}}]{% 
	  \tt {\tiny [GM$|$GM]}  
	 game_result = ABORT 
	  } 
	   } 
	   } 
	 & & \\ 
 

\end{supertabular}
}

\end{document}
