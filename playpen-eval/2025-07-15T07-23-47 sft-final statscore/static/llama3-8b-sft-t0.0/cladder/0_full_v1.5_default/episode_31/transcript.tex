
\documentclass{article}
\usepackage{colortbl}
\usepackage{makecell}
\usepackage{multirow}
\usepackage{supertabular}

\begin{document}

\newcounter{utterance}

\twocolumn

{ \footnotesize  \setcounter{utterance}{1}
\setlength{\tabcolsep}{0pt}
\begin{supertabular}{c@{$\;$}|p{.15\linewidth}@{}p{.15\linewidth}p{.15\linewidth}p{.15\linewidth}p{.15\linewidth}p{.15\linewidth}}

    \# & $\;$A & \multicolumn{4}{c}{Game Master} & $\;\:$B\\
    \hline 

    \theutterance \stepcounter{utterance}  

    & & \multicolumn{4}{p{0.6\linewidth}}{\cellcolor[rgb]{0.9,0.9,0.9}{%
	\makecell[{{p{\linewidth}}}]{% 
	  \tt {\tiny [A$\langle$GM]}  
	 You are an expert in causal inference. The following question is not a typical commonsense query, but rather a meticulously designed question created by a professor specializing in causal inference, intended to assess the students' mastery of the course content.\\ \tt Imagine a self-contained, hypothetical world with only the following conditions, and without any unmentioned factors or causal relationships: Gene has a direct effect on smoking and lung cancer. Pollution has a direct effect on smoking and lung cancer. Smoking has a direct effect on lung cancer. Pollution is unobserved. For people with nonsmoking genes and nonsmokers, the probability of lung cancer is 56%. For people with nonsmoking genes and smokers, the probability of lung cancer is 42%. For people with smoking genes and nonsmokers, the probability of lung cancer is 60%. For people with smoking genes and smokers, the probability of lung cancer is 40%. For people with nonsmoking genes and with low pollution, the probability of smoking is 76%. For people with nonsmoking genes and with high pollution, the probability of smoking is 48%. For people with smoking genes and with low pollution, the probability of smoking is 46%. For people with smoking genes and with high pollution, the probability of smoking is 21%. The overall probability of high pollution is 41%. Does gene positively affect lung cancer through smoking?\\ \tt Start your answer with 'Yes' or 'No', followed by additional reasoning or evidence to support your explanation. 
	  } 
	   } 
	   } 
	 & \\ 
 

    \theutterance \stepcounter{utterance}  

    & \multicolumn{4}{p{0.6\linewidth}}{\cellcolor[rgb]{0.8,1,0.9}{%
	\makecell[{{p{\linewidth}}}]{% 
	  \tt {\tiny [A$\rangle$GM]}  
	 To determine whether gene positively affects lung cancer through smoking, we need to analyze the given probabilities and causal relationships. Let's break down the information step by step:\\ \tt \\ \tt 1. **Gene and Smoking Relationship:**\\ \tt    - For people with nonsmoking genes and nonsmokers, the probability of lung cancer is 56%.\\ \tt    - For people with nonsmoking genes and smokers, the probability of lung cancer is 42%.\\ \tt    - For people with smoking genes and nonsmokers, the probability of lung cancer is 60%.\\ \tt    - For people with smoking genes and smokers, the probability of lung cancer is 40%.\\ \tt \\ \tt 2. **Pollution and Smoking Relationship:**\\ \tt    - For people with nonsmoking genes and with low pollution, the probability of smoking is 76%.\\ \tt    - For people with nonsmoking genes and with high pollution, the probability of smoking is 48%.\\ \tt    - For people with smoking genes and with low pollution, the probability of smoking is 46%.\\ \tt    - For people with smoking genes and with high pollution, the probability of smoking is 21%.\\ \tt \\ \tt 3. **Overall Probability of High Pollution:**\\ \tt    - The overall probability of high pollution is 41%.\\ \tt \\ \tt Now, let's analyze the effect of gene on lung cancer through smoking:\\ \tt \\ \tt - **Nonsmoking Genes:**\\ \tt   - Without considering pollution, the probability of lung cancer is higher for smokers (42%) than nonsmokers (56%) among those with nonsmoking genes. This 
	  } 
	   } 
	   } 
	 & & \\ 
 

    \theutterance \stepcounter{utterance}  

    & & & \multicolumn{2}{p{0.3\linewidth}}{\cellcolor[rgb]{0.95,0.95,0.95}{%
	\makecell[{{p{\linewidth}}}]{% 
	  \tt {\tiny [GM$|$GM]}  
	 ParseError: The response doesn't start with one of the required word ['yes', 'no'], but To 
	  } 
	   } 
	   } 
	 & & \\ 
 

    \theutterance \stepcounter{utterance}  

    & & & \multicolumn{2}{p{0.3\linewidth}}{\cellcolor[rgb]{0.95,0.95,0.95}{%
	\makecell[{{p{\linewidth}}}]{% 
	  \tt {\tiny [GM$|$GM]}  
	 game_result = ABORT 
	  } 
	   } 
	   } 
	 & & \\ 
 

\end{supertabular}
}

\end{document}
