
\documentclass{article}
\usepackage{colortbl}
\usepackage{makecell}
\usepackage{multirow}
\usepackage{supertabular}

\begin{document}

\newcounter{utterance}

\twocolumn

{ \footnotesize  \setcounter{utterance}{1}
\setlength{\tabcolsep}{0pt}
\begin{supertabular}{c@{$\;$}|p{.15\linewidth}@{}p{.15\linewidth}p{.15\linewidth}p{.15\linewidth}p{.15\linewidth}p{.15\linewidth}}

    \# & $\;$A & \multicolumn{4}{c}{Game Master} & $\;\:$B\\
    \hline 

    \theutterance \stepcounter{utterance}  

    & & \multicolumn{4}{p{0.6\linewidth}}{\cellcolor[rgb]{0.9,0.9,0.9}{%
	\makecell[{{p{\linewidth}}}]{% 
	  \tt {\tiny [A$\langle$GM]}  
	 You are an expert in causal inference. The following question is not a typical commonsense query, but rather a meticulously designed question created by a professor specializing in causal inference, intended to assess the students' mastery of the course content.\\ \tt Imagine a self-contained, hypothetical world with only the following conditions, and without any unmentioned factors or causal relationships: Gwet has a direct effect on yomx and xevu. Yomx has a direct effect on xevu. For those who are not gwet and are not yomx, the probability of xevu is 82%. For those who are not gwet and are yomx, the probability of xevu is 63%. For those who are gwet and are not yomx, the probability of xevu is 36%. For those who are gwet and are yomx, the probability of xevu is 18%. The overall probability of gwet is 41%. Will yomx increase the chance of xevu?\\ \tt Start your answer with 'Yes' or 'No', followed by additional reasoning or evidence to support your explanation. 
	  } 
	   } 
	   } 
	 & \\ 
 

    \theutterance \stepcounter{utterance}  

    & \multicolumn{4}{p{0.6\linewidth}}{\cellcolor[rgb]{0.8,1,0.9}{%
	\makecell[{{p{\linewidth}}}]{% 
	  \tt {\tiny [A$\rangle$GM]}  
	 To solve this problem, we need to determine whether the presence of yomx increases the probability of xevu. We are given the following probabilities:\\ \tt \\ \tt 1. For those who are not gwet and are not yomx, the probability of xevu is 82%.\\ \tt 2. For those who are not gwet and are yomx, the probability of xevu is 63%.\\ \tt 3. For those who are gwet and are not yomx, the probability of xevu is 36%.\\ \tt 4. For those who are gwet and are yomx, the probability of xevu is 18%.\\ \tt 5. The overall probability of gwet is 41%.\\ \tt \\ \tt We need to calculate the overall probability of xevu and compare it with the probability of xevu when yomx is present.\\ \tt \\ \tt Let's denote the probability of gwet as P(gwet) = 0.41 and the probability of yomx as P(yomx) = 1 - P(not yomx). We are given that the overall probability of xevu is not explicitly stated, but we can calculate it using the law of total probability.\\ \tt \\ \tt The law of total probability states that the overall probability of an event is the sum of the probabilities of the event given each possible cause, weighted by the probability of each cause.\\ \tt \\ \tt Let's denote the probability of xevu as P(xevu). We can write the law of total probability as 
	  } 
	   } 
	   } 
	 & & \\ 
 

    \theutterance \stepcounter{utterance}  

    & & & \multicolumn{2}{p{0.3\linewidth}}{\cellcolor[rgb]{0.95,0.95,0.95}{%
	\makecell[{{p{\linewidth}}}]{% 
	  \tt {\tiny [GM$|$GM]}  
	 ParseError: The response doesn't start with one of the required word ['yes', 'no'], but To 
	  } 
	   } 
	   } 
	 & & \\ 
 

    \theutterance \stepcounter{utterance}  

    & & & \multicolumn{2}{p{0.3\linewidth}}{\cellcolor[rgb]{0.95,0.95,0.95}{%
	\makecell[{{p{\linewidth}}}]{% 
	  \tt {\tiny [GM$|$GM]}  
	 game_result = ABORT 
	  } 
	   } 
	   } 
	 & & \\ 
 

\end{supertabular}
}

\end{document}
