
\documentclass{article}
\usepackage{colortbl}
\usepackage{makecell}
\usepackage{multirow}
\usepackage{supertabular}

\begin{document}

\newcounter{utterance}

\twocolumn

{ \footnotesize  \setcounter{utterance}{1}
\setlength{\tabcolsep}{0pt}
\begin{supertabular}{c@{$\;$}|p{.15\linewidth}@{}p{.15\linewidth}p{.15\linewidth}p{.15\linewidth}p{.15\linewidth}p{.15\linewidth}}

    \# & $\;$A & \multicolumn{4}{c}{Game Master} & $\;\:$B\\
    \hline 

    \theutterance \stepcounter{utterance}  

    & & \multicolumn{4}{p{0.6\linewidth}}{\cellcolor[rgb]{0.9,0.9,0.9}{%
	\makecell[{{p{\linewidth}}}]{% 
	  \tt {\tiny [A$\langle$GM]}  
	 You are an expert in causal inference. The following question is not a typical commonsense query, but rather a meticulously designed question created by a professor specializing in causal inference, intended to assess the students' mastery of the course content.\\ \tt Imagine a self-contained, hypothetical world with only the following conditions, and without any unmentioned factors or causal relationships: Gwet has a direct effect on yomx and xevu. Yomx has a direct effect on xevu. The overall probability of yomx is 75%. The probability of not yomx and xevu is 22%. The probability of yomx and xevu is 56%. Is the chance of xevu smaller when observing yomx?\\ \tt Start your answer with 'Yes' or 'No', followed by additional reasoning or evidence to support your explanation. 
	  } 
	   } 
	   } 
	 & \\ 
 

    \theutterance \stepcounter{utterance}  

    & \multicolumn{4}{p{0.6\linewidth}}{\cellcolor[rgb]{0.8,1,0.9}{%
	\makecell[{{p{\linewidth}}}]{% 
	  \tt {\tiny [A$\rangle$GM]}  
	 To solve this problem, we need to understand the given conditions and their implications on the relationships between the variables yomx and xevu. Let's break down the information provided:\\ \tt \\ \tt 1. **Gwet has a direct effect on yomx and xevu.**\\ \tt    - This means that the presence or absence of gwet can influence the probability of yomx and xevu occurring.\\ \tt \\ \tt 2. **Yomx has a direct effect on xevu.**\\ \tt    - This indicates that the occurrence of yomx can influence the probability of xevu.\\ \tt \\ \tt 3. **The overall probability of yomx is 75%.**\\ \tt    - This gives us a baseline probability for yomx.\\ \tt \\ \tt 4. **The probability of not yomx and xevu is 22%.**\\ \tt    - This tells us the probability of both yomx and xevu not occurring.\\ \tt \\ \tt 5. **The probability of yomx and xevu is 56%.**\\ \tt    - This gives us the probability of both yomx and xevu occurring.\\ \tt \\ \tt We need to determine if the chance of xevu is smaller when observing yomx. To do this, we can use the concept of conditional probability. The probability of xevu given yomx (P(xevu|yomx)) can be calculated using Bayes' theorem.\\ \tt \\ \tt Let's denote:\\ \tt - \( P(yomx) = 0.75 \) 
	  } 
	   } 
	   } 
	 & & \\ 
 

    \theutterance \stepcounter{utterance}  

    & & & \multicolumn{2}{p{0.3\linewidth}}{\cellcolor[rgb]{0.95,0.95,0.95}{%
	\makecell[{{p{\linewidth}}}]{% 
	  \tt {\tiny [GM$|$GM]}  
	 ParseError: The response doesn't start with one of the required word ['yes', 'no'], but To 
	  } 
	   } 
	   } 
	 & & \\ 
 

    \theutterance \stepcounter{utterance}  

    & & & \multicolumn{2}{p{0.3\linewidth}}{\cellcolor[rgb]{0.95,0.95,0.95}{%
	\makecell[{{p{\linewidth}}}]{% 
	  \tt {\tiny [GM$|$GM]}  
	 game_result = ABORT 
	  } 
	   } 
	   } 
	 & & \\ 
 

\end{supertabular}
}

\end{document}
