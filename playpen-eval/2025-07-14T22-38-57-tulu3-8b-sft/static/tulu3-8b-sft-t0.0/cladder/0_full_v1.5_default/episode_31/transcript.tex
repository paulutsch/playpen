
\documentclass{article}
\usepackage{colortbl}
\usepackage{makecell}
\usepackage{multirow}
\usepackage{supertabular}

\begin{document}

\newcounter{utterance}

\twocolumn

{ \footnotesize  \setcounter{utterance}{1}
\setlength{\tabcolsep}{0pt}
\begin{supertabular}{c@{$\;$}|p{.15\linewidth}@{}p{.15\linewidth}p{.15\linewidth}p{.15\linewidth}p{.15\linewidth}p{.15\linewidth}}

    \# & $\;$A & \multicolumn{4}{c}{Game Master} & $\;\:$B\\
    \hline 

    \theutterance \stepcounter{utterance}  

    & & \multicolumn{4}{p{0.6\linewidth}}{\cellcolor[rgb]{0.9,0.9,0.9}{%
	\makecell[{{p{\linewidth}}}]{% 
	  \tt {\tiny [A$\langle$GM]}  
	 You are an expert in causal inference. The following question is not a typical commonsense query, but rather a meticulously designed question created by a professor specializing in causal inference, intended to assess the students' mastery of the course content.\\ \tt Imagine a self-contained, hypothetical world with only the following conditions, and without any unmentioned factors or causal relationships: Gene has a direct effect on smoking and lung cancer. Pollution has a direct effect on smoking and lung cancer. Smoking has a direct effect on lung cancer. Pollution is unobserved. For people with nonsmoking genes and nonsmokers, the probability of lung cancer is 56%. For people with nonsmoking genes and smokers, the probability of lung cancer is 42%. For people with smoking genes and nonsmokers, the probability of lung cancer is 60%. For people with smoking genes and smokers, the probability of lung cancer is 40%. For people with nonsmoking genes and with low pollution, the probability of smoking is 76%. For people with nonsmoking genes and with high pollution, the probability of smoking is 48%. For people with smoking genes and with low pollution, the probability of smoking is 46%. For people with smoking genes and with high pollution, the probability of smoking is 21%. The overall probability of high pollution is 41%. Does gene positively affect lung cancer through smoking?\\ \tt Start your answer with 'Yes' or 'No', followed by additional reasoning or evidence to support your explanation. 
	  } 
	   } 
	   } 
	 & \\ 
 

    \theutterance \stepcounter{utterance}  

    & \multicolumn{4}{p{0.6\linewidth}}{\cellcolor[rgb]{0.8,1,0.9}{%
	\makecell[{{p{\linewidth}}}]{% 
	  \tt {\tiny [A$\rangle$GM]}  
	 No, gene does not positively affect lung cancer through smoking. The data provided shows that the probability of lung cancer is higher for people with nonsmoking genes and nonsmokers (56%) compared to people with nonsmoking genes and smokers (42%). This suggests that having nonsmoking genes actually reduces the risk of lung cancer, even among smokers. Additionally, the probability of smoking is higher for people with nonsmoking genes and low pollution (76%) compared to people with nonsmoking genes and high pollution (48%), indicating that pollution has a stronger effect on smoking behavior than genes. Therefore, the relationship between genes and lung cancer is not mediated by smoking, but rather, genes may have a direct protective effect against lung cancer.<|end_of_text|> 
	  } 
	   } 
	   } 
	 & & \\ 
 

    \theutterance \stepcounter{utterance}  

    & & & \multicolumn{2}{p{0.3\linewidth}}{\cellcolor[rgb]{0.95,0.95,0.95}{%
	\makecell[{{p{\linewidth}}}]{% 
	  \tt {\tiny [GM$|$GM]}  
	 No 
	  } 
	   } 
	   } 
	 & & \\ 
 

    \theutterance \stepcounter{utterance}  

    & & & \multicolumn{2}{p{0.3\linewidth}}{\cellcolor[rgb]{0.95,0.95,0.95}{%
	\makecell[{{p{\linewidth}}}]{% 
	  \tt {\tiny [GM$|$GM]}  
	 yes 
	  } 
	   } 
	   } 
	 & & \\ 
 

    \theutterance \stepcounter{utterance}  

    & & & \multicolumn{2}{p{0.3\linewidth}}{\cellcolor[rgb]{0.95,0.95,0.95}{%
	\makecell[{{p{\linewidth}}}]{% 
	  \tt {\tiny [GM$|$GM]}  
	 game_result = LOSE 
	  } 
	   } 
	   } 
	 & & \\ 
 

\end{supertabular}
}

\end{document}
