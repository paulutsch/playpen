
\documentclass{article}
\usepackage{colortbl}
\usepackage{makecell}
\usepackage{multirow}
\usepackage{supertabular}

\begin{document}

\newcounter{utterance}

\twocolumn

{ \footnotesize  \setcounter{utterance}{1}
\setlength{\tabcolsep}{0pt}
\begin{supertabular}{c@{$\;$}|p{.15\linewidth}@{}p{.15\linewidth}p{.15\linewidth}p{.15\linewidth}p{.15\linewidth}p{.15\linewidth}}

    \# & $\;$A & \multicolumn{4}{c}{Game Master} & $\;\:$B\\
    \hline 

    \theutterance \stepcounter{utterance}  

    & & \multicolumn{4}{p{0.6\linewidth}}{\cellcolor[rgb]{0.9,0.9,0.9}{%
	\makecell[{{p{\linewidth}}}]{% 
	  \tt {\tiny [A$\langle$GM]}  
	 You are an expert in causal inference. The following question is not a typical commonsense query, but rather a meticulously designed question created by a professor specializing in causal inference, intended to assess the students' mastery of the course content.\\ \tt Imagine a self-contained, hypothetical world with only the following conditions, and without any unmentioned factors or causal relationships: Rixq has a direct effect on xevu. Zuph has a direct effect on xevu. The overall probability of rixq is 40%. The probability of not rixq and xevu is 9%. The probability of rixq and xevu is 22%. Is the chance of xevu smaller when observing rixq?\\ \tt Start your answer with 'Yes' or 'No', followed by additional reasoning or evidence to support your explanation. 
	  } 
	   } 
	   } 
	 & \\ 
 

    \theutterance \stepcounter{utterance}  

    & \multicolumn{4}{p{0.6\linewidth}}{\cellcolor[rgb]{0.8,1,0.9}{%
	\makecell[{{p{\linewidth}}}]{% 
	  \tt {\tiny [A$\rangle$GM]}  
	 No, the chance of xevu is not smaller when observing rixq. To determine this, we can use the law of total probability. \\ \tt \\ \tt Let's denote the following events:\\ \tt - Rixq = R\\ \tt - Xevu = X\\ \tt - Not Rixq = ~R\\ \tt - Not Xevu = ~X\\ \tt \\ \tt We are given the following probabilities:\\ \tt - P(R) = 0.40\\ \tt - P(~R ∩ ~X) = 0.09\\ \tt - P(R ∩ X) = 0.22\\ \tt \\ \tt We want to find P(X|R), the probability of Xevu given Rixq. Using the law of total probability, we can express P(X) as follows:\\ \tt \\ \tt P(X) = P(X|R)P(R) + P(X|~R)P(~R)\\ \tt \\ \tt We also know that P(X) = P(X|R)P(R) + P(X|~R)P(~R) + P(X|R ∩ ~X)P(R ∩ ~X)\\ \tt \\ \tt Since P(X|R ∩ ~X) = 0 (because Rixq and Xevu cannot both be false), we can simplify this to:\\ \tt \\ \tt P(X) = P(X|R)P(R) + P(X|~R)P(~R)\\ \tt \\ \tt Now, we can solve for P(X|R):\\ \tt \\ \tt P(X|R) = (P(X) - P(X|~R)P(~R)) / P 
	  } 
	   } 
	   } 
	 & & \\ 
 

    \theutterance \stepcounter{utterance}  

    & & & \multicolumn{2}{p{0.3\linewidth}}{\cellcolor[rgb]{0.95,0.95,0.95}{%
	\makecell[{{p{\linewidth}}}]{% 
	  \tt {\tiny [GM$|$GM]}  
	 No 
	  } 
	   } 
	   } 
	 & & \\ 
 

    \theutterance \stepcounter{utterance}  

    & & & \multicolumn{2}{p{0.3\linewidth}}{\cellcolor[rgb]{0.95,0.95,0.95}{%
	\makecell[{{p{\linewidth}}}]{% 
	  \tt {\tiny [GM$|$GM]}  
	 no 
	  } 
	   } 
	   } 
	 & & \\ 
 

    \theutterance \stepcounter{utterance}  

    & & & \multicolumn{2}{p{0.3\linewidth}}{\cellcolor[rgb]{0.95,0.95,0.95}{%
	\makecell[{{p{\linewidth}}}]{% 
	  \tt {\tiny [GM$|$GM]}  
	 game_result = WIN 
	  } 
	   } 
	   } 
	 & & \\ 
 

\end{supertabular}
}

\end{document}
