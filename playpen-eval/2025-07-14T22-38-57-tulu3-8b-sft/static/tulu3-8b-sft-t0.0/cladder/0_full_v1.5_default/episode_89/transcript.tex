
\documentclass{article}
\usepackage{colortbl}
\usepackage{makecell}
\usepackage{multirow}
\usepackage{supertabular}

\begin{document}

\newcounter{utterance}

\twocolumn

{ \footnotesize  \setcounter{utterance}{1}
\setlength{\tabcolsep}{0pt}
\begin{supertabular}{c@{$\;$}|p{.15\linewidth}@{}p{.15\linewidth}p{.15\linewidth}p{.15\linewidth}p{.15\linewidth}p{.15\linewidth}}

    \# & $\;$A & \multicolumn{4}{c}{Game Master} & $\;\:$B\\
    \hline 

    \theutterance \stepcounter{utterance}  

    & & \multicolumn{4}{p{0.6\linewidth}}{\cellcolor[rgb]{0.9,0.9,0.9}{%
	\makecell[{{p{\linewidth}}}]{% 
	  \tt {\tiny [A$\langle$GM]}  
	 You are an expert in causal inference. The following question is not a typical commonsense query, but rather a meticulously designed question created by a professor specializing in causal inference, intended to assess the students' mastery of the course content.\\ \tt Imagine a self-contained, hypothetical world with only the following conditions, and without any unmentioned factors or causal relationships: Zuph has a direct effect on wibl. Uvzi has a direct effect on wibl. Method 1: We look at how zuph correlates with uvzi case by case according to wibl. Method 2: We look directly at how zuph correlates with uvzi in general. To understand how zuph affects uvzi, is it more correct to use the Method 1 than Method 2?\\ \tt Start your answer with 'Yes' or 'No', followed by additional reasoning or evidence to support your explanation. 
	  } 
	   } 
	   } 
	 & \\ 
 

    \theutterance \stepcounter{utterance}  

    & \multicolumn{4}{p{0.6\linewidth}}{\cellcolor[rgb]{0.8,1,0.9}{%
	\makecell[{{p{\linewidth}}}]{% 
	  \tt {\tiny [A$\rangle$GM]}  
	 No, it is not more correct to use Method 1 than Method 2 to understand how Zuph affects Uvzi. Both methods can provide insights into the relationship between Zuph and Uvzi, but they do so in different ways. Method 1 looks at the correlation between Zuph and Uvzi on a case-by-case basis according to Wibl, which can be useful for identifying specific instances where Zuph affects Uvzi. However, this method may not provide a general understanding of the relationship between Zuph and Uvzi. Method 2, on the other hand, looks directly at the correlation between Zuph and Uvzi in general, which can provide a more comprehensive understanding of their relationship. Therefore, both methods have their own strengths and weaknesses, and neither is inherently more correct than the other.<|end_of_text|> 
	  } 
	   } 
	   } 
	 & & \\ 
 

    \theutterance \stepcounter{utterance}  

    & & & \multicolumn{2}{p{0.3\linewidth}}{\cellcolor[rgb]{0.95,0.95,0.95}{%
	\makecell[{{p{\linewidth}}}]{% 
	  \tt {\tiny [GM$|$GM]}  
	 No 
	  } 
	   } 
	   } 
	 & & \\ 
 

    \theutterance \stepcounter{utterance}  

    & & & \multicolumn{2}{p{0.3\linewidth}}{\cellcolor[rgb]{0.95,0.95,0.95}{%
	\makecell[{{p{\linewidth}}}]{% 
	  \tt {\tiny [GM$|$GM]}  
	 no 
	  } 
	   } 
	   } 
	 & & \\ 
 

    \theutterance \stepcounter{utterance}  

    & & & \multicolumn{2}{p{0.3\linewidth}}{\cellcolor[rgb]{0.95,0.95,0.95}{%
	\makecell[{{p{\linewidth}}}]{% 
	  \tt {\tiny [GM$|$GM]}  
	 game_result = WIN 
	  } 
	   } 
	   } 
	 & & \\ 
 

\end{supertabular}
}

\end{document}
