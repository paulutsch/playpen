
\documentclass{article}
\usepackage{colortbl}
\usepackage{makecell}
\usepackage{multirow}
\usepackage{supertabular}

\begin{document}

\newcounter{utterance}

\twocolumn

{ \footnotesize  \setcounter{utterance}{1}
\setlength{\tabcolsep}{0pt}
\begin{supertabular}{c@{$\;$}|p{.15\linewidth}@{}p{.15\linewidth}p{.15\linewidth}p{.15\linewidth}p{.15\linewidth}p{.15\linewidth}}

    \# & $\;$A & \multicolumn{4}{c}{Game Master} & $\;\:$B\\
    \hline 

    \theutterance \stepcounter{utterance}  

    & & \multicolumn{4}{p{0.6\linewidth}}{\cellcolor[rgb]{0.9,0.9,0.9}{%
	\makecell[{{p{\linewidth}}}]{% 
	  \tt {\tiny [A$\langle$GM]}  
	 You are an expert in causal inference. The following question is not a typical commonsense query, but rather a meticulously designed question created by a professor specializing in causal inference, intended to assess the students' mastery of the course content.\\ \tt Imagine a self-contained, hypothetical world with only the following conditions, and without any unmentioned factors or causal relationships: Poverty has a direct effect on liking spicy food and cholera. Water company has a direct effect on liking spicy food. Liking spicy food has a direct effect on cholera. Poverty is unobserved. The overall probability of liking spicy food is 41%. For people who do not like spicy food, the probability of cholera contraction is 59%. For people who like spicy food, the probability of cholera contraction is 33%. Is cholera contraction more likely than cholera prevention overall?\\ \tt Start your answer with 'Yes' or 'No', followed by additional reasoning or evidence to support your explanation. 
	  } 
	   } 
	   } 
	 & \\ 
 

    \theutterance \stepcounter{utterance}  

    & \multicolumn{4}{p{0.6\linewidth}}{\cellcolor[rgb]{0.8,1,0.9}{%
	\makecell[{{p{\linewidth}}}]{% 
	  \tt {\tiny [A$\rangle$GM]}  
	 No, cholera contraction is not more likely than cholera prevention overall. \\ \tt \\ \tt To understand this, we need to consider the probabilities given in the problem. The overall probability of liking spicy food is 41%. For people who do not like spicy food, the probability of cholera contraction is 59%. For people who like spicy food, the probability of cholera contraction is 33%.\\ \tt \\ \tt We can calculate the overall probability of cholera contraction by considering both groups: those who like spicy food and those who do not. The probability of cholera contraction for the entire population can be calculated as follows:\\ \tt \\ \tt P(cholera) = P(cholera | like spicy) * P(like spicy) + P(cholera | dislike spicy) * P(dislike spicy)\\ \tt \\ \tt P(cholera) = (0.33 * 0.41) + (0.59 * 0.59)\\ \tt \\ \tt P(cholera) = 0.1353 + 0.3481\\ \tt \\ \tt P(cholera) = 0.4834\\ \tt \\ \tt So, the overall probability of cholera contraction is 48.34%. \\ \tt \\ \tt On the other hand, the overall probability of cholera prevention is 1 - P(cholera), which is:\\ \tt \\ \tt P(prevention) = 1 - 0.4834 = 0.5166\\ \tt \\ \tt Therefore, cholera prevention is more likely than cholera contraction overall.<|end_of_text|> 
	  } 
	   } 
	   } 
	 & & \\ 
 

    \theutterance \stepcounter{utterance}  

    & & & \multicolumn{2}{p{0.3\linewidth}}{\cellcolor[rgb]{0.95,0.95,0.95}{%
	\makecell[{{p{\linewidth}}}]{% 
	  \tt {\tiny [GM$|$GM]}  
	 No 
	  } 
	   } 
	   } 
	 & & \\ 
 

    \theutterance \stepcounter{utterance}  

    & & & \multicolumn{2}{p{0.3\linewidth}}{\cellcolor[rgb]{0.95,0.95,0.95}{%
	\makecell[{{p{\linewidth}}}]{% 
	  \tt {\tiny [GM$|$GM]}  
	 no 
	  } 
	   } 
	   } 
	 & & \\ 
 

    \theutterance \stepcounter{utterance}  

    & & & \multicolumn{2}{p{0.3\linewidth}}{\cellcolor[rgb]{0.95,0.95,0.95}{%
	\makecell[{{p{\linewidth}}}]{% 
	  \tt {\tiny [GM$|$GM]}  
	 game_result = WIN 
	  } 
	   } 
	   } 
	 & & \\ 
 

\end{supertabular}
}

\end{document}
