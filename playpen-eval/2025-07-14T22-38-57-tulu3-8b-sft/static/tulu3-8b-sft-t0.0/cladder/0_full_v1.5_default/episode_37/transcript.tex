
\documentclass{article}
\usepackage{colortbl}
\usepackage{makecell}
\usepackage{multirow}
\usepackage{supertabular}

\begin{document}

\newcounter{utterance}

\twocolumn

{ \footnotesize  \setcounter{utterance}{1}
\setlength{\tabcolsep}{0pt}
\begin{supertabular}{c@{$\;$}|p{.15\linewidth}@{}p{.15\linewidth}p{.15\linewidth}p{.15\linewidth}p{.15\linewidth}p{.15\linewidth}}

    \# & $\;$A & \multicolumn{4}{c}{Game Master} & $\;\:$B\\
    \hline 

    \theutterance \stepcounter{utterance}  

    & & \multicolumn{4}{p{0.6\linewidth}}{\cellcolor[rgb]{0.9,0.9,0.9}{%
	\makecell[{{p{\linewidth}}}]{% 
	  \tt {\tiny [A$\langle$GM]}  
	 You are an expert in causal inference. The following question is not a typical commonsense query, but rather a meticulously designed question created by a professor specializing in causal inference, intended to assess the students' mastery of the course content.\\ \tt Imagine a self-contained, hypothetical world with only the following conditions, and without any unmentioned factors or causal relationships: Full moon has a direct effect on sprinkler and weather. Weather has a direct effect on ground. Sprinkler has a direct effect on ground. The overall probability of full moon is 31%. The probability of no full moon and wet ground is 33%. The probability of full moon and wet ground is 19%. Is the chance of wet ground smaller when observing full moon?\\ \tt Start your answer with 'Yes' or 'No', followed by additional reasoning or evidence to support your explanation. 
	  } 
	   } 
	   } 
	 & \\ 
 

    \theutterance \stepcounter{utterance}  

    & \multicolumn{4}{p{0.6\linewidth}}{\cellcolor[rgb]{0.8,1,0.9}{%
	\makecell[{{p{\linewidth}}}]{% 
	  \tt {\tiny [A$\rangle$GM]}  
	 No, the chance of wet ground is not smaller when observing a full moon. \\ \tt \\ \tt To understand this, we need to use the principle of conditional probability. We are given the following probabilities:\\ \tt \\ \tt P(Full Moon) = 0.31\\ \tt P(No Full Moon and Wet Ground) = 0.33\\ \tt P(Full Moon and Wet Ground) = 0.19\\ \tt \\ \tt We want to find the probability of wet ground given a full moon, P(Wet Ground | Full Moon). We can use Bayes' theorem to find this:\\ \tt \\ \tt P(Wet Ground | Full Moon) = P(Full Moon and Wet Ground) / P(Full Moon)\\ \tt \\ \tt P(Wet Ground | Full Moon) = 0.19 / 0.31\\ \tt P(Wet Ground | Full Moon) ≈ 0.61\\ \tt \\ \tt So, the probability of wet ground given a full moon is approximately 0.61, or 61%. This is higher than the overall probability of wet ground, which we can find by summing the probabilities of wet ground under both conditions:\\ \tt \\ \tt P(Wet Ground) = P(No Full Moon and Wet Ground) + P(Full Moon and Wet Ground)\\ \tt P(Wet Ground) = 0.33 + 0.19\\ \tt P(Wet Ground) = 0.52\\ \tt \\ \tt So, the overall probability of wet ground is 0.52, or 52%. Since 0.61 (61%) is greater than 0.52 
	  } 
	   } 
	   } 
	 & & \\ 
 

    \theutterance \stepcounter{utterance}  

    & & & \multicolumn{2}{p{0.3\linewidth}}{\cellcolor[rgb]{0.95,0.95,0.95}{%
	\makecell[{{p{\linewidth}}}]{% 
	  \tt {\tiny [GM$|$GM]}  
	 No 
	  } 
	   } 
	   } 
	 & & \\ 
 

    \theutterance \stepcounter{utterance}  

    & & & \multicolumn{2}{p{0.3\linewidth}}{\cellcolor[rgb]{0.95,0.95,0.95}{%
	\makecell[{{p{\linewidth}}}]{% 
	  \tt {\tiny [GM$|$GM]}  
	 no 
	  } 
	   } 
	   } 
	 & & \\ 
 

    \theutterance \stepcounter{utterance}  

    & & & \multicolumn{2}{p{0.3\linewidth}}{\cellcolor[rgb]{0.95,0.95,0.95}{%
	\makecell[{{p{\linewidth}}}]{% 
	  \tt {\tiny [GM$|$GM]}  
	 game_result = WIN 
	  } 
	   } 
	   } 
	 & & \\ 
 

\end{supertabular}
}

\end{document}
