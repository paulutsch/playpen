
\documentclass{article}
\usepackage{colortbl}
\usepackage{makecell}
\usepackage{multirow}
\usepackage{supertabular}

\begin{document}

\newcounter{utterance}

\twocolumn

{ \footnotesize  \setcounter{utterance}{1}
\setlength{\tabcolsep}{0pt}
\begin{supertabular}{c@{$\;$}|p{.15\linewidth}@{}p{.15\linewidth}p{.15\linewidth}p{.15\linewidth}p{.15\linewidth}p{.15\linewidth}}

    \# & $\;$A & \multicolumn{4}{c}{Game Master} & $\;\:$B\\
    \hline 

    \theutterance \stepcounter{utterance}  

    & & \multicolumn{4}{p{0.6\linewidth}}{\cellcolor[rgb]{0.9,0.9,0.9}{%
	\makecell[{{p{\linewidth}}}]{% 
	  \tt {\tiny [A$\langle$GM]}  
	 Write a casual blog post about how the outer solar system is different from the inner solar system, and what that means for the possibility of life. Wrap your entire response with double quotation marks. Your response should contain 17 or more sentences. 
	  } 
	   } 
	   } 
	 & \\ 
 

    \theutterance \stepcounter{utterance}  

    & \multicolumn{4}{p{0.6\linewidth}}{\cellcolor[rgb]{0.8,1,0.9}{%
	\makecell[{{p{\linewidth}}}]{% 
	  \tt {\tiny [A$\rangle$GM]}  
	 "Exploring the vast expanse of our solar system reveals a fascinating contrast between its inner and outer regions, each with unique characteristics that influence the potential for life. The inner solar system, comprising the region closest to the Sun, includes the terrestrial planets: Mercury, Venus, Earth, and Mars. These planets are primarily composed of rock and metal, with relatively high surface temperatures due to their proximity to the Sun. Earth, in particular, is the only known planet in our solar system that supports life, thanks to its moderate climate, abundant water, and a protective atmosphere.\\ \tt \\ \tt In contrast, the outer solar system, beyond the asteroid belt, is dominated by gas giants like Jupiter and Saturn, and ice giants like Uranus and Neptune. These planets are composed mostly of hydrogen and helium, with thick atmospheres and powerful winds. The outer solar system also includes moons, such as Jupiter's Europa and Saturn's Enceladus, which have captured the attention of scientists due to their subsurface oceans. These oceans, shielded from the harsh radiation of space by a layer of ice, offer potential habitats for life.\\ \tt \\ \tt The differences in temperature, atmospheric composition, and geological activity between the inner and outer solar system have significant implications for the possibility of life. In the inner solar system, the harsh conditions on planets like Mercury and Venus make life as we know it unlikely. However, Earth's unique position and conditions have allowed life to flourish. Mars, while once thought to be a potential abode for life, faces challenges such as a thin atmosphere and extreme cold, which make sustaining life difficult.\\ \tt \\ \tt In the outer solar system, the presence of subsurface oceans on moons like Europa and Enceladus presents intriguing possibilities. These oceans could provide the necessary conditions for life, such as liquid water and potential chemical energy sources. The protection offered by these moons' icy crusts from cosmic radiation and solar winds could also shield any potential life forms from harmful space environments.\\ \tt \\ \tt The study of the outer solar system not only expands our understanding of planetary science but also informs our search for life beyond Earth. As we continue to explore these distant worlds, we gain insights into the diversity of planetary systems and the conditions necessary for life. The outer solar system, with its icy moons and gas giants, offers a glimpse into the potential for life in other parts of our galaxy and beyond.\\ \tt \\ \tt In conclusion, the differences between the inner and outer solar system highlight the unique conditions that can support life. While the inner solar system presents challenges for life, the outer solar system offers intriguing possibilities with its subsurface oceans and protected environments. As we delve deeper into the mysteries of our solar system, we uncover the vast potential for life in the universe, reminding us of the importance of continued exploration and discovery."<|end_of_text|> 
	  } 
	   } 
	   } 
	 & & \\ 
 

    \theutterance \stepcounter{utterance}  

    & & & \multicolumn{2}{p{0.3\linewidth}}{\cellcolor[rgb]{0.95,0.95,0.95}{%
	\makecell[{{p{\linewidth}}}]{% 
	  \tt {\tiny [GM$|$GM]}  
	 {'startend:quotation': {}, 'length_constraints:number_sentences': {'relation': 'at least', 'num_sentences': 17}} 
	  } 
	   } 
	   } 
	 & & \\ 
 

    \theutterance \stepcounter{utterance}  

    & & & \multicolumn{2}{p{0.3\linewidth}}{\cellcolor[rgb]{0.95,0.95,0.95}{%
	\makecell[{{p{\linewidth}}}]{% 
	  \tt {\tiny [GM$|$GM]}  
	 game_result = LOSE 
	  } 
	   } 
	   } 
	 & & \\ 
 

\end{supertabular}
}

\end{document}
