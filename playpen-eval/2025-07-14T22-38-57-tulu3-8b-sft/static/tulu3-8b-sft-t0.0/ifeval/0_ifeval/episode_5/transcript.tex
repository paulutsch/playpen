
\documentclass{article}
\usepackage{colortbl}
\usepackage{makecell}
\usepackage{multirow}
\usepackage{supertabular}

\begin{document}

\newcounter{utterance}

\twocolumn

{ \footnotesize  \setcounter{utterance}{1}
\setlength{\tabcolsep}{0pt}
\begin{supertabular}{c@{$\;$}|p{.15\linewidth}@{}p{.15\linewidth}p{.15\linewidth}p{.15\linewidth}p{.15\linewidth}p{.15\linewidth}}

    \# & $\;$A & \multicolumn{4}{c}{Game Master} & $\;\:$B\\
    \hline 

    \theutterance \stepcounter{utterance}  

    & & \multicolumn{4}{p{0.6\linewidth}}{\cellcolor[rgb]{0.9,0.9,0.9}{%
	\makecell[{{p{\linewidth}}}]{% 
	  \tt {\tiny [A$\langle$GM]}  
	 Write an essay about how aluminium cans are used in food storage. Don’t forget to include the keywords waste, material and meal. Have more than 30 sentences in your response. 
	  } 
	   } 
	   } 
	 & \\ 
 

    \theutterance \stepcounter{utterance}  

    & \multicolumn{4}{p{0.6\linewidth}}{\cellcolor[rgb]{0.8,1,0.9}{%
	\makecell[{{p{\linewidth}}}]{% 
	  \tt {\tiny [A$\rangle$GM]}  
	 Aluminium cans are widely used in food storage due to their lightweight, durability, and non-toxic nature. The material is highly corrosion-resistant, which makes it an ideal choice for storing food. Aluminium cans are also easy to stack and transport, making them a convenient option for storing food in bulk. The material is also recyclable, which helps reduce waste and promotes sustainability.\\ \tt \\ \tt Aluminium cans are commonly used to store a variety of food items, including canned vegetables, fruits, and soups. The material provides an airtight seal, which helps preserve the freshness of the food. The cans also protect the food from external factors such as moisture, light, and oxygen, which can cause spoilage. This makes aluminium cans an excellent choice for storing food that has a longer shelf life.\\ \tt \\ \tt In addition to canned food, aluminium cans are also used to store dry goods such as grains, beans, and pasta. The material provides a moisture-proof barrier, which helps prevent the food from becoming stale or moldy. The cans also protect the food from pests such as insects and rodents, which can contaminate the food. This makes aluminium cans an excellent choice for storing food in areas with high pest activity.\\ \tt \\ \tt Aluminium cans are also used to store homemade meals and leftovers. The material provides an airtight seal, which helps preserve the flavor and texture of the food. The cans also protect the food from bacteria and other contaminants, which can cause foodborne illness. This makes aluminium cans an excellent choice for storing food for later consumption.\\ \tt \\ \tt Despite the many benefits of aluminium cans, there are some drawbacks to consider. The material is not suitable for storing acidic or alkaline foods, as it can cause a chemical reaction that can damage the can and affect the taste of the food. Additionally, the material is not suitable for storing hot food, as it can cause the can to warp or deform.\\ \tt \\ \tt In conclusion, aluminium cans are an excellent choice for food storage due to their lightweight, durability, and non-toxic nature. The material provides an airtight seal, which helps preserve the freshness of the food. The cans also protect the food from external factors such as moisture, light, and oxygen, which can cause spoilage. The material is also recyclable, which helps reduce waste and promotes sustainability. However, it is important to consider the limitations of the material, such as its incompatibility with acidic or alkaline foods and its inability to store hot food. Overall, aluminium cans are a versatile and convenient option for storing food, but it is important to use them appropriately to ensure the safety and quality of the food.<|end_of_text|> 
	  } 
	   } 
	   } 
	 & & \\ 
 

    \theutterance \stepcounter{utterance}  

    & & & \multicolumn{2}{p{0.3\linewidth}}{\cellcolor[rgb]{0.95,0.95,0.95}{%
	\makecell[{{p{\linewidth}}}]{% 
	  \tt {\tiny [GM$|$GM]}  
	 {'keywords:existence': {'keywords': ['waste', 'material', 'meal']}, 'length_constraints:number_sentences': {'relation': 'at least', 'num_sentences': 31}} 
	  } 
	   } 
	   } 
	 & & \\ 
 

    \theutterance \stepcounter{utterance}  

    & & & \multicolumn{2}{p{0.3\linewidth}}{\cellcolor[rgb]{0.95,0.95,0.95}{%
	\makecell[{{p{\linewidth}}}]{% 
	  \tt {\tiny [GM$|$GM]}  
	 game_result = LOSE 
	  } 
	   } 
	   } 
	 & & \\ 
 

\end{supertabular}
}

\end{document}
