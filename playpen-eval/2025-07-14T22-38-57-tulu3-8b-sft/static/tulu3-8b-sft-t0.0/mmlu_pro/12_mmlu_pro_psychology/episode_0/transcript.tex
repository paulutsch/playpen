
\documentclass{article}
\usepackage{colortbl}
\usepackage{makecell}
\usepackage{multirow}
\usepackage{supertabular}

\begin{document}

\newcounter{utterance}

\twocolumn

{ \footnotesize  \setcounter{utterance}{1}
\setlength{\tabcolsep}{0pt}
\begin{supertabular}{c@{$\;$}|p{.15\linewidth}@{}p{.15\linewidth}p{.15\linewidth}p{.15\linewidth}p{.15\linewidth}p{.15\linewidth}}

    \# & $\;$A & \multicolumn{4}{c}{Game Master} & $\;\:$B\\
    \hline 

    \theutterance \stepcounter{utterance}  

    & & \multicolumn{4}{p{0.6\linewidth}}{\cellcolor[rgb]{0.9,0.9,0.9}{%
	\makecell[{{p{\linewidth}}}]{% 
	  \tt {\tiny [A$\langle$GM]}  
	 The following are multiple choice questions (with answers) about psychology. Answer only with "The answer is (X)" where X is the correct letter choice.\\ \tt \\ \tt Question:\\ \tt Pascale is interested in the processing strategies children use to learn new information. Pascale would best be classified as what type of psychologist?\\ \tt Options:\\ \tt A. social\\ \tt B. school\\ \tt C. sociocultural\\ \tt D. forensic\\ \tt E. behaviorist\\ \tt F. health\\ \tt G. clinical\\ \tt H. cognitive\\ \tt I. psychoanalytic\\ \tt J. developmental\\ \tt The answer is (H)\\ \tt \\ \tt Question:\\ \tt According to Caplan's model of consultee-centered case consultation, the consultant is primarily interested in\\ \tt Options:\\ \tt A. identifying the causes and solutions of the client's presenting problems\\ \tt B. establishing a hierarchy of authority to enable effective decision making\\ \tt C. ensuring the consultee adheres strictly to a predetermined action plan\\ \tt D. proposing multiple alternative solutions for the consultee to choose from\\ \tt E. identifying the strengths and weaknesses of the consultee's current approach\\ \tt F. presenting a single, well-defined and unambiguous course of action for the consultant to overcome skills deficits\\ \tt G. developing a comprehensive treatment plan for the client\\ \tt H. identifying and eliminating the causes of the consultee's difficulties in handling a problem\\ \tt I. focusing on the consultant's personal growth and development\\ \tt J. focusing on the relationship between the client and the consultee\\ \tt The answer is (H)\\ \tt \\ \tt Question:\\ \tt According to the Individuals with Disabilities Education Improvement Act, which of the following must an educational agency do before it changes the educational placement of a student with a disability?\\ \tt Options:\\ \tt A. Notify the parents in writing\\ \tt B. Obtain the child's consent\\ \tt C. Obtain a court order\\ \tt D. Conduct a new evaluation of the child's disability\\ \tt E. Discuss with the child's psychologist\\ \tt F. Give the child a trial period in the new environment\\ \tt G. Obtain parental consent\\ \tt H. Notify the local education authority\\ \tt I. Arrange a meeting with all teachers and administrators\\ \tt J. Obtain school board approval\\ \tt The answer is (A)\\ \tt \\ \tt Question:\\ \tt Ani believes that her attitudes and behavior play a central role in what happens to her. Such a belief is likely to be associated with\\ \tt Options:\\ \tt A. low self-esteem.\\ \tt B. a strong id.\\ \tt C. a high level of anxiety.\\ \tt D. a strong superego.\\ \tt E. high levels of self-consciousness.\\ \tt F. an external locus of control.\\ \tt G. an inferiority complex.\\ \tt H. a low level of self-awareness.\\ \tt I. low self-efficacy.\\ \tt J. an internal locus of control.\\ \tt The answer is (J)\\ \tt \\ \tt Question:\\ \tt In terms of Hofstede’s (1980) five cultural dimensions, the United States scores at the top of the scale on:\\ \tt Options:\\ \tt A. individualism and long-term orientation.\\ \tt B. individualism and power distance.\\ \tt C. uncertainty avoidance.\\ \tt D. long-term orientation.\\ \tt E. individualism.\\ \tt F. individualism and masculinity.\\ \tt G. long-term orientation and uncertainty avoidance.\\ \tt H. power distance.\\ \tt I. power distance and masculinity.\\ \tt J. N/A\\ \tt The answer is (E)\\ \tt \\ \tt Question:\\ \tt In organizations, the level of affective commitment is likely to be least predictive of which of the following?\\ \tt Options:\\ \tt A. productivity\\ \tt B. individual creativity\\ \tt C. job satisfaction\\ \tt D. company loyalty\\ \tt E. work-life balance\\ \tt F. job motivation\\ \tt G. workplace morale\\ \tt H. turnover\\ \tt I. leadership style\\ \tt J. team cohesion\\ \tt  
	  } 
	   } 
	   } 
	 & \\ 
 

    \theutterance \stepcounter{utterance}  

    & \multicolumn{4}{p{0.6\linewidth}}{\cellcolor[rgb]{0.8,1,0.9}{%
	\makecell[{{p{\linewidth}}}]{% 
	  \tt {\tiny [A$\rangle$GM]}  
	 The answer is (H)<|end_of_text|> 
	  } 
	   } 
	   } 
	 & & \\ 
 

    \theutterance \stepcounter{utterance}  

    & & & \multicolumn{2}{p{0.3\linewidth}}{\cellcolor[rgb]{0.95,0.95,0.95}{%
	\makecell[{{p{\linewidth}}}]{% 
	  \tt {\tiny [GM$|$GM]}  
	 H 
	  } 
	   } 
	   } 
	 & & \\ 
 

    \theutterance \stepcounter{utterance}  

    & & & \multicolumn{2}{p{0.3\linewidth}}{\cellcolor[rgb]{0.95,0.95,0.95}{%
	\makecell[{{p{\linewidth}}}]{% 
	  \tt {\tiny [GM$|$GM]}  
	 A 
	  } 
	   } 
	   } 
	 & & \\ 
 

    \theutterance \stepcounter{utterance}  

    & & & \multicolumn{2}{p{0.3\linewidth}}{\cellcolor[rgb]{0.95,0.95,0.95}{%
	\makecell[{{p{\linewidth}}}]{% 
	  \tt {\tiny [GM$|$GM]}  
	 game_result = LOSE 
	  } 
	   } 
	   } 
	 & & \\ 
 

\end{supertabular}
}

\end{document}
