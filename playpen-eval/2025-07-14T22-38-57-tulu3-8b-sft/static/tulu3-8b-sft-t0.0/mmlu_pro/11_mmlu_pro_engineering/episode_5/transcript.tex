
\documentclass{article}
\usepackage{colortbl}
\usepackage{makecell}
\usepackage{multirow}
\usepackage{supertabular}

\begin{document}

\newcounter{utterance}

\twocolumn

{ \footnotesize  \setcounter{utterance}{1}
\setlength{\tabcolsep}{0pt}
\begin{supertabular}{c@{$\;$}|p{.15\linewidth}@{}p{.15\linewidth}p{.15\linewidth}p{.15\linewidth}p{.15\linewidth}p{.15\linewidth}}

    \# & $\;$A & \multicolumn{4}{c}{Game Master} & $\;\:$B\\
    \hline 

    \theutterance \stepcounter{utterance}  

    & & \multicolumn{4}{p{0.6\linewidth}}{\cellcolor[rgb]{0.9,0.9,0.9}{%
	\makecell[{{p{\linewidth}}}]{% 
	  \tt {\tiny [A$\langle$GM]}  
	 The following are multiple choice questions (with answers) about engineering. Answer only with "The answer is (X)" where X is the correct letter choice.\\ \tt \\ \tt Question:\\ \tt A point pole has a strength of 4π * 10^-4 weber. The force in newtons on a point pole of 4π * 1.5 * 10^-4 weber placed at a distance of 10 cm from it will be\\ \tt Options:\\ \tt A. 25 N.\\ \tt B. 12.5 N.\\ \tt C. 15 N.\\ \tt D. 10 N.\\ \tt E. 7.5 N.\\ \tt F. 20 N.\\ \tt G. 5 N.\\ \tt H. 3.75 N.\\ \tt I. N/A\\ \tt J. N/A\\ \tt The answer is (C)\\ \tt \\ \tt Question:\\ \tt The coil of a moving coil meter has 100 turns, is 40 mm long and 30 mm wide. The control torque is 240*10-6 N-m on full scale. If magnetic flux density is 1Wb/m2 range of meter is\\ \tt Options:\\ \tt A. 2 mA.\\ \tt B. 5 mA.\\ \tt C. 1.5 mA.\\ \tt D. 0.5 mA.\\ \tt E. 6 mA.\\ \tt F. 4 mA.\\ \tt G. 3 mA.\\ \tt H. 1 mA.\\ \tt I. 2.5 mA.\\ \tt J. 3.5 mA.\\ \tt The answer is (A)\\ \tt \\ \tt Question:\\ \tt In an SR latch built from NOR gates, which condition is not allowed\\ \tt Options:\\ \tt A. S=0, R=2\\ \tt B. S=2, R=2\\ \tt C. S=1, R=1\\ \tt D. S=1, R=-1\\ \tt E. S=1, R=2\\ \tt F. S=0, R=0\\ \tt G. S=2, R=0\\ \tt H. S=1, R=0\\ \tt I. S=2, R=1\\ \tt J. S=0, R=1\\ \tt The answer is (C)\\ \tt \\ \tt Question:\\ \tt Two long parallel conductors carry 100 A. If the conductors are separated by 20 mm, the force per meter of length of each conductor will be\\ \tt Options:\\ \tt A. 100 N.\\ \tt B. 0.01 N.\\ \tt C. 10 N.\\ \tt D. 0.5 N.\\ \tt E. 0.001 N.\\ \tt F. 5 N.\\ \tt G. 1 N.\\ \tt H. 50 N.\\ \tt I. 0.1 N.\\ \tt J. 0.0001 N.\\ \tt The answer is (I)\\ \tt \\ \tt Question:\\ \tt In a 2 pole lap winding dc machine , the resistance of one conductor is 2Ω and total number of conductors is 100. Find the total resistance\\ \tt Options:\\ \tt A. 50Ω\\ \tt B. 1Ω\\ \tt C. 25Ω\\ \tt D. 200Ω\\ \tt E. 10Ω\\ \tt F. 100Ω\\ \tt G. 500Ω\\ \tt H. 150Ω\\ \tt I. 75Ω\\ \tt J. 20Ω\\ \tt The answer is (A)\\ \tt \\ \tt Question:\\ \tt Calculate the minimum power required to cool 100 lb moles / min of air from 550°R to 500°R when the surrounding temperature is 550°R.\\ \tt Options:\\ \tt A. - 1600 Btu/min or - 37.7 hp\\ \tt B. - 2100 Btu/min or - 49.8 hp\\ \tt C. - 1500 Btu/min or - 35.7 hp\\ \tt D. - 1300 Btu/min or - 30.8 hp\\ \tt E. - 2200 Btu/min or - 52.2 hp\\ \tt F. - 1900 Btu/min or - 45.0 hp\\ \tt G. - 2000 Btu/min or - 47.4 hp\\ \tt H. - 1400 Btu/min or - 33.2 hp\\ \tt I. - 1800 Btu/min or - 42.7 hp\\ \tt J. - 1700 Btu/min or - 40.7 hp\\ \tt  
	  } 
	   } 
	   } 
	 & \\ 
 

    \theutterance \stepcounter{utterance}  

    & \multicolumn{4}{p{0.6\linewidth}}{\cellcolor[rgb]{0.8,1,0.9}{%
	\makecell[{{p{\linewidth}}}]{% 
	  \tt {\tiny [A$\rangle$GM]}  
	 The answer is (C)<|end_of_text|> 
	  } 
	   } 
	   } 
	 & & \\ 
 

    \theutterance \stepcounter{utterance}  

    & & & \multicolumn{2}{p{0.3\linewidth}}{\cellcolor[rgb]{0.95,0.95,0.95}{%
	\makecell[{{p{\linewidth}}}]{% 
	  \tt {\tiny [GM$|$GM]}  
	 C 
	  } 
	   } 
	   } 
	 & & \\ 
 

    \theutterance \stepcounter{utterance}  

    & & & \multicolumn{2}{p{0.3\linewidth}}{\cellcolor[rgb]{0.95,0.95,0.95}{%
	\makecell[{{p{\linewidth}}}]{% 
	  \tt {\tiny [GM$|$GM]}  
	 A 
	  } 
	   } 
	   } 
	 & & \\ 
 

    \theutterance \stepcounter{utterance}  

    & & & \multicolumn{2}{p{0.3\linewidth}}{\cellcolor[rgb]{0.95,0.95,0.95}{%
	\makecell[{{p{\linewidth}}}]{% 
	  \tt {\tiny [GM$|$GM]}  
	 game_result = LOSE 
	  } 
	   } 
	   } 
	 & & \\ 
 

\end{supertabular}
}

\end{document}
