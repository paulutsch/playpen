
\documentclass{article}
\usepackage{colortbl}
\usepackage{makecell}
\usepackage{multirow}
\usepackage{supertabular}

\begin{document}

\newcounter{utterance}

\twocolumn

{ \footnotesize  \setcounter{utterance}{1}
\setlength{\tabcolsep}{0pt}
\begin{supertabular}{c@{$\;$}|p{.15\linewidth}@{}p{.15\linewidth}p{.15\linewidth}p{.15\linewidth}p{.15\linewidth}p{.15\linewidth}}

    \# & $\;$A & \multicolumn{4}{c}{Game Master} & $\;\:$B\\
    \hline 

    \theutterance \stepcounter{utterance}  

    & & \multicolumn{4}{p{0.6\linewidth}}{\cellcolor[rgb]{0.9,0.9,0.9}{%
	\makecell[{{p{\linewidth}}}]{% 
	  \tt {\tiny [A$\langle$GM]}  
	 You are an expert in causal inference. The following question is not a typical commonsense query, but rather a meticulously designed question created by a professor specializing in causal inference, intended to assess the students' mastery of the course content.\\ \tt Imagine a self-contained, hypothetical world with only the following conditions, and without any unmentioned factors or causal relationships: Gender has a direct effect on smoking and lung cancer. Smoking has a direct effect on tar deposit. Tar deposit has a direct effect on lung cancer. Gender is unobserved. The overall probability of smoking is 12%. For nonsmokers, the probability of lung cancer is 58%. For smokers, the probability of lung cancer is 61%. Is lung cancer more likely than absence of lung cancer overall?\\ \tt Start your answer with 'Yes' or 'No', followed by additional reasoning or evidence to support your explanation. 
	  } 
	   } 
	   } 
	 & \\ 
 

    \theutterance \stepcounter{utterance}  

    & \multicolumn{4}{p{0.6\linewidth}}{\cellcolor[rgb]{0.8,1,0.9}{%
	\makecell[{{p{\linewidth}}}]{% 
	  \tt {\tiny [A$\rangle$GM]}  
	 No, lung cancer is not more likely than its absence overall.\\ \tt \\ \tt To address this question, we need to apply the principles of causal inference, specifically the concept of counterfactuals and potential outcomes. Given the direct effects:\\ \tt \\ \tt 1. Gender → Smoking\\ \tt 2. Smoking → Tar deposit\\ \tt 3. Tar deposit → Lung cancer\\ \tt \\ \tt We can infer the following:\\ \tt \\ \tt - The probability of smoking given gender is not directly observable, but we know the overall probability of smoking is 12%.\\ \tt - The probability of lung cancer given smoking is 61%, and given nonsmoking is 58%.\\ \tt - The effect of smoking on lung cancer is mediated by tar deposit, but we don't need to consider this intermediate variable to answer the question.\\ \tt \\ \tt To determine whether lung cancer is more likely than its absence overall, we need to calculate the overall probability of lung cancer. We can do this by considering the probability of lung cancer among smokers and nonsmokers, weighted by the probability of smoking.\\ \tt \\ \tt Let P(LC) be the overall probability of lung cancer, P(S) be the probability of smoking, P(LC|S) be the probability of lung cancer given smoking, and P(LC|NS) be the probability of lung cancer given nonsmoking.\\ \tt \\ \tt We can use the law of total probability to calculate P(LC):\\ \tt \\ \tt P(LC) = P(S) * P(LC|S) + P(NS) * P(LC|NS)\\ \tt \\ \tt where P(NS) is the probability of nonsm 
	  } 
	   } 
	   } 
	 & & \\ 
 

    \theutterance \stepcounter{utterance}  

    & & & \multicolumn{2}{p{0.3\linewidth}}{\cellcolor[rgb]{0.95,0.95,0.95}{%
	\makecell[{{p{\linewidth}}}]{% 
	  \tt {\tiny [GM$|$GM]}  
	 No 
	  } 
	   } 
	   } 
	 & & \\ 
 

    \theutterance \stepcounter{utterance}  

    & & & \multicolumn{2}{p{0.3\linewidth}}{\cellcolor[rgb]{0.95,0.95,0.95}{%
	\makecell[{{p{\linewidth}}}]{% 
	  \tt {\tiny [GM$|$GM]}  
	 yes 
	  } 
	   } 
	   } 
	 & & \\ 
 

    \theutterance \stepcounter{utterance}  

    & & & \multicolumn{2}{p{0.3\linewidth}}{\cellcolor[rgb]{0.95,0.95,0.95}{%
	\makecell[{{p{\linewidth}}}]{% 
	  \tt {\tiny [GM$|$GM]}  
	 game_result = LOSE 
	  } 
	   } 
	   } 
	 & & \\ 
 

\end{supertabular}
}

\end{document}
