
\documentclass{article}
\usepackage{colortbl}
\usepackage{makecell}
\usepackage{multirow}
\usepackage{supertabular}

\begin{document}

\newcounter{utterance}

\twocolumn

{ \footnotesize  \setcounter{utterance}{1}
\setlength{\tabcolsep}{0pt}
\begin{supertabular}{c@{$\;$}|p{.15\linewidth}@{}p{.15\linewidth}p{.15\linewidth}p{.15\linewidth}p{.15\linewidth}p{.15\linewidth}}

    \# & $\;$A & \multicolumn{4}{c}{Game Master} & $\;\:$B\\
    \hline 

    \theutterance \stepcounter{utterance}  

    & & \multicolumn{4}{p{0.6\linewidth}}{\cellcolor[rgb]{0.9,0.9,0.9}{%
	\makecell[{{p{\linewidth}}}]{% 
	  \tt {\tiny [A$\langle$GM]}  
	 You are an expert in causal inference. The following question is not a typical commonsense query, but rather a meticulously designed question created by a professor specializing in causal inference, intended to assess the students' mastery of the course content.\\ \tt Imagine a self-contained, hypothetical world with only the following conditions, and without any unmentioned factors or causal relationships: Eating citrus has a direct effect on vitmain C. Vitmain C has a direct effect on scurvy. For patients not consuming citrus, the probability of scurvy is 73%. For patients consuming citrus, the probability of scurvy is 54%. Does eating citrus positively affect scurvy through vitmain C?\\ \tt Start your answer with 'Yes' or 'No', followed by additional reasoning or evidence to support your explanation. 
	  } 
	   } 
	   } 
	 & \\ 
 

    \theutterance \stepcounter{utterance}  

    & \multicolumn{4}{p{0.6\linewidth}}{\cellcolor[rgb]{0.8,1,0.9}{%
	\makecell[{{p{\linewidth}}}]{% 
	  \tt {\tiny [A$\rangle$GM]}  
	 No\\ \tt \\ \tt To determine whether eating citrus positively affects scurvy through vitamin C, we need to apply the backdoor criterion from Pearl's do-calculus. The backdoor criterion states that a path from the treatment (eating citrus) to the outcome (scurvy) is unconfounded if all backdoor paths (paths from the treatment to the outcome through intermediate variables) are blocked.\\ \tt \\ \tt In this case, the backdoor path from eating citrus to scurvy through vitamin C is not blocked. However, to assess whether this path positively affects scurvy, we need to consider the effect of eating citrus on vitamin C and the effect of vitamin C on scurvy.\\ \tt \\ \tt Given that eating citrus has a direct effect on vitamin C, we can say that eating citrus increases vitamin C. However, the effect of vitamin C on scurvy is not direct; rather, it is through the probability of scurvy given vitamin C. Specifically, the probability of scurvy is 73% for patients not consuming citrus (i.e., vitamin C = 0) and 54% for patients consuming citrus (i.e., vitamin C > 0).\\ \tt \\ \tt To determine whether the effect of eating citrus on scurvy is positive through vitamin C, we need to compare the probability of scurvy given vitamin C = 0 to the probability of scurvy given vitamin C > 0. However, the given probabilities do not provide a direct comparison. Instead, they provide the probabilities of 
	  } 
	   } 
	   } 
	 & & \\ 
 

    \theutterance \stepcounter{utterance}  

    & & & \multicolumn{2}{p{0.3\linewidth}}{\cellcolor[rgb]{0.95,0.95,0.95}{%
	\makecell[{{p{\linewidth}}}]{% 
	  \tt {\tiny [GM$|$GM]}  
	 No 
	  } 
	   } 
	   } 
	 & & \\ 
 

    \theutterance \stepcounter{utterance}  

    & & & \multicolumn{2}{p{0.3\linewidth}}{\cellcolor[rgb]{0.95,0.95,0.95}{%
	\makecell[{{p{\linewidth}}}]{% 
	  \tt {\tiny [GM$|$GM]}  
	 no 
	  } 
	   } 
	   } 
	 & & \\ 
 

    \theutterance \stepcounter{utterance}  

    & & & \multicolumn{2}{p{0.3\linewidth}}{\cellcolor[rgb]{0.95,0.95,0.95}{%
	\makecell[{{p{\linewidth}}}]{% 
	  \tt {\tiny [GM$|$GM]}  
	 game_result = WIN 
	  } 
	   } 
	   } 
	 & & \\ 
 

\end{supertabular}
}

\end{document}
