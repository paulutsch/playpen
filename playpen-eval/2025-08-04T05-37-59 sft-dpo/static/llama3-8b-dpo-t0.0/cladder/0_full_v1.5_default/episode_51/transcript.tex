
\documentclass{article}
\usepackage{colortbl}
\usepackage{makecell}
\usepackage{multirow}
\usepackage{supertabular}

\begin{document}

\newcounter{utterance}

\twocolumn

{ \footnotesize  \setcounter{utterance}{1}
\setlength{\tabcolsep}{0pt}
\begin{supertabular}{c@{$\;$}|p{.15\linewidth}@{}p{.15\linewidth}p{.15\linewidth}p{.15\linewidth}p{.15\linewidth}p{.15\linewidth}}

    \# & $\;$A & \multicolumn{4}{c}{Game Master} & $\;\:$B\\
    \hline 

    \theutterance \stepcounter{utterance}  

    & & \multicolumn{4}{p{0.6\linewidth}}{\cellcolor[rgb]{0.9,0.9,0.9}{%
	\makecell[{{p{\linewidth}}}]{% 
	  \tt {\tiny [A$\langle$GM]}  
	 You are an expert in causal inference. The following question is not a typical commonsense query, but rather a meticulously designed question created by a professor specializing in causal inference, intended to assess the students' mastery of the course content.\\ \tt Imagine a self-contained, hypothetical world with only the following conditions, and without any unmentioned factors or causal relationships: Having a brother has a direct effect on wife and alarm clock. Wife has a direct effect on alarm clock. For people who do not have a brother and wives that don't set the alarm, the probability of ringing alarm is 88%. For people who do not have a brother and wives that set the alarm, the probability of ringing alarm is 45%. For people who have a brother and wives that don't set the alarm, the probability of ringing alarm is 58%. For people who have a brother and wives that set the alarm, the probability of ringing alarm is 13%. For people who do not have a brother, the probability of alarm set by wife is 9%. For people who have a brother, the probability of alarm set by wife is 64%. Does having a brother negatively affect alarm clock through wife?\\ \tt Start your answer with 'Yes' or 'No', followed by additional reasoning or evidence to support your explanation. 
	  } 
	   } 
	   } 
	 & \\ 
 

    \theutterance \stepcounter{utterance}  

    & \multicolumn{4}{p{0.6\linewidth}}{\cellcolor[rgb]{0.8,1,0.9}{%
	\makecell[{{p{\linewidth}}}]{% 
	  \tt {\tiny [A$\rangle$GM]}  
	 No, having a brother does not negatively affect the alarm clock through the wife. \\ \tt \\ \tt To assess the causal relationship, we need to consider the effect of having a brother on the probability of the wife setting the alarm, and then the effect of the wife setting the alarm on the probability of the alarm ringing. \\ \tt \\ \tt First, we can calculate the probability of the wife setting the alarm given that the person does not have a brother (P(set alarm|no brother)) and the probability of the wife setting the alarm given that the person has a brother (P(set alarm|brother)). We are given that P(set alarm|no brother) = 9% and P(set alarm|brother) = 64%. \\ \tt \\ \tt Next, we can calculate the probability of the alarm ringing given that the wife does not set the alarm (P(ring|not set)) and the probability of the alarm ringing given that the wife sets the alarm (P(ring|set)). We are given the following probabilities: \\ \tt - P(ring|not set) = 88% for people who do not have a brother and wives that don't set the alarm\\ \tt - P(ring|set) = 45% for people who do not have a brother and wives that set the alarm\\ \tt - P(ring|not set) = 58% for people who have a brother and wives that don't set the alarm\\ \tt - P(ring|set) = 13% for people who have a brother 
	  } 
	   } 
	   } 
	 & & \\ 
 

    \theutterance \stepcounter{utterance}  

    & & & \multicolumn{2}{p{0.3\linewidth}}{\cellcolor[rgb]{0.95,0.95,0.95}{%
	\makecell[{{p{\linewidth}}}]{% 
	  \tt {\tiny [GM$|$GM]}  
	 No 
	  } 
	   } 
	   } 
	 & & \\ 
 

    \theutterance \stepcounter{utterance}  

    & & & \multicolumn{2}{p{0.3\linewidth}}{\cellcolor[rgb]{0.95,0.95,0.95}{%
	\makecell[{{p{\linewidth}}}]{% 
	  \tt {\tiny [GM$|$GM]}  
	 yes 
	  } 
	   } 
	   } 
	 & & \\ 
 

    \theutterance \stepcounter{utterance}  

    & & & \multicolumn{2}{p{0.3\linewidth}}{\cellcolor[rgb]{0.95,0.95,0.95}{%
	\makecell[{{p{\linewidth}}}]{% 
	  \tt {\tiny [GM$|$GM]}  
	 game_result = LOSE 
	  } 
	   } 
	   } 
	 & & \\ 
 

\end{supertabular}
}

\end{document}
