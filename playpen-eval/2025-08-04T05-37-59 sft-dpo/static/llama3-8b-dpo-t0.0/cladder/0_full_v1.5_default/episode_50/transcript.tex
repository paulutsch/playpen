
\documentclass{article}
\usepackage{colortbl}
\usepackage{makecell}
\usepackage{multirow}
\usepackage{supertabular}

\begin{document}

\newcounter{utterance}

\twocolumn

{ \footnotesize  \setcounter{utterance}{1}
\setlength{\tabcolsep}{0pt}
\begin{supertabular}{c@{$\;$}|p{.15\linewidth}@{}p{.15\linewidth}p{.15\linewidth}p{.15\linewidth}p{.15\linewidth}p{.15\linewidth}}

    \# & $\;$A & \multicolumn{4}{c}{Game Master} & $\;\:$B\\
    \hline 

    \theutterance \stepcounter{utterance}  

    & & \multicolumn{4}{p{0.6\linewidth}}{\cellcolor[rgb]{0.9,0.9,0.9}{%
	\makecell[{{p{\linewidth}}}]{% 
	  \tt {\tiny [A$\langle$GM]}  
	 You are an expert in causal inference. The following question is not a typical commonsense query, but rather a meticulously designed question created by a professor specializing in causal inference, intended to assess the students' mastery of the course content.\\ \tt Imagine a self-contained, hypothetical world with only the following conditions, and without any unmentioned factors or causal relationships: Having a brother has a direct effect on wife and alarm clock. Wife has a direct effect on alarm clock. The overall probability of having a brother is 1%. The probability of not having a brother and ringing alarm is 64%. The probability of having a brother and ringing alarm is 0%. Is the chance of ringing alarm larger when observing having a brother?\\ \tt Start your answer with 'Yes' or 'No', followed by additional reasoning or evidence to support your explanation. 
	  } 
	   } 
	   } 
	 & \\ 
 

    \theutterance \stepcounter{utterance}  

    & \multicolumn{4}{p{0.6\linewidth}}{\cellcolor[rgb]{0.8,1,0.9}{%
	\makecell[{{p{\linewidth}}}]{% 
	  \tt {\tiny [A$\rangle$GM]}  
	 No.\\ \tt \\ \tt To assess whether the chance of ringing alarm is larger when observing having a brother, we need to consider the causal relationships and probabilities given in the problem. \\ \tt \\ \tt First, we know that having a brother has a direct effect on the wife and alarm clock. This implies that the presence of a brother can influence the wife's behavior, which in turn affects the alarm clock. However, we also know that the probability of having a brother and ringing alarm is 0%. This suggests that having a brother does not directly cause the alarm to ring.\\ \tt \\ \tt Next, we are given that the probability of not having a brother and ringing alarm is 64%. This is a crucial piece of information, as it provides a baseline probability of the alarm ringing in the absence of a brother.\\ \tt \\ \tt To determine whether the presence of a brother increases the probability of the alarm ringing, we need to consider the probability of having a brother and not ringing alarm, as well as the probability of not having a brother and ringing alarm. Since the probability of having a brother and ringing alarm is 0%, the probability of having a brother and not ringing alarm is 1% (the overall probability of having a brother).\\ \tt \\ \tt Using the law of total probability, we can calculate the probability of ringing alarm given having a brother (P(Ring|Brother)) as follows:\\ \tt \\ \tt P(Ring|Brother) = P(Ring and Brother) / P(Brother)\\ \tt = 0% / 1%\\ \tt = 0%\\ \tt \\ \tt Similarly, 
	  } 
	   } 
	   } 
	 & & \\ 
 

    \theutterance \stepcounter{utterance}  

    & & & \multicolumn{2}{p{0.3\linewidth}}{\cellcolor[rgb]{0.95,0.95,0.95}{%
	\makecell[{{p{\linewidth}}}]{% 
	  \tt {\tiny [GM$|$GM]}  
	 No 
	  } 
	   } 
	   } 
	 & & \\ 
 

    \theutterance \stepcounter{utterance}  

    & & & \multicolumn{2}{p{0.3\linewidth}}{\cellcolor[rgb]{0.95,0.95,0.95}{%
	\makecell[{{p{\linewidth}}}]{% 
	  \tt {\tiny [GM$|$GM]}  
	 no 
	  } 
	   } 
	   } 
	 & & \\ 
 

    \theutterance \stepcounter{utterance}  

    & & & \multicolumn{2}{p{0.3\linewidth}}{\cellcolor[rgb]{0.95,0.95,0.95}{%
	\makecell[{{p{\linewidth}}}]{% 
	  \tt {\tiny [GM$|$GM]}  
	 game_result = WIN 
	  } 
	   } 
	   } 
	 & & \\ 
 

\end{supertabular}
}

\end{document}
