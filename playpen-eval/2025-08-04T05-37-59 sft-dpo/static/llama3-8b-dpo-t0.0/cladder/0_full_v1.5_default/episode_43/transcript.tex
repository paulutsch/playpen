
\documentclass{article}
\usepackage{colortbl}
\usepackage{makecell}
\usepackage{multirow}
\usepackage{supertabular}

\begin{document}

\newcounter{utterance}

\twocolumn

{ \footnotesize  \setcounter{utterance}{1}
\setlength{\tabcolsep}{0pt}
\begin{supertabular}{c@{$\;$}|p{.15\linewidth}@{}p{.15\linewidth}p{.15\linewidth}p{.15\linewidth}p{.15\linewidth}p{.15\linewidth}}

    \# & $\;$A & \multicolumn{4}{c}{Game Master} & $\;\:$B\\
    \hline 

    \theutterance \stepcounter{utterance}  

    & & \multicolumn{4}{p{0.6\linewidth}}{\cellcolor[rgb]{0.9,0.9,0.9}{%
	\makecell[{{p{\linewidth}}}]{% 
	  \tt {\tiny [A$\langle$GM]}  
	 You are an expert in causal inference. The following question is not a typical commonsense query, but rather a meticulously designed question created by a professor specializing in causal inference, intended to assess the students' mastery of the course content.\\ \tt Imagine a self-contained, hypothetical world with only the following conditions, and without any unmentioned factors or causal relationships: Ability to speak english has a direct effect on diabetes and lifespan. Smoking has a direct effect on diabetes and lifespan. Diabetes has a direct effect on lifespan. Smoking is unobserved. For people who do not speak english, the probability of long lifespan is 77%. For people who speak english, the probability of long lifespan is 45%. Will speaking english decrease the chance of long lifespan?\\ \tt Start your answer with 'Yes' or 'No', followed by additional reasoning or evidence to support your explanation. 
	  } 
	   } 
	   } 
	 & \\ 
 

    \theutterance \stepcounter{utterance}  

    & \multicolumn{4}{p{0.6\linewidth}}{\cellcolor[rgb]{0.8,1,0.9}{%
	\makecell[{{p{\linewidth}}}]{% 
	  \tt {\tiny [A$\rangle$GM]}  
	 No.\\ \tt \\ \tt To address this question, we need to apply the principles of causal inference, specifically the concept of confounding variables and the problem of selection bias. \\ \tt \\ \tt Given the conditions, we have the following causal relationships:\\ \tt \\ \tt 1. Ability to speak English (A) → Diabetes (D)\\ \tt 2. Smoking (S) → Diabetes (D)\\ \tt 3. Diabetes (D) → Lifespan (L)\\ \tt \\ \tt We are also given the following probabilities:\\ \tt \\ \tt - P(L|A=0) = 0.77 (probability of long lifespan for those who do not speak English)\\ \tt - P(L|A=1) = 0.45 (probability of long lifespan for those who speak English)\\ \tt \\ \tt The question asks whether speaking English (A=1) decreases the chance of long lifespan. To answer this, we need to consider the potential confounding effect of smoking (S), which is unobserved.\\ \tt \\ \tt Using the backdoor criterion, we can identify a path from A to L that is not confounded by S:\\ \tt \\ \tt A → D → L\\ \tt \\ \tt This path suggests that the effect of A on L is confounded by D. However, we can also consider the following path:\\ \tt \\ \tt A → D → S → L\\ \tt \\ \tt This path is confounded by S, but we can use the given probabilities to estimate the effect of A on L.\\ \tt \\ \tt Let's assume that the probability of smoking (S) is the same for both groups (A=0 and A=1). We can then use the law 
	  } 
	   } 
	   } 
	 & & \\ 
 

    \theutterance \stepcounter{utterance}  

    & & & \multicolumn{2}{p{0.3\linewidth}}{\cellcolor[rgb]{0.95,0.95,0.95}{%
	\makecell[{{p{\linewidth}}}]{% 
	  \tt {\tiny [GM$|$GM]}  
	 No 
	  } 
	   } 
	   } 
	 & & \\ 
 

    \theutterance \stepcounter{utterance}  

    & & & \multicolumn{2}{p{0.3\linewidth}}{\cellcolor[rgb]{0.95,0.95,0.95}{%
	\makecell[{{p{\linewidth}}}]{% 
	  \tt {\tiny [GM$|$GM]}  
	 yes 
	  } 
	   } 
	   } 
	 & & \\ 
 

    \theutterance \stepcounter{utterance}  

    & & & \multicolumn{2}{p{0.3\linewidth}}{\cellcolor[rgb]{0.95,0.95,0.95}{%
	\makecell[{{p{\linewidth}}}]{% 
	  \tt {\tiny [GM$|$GM]}  
	 game_result = LOSE 
	  } 
	   } 
	   } 
	 & & \\ 
 

\end{supertabular}
}

\end{document}
