
\documentclass{article}
\usepackage{colortbl}
\usepackage{makecell}
\usepackage{multirow}
\usepackage{supertabular}

\begin{document}

\newcounter{utterance}

\twocolumn

{ \footnotesize  \setcounter{utterance}{1}
\setlength{\tabcolsep}{0pt}
\begin{supertabular}{c@{$\;$}|p{.15\linewidth}@{}p{.15\linewidth}p{.15\linewidth}p{.15\linewidth}p{.15\linewidth}p{.15\linewidth}}

    \# & $\;$A & \multicolumn{4}{c}{Game Master} & $\;\:$B\\
    \hline 

    \theutterance \stepcounter{utterance}  

    & & \multicolumn{4}{p{0.6\linewidth}}{\cellcolor[rgb]{0.9,0.9,0.9}{%
	\makecell[{{p{\linewidth}}}]{% 
	  \tt {\tiny [A$\langle$GM]}  
	 You are an expert in causal inference. The following question is not a typical commonsense query, but rather a meticulously designed question created by a professor specializing in causal inference, intended to assess the students' mastery of the course content.\\ \tt Imagine a self-contained, hypothetical world with only the following conditions, and without any unmentioned factors or causal relationships: Gender has a direct effect on department competitiveness and admission status. Department competitiveness has a direct effect on admission status. The overall probability of male gender is 54%. For individuals who are not male, the probability of admission acceptance is 49%. For individuals who are male, the probability of admission acceptance is 24%. Is admission acceptance less likely than admission rejection overall?\\ \tt Start your answer with 'Yes' or 'No', followed by additional reasoning or evidence to support your explanation. 
	  } 
	   } 
	   } 
	 & \\ 
 

    \theutterance \stepcounter{utterance}  

    & \multicolumn{4}{p{0.6\linewidth}}{\cellcolor[rgb]{0.8,1,0.9}{%
	\makecell[{{p{\linewidth}}}]{% 
	  \tt {\tiny [A$\rangle$GM]}  
	 No, admission acceptance is not less likely than admission rejection overall.\\ \tt \\ \tt To understand this, let's break down the given information:\\ \tt \\ \tt 1. The probability of male gender is 54%.\\ \tt 2. For individuals who are not male (i.e., female), the probability of admission acceptance is 49%.\\ \tt 3. For individuals who are male, the probability of admission acceptance is 24%.\\ \tt \\ \tt We can calculate the overall probability of admission acceptance as follows:\\ \tt \\ \tt - Probability of being male and accepted: 0.54 * 0.24 = 0.1296\\ \tt - Probability of being female and accepted: 0.46 * 0.49 = 0.2264\\ \tt - Overall probability of admission acceptance: 0.1296 + 0.2264 = 0.356\\ \tt \\ \tt Now, let's calculate the overall probability of admission rejection:\\ \tt \\ \tt - Probability of being male and rejected: 0.54 * (1 - 0.24) = 0.54 * 0.76 = 0.4104\\ \tt - Probability of being female and rejected: 0.46 * (1 - 0.49) = 0.46 * 0.51 = 0.2346\\ \tt - Overall probability of admission rejection: 0.4104 + 0.2346 = 0.6449 (approximately)\\ \tt \\ \tt Since the overall probability of admission rejection (0.6449) is greater than the overall probability of admission acceptance (0. 
	  } 
	   } 
	   } 
	 & & \\ 
 

    \theutterance \stepcounter{utterance}  

    & & & \multicolumn{2}{p{0.3\linewidth}}{\cellcolor[rgb]{0.95,0.95,0.95}{%
	\makecell[{{p{\linewidth}}}]{% 
	  \tt {\tiny [GM$|$GM]}  
	 No 
	  } 
	   } 
	   } 
	 & & \\ 
 

    \theutterance \stepcounter{utterance}  

    & & & \multicolumn{2}{p{0.3\linewidth}}{\cellcolor[rgb]{0.95,0.95,0.95}{%
	\makecell[{{p{\linewidth}}}]{% 
	  \tt {\tiny [GM$|$GM]}  
	 yes 
	  } 
	   } 
	   } 
	 & & \\ 
 

    \theutterance \stepcounter{utterance}  

    & & & \multicolumn{2}{p{0.3\linewidth}}{\cellcolor[rgb]{0.95,0.95,0.95}{%
	\makecell[{{p{\linewidth}}}]{% 
	  \tt {\tiny [GM$|$GM]}  
	 game_result = LOSE 
	  } 
	   } 
	   } 
	 & & \\ 
 

\end{supertabular}
}

\end{document}
