
\documentclass{article}
\usepackage{colortbl}
\usepackage{makecell}
\usepackage{multirow}
\usepackage{supertabular}

\begin{document}

\newcounter{utterance}

\twocolumn

{ \footnotesize  \setcounter{utterance}{1}
\setlength{\tabcolsep}{0pt}
\begin{supertabular}{c@{$\;$}|p{.15\linewidth}@{}p{.15\linewidth}p{.15\linewidth}p{.15\linewidth}p{.15\linewidth}p{.15\linewidth}}

    \# & $\;$A & \multicolumn{4}{c}{Game Master} & $\;\:$B\\
    \hline 

    \theutterance \stepcounter{utterance}  

    & & \multicolumn{4}{p{0.6\linewidth}}{\cellcolor[rgb]{0.9,0.9,0.9}{%
	\makecell[{{p{\linewidth}}}]{% 
	  \tt {\tiny [A$\langle$GM]}  
	 You are an expert in causal inference. The following question is not a typical commonsense query, but rather a meticulously designed question created by a professor specializing in causal inference, intended to assess the students' mastery of the course content.\\ \tt Imagine a self-contained, hypothetical world with only the following conditions, and without any unmentioned factors or causal relationships: Ability to speak english has a direct effect on the forest. The smoker has a direct effect on the forest. The overall probability of speaking english is 93%. For people who do not speak english, the probability of the forest on fire is 79%. For people who speak english, the probability of the forest on fire is 65%. Is the forest on fire more likely than the forest not on fire overall?\\ \tt Start your answer with 'Yes' or 'No', followed by additional reasoning or evidence to support your explanation. 
	  } 
	   } 
	   } 
	 & \\ 
 

    \theutterance \stepcounter{utterance}  

    & \multicolumn{4}{p{0.6\linewidth}}{\cellcolor[rgb]{0.8,1,0.9}{%
	\makecell[{{p{\linewidth}}}]{% 
	  \tt {\tiny [A$\rangle$GM]}  
	 No, the forest is not more likely to be on fire overall.\\ \tt \\ \tt To answer this question, we need to apply the principles of causal inference, specifically the concept of confounding variables and the potential outcomes framework. Let's break down the given information:\\ \tt \\ \tt 1. Ability to speak English has a direct effect on the forest.\\ \tt 2. Smoking has a direct effect on the forest.\\ \tt 3. The overall probability of speaking English is 93%.\\ \tt 4. For people who do not speak English, the probability of the forest on fire is 79%.\\ \tt 5. For people who speak English, the probability of the forest on fire is 65%.\\ \tt \\ \tt We can use the potential outcomes framework to analyze this problem. Let's define two potential outcomes for the forest: on fire (Y=1) and not on fire (Y=0). We also have two treatment variables: speaking English (T=1) and not speaking English (T=0), and a confounding variable: smoking (S).\\ \tt \\ \tt We are interested in estimating the overall probability of the forest being on fire (P(Y=1)). To do this, we need to consider the joint distribution of the potential outcomes and the treatment variables.\\ \tt \\ \tt Using the law of total probability, we can write:\\ \tt \\ \tt P(Y=1) = P(Y=1|T=1) \* P(T=1) + P(Y=1|T=0) \* P(T=0)\\ \tt \\ \tt We know that P(T=1) = 0. 
	  } 
	   } 
	   } 
	 & & \\ 
 

    \theutterance \stepcounter{utterance}  

    & & & \multicolumn{2}{p{0.3\linewidth}}{\cellcolor[rgb]{0.95,0.95,0.95}{%
	\makecell[{{p{\linewidth}}}]{% 
	  \tt {\tiny [GM$|$GM]}  
	 No 
	  } 
	   } 
	   } 
	 & & \\ 
 

    \theutterance \stepcounter{utterance}  

    & & & \multicolumn{2}{p{0.3\linewidth}}{\cellcolor[rgb]{0.95,0.95,0.95}{%
	\makecell[{{p{\linewidth}}}]{% 
	  \tt {\tiny [GM$|$GM]}  
	 yes 
	  } 
	   } 
	   } 
	 & & \\ 
 

    \theutterance \stepcounter{utterance}  

    & & & \multicolumn{2}{p{0.3\linewidth}}{\cellcolor[rgb]{0.95,0.95,0.95}{%
	\makecell[{{p{\linewidth}}}]{% 
	  \tt {\tiny [GM$|$GM]}  
	 game_result = LOSE 
	  } 
	   } 
	   } 
	 & & \\ 
 

\end{supertabular}
}

\end{document}
