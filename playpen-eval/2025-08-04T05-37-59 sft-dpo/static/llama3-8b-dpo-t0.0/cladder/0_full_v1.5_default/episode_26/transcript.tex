
\documentclass{article}
\usepackage{colortbl}
\usepackage{makecell}
\usepackage{multirow}
\usepackage{supertabular}

\begin{document}

\newcounter{utterance}

\twocolumn

{ \footnotesize  \setcounter{utterance}{1}
\setlength{\tabcolsep}{0pt}
\begin{supertabular}{c@{$\;$}|p{.15\linewidth}@{}p{.15\linewidth}p{.15\linewidth}p{.15\linewidth}p{.15\linewidth}p{.15\linewidth}}

    \# & $\;$A & \multicolumn{4}{c}{Game Master} & $\;\:$B\\
    \hline 

    \theutterance \stepcounter{utterance}  

    & & \multicolumn{4}{p{0.6\linewidth}}{\cellcolor[rgb]{0.9,0.9,0.9}{%
	\makecell[{{p{\linewidth}}}]{% 
	  \tt {\tiny [A$\langle$GM]}  
	 You are an expert in causal inference. The following question is not a typical commonsense query, but rather a meticulously designed question created by a professor specializing in causal inference, intended to assess the students' mastery of the course content.\\ \tt Imagine a self-contained, hypothetical world with only the following conditions, and without any unmentioned factors or causal relationships: Gender has a direct effect on department competitiveness and admission status. Residency status has a direct effect on department competitiveness and admission status. Department competitiveness has a direct effect on admission status. Residency status is unobserved. The overall probability of male gender is 13%. For individuals who are not male, the probability of admission acceptance is 61%. For individuals who are male, the probability of admission acceptance is 65%. Is admission acceptance less likely than admission rejection overall?\\ \tt Start your answer with 'Yes' or 'No', followed by additional reasoning or evidence to support your explanation. 
	  } 
	   } 
	   } 
	 & \\ 
 

    \theutterance \stepcounter{utterance}  

    & \multicolumn{4}{p{0.6\linewidth}}{\cellcolor[rgb]{0.8,1,0.9}{%
	\makecell[{{p{\linewidth}}}]{% 
	  \tt {\tiny [A$\rangle$GM]}  
	 No, admission acceptance is not less likely than admission rejection overall.\\ \tt \\ \tt To understand this, we need to apply the principles of causal inference, specifically the concept of causal graphs and the do-calculus. Given the conditions:\\ \tt \\ \tt 1. Gender has a direct effect on department competitiveness and admission status.\\ \tt 2. Residency status has a direct effect on department competitiveness and admission status.\\ \tt 3. Department competitiveness has a direct effect on admission status.\\ \tt 4. Residency status is unobserved.\\ \tt \\ \tt We can represent these relationships using a causal graph:\\ \tt \\ \tt - Gender → Department Competitiveness\\ \tt - Residency Status → Department Competitiveness\\ \tt - Department Competitiveness → Admission Status\\ \tt \\ \tt Since Residency Status is unobserved, we cannot directly estimate its effect on Admission Status. However, we can use the do-calculus to identify the total effect of Gender on Admission Status.\\ \tt \\ \tt Let's denote:\\ \tt \\ \tt - P(A=1|G=1) as the probability of admission acceptance given male gender (65%)\\ \tt - P(A=1|G=0) as the probability of admission acceptance given non-male gender (61%)\\ \tt - P(G=1) as the probability of male gender (13%)\\ \tt \\ \tt Using the law of total probability, we can express the overall probability of admission acceptance as:\\ \tt \\ \tt P(A=1) = P(G=1) * P(A=1|G=1) + P(G=0) * P(A=1|G=0)\\ \tt \\ \tt where P(G=0) = 1 
	  } 
	   } 
	   } 
	 & & \\ 
 

    \theutterance \stepcounter{utterance}  

    & & & \multicolumn{2}{p{0.3\linewidth}}{\cellcolor[rgb]{0.95,0.95,0.95}{%
	\makecell[{{p{\linewidth}}}]{% 
	  \tt {\tiny [GM$|$GM]}  
	 No 
	  } 
	   } 
	   } 
	 & & \\ 
 

    \theutterance \stepcounter{utterance}  

    & & & \multicolumn{2}{p{0.3\linewidth}}{\cellcolor[rgb]{0.95,0.95,0.95}{%
	\makecell[{{p{\linewidth}}}]{% 
	  \tt {\tiny [GM$|$GM]}  
	 no 
	  } 
	   } 
	   } 
	 & & \\ 
 

    \theutterance \stepcounter{utterance}  

    & & & \multicolumn{2}{p{0.3\linewidth}}{\cellcolor[rgb]{0.95,0.95,0.95}{%
	\makecell[{{p{\linewidth}}}]{% 
	  \tt {\tiny [GM$|$GM]}  
	 game_result = WIN 
	  } 
	   } 
	   } 
	 & & \\ 
 

\end{supertabular}
}

\end{document}
