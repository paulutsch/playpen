
\documentclass{article}
\usepackage{colortbl}
\usepackage{makecell}
\usepackage{multirow}
\usepackage{supertabular}

\begin{document}

\newcounter{utterance}

\twocolumn

{ \footnotesize  \setcounter{utterance}{1}
\setlength{\tabcolsep}{0pt}
\begin{supertabular}{c@{$\;$}|p{.15\linewidth}@{}p{.15\linewidth}p{.15\linewidth}p{.15\linewidth}p{.15\linewidth}p{.15\linewidth}}

    \# & $\;$A & \multicolumn{4}{c}{Game Master} & $\;\:$B\\
    \hline 

    \theutterance \stepcounter{utterance}  

    & & \multicolumn{4}{p{0.6\linewidth}}{\cellcolor[rgb]{0.9,0.9,0.9}{%
	\makecell[{{p{\linewidth}}}]{% 
	  \tt {\tiny [A$\langle$GM]}  
	 You are an expert in causal inference. The following question is not a typical commonsense query, but rather a meticulously designed question created by a professor specializing in causal inference, intended to assess the students' mastery of the course content.\\ \tt Imagine a self-contained, hypothetical world with only the following conditions, and without any unmentioned factors or causal relationships: Unobserved confounders has a direct effect on education level and foot size. Proximity to a college has a direct effect on education level. Education level has a direct effect on foot size. Unobserved confounders is unobserved. The overall probability of college degree or higher is 47%. The probability of high school degree or lower and large feet is 32%. The probability of college degree or higher and large feet is 10%. Is the chance of large feet smaller when observing college degree or higher?\\ \tt Start your answer with 'Yes' or 'No', followed by additional reasoning or evidence to support your explanation. 
	  } 
	   } 
	   } 
	 & \\ 
 

    \theutterance \stepcounter{utterance}  

    & \multicolumn{4}{p{0.6\linewidth}}{\cellcolor[rgb]{0.8,1,0.9}{%
	\makecell[{{p{\linewidth}}}]{% 
	  \tt {\tiny [A$\rangle$GM]}  
	 No\\ \tt \\ \tt To answer this question, we need to apply the principles of causal inference, specifically the concept of confounding and the use of conditional probabilities. \\ \tt \\ \tt Let's denote the variables as follows:\\ \tt - U: Unobserved confounders\\ \tt - E: Education level (college degree or higher = 1, high school degree or lower = 0)\\ \tt - F: Foot size (large = 1, small = 0)\\ \tt \\ \tt Given the conditions:\\ \tt 1. U → E (U has a direct effect on E)\\ \tt 2. U → F (U has a direct effect on F)\\ \tt 3. E → F (E has a direct effect on F)\\ \tt 4. P(E=1) = 0.47\\ \tt 5. P(F=1, E=0) = 0.32\\ \tt 6. P(F=1, E=1) = 0.10\\ \tt \\ \tt We want to find P(F=1|E=1) compared to P(F=1). \\ \tt \\ \tt First, we can calculate the probability of large feet (F=1) using the law of total probability:\\ \tt P(F=1) = P(F=1, E=1) + P(F=1, E=0)\\ \tt = 0.10 + P(F=1|E=0) * P(E=0)\\ \tt Since P(E=0) = 1 - P(E=1) = 1 - 0.47 = 0.53\\ \tt P 
	  } 
	   } 
	   } 
	 & & \\ 
 

    \theutterance \stepcounter{utterance}  

    & & & \multicolumn{2}{p{0.3\linewidth}}{\cellcolor[rgb]{0.95,0.95,0.95}{%
	\makecell[{{p{\linewidth}}}]{% 
	  \tt {\tiny [GM$|$GM]}  
	 No 
	  } 
	   } 
	   } 
	 & & \\ 
 

    \theutterance \stepcounter{utterance}  

    & & & \multicolumn{2}{p{0.3\linewidth}}{\cellcolor[rgb]{0.95,0.95,0.95}{%
	\makecell[{{p{\linewidth}}}]{% 
	  \tt {\tiny [GM$|$GM]}  
	 yes 
	  } 
	   } 
	   } 
	 & & \\ 
 

    \theutterance \stepcounter{utterance}  

    & & & \multicolumn{2}{p{0.3\linewidth}}{\cellcolor[rgb]{0.95,0.95,0.95}{%
	\makecell[{{p{\linewidth}}}]{% 
	  \tt {\tiny [GM$|$GM]}  
	 game_result = LOSE 
	  } 
	   } 
	   } 
	 & & \\ 
 

\end{supertabular}
}

\end{document}
