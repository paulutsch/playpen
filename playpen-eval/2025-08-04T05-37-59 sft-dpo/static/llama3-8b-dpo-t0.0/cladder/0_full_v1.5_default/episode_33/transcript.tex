
\documentclass{article}
\usepackage{colortbl}
\usepackage{makecell}
\usepackage{multirow}
\usepackage{supertabular}

\begin{document}

\newcounter{utterance}

\twocolumn

{ \footnotesize  \setcounter{utterance}{1}
\setlength{\tabcolsep}{0pt}
\begin{supertabular}{c@{$\;$}|p{.15\linewidth}@{}p{.15\linewidth}p{.15\linewidth}p{.15\linewidth}p{.15\linewidth}p{.15\linewidth}}

    \# & $\;$A & \multicolumn{4}{c}{Game Master} & $\;\:$B\\
    \hline 

    \theutterance \stepcounter{utterance}  

    & & \multicolumn{4}{p{0.6\linewidth}}{\cellcolor[rgb]{0.9,0.9,0.9}{%
	\makecell[{{p{\linewidth}}}]{% 
	  \tt {\tiny [A$\langle$GM]}  
	 You are an expert in causal inference. The following question is not a typical commonsense query, but rather a meticulously designed question created by a professor specializing in causal inference, intended to assess the students' mastery of the course content.\\ \tt Imagine a self-contained, hypothetical world with only the following conditions, and without any unmentioned factors or causal relationships: Poverty has a direct effect on water quality and cholera. Water company has a direct effect on water quality. Water quality has a direct effect on cholera. Poverty is unobserved. We know that high poverty or global water company causes clean water. high poverty or clean water causes cholera contraction. We observed the person is served by a global water company and the region has low poverty. Would the person gets cholera if polluted water instead of clean water?\\ \tt Start your answer with 'Yes' or 'No', followed by additional reasoning or evidence to support your explanation. 
	  } 
	   } 
	   } 
	 & \\ 
 

    \theutterance \stepcounter{utterance}  

    & \multicolumn{4}{p{0.6\linewidth}}{\cellcolor[rgb]{0.8,1,0.9}{%
	\makecell[{{p{\linewidth}}}]{% 
	  \tt {\tiny [A$\rangle$GM]}  
	 No\\ \tt \\ \tt To answer this question, we need to apply the principles of causal inference, specifically the concept of counterfactuals and the backdoor criterion. \\ \tt \\ \tt Given the conditions:\\ \tt \\ \tt 1. Poverty has a direct effect on water quality and cholera.\\ \tt 2. Water company has a direct effect on water quality.\\ \tt 3. Water quality has a direct effect on cholera.\\ \tt 4. Poverty is unobserved.\\ \tt 5. High poverty or global water company causes clean water.\\ \tt 6. High poverty or clean water causes cholera contraction.\\ \tt \\ \tt We are given that the person is served by a global water company and the region has low poverty. \\ \tt \\ \tt To determine the effect of polluted water on the person, we need to consider the counterfactual scenario where the person is exposed to polluted water instead of clean water. \\ \tt \\ \tt Since the person is served by a global water company, we know that this is a sufficient condition for clean water (condition 5). Therefore, the person would have clean water regardless of the region's poverty level.\\ \tt \\ \tt Now, let's consider the effect of clean water on cholera contraction (condition 6). We know that high poverty or clean water causes cholera contraction. However, since the region has low poverty, the effect of clean water on cholera contraction is the relevant one. \\ \tt \\ \tt Given that the person has clean water, we can conclude that the person would not get cholera due to the direct effect of clean water on cholera (condition 3).\\ \tt \\ \tt However, we need to 
	  } 
	   } 
	   } 
	 & & \\ 
 

    \theutterance \stepcounter{utterance}  

    & & & \multicolumn{2}{p{0.3\linewidth}}{\cellcolor[rgb]{0.95,0.95,0.95}{%
	\makecell[{{p{\linewidth}}}]{% 
	  \tt {\tiny [GM$|$GM]}  
	 No 
	  } 
	   } 
	   } 
	 & & \\ 
 

    \theutterance \stepcounter{utterance}  

    & & & \multicolumn{2}{p{0.3\linewidth}}{\cellcolor[rgb]{0.95,0.95,0.95}{%
	\makecell[{{p{\linewidth}}}]{% 
	  \tt {\tiny [GM$|$GM]}  
	 no 
	  } 
	   } 
	   } 
	 & & \\ 
 

    \theutterance \stepcounter{utterance}  

    & & & \multicolumn{2}{p{0.3\linewidth}}{\cellcolor[rgb]{0.95,0.95,0.95}{%
	\makecell[{{p{\linewidth}}}]{% 
	  \tt {\tiny [GM$|$GM]}  
	 game_result = WIN 
	  } 
	   } 
	   } 
	 & & \\ 
 

\end{supertabular}
}

\end{document}
