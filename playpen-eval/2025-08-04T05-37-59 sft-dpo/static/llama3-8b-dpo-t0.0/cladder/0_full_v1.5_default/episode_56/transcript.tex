
\documentclass{article}
\usepackage{colortbl}
\usepackage{makecell}
\usepackage{multirow}
\usepackage{supertabular}

\begin{document}

\newcounter{utterance}

\twocolumn

{ \footnotesize  \setcounter{utterance}{1}
\setlength{\tabcolsep}{0pt}
\begin{supertabular}{c@{$\;$}|p{.15\linewidth}@{}p{.15\linewidth}p{.15\linewidth}p{.15\linewidth}p{.15\linewidth}p{.15\linewidth}}

    \# & $\;$A & \multicolumn{4}{c}{Game Master} & $\;\:$B\\
    \hline 

    \theutterance \stepcounter{utterance}  

    & & \multicolumn{4}{p{0.6\linewidth}}{\cellcolor[rgb]{0.9,0.9,0.9}{%
	\makecell[{{p{\linewidth}}}]{% 
	  \tt {\tiny [A$\langle$GM]}  
	 You are an expert in causal inference. The following question is not a typical commonsense query, but rather a meticulously designed question created by a professor specializing in causal inference, intended to assess the students' mastery of the course content.\\ \tt Imagine a self-contained, hypothetical world with only the following conditions, and without any unmentioned factors or causal relationships: Tanning salon treatment has a direct effect on black hair. Going to the beach has a direct effect on black hair. The overall probability of tanning salon treatment is 99%. For people not using tanning salon treatments, the probability of black hair is 66%. For people who went to tanning salons, the probability of black hair is 24%. Is black hair less likely than blond hair overall?\\ \tt Start your answer with 'Yes' or 'No', followed by additional reasoning or evidence to support your explanation. 
	  } 
	   } 
	   } 
	 & \\ 
 

    \theutterance \stepcounter{utterance}  

    & \multicolumn{4}{p{0.6\linewidth}}{\cellcolor[rgb]{0.8,1,0.9}{%
	\makecell[{{p{\linewidth}}}]{% 
	  \tt {\tiny [A$\rangle$GM]}  
	 No\\ \tt \\ \tt To answer this question, we need to apply the principles of causal inference, specifically the concept of confounding variables and the potential outcomes framework. \\ \tt \\ \tt Let's denote the treatment as T (tanning salon treatment), the outcome as Y (black hair), and the alternative outcome as Y' (blond hair). We are given the following information:\\ \tt \\ \tt 1. P(T) = 0.99 (the probability of tanning salon treatment)\\ \tt 2. P(Y|not T) = 0.66 (the probability of black hair for people not using tanning salon treatments)\\ \tt 3. P(Y|T) = 0.24 (the probability of black hair for people who went to tanning salons)\\ \tt \\ \tt We are asked to compare the overall probability of black hair (Y) with the overall probability of blond hair (Y'). To do this, we need to calculate the overall probability of Y and Y'.\\ \tt \\ \tt First, let's calculate the probability of Y for people who use tanning salon treatments (P(Y|T)) and those who do not (P(Y|not T)). We can then use these probabilities to calculate the overall probability of Y.\\ \tt \\ \tt P(Y|T) = 0.24\\ \tt P(Y|not T) = 0.66\\ \tt \\ \tt Since P(T) = 0.99, the probability of not using tanning salon treatments is P(not T) = 1 - P(T) = 1 - 0.99 = 0 
	  } 
	   } 
	   } 
	 & & \\ 
 

    \theutterance \stepcounter{utterance}  

    & & & \multicolumn{2}{p{0.3\linewidth}}{\cellcolor[rgb]{0.95,0.95,0.95}{%
	\makecell[{{p{\linewidth}}}]{% 
	  \tt {\tiny [GM$|$GM]}  
	 No 
	  } 
	   } 
	   } 
	 & & \\ 
 

    \theutterance \stepcounter{utterance}  

    & & & \multicolumn{2}{p{0.3\linewidth}}{\cellcolor[rgb]{0.95,0.95,0.95}{%
	\makecell[{{p{\linewidth}}}]{% 
	  \tt {\tiny [GM$|$GM]}  
	 yes 
	  } 
	   } 
	   } 
	 & & \\ 
 

    \theutterance \stepcounter{utterance}  

    & & & \multicolumn{2}{p{0.3\linewidth}}{\cellcolor[rgb]{0.95,0.95,0.95}{%
	\makecell[{{p{\linewidth}}}]{% 
	  \tt {\tiny [GM$|$GM]}  
	 game_result = LOSE 
	  } 
	   } 
	   } 
	 & & \\ 
 

\end{supertabular}
}

\end{document}
