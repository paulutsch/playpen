
\documentclass{article}
\usepackage{colortbl}
\usepackage{makecell}
\usepackage{multirow}
\usepackage{supertabular}

\begin{document}

\newcounter{utterance}

\twocolumn

{ \footnotesize  \setcounter{utterance}{1}
\setlength{\tabcolsep}{0pt}
\begin{supertabular}{c@{$\;$}|p{.15\linewidth}@{}p{.15\linewidth}p{.15\linewidth}p{.15\linewidth}p{.15\linewidth}p{.15\linewidth}}

    \# & $\;$A & \multicolumn{4}{c}{Game Master} & $\;\:$B\\
    \hline 

    \theutterance \stepcounter{utterance}  

    & & \multicolumn{4}{p{0.6\linewidth}}{\cellcolor[rgb]{0.9,0.9,0.9}{%
	\makecell[{{p{\linewidth}}}]{% 
	  \tt {\tiny [A$\langle$GM]}  
	 You are an expert in causal inference. The following question is not a typical commonsense query, but rather a meticulously designed question created by a professor specializing in causal inference, intended to assess the students' mastery of the course content.\\ \tt Imagine a self-contained, hypothetical world with only the following conditions, and without any unmentioned factors or causal relationships: Rixq has a direct effect on xevu. Zuph has a direct effect on xevu. The overall probability of rixq is 40%. The probability of not rixq and xevu is 9%. The probability of rixq and xevu is 22%. Is the chance of xevu smaller when observing rixq?\\ \tt Start your answer with 'Yes' or 'No', followed by additional reasoning or evidence to support your explanation. 
	  } 
	   } 
	   } 
	 & \\ 
 

    \theutterance \stepcounter{utterance}  

    & \multicolumn{4}{p{0.6\linewidth}}{\cellcolor[rgb]{0.8,1,0.9}{%
	\makecell[{{p{\linewidth}}}]{% 
	  \tt {\tiny [A$\rangle$GM]}  
	 No, the chance of xevu is not smaller when observing rixq.\\ \tt \\ \tt To answer this question, we need to apply the principles of causal inference, specifically the concept of conditional probability and the backdoor criterion. \\ \tt \\ \tt Let's denote the probability of xevu as P(xevu), the probability of rixq as P(rixq), and the probability of zuph as P(zuph). We are given the following information:\\ \tt \\ \tt 1. P(rixq) = 0.4\\ \tt 2. P(¬rixq ∧ xevu) = 0.09\\ \tt 3. P(rixq ∧ xevu) = 0.22\\ \tt \\ \tt We want to determine if the chance of xevu is smaller when observing rixq, i.e., P(xevu | rixq) ≤ P(xevu).\\ \tt \\ \tt To do this, we can use the law of total probability to calculate P(xevu | rixq):\\ \tt \\ \tt P(xevu | rixq) = P(xevu ∧ rixq) / P(rixq)\\ \tt = 0.22 / 0.4\\ \tt = 0.55\\ \tt \\ \tt Now, we need to compare this with P(xevu). To calculate P(xevu), we can use the law of total probability again:\\ \tt \\ \tt P(xevu) = P(xevu ∧ rixq) + P(xevu ∧ ¬rixq)\\ \tt = 0 
	  } 
	   } 
	   } 
	 & & \\ 
 

    \theutterance \stepcounter{utterance}  

    & & & \multicolumn{2}{p{0.3\linewidth}}{\cellcolor[rgb]{0.95,0.95,0.95}{%
	\makecell[{{p{\linewidth}}}]{% 
	  \tt {\tiny [GM$|$GM]}  
	 No 
	  } 
	   } 
	   } 
	 & & \\ 
 

    \theutterance \stepcounter{utterance}  

    & & & \multicolumn{2}{p{0.3\linewidth}}{\cellcolor[rgb]{0.95,0.95,0.95}{%
	\makecell[{{p{\linewidth}}}]{% 
	  \tt {\tiny [GM$|$GM]}  
	 no 
	  } 
	   } 
	   } 
	 & & \\ 
 

    \theutterance \stepcounter{utterance}  

    & & & \multicolumn{2}{p{0.3\linewidth}}{\cellcolor[rgb]{0.95,0.95,0.95}{%
	\makecell[{{p{\linewidth}}}]{% 
	  \tt {\tiny [GM$|$GM]}  
	 game_result = WIN 
	  } 
	   } 
	   } 
	 & & \\ 
 

\end{supertabular}
}

\end{document}
