
\documentclass{article}
\usepackage{colortbl}
\usepackage{makecell}
\usepackage{multirow}
\usepackage{supertabular}

\begin{document}

\newcounter{utterance}

\twocolumn

{ \footnotesize  \setcounter{utterance}{1}
\setlength{\tabcolsep}{0pt}
\begin{supertabular}{c@{$\;$}|p{.15\linewidth}@{}p{.15\linewidth}p{.15\linewidth}p{.15\linewidth}p{.15\linewidth}p{.15\linewidth}}

    \# & $\;$A & \multicolumn{4}{c}{Game Master} & $\;\:$B\\
    \hline 

    \theutterance \stepcounter{utterance}  

    & & \multicolumn{4}{p{0.6\linewidth}}{\cellcolor[rgb]{0.9,0.9,0.9}{%
	\makecell[{{p{\linewidth}}}]{% 
	  \tt {\tiny [A$\langle$GM]}  
	 You are an expert in causal inference. The following question is not a typical commonsense query, but rather a meticulously designed question created by a professor specializing in causal inference, intended to assess the students' mastery of the course content.\\ \tt Imagine a self-contained, hypothetical world with only the following conditions, and without any unmentioned factors or causal relationships: Zuph has a direct effect on rixq and xevu. Rixq has a direct effect on xevu. For those who are not zuph and are not rixq, the probability of xevu is 15%. For those who are not zuph and are rixq, the probability of xevu is 50%. For those who are zuph and are not rixq, the probability of xevu is 32%. For those who are zuph and are rixq, the probability of xevu is 85%. The overall probability of zuph is 84%. Will rixq increase the chance of xevu?\\ \tt Start your answer with 'Yes' or 'No', followed by additional reasoning or evidence to support your explanation. 
	  } 
	   } 
	   } 
	 & \\ 
 

    \theutterance \stepcounter{utterance}  

    & \multicolumn{4}{p{0.6\linewidth}}{\cellcolor[rgb]{0.8,1,0.9}{%
	\makecell[{{p{\linewidth}}}]{% 
	  \tt {\tiny [A$\rangle$GM]}  
	 No, rixq does not increase the chance of xevu.\\ \tt \\ \tt To assess this, we can use the concept of causal mediation analysis and the backdoor criterion from Pearl's do-calculus. We need to determine whether the effect of rixq on xevu is mediated by zuph or if there's a direct effect of rixq on xevu.\\ \tt \\ \tt First, let's calculate the probability of xevu for those who are rixq, regardless of their zuph status. We can use the law of total probability:\\ \tt \\ \tt P(xevu | rixq) = P(xevu | rixq, zuph) * P(zuph | rixq) + P(xevu | rixq, not zuph) * P(not zuph | rixq)\\ \tt \\ \tt Given that P(xevu | rixq, zuph) = 0.85, P(zuph | rixq) is not directly provided, but we can calculate it using the law of total probability:\\ \tt \\ \tt P(zuph | rixq) = P(zuph and rixq) / P(rixq)\\ \tt Since P(zuph and rixq) is not directly provided, we need to calculate it using the law of total probability:\\ \tt P(zuph and rixq) = P(xevu | zuph and rixq) * P(zuph) 
	  } 
	   } 
	   } 
	 & & \\ 
 

    \theutterance \stepcounter{utterance}  

    & & & \multicolumn{2}{p{0.3\linewidth}}{\cellcolor[rgb]{0.95,0.95,0.95}{%
	\makecell[{{p{\linewidth}}}]{% 
	  \tt {\tiny [GM$|$GM]}  
	 No 
	  } 
	   } 
	   } 
	 & & \\ 
 

    \theutterance \stepcounter{utterance}  

    & & & \multicolumn{2}{p{0.3\linewidth}}{\cellcolor[rgb]{0.95,0.95,0.95}{%
	\makecell[{{p{\linewidth}}}]{% 
	  \tt {\tiny [GM$|$GM]}  
	 yes 
	  } 
	   } 
	   } 
	 & & \\ 
 

    \theutterance \stepcounter{utterance}  

    & & & \multicolumn{2}{p{0.3\linewidth}}{\cellcolor[rgb]{0.95,0.95,0.95}{%
	\makecell[{{p{\linewidth}}}]{% 
	  \tt {\tiny [GM$|$GM]}  
	 game_result = LOSE 
	  } 
	   } 
	   } 
	 & & \\ 
 

\end{supertabular}
}

\end{document}
