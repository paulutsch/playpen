
\documentclass{article}
\usepackage{colortbl}
\usepackage{makecell}
\usepackage{multirow}
\usepackage{supertabular}

\begin{document}

\newcounter{utterance}

\twocolumn

{ \footnotesize  \setcounter{utterance}{1}
\setlength{\tabcolsep}{0pt}
\begin{supertabular}{c@{$\;$}|p{.15\linewidth}@{}p{.15\linewidth}p{.15\linewidth}p{.15\linewidth}p{.15\linewidth}p{.15\linewidth}}

    \# & $\;$A & \multicolumn{4}{c}{Game Master} & $\;\:$B\\
    \hline 

    \theutterance \stepcounter{utterance}  

    & & \multicolumn{4}{p{0.6\linewidth}}{\cellcolor[rgb]{0.9,0.9,0.9}{%
	\makecell[{{p{\linewidth}}}]{% 
	  \tt {\tiny [A$\langle$GM]}  
	 You are an expert in causal inference. The following question is not a typical commonsense query, but rather a meticulously designed question created by a professor specializing in causal inference, intended to assess the students' mastery of the course content.\\ \tt Imagine a self-contained, hypothetical world with only the following conditions, and without any unmentioned factors or causal relationships: Jyka has a direct effect on zory and zuph. Zuph has a direct effect on glimx. Zory has a direct effect on glimx. We know that jyka causes not zuph and zory. zuph and zory causes glimx. We observed an individual is jyka. Would an individual is glimx if zuph instead of not zuph?\\ \tt Start your answer with 'Yes' or 'No', followed by additional reasoning or evidence to support your explanation. 
	  } 
	   } 
	   } 
	 & \\ 
 

    \theutterance \stepcounter{utterance}  

    & \multicolumn{4}{p{0.6\linewidth}}{\cellcolor[rgb]{0.8,1,0.9}{%
	\makecell[{{p{\linewidth}}}]{% 
	  \tt {\tiny [A$\rangle$GM]}  
	 No\\ \tt \\ \tt The given conditions imply a causal graph with the following structure:\\ \tt \\ \tt - Jyka → Zory\\ \tt - Jyka → Zuph\\ \tt - Zuph → Glimx\\ \tt - Zory → Glimx\\ \tt \\ \tt We are given that Jyka causes both Zuph and Zory, and both Zuph and Zory cause Glimx. We are also given that we observe an individual with Jyka.\\ \tt \\ \tt To answer the question, we need to consider the counterfactual scenario where Zuph is present instead of absent. \\ \tt \\ \tt Since Zuph is a child of Jyka, and we observe Jyka, the presence of Zuph is already implied by the observed Jyka. Therefore, the counterfactual scenario where Zuph is present instead of absent does not change the causal effect of Jyka on Glimx.\\ \tt \\ \tt However, the presence of Zuph does not block the effect of Zory on Glimx, as Zuph and Zory are independent causes of Glimx. \\ \tt \\ \tt To determine whether an individual is Glimx, we need to consider the effect of Zory on Glimx. Since Zory is a child of Jyka, and we observe Jyka, the presence of Zory is also implied by the observed Jyka.\\ \tt \\ \tt Given that Zory causes Glimx, and we observe Jyka, the individual is Glimx regardless of 
	  } 
	   } 
	   } 
	 & & \\ 
 

    \theutterance \stepcounter{utterance}  

    & & & \multicolumn{2}{p{0.3\linewidth}}{\cellcolor[rgb]{0.95,0.95,0.95}{%
	\makecell[{{p{\linewidth}}}]{% 
	  \tt {\tiny [GM$|$GM]}  
	 No 
	  } 
	   } 
	   } 
	 & & \\ 
 

    \theutterance \stepcounter{utterance}  

    & & & \multicolumn{2}{p{0.3\linewidth}}{\cellcolor[rgb]{0.95,0.95,0.95}{%
	\makecell[{{p{\linewidth}}}]{% 
	  \tt {\tiny [GM$|$GM]}  
	 no 
	  } 
	   } 
	   } 
	 & & \\ 
 

    \theutterance \stepcounter{utterance}  

    & & & \multicolumn{2}{p{0.3\linewidth}}{\cellcolor[rgb]{0.95,0.95,0.95}{%
	\makecell[{{p{\linewidth}}}]{% 
	  \tt {\tiny [GM$|$GM]}  
	 game_result = WIN 
	  } 
	   } 
	   } 
	 & & \\ 
 

\end{supertabular}
}

\end{document}
