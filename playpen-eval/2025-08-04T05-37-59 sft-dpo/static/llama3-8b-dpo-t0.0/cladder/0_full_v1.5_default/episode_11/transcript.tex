
\documentclass{article}
\usepackage{colortbl}
\usepackage{makecell}
\usepackage{multirow}
\usepackage{supertabular}

\begin{document}

\newcounter{utterance}

\twocolumn

{ \footnotesize  \setcounter{utterance}{1}
\setlength{\tabcolsep}{0pt}
\begin{supertabular}{c@{$\;$}|p{.15\linewidth}@{}p{.15\linewidth}p{.15\linewidth}p{.15\linewidth}p{.15\linewidth}p{.15\linewidth}}

    \# & $\;$A & \multicolumn{4}{c}{Game Master} & $\;\:$B\\
    \hline 

    \theutterance \stepcounter{utterance}  

    & & \multicolumn{4}{p{0.6\linewidth}}{\cellcolor[rgb]{0.9,0.9,0.9}{%
	\makecell[{{p{\linewidth}}}]{% 
	  \tt {\tiny [A$\langle$GM]}  
	 You are an expert in causal inference. The following question is not a typical commonsense query, but rather a meticulously designed question created by a professor specializing in causal inference, intended to assess the students' mastery of the course content.\\ \tt Imagine a self-contained, hypothetical world with only the following conditions, and without any unmentioned factors or causal relationships: Obesity has a direct effect on diabetes and lifespan. Smoking has a direct effect on diabetes and lifespan. Diabetes has a direct effect on lifespan. Smoking is unobserved. For normal weight people and without diabetes, the probability of long lifespan is 91%. For normal weight people and with diabetes, the probability of long lifespan is 71%. For obese people and without diabetes, the probability of long lifespan is 69%. For obese people and with diabetes, the probability of long lifespan is 43%. For normal weight people and nonsmokers, the probability of having diabetes is 28%. For normal weight people and smokers, the probability of having diabetes is 20%. For obese people and nonsmokers, the probability of having diabetes is 57%. For obese people and smokers, the probability of having diabetes is 70%. The overall probability of smoker is 8%. If we disregard the mediation effect through diabetes, would obesity negatively affect lifespan?\\ \tt Start your answer with 'Yes' or 'No', followed by additional reasoning or evidence to support your explanation. 
	  } 
	   } 
	   } 
	 & \\ 
 

    \theutterance \stepcounter{utterance}  

    & \multicolumn{4}{p{0.6\linewidth}}{\cellcolor[rgb]{0.8,1,0.9}{%
	\makecell[{{p{\linewidth}}}]{% 
	  \tt {\tiny [A$\rangle$GM]}  
	 No, obesity would not negatively affect lifespan when disregarding the mediation effect through diabetes.\\ \tt \\ \tt To address this question, we can use the counterfactual framework and the potential outcomes approach. Let's define the potential outcomes for an individual with obesity (O=1) and without obesity (O=0) as follows:\\ \tt \\ \tt - Y(1) = Lifespan if the individual has obesity\\ \tt - Y(0) = Lifespan if the individual does not have obesity\\ \tt \\ \tt We are interested in the causal effect of obesity on lifespan, which can be represented as:\\ \tt \\ \tt E[Y(1) - Y(0)]\\ \tt \\ \tt To estimate this effect, we can use the front-door adjustment, which involves identifying a set of variables that are affected by the treatment (obesity) and have a direct effect on the outcome (lifespan). In this case, the relevant variables are diabetes (D) and smoking (S), which are both affected by obesity.\\ \tt \\ \tt We can use the following formula to estimate the causal effect:\\ \tt \\ \tt E[Y(1) - Y(0)] = E[Y(1)|O=1, D=0, S=0] - E[Y(0)|O=0, D=0, S=0]\\ \tt \\ \tt Using the given probabilities, we can calculate the following:\\ \tt \\ \tt - E[Y(1)|O=1, D=0, S=0] = 0.69 (probability of long lifespan for obese people without diabetes and without smoking)\\ \tt - E[Y( 
	  } 
	   } 
	   } 
	 & & \\ 
 

    \theutterance \stepcounter{utterance}  

    & & & \multicolumn{2}{p{0.3\linewidth}}{\cellcolor[rgb]{0.95,0.95,0.95}{%
	\makecell[{{p{\linewidth}}}]{% 
	  \tt {\tiny [GM$|$GM]}  
	 No 
	  } 
	   } 
	   } 
	 & & \\ 
 

    \theutterance \stepcounter{utterance}  

    & & & \multicolumn{2}{p{0.3\linewidth}}{\cellcolor[rgb]{0.95,0.95,0.95}{%
	\makecell[{{p{\linewidth}}}]{% 
	  \tt {\tiny [GM$|$GM]}  
	 yes 
	  } 
	   } 
	   } 
	 & & \\ 
 

    \theutterance \stepcounter{utterance}  

    & & & \multicolumn{2}{p{0.3\linewidth}}{\cellcolor[rgb]{0.95,0.95,0.95}{%
	\makecell[{{p{\linewidth}}}]{% 
	  \tt {\tiny [GM$|$GM]}  
	 game_result = LOSE 
	  } 
	   } 
	   } 
	 & & \\ 
 

\end{supertabular}
}

\end{document}
