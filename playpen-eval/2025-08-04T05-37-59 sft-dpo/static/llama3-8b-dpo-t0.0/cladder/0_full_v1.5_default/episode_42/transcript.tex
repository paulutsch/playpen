
\documentclass{article}
\usepackage{colortbl}
\usepackage{makecell}
\usepackage{multirow}
\usepackage{supertabular}

\begin{document}

\newcounter{utterance}

\twocolumn

{ \footnotesize  \setcounter{utterance}{1}
\setlength{\tabcolsep}{0pt}
\begin{supertabular}{c@{$\;$}|p{.15\linewidth}@{}p{.15\linewidth}p{.15\linewidth}p{.15\linewidth}p{.15\linewidth}p{.15\linewidth}}

    \# & $\;$A & \multicolumn{4}{c}{Game Master} & $\;\:$B\\
    \hline 

    \theutterance \stepcounter{utterance}  

    & & \multicolumn{4}{p{0.6\linewidth}}{\cellcolor[rgb]{0.9,0.9,0.9}{%
	\makecell[{{p{\linewidth}}}]{% 
	  \tt {\tiny [A$\langle$GM]}  
	 You are an expert in causal inference. The following question is not a typical commonsense query, but rather a meticulously designed question created by a professor specializing in causal inference, intended to assess the students' mastery of the course content.\\ \tt Imagine a self-contained, hypothetical world with only the following conditions, and without any unmentioned factors or causal relationships: Gender has a direct effect on department competitiveness and peanut allergy. Department competitiveness has a direct effect on peanut allergy. For individuals who are not male, the probability of being allergic to peanuts is 48%. For individuals who are male, the probability of being allergic to peanuts is 36%. For individuals who are male, would it be more likely to see being allergic to peanuts if the individual had not been male?\\ \tt Start your answer with 'Yes' or 'No', followed by additional reasoning or evidence to support your explanation. 
	  } 
	   } 
	   } 
	 & \\ 
 

    \theutterance \stepcounter{utterance}  

    & \multicolumn{4}{p{0.6\linewidth}}{\cellcolor[rgb]{0.8,1,0.9}{%
	\makecell[{{p{\linewidth}}}]{% 
	  \tt {\tiny [A$\rangle$GM]}  
	 No.\\ \tt \\ \tt To address this question, we need to apply the concept of counterfactuals and potential outcomes. The question essentially asks: "If an individual who is male were not male, would it be more likely for them to be allergic to peanuts?"\\ \tt \\ \tt Let's denote the potential outcome of being allergic to peanuts as Y, the gender as X, and the counterfactual gender as X'. We are interested in the probability of Y given X' (i.e., P(Y|X')) for individuals who are male (X=1).\\ \tt \\ \tt We know that P(Y|X=0) = 0.48 (probability of being allergic to peanuts for individuals who are not male) and P(Y|X=1) = 0.36 (probability of being allergic to peanuts for individuals who are male).\\ \tt \\ \tt To answer the question, we need to consider the potential outcome of Y given X' (i.e., P(Y|X'=0) for individuals who are male). This is a counterfactual quantity that cannot be directly observed. However, we can use the concept of causal graphs and the back-door criterion to identify a valid adjustment set.\\ \tt \\ \tt In this case, the causal graph is:\\ \tt \\ \tt - X → Y\\ \tt - X → Department Competitiveness (D)\\ \tt - D → Y\\ \tt \\ \tt The back-door criterion suggests that we need to adjust for the common effect D to identify the causal effect of X on Y. However, the question asks about the counterfactual probability of Y 
	  } 
	   } 
	   } 
	 & & \\ 
 

    \theutterance \stepcounter{utterance}  

    & & & \multicolumn{2}{p{0.3\linewidth}}{\cellcolor[rgb]{0.95,0.95,0.95}{%
	\makecell[{{p{\linewidth}}}]{% 
	  \tt {\tiny [GM$|$GM]}  
	 No 
	  } 
	   } 
	   } 
	 & & \\ 
 

    \theutterance \stepcounter{utterance}  

    & & & \multicolumn{2}{p{0.3\linewidth}}{\cellcolor[rgb]{0.95,0.95,0.95}{%
	\makecell[{{p{\linewidth}}}]{% 
	  \tt {\tiny [GM$|$GM]}  
	 yes 
	  } 
	   } 
	   } 
	 & & \\ 
 

    \theutterance \stepcounter{utterance}  

    & & & \multicolumn{2}{p{0.3\linewidth}}{\cellcolor[rgb]{0.95,0.95,0.95}{%
	\makecell[{{p{\linewidth}}}]{% 
	  \tt {\tiny [GM$|$GM]}  
	 game_result = LOSE 
	  } 
	   } 
	   } 
	 & & \\ 
 

\end{supertabular}
}

\end{document}
