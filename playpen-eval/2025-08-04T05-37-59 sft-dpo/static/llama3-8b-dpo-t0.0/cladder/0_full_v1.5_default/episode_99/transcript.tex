
\documentclass{article}
\usepackage{colortbl}
\usepackage{makecell}
\usepackage{multirow}
\usepackage{supertabular}

\begin{document}

\newcounter{utterance}

\twocolumn

{ \footnotesize  \setcounter{utterance}{1}
\setlength{\tabcolsep}{0pt}
\begin{supertabular}{c@{$\;$}|p{.15\linewidth}@{}p{.15\linewidth}p{.15\linewidth}p{.15\linewidth}p{.15\linewidth}p{.15\linewidth}}

    \# & $\;$A & \multicolumn{4}{c}{Game Master} & $\;\:$B\\
    \hline 

    \theutterance \stepcounter{utterance}  

    & & \multicolumn{4}{p{0.6\linewidth}}{\cellcolor[rgb]{0.9,0.9,0.9}{%
	\makecell[{{p{\linewidth}}}]{% 
	  \tt {\tiny [A$\langle$GM]}  
	 You are an expert in causal inference. The following question is not a typical commonsense query, but rather a meticulously designed question created by a professor specializing in causal inference, intended to assess the students' mastery of the course content.\\ \tt Imagine a self-contained, hypothetical world with only the following conditions, and without any unmentioned factors or causal relationships: Zory has a direct effect on zuph and glimx. Zuph has a direct effect on jyka. Jyka has a direct effect on glimx. Zory is unobserved. We know that zory causes zuph. zuph causes not jyka. zory or jyka causes glimx. We observed an individual is zory. Would an individual is not glimx if not zuph instead of zuph?\\ \tt Start your answer with 'Yes' or 'No', followed by additional reasoning or evidence to support your explanation. 
	  } 
	   } 
	   } 
	 & \\ 
 

    \theutterance \stepcounter{utterance}  

    & \multicolumn{4}{p{0.6\linewidth}}{\cellcolor[rgb]{0.8,1,0.9}{%
	\makecell[{{p{\linewidth}}}]{% 
	  \tt {\tiny [A$\rangle$GM]}  
	 No\\ \tt \\ \tt To answer this question, we need to carefully analyze the given causal relationships and conditions. \\ \tt \\ \tt First, we know that Zory is unobserved, but we are given that Zory causes Zuph. This implies a causal relationship between Zory and Zuph, but we cannot directly observe Zory.\\ \tt \\ \tt We are also given that Zuph causes not Jyka, which means that if Zuph is present, Jyka is not present, and vice versa.\\ \tt \\ \tt Furthermore, we know that Zory or Jyka causes Glimx. This is a disjunctive causal relationship, indicating that either Zory or Jyka (or both) can cause Glimx.\\ \tt \\ \tt Given that we observed an individual is Zory, we need to determine whether an individual is not Glimx if not Zuph instead of Zuph.\\ \tt \\ \tt To do this, we need to consider the possible causal pathways that lead to Glimx. There are two possible pathways:\\ \tt \\ \tt 1. Zory → Zuph → not Jyka → Glimx (this pathway is blocked because Zory causes Zuph, but Zuph causes not Jyka)\\ \tt 2. Zory → Glimx (this pathway is direct and unblocked)\\ \tt 3. Jyka → Glimx (this pathway is blocked because we observed an individual is Zory, and Zory causes Zuph, which in turn causes not Jyka) 
	  } 
	   } 
	   } 
	 & & \\ 
 

    \theutterance \stepcounter{utterance}  

    & & & \multicolumn{2}{p{0.3\linewidth}}{\cellcolor[rgb]{0.95,0.95,0.95}{%
	\makecell[{{p{\linewidth}}}]{% 
	  \tt {\tiny [GM$|$GM]}  
	 No 
	  } 
	   } 
	   } 
	 & & \\ 
 

    \theutterance \stepcounter{utterance}  

    & & & \multicolumn{2}{p{0.3\linewidth}}{\cellcolor[rgb]{0.95,0.95,0.95}{%
	\makecell[{{p{\linewidth}}}]{% 
	  \tt {\tiny [GM$|$GM]}  
	 no 
	  } 
	   } 
	   } 
	 & & \\ 
 

    \theutterance \stepcounter{utterance}  

    & & & \multicolumn{2}{p{0.3\linewidth}}{\cellcolor[rgb]{0.95,0.95,0.95}{%
	\makecell[{{p{\linewidth}}}]{% 
	  \tt {\tiny [GM$|$GM]}  
	 game_result = WIN 
	  } 
	   } 
	   } 
	 & & \\ 
 

\end{supertabular}
}

\end{document}
