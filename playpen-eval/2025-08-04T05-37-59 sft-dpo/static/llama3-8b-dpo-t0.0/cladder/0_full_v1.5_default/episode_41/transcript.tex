
\documentclass{article}
\usepackage{colortbl}
\usepackage{makecell}
\usepackage{multirow}
\usepackage{supertabular}

\begin{document}

\newcounter{utterance}

\twocolumn

{ \footnotesize  \setcounter{utterance}{1}
\setlength{\tabcolsep}{0pt}
\begin{supertabular}{c@{$\;$}|p{.15\linewidth}@{}p{.15\linewidth}p{.15\linewidth}p{.15\linewidth}p{.15\linewidth}p{.15\linewidth}}

    \# & $\;$A & \multicolumn{4}{c}{Game Master} & $\;\:$B\\
    \hline 

    \theutterance \stepcounter{utterance}  

    & & \multicolumn{4}{p{0.6\linewidth}}{\cellcolor[rgb]{0.9,0.9,0.9}{%
	\makecell[{{p{\linewidth}}}]{% 
	  \tt {\tiny [A$\langle$GM]}  
	 You are an expert in causal inference. The following question is not a typical commonsense query, but rather a meticulously designed question created by a professor specializing in causal inference, intended to assess the students' mastery of the course content.\\ \tt Imagine a self-contained, hypothetical world with only the following conditions, and without any unmentioned factors or causal relationships: Education level has a direct effect on skill. Skill has a direct effect on black hair. The overall probability of college degree or higher is 6%. The probability of high school degree or lower and black hair is 25%. The probability of college degree or higher and black hair is 3%. Is the chance of black hair smaller when observing college degree or higher?\\ \tt Start your answer with 'Yes' or 'No', followed by additional reasoning or evidence to support your explanation. 
	  } 
	   } 
	   } 
	 & \\ 
 

    \theutterance \stepcounter{utterance}  

    & \multicolumn{4}{p{0.6\linewidth}}{\cellcolor[rgb]{0.8,1,0.9}{%
	\makecell[{{p{\linewidth}}}]{% 
	  \tt {\tiny [A$\rangle$GM]}  
	 No, the chance of black hair is not smaller when observing college degree or higher.\\ \tt \\ \tt To answer this question, we need to apply the principles of causal inference, specifically the concept of conditional probability and the backdoor criterion. \\ \tt \\ \tt Let's denote the variables as follows: E (Education level), S (Skill), and B (Black hair). We are given the following information:\\ \tt \\ \tt 1. P(E ≥ college) = 0.06 (probability of college degree or higher)\\ \tt 2. P(E ≤ high school) = 1 - P(E ≥ college) = 1 - 0.06 = 0.94 (probability of high school degree or lower)\\ \tt 3. P(E ≤ high school, B) = 0.25 (probability of high school degree or lower and black hair)\\ \tt 4. P(E ≥ college, B) = 0.03 (probability of college degree or higher and black hair)\\ \tt \\ \tt We want to compare the probability of black hair (B) when observing college degree or higher (E ≥ college) to the overall probability of black hair. To do this, we need to calculate the probability of black hair given college degree or higher, P(B|E ≥ college).\\ \tt \\ \tt Using the law of total probability, we can write:\\ \tt \\ \tt P(B|E ≥ college) = P(B, E ≥ college) / P(E ≥ college)\\ \tt = P(E ≥ college, B) / P(E ≥ college)\\ \tt = 0.03 / 0.06 
	  } 
	   } 
	   } 
	 & & \\ 
 

    \theutterance \stepcounter{utterance}  

    & & & \multicolumn{2}{p{0.3\linewidth}}{\cellcolor[rgb]{0.95,0.95,0.95}{%
	\makecell[{{p{\linewidth}}}]{% 
	  \tt {\tiny [GM$|$GM]}  
	 No 
	  } 
	   } 
	   } 
	 & & \\ 
 

    \theutterance \stepcounter{utterance}  

    & & & \multicolumn{2}{p{0.3\linewidth}}{\cellcolor[rgb]{0.95,0.95,0.95}{%
	\makecell[{{p{\linewidth}}}]{% 
	  \tt {\tiny [GM$|$GM]}  
	 no 
	  } 
	   } 
	   } 
	 & & \\ 
 

    \theutterance \stepcounter{utterance}  

    & & & \multicolumn{2}{p{0.3\linewidth}}{\cellcolor[rgb]{0.95,0.95,0.95}{%
	\makecell[{{p{\linewidth}}}]{% 
	  \tt {\tiny [GM$|$GM]}  
	 game_result = WIN 
	  } 
	   } 
	   } 
	 & & \\ 
 

\end{supertabular}
}

\end{document}
