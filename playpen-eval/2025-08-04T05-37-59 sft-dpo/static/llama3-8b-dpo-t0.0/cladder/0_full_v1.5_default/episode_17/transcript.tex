
\documentclass{article}
\usepackage{colortbl}
\usepackage{makecell}
\usepackage{multirow}
\usepackage{supertabular}

\begin{document}

\newcounter{utterance}

\twocolumn

{ \footnotesize  \setcounter{utterance}{1}
\setlength{\tabcolsep}{0pt}
\begin{supertabular}{c@{$\;$}|p{.15\linewidth}@{}p{.15\linewidth}p{.15\linewidth}p{.15\linewidth}p{.15\linewidth}p{.15\linewidth}}

    \# & $\;$A & \multicolumn{4}{c}{Game Master} & $\;\:$B\\
    \hline 

    \theutterance \stepcounter{utterance}  

    & & \multicolumn{4}{p{0.6\linewidth}}{\cellcolor[rgb]{0.9,0.9,0.9}{%
	\makecell[{{p{\linewidth}}}]{% 
	  \tt {\tiny [A$\langle$GM]}  
	 You are an expert in causal inference. The following question is not a typical commonsense query, but rather a meticulously designed question created by a professor specializing in causal inference, intended to assess the students' mastery of the course content.\\ \tt Imagine a self-contained, hypothetical world with only the following conditions, and without any unmentioned factors or causal relationships: Husband has a direct effect on wife and alarm clock. Wife has a direct effect on alarm clock. The overall probability of alarm set by husband is 8%. For husbands that don't set the alarm, the probability of ringing alarm is 75%. For husbands that set the alarm, the probability of ringing alarm is 16%. Is ringing alarm more likely than silent alarm overall?\\ \tt Start your answer with 'Yes' or 'No', followed by additional reasoning or evidence to support your explanation. 
	  } 
	   } 
	   } 
	 & \\ 
 

    \theutterance \stepcounter{utterance}  

    & \multicolumn{4}{p{0.6\linewidth}}{\cellcolor[rgb]{0.8,1,0.9}{%
	\makecell[{{p{\linewidth}}}]{% 
	  \tt {\tiny [A$\rangle$GM]}  
	 No, ringing alarm is not more likely than silent alarm overall.\\ \tt \\ \tt To answer this question, we need to apply the principles of causal inference, specifically the concept of counterfactuals and potential outcomes. Let's break down the information given:\\ \tt \\ \tt 1. The probability of alarm set by husband is 8%.\\ \tt 2. For husbands that don't set the alarm, the probability of ringing alarm is 75%.\\ \tt 3. For husbands that set the alarm, the probability of ringing alarm is 16%.\\ \tt \\ \tt We want to compare the overall probability of a ringing alarm to a silent alarm. To do this, we need to consider the potential outcomes for each husband.\\ \tt \\ \tt Let's define two potential outcomes for each husband:\\ \tt \\ \tt - A: Alarm is set\\ \tt - ~A: Alarm is not set\\ \tt \\ \tt For each outcome, we can calculate the probability of a ringing alarm:\\ \tt \\ \tt - P(Ringing|~A) = 75% (given in the problem statement)\\ \tt - P(Ringing|A) = 16% (given in the problem statement)\\ \tt \\ \tt We also know that the probability of an alarm being set is 8%, so the probability of an alarm not being set is 92% (1 - 0.08).\\ \tt \\ \tt Now, we can calculate the overall probability of a ringing alarm by considering both potential outcomes:\\ \tt \\ \tt P(Ringing) = P(Ringing|~A) \* P(~A) + P(Ringing|A) \* P(A)\\ \tt = 0.75 \* 0 
	  } 
	   } 
	   } 
	 & & \\ 
 

    \theutterance \stepcounter{utterance}  

    & & & \multicolumn{2}{p{0.3\linewidth}}{\cellcolor[rgb]{0.95,0.95,0.95}{%
	\makecell[{{p{\linewidth}}}]{% 
	  \tt {\tiny [GM$|$GM]}  
	 No 
	  } 
	   } 
	   } 
	 & & \\ 
 

    \theutterance \stepcounter{utterance}  

    & & & \multicolumn{2}{p{0.3\linewidth}}{\cellcolor[rgb]{0.95,0.95,0.95}{%
	\makecell[{{p{\linewidth}}}]{% 
	  \tt {\tiny [GM$|$GM]}  
	 yes 
	  } 
	   } 
	   } 
	 & & \\ 
 

    \theutterance \stepcounter{utterance}  

    & & & \multicolumn{2}{p{0.3\linewidth}}{\cellcolor[rgb]{0.95,0.95,0.95}{%
	\makecell[{{p{\linewidth}}}]{% 
	  \tt {\tiny [GM$|$GM]}  
	 game_result = LOSE 
	  } 
	   } 
	   } 
	 & & \\ 
 

\end{supertabular}
}

\end{document}
