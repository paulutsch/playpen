
\documentclass{article}
\usepackage{colortbl}
\usepackage{makecell}
\usepackage{multirow}
\usepackage{supertabular}

\begin{document}

\newcounter{utterance}

\twocolumn

{ \footnotesize  \setcounter{utterance}{1}
\setlength{\tabcolsep}{0pt}
\begin{supertabular}{c@{$\;$}|p{.15\linewidth}@{}p{.15\linewidth}p{.15\linewidth}p{.15\linewidth}p{.15\linewidth}p{.15\linewidth}}

    \# & $\;$A & \multicolumn{4}{c}{Game Master} & $\;\:$B\\
    \hline 

    \theutterance \stepcounter{utterance}  

    & & \multicolumn{4}{p{0.6\linewidth}}{\cellcolor[rgb]{0.9,0.9,0.9}{%
	\makecell[{{p{\linewidth}}}]{% 
	  \tt {\tiny [A$\langle$GM]}  
	 You are an expert in causal inference. The following question is not a typical commonsense query, but rather a meticulously designed question created by a professor specializing in causal inference, intended to assess the students' mastery of the course content.\\ \tt Imagine a self-contained, hypothetical world with only the following conditions, and without any unmentioned factors or causal relationships: My decision has a direct effect on penguin mood and curly hair. Penguin mood has a direct effect on curly hair. For those who choose to take the stairs, the probability of curly hair is 70%. For those who choose to take the elevator, the probability of curly hair is 17%. For those who choose to take the elevator, would it be less likely to see curly hair if I had taken the stairs?\\ \tt Start your answer with 'Yes' or 'No', followed by additional reasoning or evidence to support your explanation. 
	  } 
	   } 
	   } 
	 & \\ 
 

    \theutterance \stepcounter{utterance}  

    & \multicolumn{4}{p{0.6\linewidth}}{\cellcolor[rgb]{0.8,1,0.9}{%
	\makecell[{{p{\linewidth}}}]{% 
	  \tt {\tiny [A$\rangle$GM]}  
	 No.\\ \tt \\ \tt To address this question, we need to apply the principles of causal inference, specifically the concept of counterfactuals and the backdoor criterion. \\ \tt \\ \tt Let's denote the following variables:\\ \tt - D: My decision (taking the stairs or the elevator)\\ \tt - C: Curly hair\\ \tt - P: Penguin mood\\ \tt \\ \tt We are given the following direct causal relationships:\\ \tt - D → P (My decision affects penguin mood)\\ \tt - P → C (Penguin mood affects curly hair)\\ \tt \\ \tt We are also given the following conditional probabilities:\\ \tt - P(C|D=elevator) = 0.17 (Probability of curly hair given that I take the elevator)\\ \tt - P(C|D=stairs) = 0.70 (Probability of curly hair given that I take the stairs)\\ \tt \\ \tt The question asks whether it would be less likely to see curly hair if I had taken the stairs, given that I actually took the elevator. This is a counterfactual question, and we need to consider the potential outcomes under different scenarios.\\ \tt \\ \tt To answer this question, we need to consider the causal effect of my decision on curly hair, mediated by penguin mood. We can use the backdoor criterion to identify the causal effect.\\ \tt \\ \tt The backdoor path from D to C is P. To block this path, we need to condition on P. However, we don't have direct information about the probability of penguin mood given my decision. Instead, we can use the given conditional probabilities to estimate the causal effect. 
	  } 
	   } 
	   } 
	 & & \\ 
 

    \theutterance \stepcounter{utterance}  

    & & & \multicolumn{2}{p{0.3\linewidth}}{\cellcolor[rgb]{0.95,0.95,0.95}{%
	\makecell[{{p{\linewidth}}}]{% 
	  \tt {\tiny [GM$|$GM]}  
	 No 
	  } 
	   } 
	   } 
	 & & \\ 
 

    \theutterance \stepcounter{utterance}  

    & & & \multicolumn{2}{p{0.3\linewidth}}{\cellcolor[rgb]{0.95,0.95,0.95}{%
	\makecell[{{p{\linewidth}}}]{% 
	  \tt {\tiny [GM$|$GM]}  
	 no 
	  } 
	   } 
	   } 
	 & & \\ 
 

    \theutterance \stepcounter{utterance}  

    & & & \multicolumn{2}{p{0.3\linewidth}}{\cellcolor[rgb]{0.95,0.95,0.95}{%
	\makecell[{{p{\linewidth}}}]{% 
	  \tt {\tiny [GM$|$GM]}  
	 game_result = WIN 
	  } 
	   } 
	   } 
	 & & \\ 
 

\end{supertabular}
}

\end{document}
