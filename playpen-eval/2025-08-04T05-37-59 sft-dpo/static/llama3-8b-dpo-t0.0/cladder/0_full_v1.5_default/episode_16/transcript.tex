
\documentclass{article}
\usepackage{colortbl}
\usepackage{makecell}
\usepackage{multirow}
\usepackage{supertabular}

\begin{document}

\newcounter{utterance}

\twocolumn

{ \footnotesize  \setcounter{utterance}{1}
\setlength{\tabcolsep}{0pt}
\begin{supertabular}{c@{$\;$}|p{.15\linewidth}@{}p{.15\linewidth}p{.15\linewidth}p{.15\linewidth}p{.15\linewidth}p{.15\linewidth}}

    \# & $\;$A & \multicolumn{4}{c}{Game Master} & $\;\:$B\\
    \hline 

    \theutterance \stepcounter{utterance}  

    & & \multicolumn{4}{p{0.6\linewidth}}{\cellcolor[rgb]{0.9,0.9,0.9}{%
	\makecell[{{p{\linewidth}}}]{% 
	  \tt {\tiny [A$\langle$GM]}  
	 You are an expert in causal inference. The following question is not a typical commonsense query, but rather a meticulously designed question created by a professor specializing in causal inference, intended to assess the students' mastery of the course content.\\ \tt Imagine a self-contained, hypothetical world with only the following conditions, and without any unmentioned factors or causal relationships: Husband has a direct effect on wife and alarm clock. Wife has a direct effect on alarm clock. The overall probability of alarm set by husband is 17%. The probability of alarm not set by husband and ringing alarm is 69%. The probability of alarm set by husband and ringing alarm is 3%. Is the chance of ringing alarm smaller when observing alarm set by husband?\\ \tt Start your answer with 'Yes' or 'No', followed by additional reasoning or evidence to support your explanation. 
	  } 
	   } 
	   } 
	 & \\ 
 

    \theutterance \stepcounter{utterance}  

    & \multicolumn{4}{p{0.6\linewidth}}{\cellcolor[rgb]{0.8,1,0.9}{%
	\makecell[{{p{\linewidth}}}]{% 
	  \tt {\tiny [A$\rangle$GM]}  
	 No.\\ \tt \\ \tt To answer this question, we need to apply the concept of causal inference, specifically the backdoor criterion and the frontdoor adjustment. \\ \tt \\ \tt Let's define the variables:\\ \tt - A: Alarm set by husband\\ \tt - R: Alarm ringing\\ \tt - H: Husband's action\\ \tt - W: Wife's action\\ \tt \\ \tt Given conditions:\\ \tt - H → A (Husband has a direct effect on alarm set by husband)\\ \tt - H → W (Husband has a direct effect on wife)\\ \tt - W → A (Wife has a direct effect on alarm set by husband)\\ \tt \\ \tt We are asked to compare the probability of R given A (alarm ringing given alarm set by husband) with the probability of R (alarm ringing in general).\\ \tt \\ \tt To do this, we need to apply the frontdoor adjustment formula, which is P(R|A) = P(R|do(A)) = P(R,H|A) / P(H|A).\\ \tt \\ \tt Given probabilities:\\ \tt - P(A) = 0.17 (probability of alarm set by husband)\\ \tt - P(R, not A) = 0.69 (probability of alarm not set by husband and ringing alarm)\\ \tt - P(A, R) = 0.03 (probability of alarm set by husband and ringing alarm)\\ \tt \\ \tt We can calculate P(R, A) as P(A, R) = P(A) * P(R|A), but we don't know P(R|A) yet. However, we can calculate P(R, H) as P 
	  } 
	   } 
	   } 
	 & & \\ 
 

    \theutterance \stepcounter{utterance}  

    & & & \multicolumn{2}{p{0.3\linewidth}}{\cellcolor[rgb]{0.95,0.95,0.95}{%
	\makecell[{{p{\linewidth}}}]{% 
	  \tt {\tiny [GM$|$GM]}  
	 No 
	  } 
	   } 
	   } 
	 & & \\ 
 

    \theutterance \stepcounter{utterance}  

    & & & \multicolumn{2}{p{0.3\linewidth}}{\cellcolor[rgb]{0.95,0.95,0.95}{%
	\makecell[{{p{\linewidth}}}]{% 
	  \tt {\tiny [GM$|$GM]}  
	 yes 
	  } 
	   } 
	   } 
	 & & \\ 
 

    \theutterance \stepcounter{utterance}  

    & & & \multicolumn{2}{p{0.3\linewidth}}{\cellcolor[rgb]{0.95,0.95,0.95}{%
	\makecell[{{p{\linewidth}}}]{% 
	  \tt {\tiny [GM$|$GM]}  
	 game_result = LOSE 
	  } 
	   } 
	   } 
	 & & \\ 
 

\end{supertabular}
}

\end{document}
