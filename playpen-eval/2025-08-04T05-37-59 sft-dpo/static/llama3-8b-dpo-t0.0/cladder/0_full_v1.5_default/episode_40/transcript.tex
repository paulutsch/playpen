
\documentclass{article}
\usepackage{colortbl}
\usepackage{makecell}
\usepackage{multirow}
\usepackage{supertabular}

\begin{document}

\newcounter{utterance}

\twocolumn

{ \footnotesize  \setcounter{utterance}{1}
\setlength{\tabcolsep}{0pt}
\begin{supertabular}{c@{$\;$}|p{.15\linewidth}@{}p{.15\linewidth}p{.15\linewidth}p{.15\linewidth}p{.15\linewidth}p{.15\linewidth}}

    \# & $\;$A & \multicolumn{4}{c}{Game Master} & $\;\:$B\\
    \hline 

    \theutterance \stepcounter{utterance}  

    & & \multicolumn{4}{p{0.6\linewidth}}{\cellcolor[rgb]{0.9,0.9,0.9}{%
	\makecell[{{p{\linewidth}}}]{% 
	  \tt {\tiny [A$\langle$GM]}  
	 You are an expert in causal inference. The following question is not a typical commonsense query, but rather a meticulously designed question created by a professor specializing in causal inference, intended to assess the students' mastery of the course content.\\ \tt Imagine a self-contained, hypothetical world with only the following conditions, and without any unmentioned factors or causal relationships: Appearance has a direct effect on fame. Lip thickness has a direct effect on fame. Method 1: We look at how appearance correlates with lip thickness case by case according to fame. Method 2: We look directly at how appearance correlates with lip thickness in general. To understand how appearance affects lip thickness, is it more correct to use the Method 1 than Method 2?\\ \tt Start your answer with 'Yes' or 'No', followed by additional reasoning or evidence to support your explanation. 
	  } 
	   } 
	   } 
	 & \\ 
 

    \theutterance \stepcounter{utterance}  

    & \multicolumn{4}{p{0.6\linewidth}}{\cellcolor[rgb]{0.8,1,0.9}{%
	\makecell[{{p{\linewidth}}}]{% 
	  \tt {\tiny [A$\rangle$GM]}  
	 No, it is not more correct to use Method 1 than Method 2 to understand how appearance affects lip thickness.\\ \tt \\ \tt In this hypothetical world, we are trying to estimate the causal effect of appearance on lip thickness. However, the problem statement reveals that there is a direct effect of lip thickness on fame, and a direct effect of appearance on fame. This creates a problem of confounding by the mediator (fame) when using Method 1, which involves examining the correlation between appearance and lip thickness case by case according to fame.\\ \tt \\ \tt When we look at the correlation between appearance and lip thickness case by case according to fame (Method 1), we are essentially comparing the lip thickness of individuals with different appearances within each level of fame. However, since lip thickness directly affects fame, individuals with thicker lips may be more likely to be famous, and individuals with thinner lips may be less likely to be famous. This creates a selection bias, where the observed correlation between appearance and lip thickness is influenced by the confounding effect of fame.\\ \tt \\ \tt In contrast, Method 2 involves examining the correlation between appearance and lip thickness in general, without conditioning on fame. This approach is more suitable for estimating the causal effect of appearance on lip thickness because it does not involve conditioning on a mediator that is directly affected by the outcome variable (lip thickness).\\ \tt \\ \tt To estimate the causal effect of appearance on lip thickness, we need to use a method that can control for the confounding effect of fame, such as instrumental variables or regression adjustment 
	  } 
	   } 
	   } 
	 & & \\ 
 

    \theutterance \stepcounter{utterance}  

    & & & \multicolumn{2}{p{0.3\linewidth}}{\cellcolor[rgb]{0.95,0.95,0.95}{%
	\makecell[{{p{\linewidth}}}]{% 
	  \tt {\tiny [GM$|$GM]}  
	 No 
	  } 
	   } 
	   } 
	 & & \\ 
 

    \theutterance \stepcounter{utterance}  

    & & & \multicolumn{2}{p{0.3\linewidth}}{\cellcolor[rgb]{0.95,0.95,0.95}{%
	\makecell[{{p{\linewidth}}}]{% 
	  \tt {\tiny [GM$|$GM]}  
	 no 
	  } 
	   } 
	   } 
	 & & \\ 
 

    \theutterance \stepcounter{utterance}  

    & & & \multicolumn{2}{p{0.3\linewidth}}{\cellcolor[rgb]{0.95,0.95,0.95}{%
	\makecell[{{p{\linewidth}}}]{% 
	  \tt {\tiny [GM$|$GM]}  
	 game_result = WIN 
	  } 
	   } 
	   } 
	 & & \\ 
 

\end{supertabular}
}

\end{document}
