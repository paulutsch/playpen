
\documentclass{article}
\usepackage{colortbl}
\usepackage{makecell}
\usepackage{multirow}
\usepackage{supertabular}

\begin{document}

\newcounter{utterance}

\twocolumn

{ \footnotesize  \setcounter{utterance}{1}
\setlength{\tabcolsep}{0pt}
\begin{supertabular}{c@{$\;$}|p{.15\linewidth}@{}p{.15\linewidth}p{.15\linewidth}p{.15\linewidth}p{.15\linewidth}p{.15\linewidth}}

    \# & $\;$A & \multicolumn{4}{c}{Game Master} & $\;\:$B\\
    \hline 

    \theutterance \stepcounter{utterance}  

    & & \multicolumn{4}{p{0.6\linewidth}}{\cellcolor[rgb]{0.9,0.9,0.9}{%
	\makecell[{{p{\linewidth}}}]{% 
	  \tt {\tiny [A$\langle$GM]}  
	 You are an expert in causal inference. The following question is not a typical commonsense query, but rather a meticulously designed question created by a professor specializing in causal inference, intended to assess the students' mastery of the course content.\\ \tt Imagine a self-contained, hypothetical world with only the following conditions, and without any unmentioned factors or causal relationships: Jyka has a direct effect on gyzp and hwax. Hwax has a direct effect on lirg. Gyzp has a direct effect on lirg. For those who are not jyka, the probability of lirg is 63%. For those who are jyka, the probability of lirg is 49%. For those who are jyka, would it be more likely to see lirg if the individual was not jyka?\\ \tt Start your answer with 'Yes' or 'No', followed by additional reasoning or evidence to support your explanation. 
	  } 
	   } 
	   } 
	 & \\ 
 

    \theutterance \stepcounter{utterance}  

    & \multicolumn{4}{p{0.6\linewidth}}{\cellcolor[rgb]{0.8,1,0.9}{%
	\makecell[{{p{\linewidth}}}]{% 
	  \tt {\tiny [A$\rangle$GM]}  
	 No.\\ \tt \\ \tt To address this question, we need to apply the principles of causal inference, specifically the concept of counterfactuals. We are asked to compare the probability of lirg in a hypothetical scenario where an individual who is jyka is not jyka.\\ \tt \\ \tt Let's denote the probability of lirg given jyka as P(lirg|jyka) = 0.49 and the probability of lirg given not jyka as P(lirg|not jyka) = 0.63.\\ \tt \\ \tt We are interested in the probability of lirg in a counterfactual scenario where an individual who is jyka is not jyka, denoted as P(lirg|jyka, not jyka). To estimate this, we need to consider the causal relationships between the variables.\\ \tt \\ \tt Given the direct effects:\\ \tt - Jyka → gyzp and hwax\\ \tt - Hwax → lirg\\ \tt - Gyzp → lirg\\ \tt \\ \tt We can use the law of total probability to decompose the probability of lirg given jyka into the contributions of gyzp and hwax:\\ \tt \\ \tt P(lirg|jyka) = P(lirg|jyka, gyzp)P(gyzp|jyka) + P(lirg|jyka, hwax)P(hwax|jyka)\\ \tt \\ \tt Since we don't have 
	  } 
	   } 
	   } 
	 & & \\ 
 

    \theutterance \stepcounter{utterance}  

    & & & \multicolumn{2}{p{0.3\linewidth}}{\cellcolor[rgb]{0.95,0.95,0.95}{%
	\makecell[{{p{\linewidth}}}]{% 
	  \tt {\tiny [GM$|$GM]}  
	 No 
	  } 
	   } 
	   } 
	 & & \\ 
 

    \theutterance \stepcounter{utterance}  

    & & & \multicolumn{2}{p{0.3\linewidth}}{\cellcolor[rgb]{0.95,0.95,0.95}{%
	\makecell[{{p{\linewidth}}}]{% 
	  \tt {\tiny [GM$|$GM]}  
	 yes 
	  } 
	   } 
	   } 
	 & & \\ 
 

    \theutterance \stepcounter{utterance}  

    & & & \multicolumn{2}{p{0.3\linewidth}}{\cellcolor[rgb]{0.95,0.95,0.95}{%
	\makecell[{{p{\linewidth}}}]{% 
	  \tt {\tiny [GM$|$GM]}  
	 game_result = LOSE 
	  } 
	   } 
	   } 
	 & & \\ 
 

\end{supertabular}
}

\end{document}
