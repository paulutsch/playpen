
\documentclass{article}
\usepackage{colortbl}
\usepackage{makecell}
\usepackage{multirow}
\usepackage{supertabular}

\begin{document}

\newcounter{utterance}

\twocolumn

{ \footnotesize  \setcounter{utterance}{1}
\setlength{\tabcolsep}{0pt}
\begin{supertabular}{c@{$\;$}|p{.15\linewidth}@{}p{.15\linewidth}p{.15\linewidth}p{.15\linewidth}p{.15\linewidth}p{.15\linewidth}}

    \# & $\;$A & \multicolumn{4}{c}{Game Master} & $\;\:$B\\
    \hline 

    \theutterance \stepcounter{utterance}  

    & & \multicolumn{4}{p{0.6\linewidth}}{\cellcolor[rgb]{0.9,0.9,0.9}{%
	\makecell[{{p{\linewidth}}}]{% 
	  \tt {\tiny [A$\langle$GM]}  
	 You are an expert in causal inference. The following question is not a typical commonsense query, but rather a meticulously designed question created by a professor specializing in causal inference, intended to assess the students' mastery of the course content.\\ \tt Imagine a self-contained, hypothetical world with only the following conditions, and without any unmentioned factors or causal relationships: CEO has a direct effect on director and manager. Manager has a direct effect on foot size. Director has a direct effect on foot size. For managers who don't sign termination letters and directors who don't sign termination letters, the probability of large feet is 81%. For managers who don't sign termination letters and directors who sign termination letters, the probability of large feet is 46%. For managers who sign termination letters and directors who don't sign termination letters, the probability of large feet is 49%. For managers who sign termination letters and directors who sign termination letters, the probability of large feet is 10%. For managers who don't sign termination letters, the probability of director signing the termination letter is 15%. For managers who sign termination letters, the probability of director signing the termination letter is 39%. For managers who sign termination letters, would it be less likely to see large feet if the manager had signed the termination letter?\\ \tt Start your answer with 'Yes' or 'No', followed by additional reasoning or evidence to support your explanation. 
	  } 
	   } 
	   } 
	 & \\ 
 

    \theutterance \stepcounter{utterance}  

    & \multicolumn{4}{p{0.6\linewidth}}{\cellcolor[rgb]{0.8,1,0.9}{%
	\makecell[{{p{\linewidth}}}]{% 
	  \tt {\tiny [A$\rangle$GM]}  
	 No, it would not be less likely to see large feet if the manager had signed the termination letter.\\ \tt \\ \tt To answer this question, we need to apply the principles of causal inference, specifically the concept of counterfactuals and potential outcomes. We are interested in the probability of large feet among managers who sign termination letters, compared to the probability of large feet among managers who do not sign termination letters.\\ \tt \\ \tt Let's define the following variables:\\ \tt \\ \tt - M: Manager signs termination letter (M=1) or does not sign termination letter (M=0)\\ \tt - D: Director signs termination letter (D=1) or does not sign termination letter (D=0)\\ \tt - L: Large feet (L=1) or not large feet (L=0)\\ \tt \\ \tt We are given the following probabilities:\\ \tt \\ \tt - P(L=1|M=0, D=0) = 0.81\\ \tt - P(L=1|M=0, D=1) = 0.46\\ \tt - P(L=1|M=1, D=0) = 0.49\\ \tt - P(L=1|M=1, D=1) = 0.10\\ \tt - P(D=1|M=0) = 0.15\\ \tt - P(D=1|M=1) = 0.39\\ \tt \\ \tt To estimate the probability of large feet among managers who sign termination letters, we need to consider the potential outcomes under different scenarios. We can use the law of total probability to calculate 
	  } 
	   } 
	   } 
	 & & \\ 
 

    \theutterance \stepcounter{utterance}  

    & & & \multicolumn{2}{p{0.3\linewidth}}{\cellcolor[rgb]{0.95,0.95,0.95}{%
	\makecell[{{p{\linewidth}}}]{% 
	  \tt {\tiny [GM$|$GM]}  
	 No 
	  } 
	   } 
	   } 
	 & & \\ 
 

    \theutterance \stepcounter{utterance}  

    & & & \multicolumn{2}{p{0.3\linewidth}}{\cellcolor[rgb]{0.95,0.95,0.95}{%
	\makecell[{{p{\linewidth}}}]{% 
	  \tt {\tiny [GM$|$GM]}  
	 no 
	  } 
	   } 
	   } 
	 & & \\ 
 

    \theutterance \stepcounter{utterance}  

    & & & \multicolumn{2}{p{0.3\linewidth}}{\cellcolor[rgb]{0.95,0.95,0.95}{%
	\makecell[{{p{\linewidth}}}]{% 
	  \tt {\tiny [GM$|$GM]}  
	 game_result = WIN 
	  } 
	   } 
	   } 
	 & & \\ 
 

\end{supertabular}
}

\end{document}
