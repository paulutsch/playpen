
\documentclass{article}
\usepackage{colortbl}
\usepackage{makecell}
\usepackage{multirow}
\usepackage{supertabular}

\begin{document}

\newcounter{utterance}

\twocolumn

{ \footnotesize  \setcounter{utterance}{1}
\setlength{\tabcolsep}{0pt}
\begin{supertabular}{c@{$\;$}|p{.15\linewidth}@{}p{.15\linewidth}p{.15\linewidth}p{.15\linewidth}p{.15\linewidth}p{.15\linewidth}}

    \# & $\;$A & \multicolumn{4}{c}{Game Master} & $\;\:$B\\
    \hline 

    \theutterance \stepcounter{utterance}  

    & & \multicolumn{4}{p{0.6\linewidth}}{\cellcolor[rgb]{0.9,0.9,0.9}{%
	\makecell[{{p{\linewidth}}}]{% 
	  \tt {\tiny [A$\langle$GM]}  
	 You are an expert in causal inference. The following question is not a typical commonsense query, but rather a meticulously designed question created by a professor specializing in causal inference, intended to assess the students' mastery of the course content.\\ \tt Imagine a self-contained, hypothetical world with only the following conditions, and without any unmentioned factors or causal relationships: Swoy has a direct effect on rixq and xevu. Rixq has a direct effect on zuph. Zuph has a direct effect on xevu. Swoy is unobserved. For those who are not rixq and are not zuph, the probability of xevu is 46%. For those who are not rixq and are zuph, the probability of xevu is 41%. For those who are rixq and are not zuph, the probability of xevu is 66%. For those who are rixq and are zuph, the probability of xevu is 60%. For those who are not rixq, the probability of zuph is 61%. For those who are rixq, the probability of zuph is 34%. For those who are rixq, would it be more likely to see xevu if the individual was not rixq?\\ \tt Start your answer with 'Yes' or 'No', followed by additional reasoning or evidence to support your explanation. 
	  } 
	   } 
	   } 
	 & \\ 
 

    \theutterance \stepcounter{utterance}  

    & \multicolumn{4}{p{0.6\linewidth}}{\cellcolor[rgb]{0.8,1,0.9}{%
	\makecell[{{p{\linewidth}}}]{% 
	  \tt {\tiny [A$\rangle$GM]}  
	 No, it would not be more likely to see xevu if the individual was not rixq.\\ \tt \\ \tt To address this question, we need to apply the principles of causal inference, specifically the concept of confounding variables and the problem of selection bias. Here's a step-by-step breakdown:\\ \tt \\ \tt 1. **Identify the confounding variable**: In this scenario, zuph is a confounding variable because it affects both the exposure (being rixq) and the outcome (xevu). The probability of xevu varies depending on the combination of being rixq and zuph.\\ \tt \\ \tt 2. **Estimate the causal effect**: We want to estimate the causal effect of being rixq on the probability of xevu. To do this, we need to compare the probability of xevu among those who are rixq to those who are not rixq, while controlling for the confounding variable zuph.\\ \tt \\ \tt 3. **Apply the principle of conditional probability**: We can use the law of total probability to estimate the probability of xevu among those who are rixq. Let's denote the probability of xevu among those who are rixq and zuph as P(xevu|rixq, zuph), and the probability of xevu among those who are rixq and not zuph as P(xevu|rixq, not zuph). We can then estimate the overall probability of 
	  } 
	   } 
	   } 
	 & & \\ 
 

    \theutterance \stepcounter{utterance}  

    & & & \multicolumn{2}{p{0.3\linewidth}}{\cellcolor[rgb]{0.95,0.95,0.95}{%
	\makecell[{{p{\linewidth}}}]{% 
	  \tt {\tiny [GM$|$GM]}  
	 No 
	  } 
	   } 
	   } 
	 & & \\ 
 

    \theutterance \stepcounter{utterance}  

    & & & \multicolumn{2}{p{0.3\linewidth}}{\cellcolor[rgb]{0.95,0.95,0.95}{%
	\makecell[{{p{\linewidth}}}]{% 
	  \tt {\tiny [GM$|$GM]}  
	 no 
	  } 
	   } 
	   } 
	 & & \\ 
 

    \theutterance \stepcounter{utterance}  

    & & & \multicolumn{2}{p{0.3\linewidth}}{\cellcolor[rgb]{0.95,0.95,0.95}{%
	\makecell[{{p{\linewidth}}}]{% 
	  \tt {\tiny [GM$|$GM]}  
	 game_result = WIN 
	  } 
	   } 
	   } 
	 & & \\ 
 

\end{supertabular}
}

\end{document}
