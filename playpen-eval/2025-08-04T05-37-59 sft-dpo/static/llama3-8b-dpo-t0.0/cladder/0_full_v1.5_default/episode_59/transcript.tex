
\documentclass{article}
\usepackage{colortbl}
\usepackage{makecell}
\usepackage{multirow}
\usepackage{supertabular}

\begin{document}

\newcounter{utterance}

\twocolumn

{ \footnotesize  \setcounter{utterance}{1}
\setlength{\tabcolsep}{0pt}
\begin{supertabular}{c@{$\;$}|p{.15\linewidth}@{}p{.15\linewidth}p{.15\linewidth}p{.15\linewidth}p{.15\linewidth}p{.15\linewidth}}

    \# & $\;$A & \multicolumn{4}{c}{Game Master} & $\;\:$B\\
    \hline 

    \theutterance \stepcounter{utterance}  

    & & \multicolumn{4}{p{0.6\linewidth}}{\cellcolor[rgb]{0.9,0.9,0.9}{%
	\makecell[{{p{\linewidth}}}]{% 
	  \tt {\tiny [A$\langle$GM]}  
	 You are an expert in causal inference. The following question is not a typical commonsense query, but rather a meticulously designed question created by a professor specializing in causal inference, intended to assess the students' mastery of the course content.\\ \tt Imagine a self-contained, hypothetical world with only the following conditions, and without any unmentioned factors or causal relationships: Unobserved confounders has a direct effect on having a sister and cholesterol level. Treatment assignment has a direct effect on having a sister. Having a sister has a direct effect on cholesterol level. Unobserved confounders is unobserved. The overall probability of having a sister is 20%. The probability of not having a sister and low cholesterol is 65%. The probability of having a sister and low cholesterol is 14%. Is the chance of low cholesterol larger when observing having a sister?\\ \tt Start your answer with 'Yes' or 'No', followed by additional reasoning or evidence to support your explanation. 
	  } 
	   } 
	   } 
	 & \\ 
 

    \theutterance \stepcounter{utterance}  

    & \multicolumn{4}{p{0.6\linewidth}}{\cellcolor[rgb]{0.8,1,0.9}{%
	\makecell[{{p{\linewidth}}}]{% 
	  \tt {\tiny [A$\rangle$GM]}  
	 No\\ \tt \\ \tt To answer this question, we need to apply the principles of causal inference, specifically the concept of confounding and the causal graph. \\ \tt \\ \tt Let's denote the variables as follows:\\ \tt - S: Having a sister\\ \tt - C: Cholesterol level (low or high)\\ \tt - U: Unobserved confounders\\ \tt - T: Treatment assignment\\ \tt \\ \tt The causal graph can be represented as follows:\\ \tt - U → S\\ \tt - U → C\\ \tt - T → S\\ \tt - S → C\\ \tt \\ \tt We are interested in the effect of observing S (having a sister) on the probability of C (low cholesterol). To do this, we need to compare the conditional probability of C given S (P(C|S)) with the marginal probability of C (P(C)).\\ \tt \\ \tt Using the law of total probability, we can express P(C|S) as follows:\\ \tt P(C|S) = P(C|S,T)P(T|S) + P(C|S,~T)P(~T|S)\\ \tt \\ \tt We know that P(S) = 0.2, P(~S) = 0.8, P(C,~S) = 0.65, and P(C,S) = 0.14. We also know that P(S|~T) = 0.2 (since the probability of having a sister is independent of treatment assignment given the unobserved confounders).\\ \tt \\ \tt We can calculate P(C|S) as follows:\\ \tt P(C 
	  } 
	   } 
	   } 
	 & & \\ 
 

    \theutterance \stepcounter{utterance}  

    & & & \multicolumn{2}{p{0.3\linewidth}}{\cellcolor[rgb]{0.95,0.95,0.95}{%
	\makecell[{{p{\linewidth}}}]{% 
	  \tt {\tiny [GM$|$GM]}  
	 No 
	  } 
	   } 
	   } 
	 & & \\ 
 

    \theutterance \stepcounter{utterance}  

    & & & \multicolumn{2}{p{0.3\linewidth}}{\cellcolor[rgb]{0.95,0.95,0.95}{%
	\makecell[{{p{\linewidth}}}]{% 
	  \tt {\tiny [GM$|$GM]}  
	 no 
	  } 
	   } 
	   } 
	 & & \\ 
 

    \theutterance \stepcounter{utterance}  

    & & & \multicolumn{2}{p{0.3\linewidth}}{\cellcolor[rgb]{0.95,0.95,0.95}{%
	\makecell[{{p{\linewidth}}}]{% 
	  \tt {\tiny [GM$|$GM]}  
	 game_result = WIN 
	  } 
	   } 
	   } 
	 & & \\ 
 

\end{supertabular}
}

\end{document}
