
\documentclass{article}
\usepackage{colortbl}
\usepackage{makecell}
\usepackage{multirow}
\usepackage{supertabular}

\begin{document}

\newcounter{utterance}

\twocolumn

{ \footnotesize  \setcounter{utterance}{1}
\setlength{\tabcolsep}{0pt}
\begin{supertabular}{c@{$\;$}|p{.15\linewidth}@{}p{.15\linewidth}p{.15\linewidth}p{.15\linewidth}p{.15\linewidth}p{.15\linewidth}}

    \# & $\;$A & \multicolumn{4}{c}{Game Master} & $\;\:$B\\
    \hline 

    \theutterance \stepcounter{utterance}  

    & & \multicolumn{4}{p{0.6\linewidth}}{\cellcolor[rgb]{0.9,0.9,0.9}{%
	\makecell[{{p{\linewidth}}}]{% 
	  \tt {\tiny [A$\langle$GM]}  
	 You are an expert in causal inference. The following question is not a typical commonsense query, but rather a meticulously designed question created by a professor specializing in causal inference, intended to assess the students' mastery of the course content.\\ \tt Imagine a self-contained, hypothetical world with only the following conditions, and without any unmentioned factors or causal relationships: CEO has a direct effect on director and having visited England. Having visited England has a direct effect on employee. Director has a direct effect on employee. For CEOs who fire employees and have not visited England, the probability of employee being fired is 21%. For CEOs who fire employees and have visited England, the probability of employee being fired is 51%. For CEOs who fire employees and have not visited England, the probability of employee being fired is 35%. For CEOs who fire employees and have visited England, the probability of employee being fired is 66%. The overall probability of CEO's decision to fire the employee is 28%. Will having visited England increase the chance of employee being fired?\\ \tt Start your answer with 'Yes' or 'No', followed by additional reasoning or evidence to support your explanation. 
	  } 
	   } 
	   } 
	 & \\ 
 

    \theutterance \stepcounter{utterance}  

    & \multicolumn{4}{p{0.6\linewidth}}{\cellcolor[rgb]{0.8,1,0.9}{%
	\makecell[{{p{\linewidth}}}]{% 
	  \tt {\tiny [A$\rangle$GM]}  
	 No, having visited England will not increase the chance of employee being fired.\\ \tt \\ \tt To answer this question, we need to apply the concept of causal mediation analysis. We are given the following direct effects:\\ \tt \\ \tt - CEO → Director\\ \tt - CEO → Having visited England\\ \tt - Having visited England → Employee\\ \tt - Director → Employee\\ \tt \\ \tt We are also given the following conditional probabilities:\\ \tt \\ \tt - P(Fired | Fired, Not Visited England, Fired by CEO) = 0.35\\ \tt - P(Fired | Fired, Visited England, Fired by CEO) = 0.66\\ \tt - P(Fired | Fired, Not Visited England, Not Fired by CEO) = 0.21 (implied, but not directly given)\\ \tt - P(Fired | Fired, Visited England, Not Fired by CEO) = 0.51 (implied, but not directly given)\\ \tt \\ \tt However, we are not given the direct effect of CEO on employee. We can, however, use the law of total probability to find the overall probability of employee being fired.\\ \tt \\ \tt Let's denote the probability of CEO firing the employee as P(Fired by CEO). We are given that P(Fired by CEO) = 0.28.\\ \tt \\ \tt We can use the law of total probability to find the overall probability of employee being fired:\\ \tt \\ \tt P(Fired) = P(Fired | Fired by CEO) * P(Fired by CEO) + P(Fired | Not Fired by CEO) * P(Not Fired by CEO) 
	  } 
	   } 
	   } 
	 & & \\ 
 

    \theutterance \stepcounter{utterance}  

    & & & \multicolumn{2}{p{0.3\linewidth}}{\cellcolor[rgb]{0.95,0.95,0.95}{%
	\makecell[{{p{\linewidth}}}]{% 
	  \tt {\tiny [GM$|$GM]}  
	 No 
	  } 
	   } 
	   } 
	 & & \\ 
 

    \theutterance \stepcounter{utterance}  

    & & & \multicolumn{2}{p{0.3\linewidth}}{\cellcolor[rgb]{0.95,0.95,0.95}{%
	\makecell[{{p{\linewidth}}}]{% 
	  \tt {\tiny [GM$|$GM]}  
	 yes 
	  } 
	   } 
	   } 
	 & & \\ 
 

    \theutterance \stepcounter{utterance}  

    & & & \multicolumn{2}{p{0.3\linewidth}}{\cellcolor[rgb]{0.95,0.95,0.95}{%
	\makecell[{{p{\linewidth}}}]{% 
	  \tt {\tiny [GM$|$GM]}  
	 game_result = LOSE 
	  } 
	   } 
	   } 
	 & & \\ 
 

\end{supertabular}
}

\end{document}
