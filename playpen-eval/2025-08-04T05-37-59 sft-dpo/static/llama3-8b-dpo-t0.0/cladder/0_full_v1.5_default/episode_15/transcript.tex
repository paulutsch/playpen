
\documentclass{article}
\usepackage{colortbl}
\usepackage{makecell}
\usepackage{multirow}
\usepackage{supertabular}

\begin{document}

\newcounter{utterance}

\twocolumn

{ \footnotesize  \setcounter{utterance}{1}
\setlength{\tabcolsep}{0pt}
\begin{supertabular}{c@{$\;$}|p{.15\linewidth}@{}p{.15\linewidth}p{.15\linewidth}p{.15\linewidth}p{.15\linewidth}p{.15\linewidth}}

    \# & $\;$A & \multicolumn{4}{c}{Game Master} & $\;\:$B\\
    \hline 

    \theutterance \stepcounter{utterance}  

    & & \multicolumn{4}{p{0.6\linewidth}}{\cellcolor[rgb]{0.9,0.9,0.9}{%
	\makecell[{{p{\linewidth}}}]{% 
	  \tt {\tiny [A$\langle$GM]}  
	 You are an expert in causal inference. The following question is not a typical commonsense query, but rather a meticulously designed question created by a professor specializing in causal inference, intended to assess the students' mastery of the course content.\\ \tt Imagine a self-contained, hypothetical world with only the following conditions, and without any unmentioned factors or causal relationships: Husband has a direct effect on wife and alarm clock. Wife has a direct effect on alarm clock. For husbands that don't set the alarm, the probability of ringing alarm is 78%. For husbands that set the alarm, the probability of ringing alarm is 32%. For husbands that set the alarm, would it be less likely to see ringing alarm if the husband had not set the alarm?\\ \tt Start your answer with 'Yes' or 'No', followed by additional reasoning or evidence to support your explanation. 
	  } 
	   } 
	   } 
	 & \\ 
 

    \theutterance \stepcounter{utterance}  

    & \multicolumn{4}{p{0.6\linewidth}}{\cellcolor[rgb]{0.8,1,0.9}{%
	\makecell[{{p{\linewidth}}}]{% 
	  \tt {\tiny [A$\rangle$GM]}  
	 No.\\ \tt \\ \tt To answer this question, we need to consider the causal relationships and conditional probabilities given in the problem. Let's denote the event of a husband setting the alarm as S, and the event of the alarm ringing as R.\\ \tt \\ \tt We are given two conditional probabilities:\\ \tt \\ \tt 1. P(R|S) = 0.32 (probability of the alarm ringing given that the husband sets the alarm)\\ \tt 2. P(R|~S) = 0.78 (probability of the alarm ringing given that the husband does not set the alarm)\\ \tt \\ \tt The question asks whether it would be less likely to see the alarm ringing if the husband had not set the alarm, given that he did set the alarm. In other words, we want to compare P(R|~S) and P(R|S, ~S).\\ \tt \\ \tt To calculate P(R|S, ~S), we need to use the law of total probability. We can write:\\ \tt \\ \tt P(R|S, ~S) = P(R|S) * P(~S|S) + P(R|~S) * P(~S|~S)\\ \tt \\ \tt Since the husband either sets or does not set the alarm, P(~S|~S) = 1. Therefore:\\ \tt \\ \tt P(R|S, ~S) = P(R|S) * P(~S|S) + P(R|~S)\\ \tt \\ \tt We know that P(R|S) = 0.32 and P(R|~S) = 0.78 
	  } 
	   } 
	   } 
	 & & \\ 
 

    \theutterance \stepcounter{utterance}  

    & & & \multicolumn{2}{p{0.3\linewidth}}{\cellcolor[rgb]{0.95,0.95,0.95}{%
	\makecell[{{p{\linewidth}}}]{% 
	  \tt {\tiny [GM$|$GM]}  
	 No 
	  } 
	   } 
	   } 
	 & & \\ 
 

    \theutterance \stepcounter{utterance}  

    & & & \multicolumn{2}{p{0.3\linewidth}}{\cellcolor[rgb]{0.95,0.95,0.95}{%
	\makecell[{{p{\linewidth}}}]{% 
	  \tt {\tiny [GM$|$GM]}  
	 no 
	  } 
	   } 
	   } 
	 & & \\ 
 

    \theutterance \stepcounter{utterance}  

    & & & \multicolumn{2}{p{0.3\linewidth}}{\cellcolor[rgb]{0.95,0.95,0.95}{%
	\makecell[{{p{\linewidth}}}]{% 
	  \tt {\tiny [GM$|$GM]}  
	 game_result = WIN 
	  } 
	   } 
	   } 
	 & & \\ 
 

\end{supertabular}
}

\end{document}
