
\documentclass{article}
\usepackage{colortbl}
\usepackage{makecell}
\usepackage{multirow}
\usepackage{supertabular}

\begin{document}

\newcounter{utterance}

\twocolumn

{ \footnotesize  \setcounter{utterance}{1}
\setlength{\tabcolsep}{0pt}
\begin{supertabular}{c@{$\;$}|p{.15\linewidth}@{}p{.15\linewidth}p{.15\linewidth}p{.15\linewidth}p{.15\linewidth}p{.15\linewidth}}

    \# & $\;$A & \multicolumn{4}{c}{Game Master} & $\;\:$B\\
    \hline 

    \theutterance \stepcounter{utterance}  

    & & \multicolumn{4}{p{0.6\linewidth}}{\cellcolor[rgb]{0.9,0.9,0.9}{%
	\makecell[{{p{\linewidth}}}]{% 
	  \tt {\tiny [A$\langle$GM]}  
	 You are an expert in causal inference. The following question is not a typical commonsense query, but rather a meticulously designed question created by a professor specializing in causal inference, intended to assess the students' mastery of the course content.\\ \tt Imagine a self-contained, hypothetical world with only the following conditions, and without any unmentioned factors or causal relationships: Parents' intelligence has a direct effect on parents' social status and child's intelligence. Other unobserved factors has a direct effect on parents' social status and child's intelligence. Parents' social status has a direct effect on child's intelligence. Other unobserved factors is unobserved. We know that intelligent parents and confounder active causes high parental social status. intelligent parents and confounder active and high parental social status causes intelligent child. We observed confounder active. Would the child is intelligent if intelligent parents instead of unintelligent parents?\\ \tt Start your answer with 'Yes' or 'No', followed by additional reasoning or evidence to support your explanation. 
	  } 
	   } 
	   } 
	 & \\ 
 

    \theutterance \stepcounter{utterance}  

    & \multicolumn{4}{p{0.6\linewidth}}{\cellcolor[rgb]{0.8,1,0.9}{%
	\makecell[{{p{\linewidth}}}]{% 
	  \tt {\tiny [A$\rangle$GM]}  
	 Yes.\\ \tt \\ \tt To address this question, we need to apply the principles of causal inference, specifically the concept of counterfactuals and the backdoor criterion. \\ \tt \\ \tt First, let's define the variables and their relationships:\\ \tt \\ \tt - Intelligent parents (IP): parents with high intelligence\\ \tt - Unintelligent parents (UP): parents with low intelligence\\ \tt - Confounder active (CA): the confounder is active (i.e., it has an effect on both parents' social status and child's intelligence)\\ \tt - High parental social status (HPS): parents have a high social status\\ \tt - Low parental social status (LPS): parents have a low social status\\ \tt - Intelligent child (IC): child has high intelligence\\ \tt - Unintelligent child (UC): child has low intelligence\\ \tt \\ \tt We are given the following causal relationships:\\ \tt \\ \tt 1. IP → HPS (Parents' intelligence has a direct effect on parents' social status)\\ \tt 2. CA → HPS (Other unobserved factors has a direct effect on parents' social status)\\ \tt 3. IP, CA → IC (Intelligent parents and confounder active causes intelligent child)\\ \tt 4. IP, CA, HPS → IC (Intelligent parents, confounder active, and high parental social status causes intelligent child)\\ \tt \\ \tt We are also given that we know CA is active.\\ \tt \\ \tt Now, we want to know if the child would be intelligent if intelligent parents instead of unintelligent parents. In other words, we want to compare the outcome IC when IP is true (int 
	  } 
	   } 
	   } 
	 & & \\ 
 

    \theutterance \stepcounter{utterance}  

    & & & \multicolumn{2}{p{0.3\linewidth}}{\cellcolor[rgb]{0.95,0.95,0.95}{%
	\makecell[{{p{\linewidth}}}]{% 
	  \tt {\tiny [GM$|$GM]}  
	 Yes 
	  } 
	   } 
	   } 
	 & & \\ 
 

    \theutterance \stepcounter{utterance}  

    & & & \multicolumn{2}{p{0.3\linewidth}}{\cellcolor[rgb]{0.95,0.95,0.95}{%
	\makecell[{{p{\linewidth}}}]{% 
	  \tt {\tiny [GM$|$GM]}  
	 yes 
	  } 
	   } 
	   } 
	 & & \\ 
 

    \theutterance \stepcounter{utterance}  

    & & & \multicolumn{2}{p{0.3\linewidth}}{\cellcolor[rgb]{0.95,0.95,0.95}{%
	\makecell[{{p{\linewidth}}}]{% 
	  \tt {\tiny [GM$|$GM]}  
	 game_result = WIN 
	  } 
	   } 
	   } 
	 & & \\ 
 

\end{supertabular}
}

\end{document}
