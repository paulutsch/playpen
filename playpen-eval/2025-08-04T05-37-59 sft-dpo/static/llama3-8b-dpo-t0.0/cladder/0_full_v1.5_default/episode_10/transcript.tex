
\documentclass{article}
\usepackage{colortbl}
\usepackage{makecell}
\usepackage{multirow}
\usepackage{supertabular}

\begin{document}

\newcounter{utterance}

\twocolumn

{ \footnotesize  \setcounter{utterance}{1}
\setlength{\tabcolsep}{0pt}
\begin{supertabular}{c@{$\;$}|p{.15\linewidth}@{}p{.15\linewidth}p{.15\linewidth}p{.15\linewidth}p{.15\linewidth}p{.15\linewidth}}

    \# & $\;$A & \multicolumn{4}{c}{Game Master} & $\;\:$B\\
    \hline 

    \theutterance \stepcounter{utterance}  

    & & \multicolumn{4}{p{0.6\linewidth}}{\cellcolor[rgb]{0.9,0.9,0.9}{%
	\makecell[{{p{\linewidth}}}]{% 
	  \tt {\tiny [A$\langle$GM]}  
	 You are an expert in causal inference. The following question is not a typical commonsense query, but rather a meticulously designed question created by a professor specializing in causal inference, intended to assess the students' mastery of the course content.\\ \tt Imagine a self-contained, hypothetical world with only the following conditions, and without any unmentioned factors or causal relationships: Smoking has a direct effect on effort and college admission. Effort has a direct effect on college admission. For nonsmokers, the probability of college admission is 76%. For smokers, the probability of college admission is 22%. For smokers, would it be less likely to see college admission if the student had been a nonsmoker?\\ \tt Start your answer with 'Yes' or 'No', followed by additional reasoning or evidence to support your explanation. 
	  } 
	   } 
	   } 
	 & \\ 
 

    \theutterance \stepcounter{utterance}  

    & \multicolumn{4}{p{0.6\linewidth}}{\cellcolor[rgb]{0.8,1,0.9}{%
	\makecell[{{p{\linewidth}}}]{% 
	  \tt {\tiny [A$\rangle$GM]}  
	 No.\\ \tt \\ \tt To address this question, we need to consider the causal relationships and their implications. Let's denote the variables as follows: \\ \tt \\ \tt - S: Smoking status (1 for smokers, 0 for nonsmokers)\\ \tt - E: Effort (1 for high effort, 0 for low effort)\\ \tt - A: College admission (1 for admitted, 0 for not admitted)\\ \tt \\ \tt Given conditions:\\ \tt 1. Smoking has a direct effect on effort (S → E).\\ \tt 2. Effort has a direct effect on college admission (E → A).\\ \tt 3. For nonsmokers (S = 0), the probability of college admission (A) is 76% (P(A|S=0) = 0.76).\\ \tt 4. For smokers (S = 1), the probability of college admission (A) is 22% (P(A|S=1) = 0.22).\\ \tt \\ \tt We are asked to compare the probability of college admission for smokers (S = 1) with what it would be if they were nonsmokers (S = 0). This is a counterfactual question, which can be addressed using the potential outcomes framework.\\ \tt \\ \tt Let's denote the potential outcomes as follows:\\ \tt - A0: College admission if the individual is a nonsmoker (S = 0)\\ \tt - A1: College admission if the individual is a smoker (S = 1)\\ \tt \\ \tt We are interested in the probability of A0 for individuals who are smokers (S 
	  } 
	   } 
	   } 
	 & & \\ 
 

    \theutterance \stepcounter{utterance}  

    & & & \multicolumn{2}{p{0.3\linewidth}}{\cellcolor[rgb]{0.95,0.95,0.95}{%
	\makecell[{{p{\linewidth}}}]{% 
	  \tt {\tiny [GM$|$GM]}  
	 No 
	  } 
	   } 
	   } 
	 & & \\ 
 

    \theutterance \stepcounter{utterance}  

    & & & \multicolumn{2}{p{0.3\linewidth}}{\cellcolor[rgb]{0.95,0.95,0.95}{%
	\makecell[{{p{\linewidth}}}]{% 
	  \tt {\tiny [GM$|$GM]}  
	 no 
	  } 
	   } 
	   } 
	 & & \\ 
 

    \theutterance \stepcounter{utterance}  

    & & & \multicolumn{2}{p{0.3\linewidth}}{\cellcolor[rgb]{0.95,0.95,0.95}{%
	\makecell[{{p{\linewidth}}}]{% 
	  \tt {\tiny [GM$|$GM]}  
	 game_result = WIN 
	  } 
	   } 
	   } 
	 & & \\ 
 

\end{supertabular}
}

\end{document}
