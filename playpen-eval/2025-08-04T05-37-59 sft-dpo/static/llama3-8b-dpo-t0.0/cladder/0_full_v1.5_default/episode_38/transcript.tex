
\documentclass{article}
\usepackage{colortbl}
\usepackage{makecell}
\usepackage{multirow}
\usepackage{supertabular}

\begin{document}

\newcounter{utterance}

\twocolumn

{ \footnotesize  \setcounter{utterance}{1}
\setlength{\tabcolsep}{0pt}
\begin{supertabular}{c@{$\;$}|p{.15\linewidth}@{}p{.15\linewidth}p{.15\linewidth}p{.15\linewidth}p{.15\linewidth}p{.15\linewidth}}

    \# & $\;$A & \multicolumn{4}{c}{Game Master} & $\;\:$B\\
    \hline 

    \theutterance \stepcounter{utterance}  

    & & \multicolumn{4}{p{0.6\linewidth}}{\cellcolor[rgb]{0.9,0.9,0.9}{%
	\makecell[{{p{\linewidth}}}]{% 
	  \tt {\tiny [A$\langle$GM]}  
	 You are an expert in causal inference. The following question is not a typical commonsense query, but rather a meticulously designed question created by a professor specializing in causal inference, intended to assess the students' mastery of the course content.\\ \tt Imagine a self-contained, hypothetical world with only the following conditions, and without any unmentioned factors or causal relationships: Solar eclipse has a direct effect on Alice arriving to school. Traffic has a direct effect on Alice arriving to school. Method 1: We look at how solar eclipse correlates with Alice arriving to school case by case according to traffic. Method 2: We look directly at how solar eclipse correlates with Alice arriving to school in general. To understand how solar eclipse affects Alice arriving to school, is it more correct to use the Method 1 than Method 2?\\ \tt Start your answer with 'Yes' or 'No', followed by additional reasoning or evidence to support your explanation. 
	  } 
	   } 
	   } 
	 & \\ 
 

    \theutterance \stepcounter{utterance}  

    & \multicolumn{4}{p{0.6\linewidth}}{\cellcolor[rgb]{0.8,1,0.9}{%
	\makecell[{{p{\linewidth}}}]{% 
	  \tt {\tiny [A$\rangle$GM]}  
	 No, it is not more correct to use Method 1 than Method 2 to understand how solar eclipse affects Alice arriving to school.\\ \tt \\ \tt The reason is that Method 1, which examines the correlation between solar eclipse and Alice's arrival time case by case according to traffic, is subject to confounding by the common cause of traffic. In this hypothetical world, traffic has a direct effect on both the solar eclipse (e.g., a solar eclipse might cause people to stay home, leading to less traffic) and Alice's arrival time. This creates a confounding variable, where the observed correlation between solar eclipse and Alice's arrival time is not due to the causal effect of the solar eclipse, but rather due to the shared effect of traffic.\\ \tt \\ \tt In contrast, Method 2 directly examines the correlation between solar eclipse and Alice's arrival time in general, without considering the effect of traffic. This approach is more likely to capture the causal effect of the solar eclipse on Alice's arrival time, as it does not involve confounding by a common cause.\\ \tt \\ \tt This problem is a classic example of the "confounding by common cause" problem in causal inference, and it highlights the importance of considering the causal structure of the system when designing observational studies. By using Method 2, we can more accurately estimate the causal effect of the solar eclipse on Alice's arrival time, without being misled by the confounding effect of traffic. 
	  } 
	   } 
	   } 
	 & & \\ 
 

    \theutterance \stepcounter{utterance}  

    & & & \multicolumn{2}{p{0.3\linewidth}}{\cellcolor[rgb]{0.95,0.95,0.95}{%
	\makecell[{{p{\linewidth}}}]{% 
	  \tt {\tiny [GM$|$GM]}  
	 No 
	  } 
	   } 
	   } 
	 & & \\ 
 

    \theutterance \stepcounter{utterance}  

    & & & \multicolumn{2}{p{0.3\linewidth}}{\cellcolor[rgb]{0.95,0.95,0.95}{%
	\makecell[{{p{\linewidth}}}]{% 
	  \tt {\tiny [GM$|$GM]}  
	 no 
	  } 
	   } 
	   } 
	 & & \\ 
 

    \theutterance \stepcounter{utterance}  

    & & & \multicolumn{2}{p{0.3\linewidth}}{\cellcolor[rgb]{0.95,0.95,0.95}{%
	\makecell[{{p{\linewidth}}}]{% 
	  \tt {\tiny [GM$|$GM]}  
	 game_result = WIN 
	  } 
	   } 
	   } 
	 & & \\ 
 

\end{supertabular}
}

\end{document}
