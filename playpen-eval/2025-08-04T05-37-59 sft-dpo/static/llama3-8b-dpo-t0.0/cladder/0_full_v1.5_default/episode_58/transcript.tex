
\documentclass{article}
\usepackage{colortbl}
\usepackage{makecell}
\usepackage{multirow}
\usepackage{supertabular}

\begin{document}

\newcounter{utterance}

\twocolumn

{ \footnotesize  \setcounter{utterance}{1}
\setlength{\tabcolsep}{0pt}
\begin{supertabular}{c@{$\;$}|p{.15\linewidth}@{}p{.15\linewidth}p{.15\linewidth}p{.15\linewidth}p{.15\linewidth}p{.15\linewidth}}

    \# & $\;$A & \multicolumn{4}{c}{Game Master} & $\;\:$B\\
    \hline 

    \theutterance \stepcounter{utterance}  

    & & \multicolumn{4}{p{0.6\linewidth}}{\cellcolor[rgb]{0.9,0.9,0.9}{%
	\makecell[{{p{\linewidth}}}]{% 
	  \tt {\tiny [A$\langle$GM]}  
	 You are an expert in causal inference. The following question is not a typical commonsense query, but rather a meticulously designed question created by a professor specializing in causal inference, intended to assess the students' mastery of the course content.\\ \tt Imagine a self-contained, hypothetical world with only the following conditions, and without any unmentioned factors or causal relationships: Vaccination status has a direct effect on vaccination reaction and getting smallpox. Getting smallpox has a direct effect on black hair. Vaccination reaction has a direct effect on black hair. For unvaccinated individuals, the probability of black hair is 27%. For vaccinated individuals, the probability of black hair is 37%. Will vaccination decrease the chance of black hair?\\ \tt Start your answer with 'Yes' or 'No', followed by additional reasoning or evidence to support your explanation. 
	  } 
	   } 
	   } 
	 & \\ 
 

    \theutterance \stepcounter{utterance}  

    & \multicolumn{4}{p{0.6\linewidth}}{\cellcolor[rgb]{0.8,1,0.9}{%
	\makecell[{{p{\linewidth}}}]{% 
	  \tt {\tiny [A$\rangle$GM]}  
	 No\\ \tt \\ \tt To answer this question, we need to apply the principles of causal inference, specifically the concept of confounding variables and the backdoor criterion. \\ \tt \\ \tt Let's denote the variables as follows:\\ \tt - V: Vaccination status (0 = unvaccinated, 1 = vaccinated)\\ \tt - VR: Vaccination reaction\\ \tt - S: Getting smallpox\\ \tt - BH: Black hair\\ \tt \\ \tt We are given the following direct causal relationships:\\ \tt - V → VR\\ \tt - S → BH\\ \tt - VR → BH\\ \tt \\ \tt We are also given the following probabilities:\\ \tt - P(BH|V=0) = 0.27\\ \tt - P(BH|V=1) = 0.37\\ \tt \\ \tt The question asks whether vaccination (V=1) decreases the chance of black hair compared to not being vaccinated (V=0). To answer this, we need to compare the total effect of V on BH, which can be decomposed into the direct effect and the indirect effect through S.\\ \tt \\ \tt The direct effect of V on BH is given by the causal relationship VR → BH. However, we don't have direct information about this effect. \\ \tt \\ \tt The indirect effect of V on BH through S can be calculated using the backdoor criterion. We need to condition on the common cause S to block the backdoor path from V to BH. \\ \tt \\ \tt First, we need to calculate the probability of getting smallpox given vaccination status:\\ \tt - P(S|V=0) = 1 - P(S| 
	  } 
	   } 
	   } 
	 & & \\ 
 

    \theutterance \stepcounter{utterance}  

    & & & \multicolumn{2}{p{0.3\linewidth}}{\cellcolor[rgb]{0.95,0.95,0.95}{%
	\makecell[{{p{\linewidth}}}]{% 
	  \tt {\tiny [GM$|$GM]}  
	 No 
	  } 
	   } 
	   } 
	 & & \\ 
 

    \theutterance \stepcounter{utterance}  

    & & & \multicolumn{2}{p{0.3\linewidth}}{\cellcolor[rgb]{0.95,0.95,0.95}{%
	\makecell[{{p{\linewidth}}}]{% 
	  \tt {\tiny [GM$|$GM]}  
	 no 
	  } 
	   } 
	   } 
	 & & \\ 
 

    \theutterance \stepcounter{utterance}  

    & & & \multicolumn{2}{p{0.3\linewidth}}{\cellcolor[rgb]{0.95,0.95,0.95}{%
	\makecell[{{p{\linewidth}}}]{% 
	  \tt {\tiny [GM$|$GM]}  
	 game_result = WIN 
	  } 
	   } 
	   } 
	 & & \\ 
 

\end{supertabular}
}

\end{document}
