
\documentclass{article}
\usepackage{colortbl}
\usepackage{makecell}
\usepackage{multirow}
\usepackage{supertabular}

\begin{document}

\newcounter{utterance}

\twocolumn

{ \footnotesize  \setcounter{utterance}{1}
\setlength{\tabcolsep}{0pt}
\begin{supertabular}{c@{$\;$}|p{.15\linewidth}@{}p{.15\linewidth}p{.15\linewidth}p{.15\linewidth}p{.15\linewidth}p{.15\linewidth}}

    \# & $\;$A & \multicolumn{4}{c}{Game Master} & $\;\:$B\\
    \hline 

    \theutterance \stepcounter{utterance}  

    & & \multicolumn{4}{p{0.6\linewidth}}{\cellcolor[rgb]{0.9,0.9,0.9}{%
	\makecell[{{p{\linewidth}}}]{% 
	  \tt {\tiny [A$\langle$GM]}  
	 You are an expert in causal inference. The following question is not a typical commonsense query, but rather a meticulously designed question created by a professor specializing in causal inference, intended to assess the students' mastery of the course content.\\ \tt Imagine a self-contained, hypothetical world with only the following conditions, and without any unmentioned factors or causal relationships: Zory has a direct effect on zuph and glimx. Zuph has a direct effect on jyka. Jyka has a direct effect on glimx. Zory is unobserved. The overall probability of zuph is 8%. The probability of not zuph and glimx is 58%. The probability of zuph and glimx is 1%. Is the chance of glimx smaller when observing zuph?\\ \tt Start your answer with 'Yes' or 'No', followed by additional reasoning or evidence to support your explanation. 
	  } 
	   } 
	   } 
	 & \\ 
 

    \theutterance \stepcounter{utterance}  

    & \multicolumn{4}{p{0.6\linewidth}}{\cellcolor[rgb]{0.8,1,0.9}{%
	\makecell[{{p{\linewidth}}}]{% 
	  \tt {\tiny [A$\rangle$GM]}  
	 No, the chance of glimx is not smaller when observing zuph.\\ \tt \\ \tt To analyze this problem, we need to apply the principles of causal inference, specifically the concept of confounding variables and the backdoor criterion. Since Zory is unobserved, it acts as a confounding variable that affects both zuph and glimx. \\ \tt \\ \tt We can use the backdoor criterion to identify the causal relationships between the variables. The backdoor path from Zory to glimx is Zory → zuph → glimx, and the backdoor path from Zory to glimx is Zory → glimx. \\ \tt \\ \tt Given the information provided, we can calculate the probability of glimx using the law of total probability:\\ \tt \\ \tt P(glimx) = P(glimx|zuph) * P(zuph) + P(glimx|not zuph)\\ \tt \\ \tt We know that P(zuph) = 8% and P(not zuph and glimx) = 58%, which implies P(glimx|not zuph) = 42% (since P(glimx or not glimx) = 1). We also know that P(zuph and glimx) = 1%.\\ \tt \\ \tt Using the law of total probability, we can calculate P(glimx|zuph) as follows:\\ \tt \\ \tt P(glimx|zuph) = P(zuph and glim 
	  } 
	   } 
	   } 
	 & & \\ 
 

    \theutterance \stepcounter{utterance}  

    & & & \multicolumn{2}{p{0.3\linewidth}}{\cellcolor[rgb]{0.95,0.95,0.95}{%
	\makecell[{{p{\linewidth}}}]{% 
	  \tt {\tiny [GM$|$GM]}  
	 No 
	  } 
	   } 
	   } 
	 & & \\ 
 

    \theutterance \stepcounter{utterance}  

    & & & \multicolumn{2}{p{0.3\linewidth}}{\cellcolor[rgb]{0.95,0.95,0.95}{%
	\makecell[{{p{\linewidth}}}]{% 
	  \tt {\tiny [GM$|$GM]}  
	 yes 
	  } 
	   } 
	   } 
	 & & \\ 
 

    \theutterance \stepcounter{utterance}  

    & & & \multicolumn{2}{p{0.3\linewidth}}{\cellcolor[rgb]{0.95,0.95,0.95}{%
	\makecell[{{p{\linewidth}}}]{% 
	  \tt {\tiny [GM$|$GM]}  
	 game_result = LOSE 
	  } 
	   } 
	   } 
	 & & \\ 
 

\end{supertabular}
}

\end{document}
