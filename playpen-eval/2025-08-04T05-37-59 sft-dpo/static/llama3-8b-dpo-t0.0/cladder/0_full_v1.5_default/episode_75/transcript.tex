
\documentclass{article}
\usepackage{colortbl}
\usepackage{makecell}
\usepackage{multirow}
\usepackage{supertabular}

\begin{document}

\newcounter{utterance}

\twocolumn

{ \footnotesize  \setcounter{utterance}{1}
\setlength{\tabcolsep}{0pt}
\begin{supertabular}{c@{$\;$}|p{.15\linewidth}@{}p{.15\linewidth}p{.15\linewidth}p{.15\linewidth}p{.15\linewidth}p{.15\linewidth}}

    \# & $\;$A & \multicolumn{4}{c}{Game Master} & $\;\:$B\\
    \hline 

    \theutterance \stepcounter{utterance}  

    & & \multicolumn{4}{p{0.6\linewidth}}{\cellcolor[rgb]{0.9,0.9,0.9}{%
	\makecell[{{p{\linewidth}}}]{% 
	  \tt {\tiny [A$\langle$GM]}  
	 You are an expert in causal inference. The following question is not a typical commonsense query, but rather a meticulously designed question created by a professor specializing in causal inference, intended to assess the students' mastery of the course content.\\ \tt Imagine a self-contained, hypothetical world with only the following conditions, and without any unmentioned factors or causal relationships: Jyka has a direct effect on yupt. Yupt has a direct effect on kwox. The overall probability of jyka is 25%. The probability of not jyka and kwox is 71%. The probability of jyka and kwox is 24%. Is the chance of kwox smaller when observing jyka?\\ \tt Start your answer with 'Yes' or 'No', followed by additional reasoning or evidence to support your explanation. 
	  } 
	   } 
	   } 
	 & \\ 
 

    \theutterance \stepcounter{utterance}  

    & \multicolumn{4}{p{0.6\linewidth}}{\cellcolor[rgb]{0.8,1,0.9}{%
	\makecell[{{p{\linewidth}}}]{% 
	  \tt {\tiny [A$\rangle$GM]}  
	 No, the chance of kwox is not smaller when observing jyka.\\ \tt \\ \tt To assess this, we need to apply the principles of causal inference, specifically the concept of conditional probability and the backdoor criterion. \\ \tt \\ \tt First, let's establish the causal relationships and their corresponding probabilities:\\ \tt \\ \tt - P(Jyka) = 0.25\\ \tt - P(Not Jyka and Kwox) = 0.71\\ \tt - P(Jyka and Kwox) = 0.24\\ \tt \\ \tt We are interested in the conditional probability of Kwox given Jyka, denoted as P(Kwox|Jyka). To calculate this, we need to consider the probability of Kwox in two scenarios: when Jyka is present and when Jyka is absent.\\ \tt \\ \tt From the given information, we can infer the following:\\ \tt \\ \tt - P(Kwox|Not Jyka) = P(Not Jyka and Kwox) / P(Not Jyka) = 0.71 / (1 - 0.25) = 0.71 / 0.75 = 0.9467\\ \tt - P(Kwox|Jyka) = P(Jyka and Kwox) / P(Jyka) = 0.24 / 0.25 = 0.96\\ \tt \\ \tt Comparing the two conditional probabilities, we see that P(Kwox|Jyka) = 0 
	  } 
	   } 
	   } 
	 & & \\ 
 

    \theutterance \stepcounter{utterance}  

    & & & \multicolumn{2}{p{0.3\linewidth}}{\cellcolor[rgb]{0.95,0.95,0.95}{%
	\makecell[{{p{\linewidth}}}]{% 
	  \tt {\tiny [GM$|$GM]}  
	 No 
	  } 
	   } 
	   } 
	 & & \\ 
 

    \theutterance \stepcounter{utterance}  

    & & & \multicolumn{2}{p{0.3\linewidth}}{\cellcolor[rgb]{0.95,0.95,0.95}{%
	\makecell[{{p{\linewidth}}}]{% 
	  \tt {\tiny [GM$|$GM]}  
	 no 
	  } 
	   } 
	   } 
	 & & \\ 
 

    \theutterance \stepcounter{utterance}  

    & & & \multicolumn{2}{p{0.3\linewidth}}{\cellcolor[rgb]{0.95,0.95,0.95}{%
	\makecell[{{p{\linewidth}}}]{% 
	  \tt {\tiny [GM$|$GM]}  
	 game_result = WIN 
	  } 
	   } 
	   } 
	 & & \\ 
 

\end{supertabular}
}

\end{document}
