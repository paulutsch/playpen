
\documentclass{article}
\usepackage{colortbl}
\usepackage{makecell}
\usepackage{multirow}
\usepackage{supertabular}

\begin{document}

\newcounter{utterance}

\twocolumn

{ \footnotesize  \setcounter{utterance}{1}
\setlength{\tabcolsep}{0pt}
\begin{supertabular}{c@{$\;$}|p{.15\linewidth}@{}p{.15\linewidth}p{.15\linewidth}p{.15\linewidth}p{.15\linewidth}p{.15\linewidth}}

    \# & $\;$A & \multicolumn{4}{c}{Game Master} & $\;\:$B\\
    \hline 

    \theutterance \stepcounter{utterance}  

    & & \multicolumn{4}{p{0.6\linewidth}}{\cellcolor[rgb]{0.9,0.9,0.9}{%
	\makecell[{{p{\linewidth}}}]{% 
	  \tt {\tiny [A$\langle$GM]}  
	 You are an expert in causal inference. The following question is not a typical commonsense query, but rather a meticulously designed question created by a professor specializing in causal inference, intended to assess the students' mastery of the course content.\\ \tt Imagine a self-contained, hypothetical world with only the following conditions, and without any unmentioned factors or causal relationships: Jyka has a direct effect on lirg. Hwax has a direct effect on lirg. The overall probability of jyka is 14%. For those who are not jyka, the probability of lirg is 71%. For those who are jyka, the probability of lirg is 66%. Is lirg less likely than not lirg overall?\\ \tt Start your answer with 'Yes' or 'No', followed by additional reasoning or evidence to support your explanation. 
	  } 
	   } 
	   } 
	 & \\ 
 

    \theutterance \stepcounter{utterance}  

    & \multicolumn{4}{p{0.6\linewidth}}{\cellcolor[rgb]{0.8,1,0.9}{%
	\makecell[{{p{\linewidth}}}]{% 
	  \tt {\tiny [A$\rangle$GM]}  
	 No\\ \tt \\ \tt To determine whether lirg is less likely than not lirg overall, we need to calculate the overall probability of lirg and compare it to the probability of not lirg. \\ \tt \\ \tt First, let's break down the given probabilities:\\ \tt \\ \tt - The probability of Jyka (J) is 14%.\\ \tt - For those who are not Jyka (not J), the probability of Lirg (L) is 71%.\\ \tt - For those who are Jyka (J), the probability of Lirg (L) is 66%.\\ \tt \\ \tt We can use the law of total probability to calculate the overall probability of Lirg (P(L)). \\ \tt \\ \tt P(L) = P(J) * P(L|J) + P(not J) * P(L|not J)\\ \tt      = 0.14 * 0.66 + (1 - 0.14) * 0.71\\ \tt      = 0.0894 + 0.6166\\ \tt      = 0.706\\ \tt \\ \tt Now, we need to calculate the overall probability of not Lirg (P(not L)). Since P(L) + P(not L) = 1, we can find P(not L) as follows:\\ \tt \\ \tt P(not L) = 1 - P(L)\\ \tt           = 1 - 0.706\\ \tt           = 0.294\\ \tt \\ \tt Since 0.706 (P(L)) is greater than 0.294 (P(not L 
	  } 
	   } 
	   } 
	 & & \\ 
 

    \theutterance \stepcounter{utterance}  

    & & & \multicolumn{2}{p{0.3\linewidth}}{\cellcolor[rgb]{0.95,0.95,0.95}{%
	\makecell[{{p{\linewidth}}}]{% 
	  \tt {\tiny [GM$|$GM]}  
	 No 
	  } 
	   } 
	   } 
	 & & \\ 
 

    \theutterance \stepcounter{utterance}  

    & & & \multicolumn{2}{p{0.3\linewidth}}{\cellcolor[rgb]{0.95,0.95,0.95}{%
	\makecell[{{p{\linewidth}}}]{% 
	  \tt {\tiny [GM$|$GM]}  
	 no 
	  } 
	   } 
	   } 
	 & & \\ 
 

    \theutterance \stepcounter{utterance}  

    & & & \multicolumn{2}{p{0.3\linewidth}}{\cellcolor[rgb]{0.95,0.95,0.95}{%
	\makecell[{{p{\linewidth}}}]{% 
	  \tt {\tiny [GM$|$GM]}  
	 game_result = WIN 
	  } 
	   } 
	   } 
	 & & \\ 
 

\end{supertabular}
}

\end{document}
