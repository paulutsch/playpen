
\documentclass{article}
\usepackage{colortbl}
\usepackage{makecell}
\usepackage{multirow}
\usepackage{supertabular}

\begin{document}

\newcounter{utterance}

\twocolumn

{ \footnotesize  \setcounter{utterance}{1}
\setlength{\tabcolsep}{0pt}
\begin{supertabular}{c@{$\;$}|p{.15\linewidth}@{}p{.15\linewidth}p{.15\linewidth}p{.15\linewidth}p{.15\linewidth}p{.15\linewidth}}

    \# & $\;$A & \multicolumn{4}{c}{Game Master} & $\;\:$B\\
    \hline 

    \theutterance \stepcounter{utterance}  

    & & \multicolumn{4}{p{0.6\linewidth}}{\cellcolor[rgb]{0.9,0.9,0.9}{%
	\makecell[{{p{\linewidth}}}]{% 
	  \tt {\tiny [A$\langle$GM]}  
	 You are an expert in causal inference. The following question is not a typical commonsense query, but rather a meticulously designed question created by a professor specializing in causal inference, intended to assess the students' mastery of the course content.\\ \tt Imagine a self-contained, hypothetical world with only the following conditions, and without any unmentioned factors or causal relationships: Poverty has a direct effect on liking spicy food and cholera. Water company has a direct effect on liking spicy food. Liking spicy food has a direct effect on cholera. Poverty is unobserved. The overall probability of liking spicy food is 41%. For people who do not like spicy food, the probability of cholera contraction is 59%. For people who like spicy food, the probability of cholera contraction is 33%. Is cholera contraction more likely than cholera prevention overall?\\ \tt Start your answer with 'Yes' or 'No', followed by additional reasoning or evidence to support your explanation. 
	  } 
	   } 
	   } 
	 & \\ 
 

    \theutterance \stepcounter{utterance}  

    & \multicolumn{4}{p{0.6\linewidth}}{\cellcolor[rgb]{0.8,1,0.9}{%
	\makecell[{{p{\linewidth}}}]{% 
	  \tt {\tiny [A$\rangle$GM]}  
	 No, cholera contraction is not more likely than cholera prevention overall.\\ \tt \\ \tt To address this question, we need to apply the principles of causal inference, specifically the concept of counterfactuals and potential outcomes. Let's denote the unobserved variable as P (poverty), the variable for liking spicy food as S, and the variable for cholera contraction as C.\\ \tt \\ \tt Given the direct effects:\\ \tt \\ \tt - P → S (Poverty affects liking spicy food)\\ \tt - P → C (Poverty affects cholera)\\ \tt - S → C (Liking spicy food affects cholera)\\ \tt - W → S (Water company affects liking spicy food)\\ \tt \\ \tt We are also given the following probabilities:\\ \tt \\ \tt - P(S) = 0.41 (probability of liking spicy food)\\ \tt - P(C|S') = 0.59 (probability of cholera contraction given not liking spicy food)\\ \tt - P(C|S) = 0.33 (probability of cholera contraction given liking spicy food)\\ \tt \\ \tt To determine whether cholera contraction is more likely than cholera prevention overall, we need to calculate the overall probability of cholera contraction and compare it to the overall probability of cholera prevention.\\ \tt \\ \tt Let's denote the probability of cholera prevention as P(¬C). We can calculate P(¬C) as 1 - P(C).\\ \tt \\ \tt To calculate P(C), we need to consider the potential outcomes for individuals who like spicy food (S) and those who do not like spicy food (S'). We can use 
	  } 
	   } 
	   } 
	 & & \\ 
 

    \theutterance \stepcounter{utterance}  

    & & & \multicolumn{2}{p{0.3\linewidth}}{\cellcolor[rgb]{0.95,0.95,0.95}{%
	\makecell[{{p{\linewidth}}}]{% 
	  \tt {\tiny [GM$|$GM]}  
	 No 
	  } 
	   } 
	   } 
	 & & \\ 
 

    \theutterance \stepcounter{utterance}  

    & & & \multicolumn{2}{p{0.3\linewidth}}{\cellcolor[rgb]{0.95,0.95,0.95}{%
	\makecell[{{p{\linewidth}}}]{% 
	  \tt {\tiny [GM$|$GM]}  
	 no 
	  } 
	   } 
	   } 
	 & & \\ 
 

    \theutterance \stepcounter{utterance}  

    & & & \multicolumn{2}{p{0.3\linewidth}}{\cellcolor[rgb]{0.95,0.95,0.95}{%
	\makecell[{{p{\linewidth}}}]{% 
	  \tt {\tiny [GM$|$GM]}  
	 game_result = WIN 
	  } 
	   } 
	   } 
	 & & \\ 
 

\end{supertabular}
}

\end{document}
