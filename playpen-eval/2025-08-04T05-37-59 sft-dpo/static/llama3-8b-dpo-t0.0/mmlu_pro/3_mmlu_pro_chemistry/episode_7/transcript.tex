
\documentclass{article}
\usepackage{colortbl}
\usepackage{makecell}
\usepackage{multirow}
\usepackage{supertabular}

\begin{document}

\newcounter{utterance}

\twocolumn

{ \footnotesize  \setcounter{utterance}{1}
\setlength{\tabcolsep}{0pt}
\begin{supertabular}{c@{$\;$}|p{.15\linewidth}@{}p{.15\linewidth}p{.15\linewidth}p{.15\linewidth}p{.15\linewidth}p{.15\linewidth}}

    \# & $\;$A & \multicolumn{4}{c}{Game Master} & $\;\:$B\\
    \hline 

    \theutterance \stepcounter{utterance}  

    & & \multicolumn{4}{p{0.6\linewidth}}{\cellcolor[rgb]{0.9,0.9,0.9}{%
	\makecell[{{p{\linewidth}}}]{% 
	  \tt {\tiny [A$\langle$GM]}  
	 The following are multiple choice questions (with answers) about chemistry. Answer only with "The answer is (X)" where X is the correct letter choice.\\ \tt \\ \tt Question:\\ \tt Predict the number of lines in the EPR spectrum of a solution of 13C-labelled methyl radical (13CH3•), assuming the lines do not overlap.\\ \tt Options:\\ \tt A. 10\\ \tt B. 8\\ \tt C. 4\\ \tt D. 20\\ \tt E. 12\\ \tt F. 3\\ \tt G. 16\\ \tt H. 5\\ \tt I. 24\\ \tt J. 6\\ \tt The answer is (B)\\ \tt \\ \tt Question:\\ \tt Which of the following lists the hydrides of group-14 elements in order of thermal stability, from lowest to highest?\\ \tt Options:\\ \tt A. GeH4 < SnH4 < PbH4 < SiH4 < CH4\\ \tt B. SiH4 < GeH4 < SnH4 < PbH4 < CH4\\ \tt C. PbH4 < CH4 < SnH4 < GeH4 < SiH4\\ \tt D. PbH4 < SnH4 < CH4 < GeH4 < SiH4\\ \tt E. SnH4 < GeH4 < SiH4 < PbH4 < CH4\\ \tt F. CH4 < GeH4 < SnH4 < PbH4 < SiH4\\ \tt G. SiH4 < SnH4 < PbH4 < GeH4 < CH4\\ \tt H. CH4 < SiH4 < GeH4 < SnH4 < PbH4\\ \tt I. CH4 < PbH4 < GeH4 < SnH4 < SiH4\\ \tt J. PbH4 < SnH4 < GeH4 < SiH4 < CH4\\ \tt The answer is (J)\\ \tt \\ \tt Question:\\ \tt Which of the following is considered an acid anhydride?\\ \tt Options:\\ \tt A. H2SO3\\ \tt B. NaCl\\ \tt C. CH4\\ \tt D. HNO3\\ \tt E. SO2\\ \tt F. Al(NO3)3\\ \tt G. CaCO3\\ \tt H. H2O\\ \tt I. HCl\\ \tt J. N/A\\ \tt The answer is (E)\\ \tt \\ \tt Question:\\ \tt A new compound is synthesized and found to be a monoprotic acid with a molar mass of 248 g/mol. When 0.0050 mol of this acid are dissolved in 0.500 L of water, the pH is measured as 3.89. What is the pKa of this acid?\\ \tt Options:\\ \tt A. 5.78\\ \tt B. 4.78\\ \tt C. 4.56\\ \tt D. 6.89\\ \tt E. 7.78\\ \tt F. 3.89\\ \tt G. 1.23\\ \tt H. 2.89\\ \tt I. 2.33\\ \tt J. 5.33\\ \tt The answer is (A)\\ \tt \\ \tt Question:\\ \tt A solution contains 2.00 mole of acetic acid, CH3COOH, and 1.00 mole of calcium acetate, Ca(CH3COO)2. The solution is able to resist the addition of a small amount of strong acid or strong base with only minor changes in the pH of the solution. Larger quantities of strong acid or strong base can cause a significant change in pH. How many moles of nitric acid, HNO3, may be added before the pH begins to change significantly?\\ \tt Options:\\ \tt A. 0.250 mole\\ \tt B. 0.500 mole\\ \tt C. 3.00 mole\\ \tt D. 1.00 mole\\ \tt E. 3.50 mole\\ \tt F. 1.50 mole\\ \tt G. 2.50 mole\\ \tt H. 4.00 mole\\ \tt I. 0.750 mole\\ \tt J. 2.00 mole\\ \tt The answer is (J)\\ \tt \\ \tt Question:\\ \tt Determine the H+(aq) concentration in 1.0 M phenol, C6H5OH, solution. (The Ka for phenol is 1 × 10^-10.)\\ \tt Options:\\ \tt A. 1 × 10^-3 M\\ \tt B. 1 × 10^-7 M\\ \tt C. 1 × 10^-4 M\\ \tt D. 1 × 10^-6 M\\ \tt E. 1 × 10^-5 M\\ \tt F. 1 × 10^-8 M\\ \tt G. 1 × 10^-1 M\\ \tt H. 1 × 10^-9 M\\ \tt I. 1 × 10^-2 M\\ \tt J. 1 × 10^-10 M\\ \tt  
	  } 
	   } 
	   } 
	 & \\ 
 

    \theutterance \stepcounter{utterance}  

    & \multicolumn{4}{p{0.6\linewidth}}{\cellcolor[rgb]{0.8,1,0.9}{%
	\makecell[{{p{\linewidth}}}]{% 
	  \tt {\tiny [A$\rangle$GM]}  
	 The answer is (H) 
	  } 
	   } 
	   } 
	 & & \\ 
 

    \theutterance \stepcounter{utterance}  

    & & & \multicolumn{2}{p{0.3\linewidth}}{\cellcolor[rgb]{0.95,0.95,0.95}{%
	\makecell[{{p{\linewidth}}}]{% 
	  \tt {\tiny [GM$|$GM]}  
	 H 
	  } 
	   } 
	   } 
	 & & \\ 
 

    \theutterance \stepcounter{utterance}  

    & & & \multicolumn{2}{p{0.3\linewidth}}{\cellcolor[rgb]{0.95,0.95,0.95}{%
	\makecell[{{p{\linewidth}}}]{% 
	  \tt {\tiny [GM$|$GM]}  
	 E 
	  } 
	   } 
	   } 
	 & & \\ 
 

    \theutterance \stepcounter{utterance}  

    & & & \multicolumn{2}{p{0.3\linewidth}}{\cellcolor[rgb]{0.95,0.95,0.95}{%
	\makecell[{{p{\linewidth}}}]{% 
	  \tt {\tiny [GM$|$GM]}  
	 game_result = LOSE 
	  } 
	   } 
	   } 
	 & & \\ 
 

\end{supertabular}
}

\end{document}
