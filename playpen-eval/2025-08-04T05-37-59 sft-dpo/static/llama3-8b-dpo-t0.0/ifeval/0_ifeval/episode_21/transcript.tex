
\documentclass{article}
\usepackage{colortbl}
\usepackage{makecell}
\usepackage{multirow}
\usepackage{supertabular}

\begin{document}

\newcounter{utterance}

\twocolumn

{ \footnotesize  \setcounter{utterance}{1}
\setlength{\tabcolsep}{0pt}
\begin{supertabular}{c@{$\;$}|p{.15\linewidth}@{}p{.15\linewidth}p{.15\linewidth}p{.15\linewidth}p{.15\linewidth}p{.15\linewidth}}

    \# & $\;$A & \multicolumn{4}{c}{Game Master} & $\;\:$B\\
    \hline 

    \theutterance \stepcounter{utterance}  

    & & \multicolumn{4}{p{0.6\linewidth}}{\cellcolor[rgb]{0.9,0.9,0.9}{%
	\makecell[{{p{\linewidth}}}]{% 
	  \tt {\tiny [A$\langle$GM]}  
	 What are the uses of poppy seeds? Your answer should have exactly 7 paragraphs and the last paragraph must start with the word "Summary". Each paragraph should be separated by two new lines. 
	  } 
	   } 
	   } 
	 & \\ 
 

    \theutterance \stepcounter{utterance}  

    & \multicolumn{4}{p{0.6\linewidth}}{\cellcolor[rgb]{0.8,1,0.9}{%
	\makecell[{{p{\linewidth}}}]{% 
	  \tt {\tiny [A$\rangle$GM]}  
	 Poppy seeds are derived from the opium poppy plant (Papaver somniferum) and have been used for various purposes for thousands of years. They are a rich source of nutrients, including protein, fiber, and minerals like calcium, iron, and zinc. The seeds are also a good source of healthy fats and contain antioxidants that can help protect against cell damage and inflammation.\\ \tt \\ \tt \\ \tt The most common use of poppy seeds is as a spice or ingredient in baking. They are often used to add flavor and texture to bread, cakes, and pastries. Poppy seeds are particularly popular in Eastern European and Middle Eastern cuisine, where they are used to make traditional dishes like bagels, bialys, and strudels. They can also be used as a topping for yogurt, oatmeal, or salads.\\ \tt \\ \tt \\ \tt In addition to their culinary uses, poppy seeds have also been used for medicinal purposes. The seeds contain a compound called codeine, which has been used to treat pain and coughs. However, it's worth noting that the amount of codeine in poppy seeds is relatively small, and they are not a reliable source of this medication. Poppy seeds have also been used to treat digestive issues, such as constipation and diarrhea.\\ \tt \\ \tt \\ \tt Poppy seeds have also been used in traditional medicine for their anti-inflammatory and antioxidant properties. They have been used to treat a range of conditions, including arthritis, skin conditions, and respiratory problems. Some studies have also suggested that poppy seeds may have a role in reducing the risk of certain diseases, such as heart disease and cancer.\\ \tt \\ \tt \\ \tt In some cultures, poppy seeds are used as a natural remedy for anxiety and insomnia. The seeds contain a compound called thebaine, which has been shown to have a sedative effect. However, it's worth noting that the amount of thebaine in poppy seeds is relatively small, and they are not a reliable source of this medication.\\ \tt \\ \tt \\ \tt Poppy seeds are also used in traditional rituals and ceremonies. In some cultures, the seeds are used as a symbol of fertility and prosperity. They are often used in wedding ceremonies and other celebrations to bring good luck and happiness.\\ \tt \\ \tt \\ \tt Summary: In conclusion, poppy seeds have a range of uses, from culinary and medicinal to traditional and cultural. While they are not a reliable source of certain medications, they do contain a range of nutrients and compounds that may have health benefits. As with any food or ingredient, it's essential to use poppy seeds in moderation and to be aware of their potential effects. 
	  } 
	   } 
	   } 
	 & & \\ 
 

    \theutterance \stepcounter{utterance}  

    & & & \multicolumn{2}{p{0.3\linewidth}}{\cellcolor[rgb]{0.95,0.95,0.95}{%
	\makecell[{{p{\linewidth}}}]{% 
	  \tt {\tiny [GM$|$GM]}  
	 {'length_constraints:nth_paragraph_first_word': {'num_paragraphs': 7, 'first_word': 'summary', 'nth_paragraph': 7}} 
	  } 
	   } 
	   } 
	 & & \\ 
 

    \theutterance \stepcounter{utterance}  

    & & & \multicolumn{2}{p{0.3\linewidth}}{\cellcolor[rgb]{0.95,0.95,0.95}{%
	\makecell[{{p{\linewidth}}}]{% 
	  \tt {\tiny [GM$|$GM]}  
	 game_result = LOSE 
	  } 
	   } 
	   } 
	 & & \\ 
 

\end{supertabular}
}

\end{document}
