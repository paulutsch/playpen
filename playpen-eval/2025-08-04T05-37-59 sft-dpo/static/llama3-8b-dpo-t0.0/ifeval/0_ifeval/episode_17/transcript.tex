
\documentclass{article}
\usepackage{colortbl}
\usepackage{makecell}
\usepackage{multirow}
\usepackage{supertabular}

\begin{document}

\newcounter{utterance}

\twocolumn

{ \footnotesize  \setcounter{utterance}{1}
\setlength{\tabcolsep}{0pt}
\begin{supertabular}{c@{$\;$}|p{.15\linewidth}@{}p{.15\linewidth}p{.15\linewidth}p{.15\linewidth}p{.15\linewidth}p{.15\linewidth}}

    \# & $\;$A & \multicolumn{4}{c}{Game Master} & $\;\:$B\\
    \hline 

    \theutterance \stepcounter{utterance}  

    & & \multicolumn{4}{p{0.6\linewidth}}{\cellcolor[rgb]{0.9,0.9,0.9}{%
	\makecell[{{p{\linewidth}}}]{% 
	  \tt {\tiny [A$\langle$GM]}  
	 The Jimenez family has 20 members. Each member has at least one pet. If there are 32 pets in total, what is the maximum number of members that can possibly have 3 pets? Solve this step by step, and wrap your entire response (reasoning process and the final answer) with double quotation marks. 
	  } 
	   } 
	   } 
	 & \\ 
 

    \theutterance \stepcounter{utterance}  

    & \multicolumn{4}{p{0.6\linewidth}}{\cellcolor[rgb]{0.8,1,0.9}{%
	\makecell[{{p{\linewidth}}}]{% 
	  \tt {\tiny [A$\rangle$GM]}  
	 "Let's break down the problem step by step:\\ \tt \\ \tt 1. The total number of members in the Jimenez family is 20.\\ \tt 2. Each member has at least one pet, and there are 32 pets in total.\\ \tt 3. We want to find the maximum number of members that can possibly have 3 pets.\\ \tt \\ \tt Let's assume that 'x' members have 3 pets, 'y' members have 2 pets, and 'z' members have 1 pet. Since there are 20 members in total, we can write the equation:\\ \tt x + y + z = 20\\ \tt \\ \tt We also know that the total number of pets is 32. Since each pet belongs to one member, we can write the equation:\\ \tt 3x + 2y + z = 32\\ \tt \\ \tt Now, we want to maximize the value of 'x', which represents the number of members with 3 pets. To do this, we need to minimize the values of 'y' and 'z'.\\ \tt \\ \tt Since each member has at least one pet, 'z' cannot be 0. Let's assume 'z' is 1, which means one member has only 1 pet. This is the minimum possible value for 'z'.\\ \tt \\ \tt Now, we can rewrite the first equation as:\\ \tt x + y + 1 = 20\\ \tt x + y = 19\\ \tt \\ \tt Substituting 'z' = 1 into the second equation, we get:\\ \tt 3x + 2y + 1 = 32\\ \tt 3x + 2y = 31\\ \tt \\ \tt Now we have two equations:\\ \tt x + y = 19\\ \tt 3x + 2y = 31\\ \tt \\ \tt We can solve these equations by multiplying the first equation by 2 and subtracting it from the second equation:\\ \tt (3x + 2y) - 2(x + y) = 31 - 2(19)\\ \tt 3x + 2y - 2x - 2y = 31 - 38\\ \tt x = -7\\ \tt \\ \tt However, 'x' cannot be negative, since it represents the number of members. This means our assumption of 'z' = 1 is not valid.\\ \tt \\ \tt Let's try 'z' = 2, which means two members have only 1 pet. This is the next possible value for 'z'.\\ \tt \\ \tt Now, we can rewrite the first equation as:\\ \tt x + y + 2 = 20\\ \tt x + y = 18\\ \tt \\ \tt Substituting 'z' = 2 into the second equation, we get:\\ \tt 3x + 2y + 2 = 32\\ \tt 3x + 2y = 30\\ \tt \\ \tt Now we have two equations:\\ \tt x + y = 18\\ \tt 3x + 2y = 30\\ \tt \\ \tt We can solve these equations by multiplying the first equation by 2 and subtracting it from the second equation:\\ \tt (3x + 2y) - 2(x + y) = 30 - 2(18)\\ \tt 3x + 2y - 2x - 2y = 30 - 36\\ \tt x = -6\\ \tt \\ \tt However, 'x' cannot be negative, since it represents the number of members. This means our assumption of 'z' = 2 is not valid.\\ \tt \\ \tt Let's try 'z' = 3, which means three members have only 1 pet. This is the next possible value for 'z'.\\ \tt \\ \tt Now, we can rewrite the first equation as:\\ \tt x + y + 3 = 20\\ \tt x + y = 17\\ \tt \\ \tt Substituting 'z' = 3 into the second equation, we get:\\ \tt 3x + 2y + 3 = 32\\ \tt 3x + 2y = 29\\ \tt \\ \tt Now we have two equations:\\ \tt x + y = 17\\ \tt 3x + 2y = 29\\ \tt \\ \tt We can solve these equations by multiplying the first equation by 2 and subtracting it from the second equation:\\ \tt (3x + 2y) - 2(x + y) = 29 - 2(17)\\ \tt 3x + 2y - 2x - 2y = 29 - 34\\ \tt x = -5\\ \tt \\ \tt However, 'x' cannot be negative, since it represents the number of members. This means our assumption of 'z' = 3 is not valid.\\ \tt \\ \tt Let's try 'z' = 4, which means four members have only 1 pet. This is the next possible value for 'z'.\\ \tt \\ \tt Now, we can rewrite the first equation as:\\ \tt x + y + 4 = 20\\ \tt x + y = 16\\ \tt \\ \tt Substituting 'z' = 4 into the second equation, we get:\\ \tt 3x + 2y + 4 = 32\\ \tt 3x + 2y = 28\\ \tt \\ \tt Now we have two equations:\\ \tt x + y = 16\\ \tt 3x + 2y = 28\\ \tt \\ \tt We can solve these equations by multiplying the first equation by 2 and subtracting it from the second equation:\\ \tt (3x + 2y) - 2(x + y) = 28 - 2(16)\\ \tt 3x + 2y - 2x - 2y = 28 - 32\\ \tt x = -4\\ \tt \\ \tt However, 'x' cannot be negative, since it represents the number of members. This means our assumption of 'z' = 4 is not valid.\\ \tt \\ \tt Let's try 'z' = 5, which means five members have only 1 pet. This is the next possible value for 'z'.\\ \tt \\ \tt Now, we can rewrite the first equation as:\\ \tt x + y + 5 = 20\\ \tt x + y = 15\\ \tt \\ \tt Substituting 'z' = 5 into the second equation, we get:\\ \tt 3x + 2y + 5 = 32\\ \tt 3x + 2y = 27\\ \tt \\ \tt Now we have two equations:\\ \tt x + y = 15\\ \tt 3x + 2y = 27\\ \tt \\ \tt We can solve these equations by multiplying the first equation by 2 and subtracting it from the second equation:\\ \tt (3x + 2y) - 2(x + y) = 27 - 2(15) 
	  } 
	   } 
	   } 
	 & & \\ 
 

    \theutterance \stepcounter{utterance}  

    & & & \multicolumn{2}{p{0.3\linewidth}}{\cellcolor[rgb]{0.95,0.95,0.95}{%
	\makecell[{{p{\linewidth}}}]{% 
	  \tt {\tiny [GM$|$GM]}  
	 {'startend:quotation': {}} 
	  } 
	   } 
	   } 
	 & & \\ 
 

    \theutterance \stepcounter{utterance}  

    & & & \multicolumn{2}{p{0.3\linewidth}}{\cellcolor[rgb]{0.95,0.95,0.95}{%
	\makecell[{{p{\linewidth}}}]{% 
	  \tt {\tiny [GM$|$GM]}  
	 game_result = LOSE 
	  } 
	   } 
	   } 
	 & & \\ 
 

\end{supertabular}
}

\end{document}
