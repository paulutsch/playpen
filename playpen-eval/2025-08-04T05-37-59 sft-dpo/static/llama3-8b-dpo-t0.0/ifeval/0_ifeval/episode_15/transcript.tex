
\documentclass{article}
\usepackage{colortbl}
\usepackage{makecell}
\usepackage{multirow}
\usepackage{supertabular}

\begin{document}

\newcounter{utterance}

\twocolumn

{ \footnotesize  \setcounter{utterance}{1}
\setlength{\tabcolsep}{0pt}
\begin{supertabular}{c@{$\;$}|p{.15\linewidth}@{}p{.15\linewidth}p{.15\linewidth}p{.15\linewidth}p{.15\linewidth}p{.15\linewidth}}

    \# & $\;$A & \multicolumn{4}{c}{Game Master} & $\;\:$B\\
    \hline 

    \theutterance \stepcounter{utterance}  

    & & \multicolumn{4}{p{0.6\linewidth}}{\cellcolor[rgb]{0.9,0.9,0.9}{%
	\makecell[{{p{\linewidth}}}]{% 
	  \tt {\tiny [A$\langle$GM]}  
	 Write an interesting and funny article about the biology of a banana peel. In your response, the word disappointed should appear at least 2 times, and at least six section should be highlighted with markdown,  i.e *banana peel*. 
	  } 
	   } 
	   } 
	 & \\ 
 

    \theutterance \stepcounter{utterance}  

    & \multicolumn{4}{p{0.6\linewidth}}{\cellcolor[rgb]{0.8,1,0.9}{%
	\makecell[{{p{\linewidth}}}]{% 
	  \tt {\tiny [A$\rangle$GM]}  
	 **The Slippery Truth About Banana Peels**\\ \tt \\ \tt As we peel back the layers (pun intended) of the humble banana, we often overlook the fascinating biology behind its discarded skin. But fear not, dear readers, for we're about to dive into the juicy world of *banana peel* biology and uncover the secrets that lie beneath.\\ \tt \\ \tt ### **The Evolutionary Advantage of a Slippery Skin**\\ \tt \\ \tt Banana peels are notorious for their slipperiness, and it's not just because they're a favorite among monkeys. In reality, the peel's slick surface is a result of a clever evolutionary adaptation. The high concentration of starch and cellulose in the peel creates a hydrophobic (water-repelling) coating that helps the banana stay fresh for longer. It's like nature's own non-stick pan. But don't be disappointed – this clever trick doesn't make the peel any less prone to slipping out of your hands.\\ \tt \\ \tt ### **The Peel's Protective Layer**\\ \tt \\ \tt The *banana peel* is more than just a tasty snack for your average fruit fly. It's a protective barrier that shields the delicate fruit beneath from the harsh outside world. The peel's thick, waxy coating helps to prevent water loss and damage from UV radiation, keeping the banana fresh and healthy. But don't be too impressed – this protective layer is no match for the ravages of time, and your banana will eventually turn brown and mushy. Sorry, not sorry.\\ \tt \\ \tt ### **The Amazing World of Cellulose**\\ \tt \\ \tt Cellulose is the primary component of plant cell walls, and the *banana peel* is no exception. This tough, fibrous material provides structure and support to the peel, allowing it to withstand the rigors of daily life (i.e., being handled by humans and monkeys alike). But don't be disappointed – cellulose is also responsible for the peel's infamous slipperiness. It's like nature's own Teflon coating.\\ \tt \\ \tt ### **The Peel's Role in Fruit Development**\\ \tt \\ \tt As the banana grows, the *banana peel* plays a crucial role in its development. The peel helps to regulate the fruit's water content, preventing it from becoming too dry or too wet. It's like a built-in thermostat, keeping the banana at the perfect temperature for optimal ripening. But don't be too excited – this process is slow and laborious, and your banana will eventually turn from green to yellow to brown. Sorry, not sorry.\\ \tt \\ \tt ### **The Environmental Impact of Banana Peels**\\ \tt \\ \tt Banana peels are a significant contributor to the world's waste problem, with millions of tons ending up in landfills each year. But don't be disappointed – this problem has a solution. Banana peels can be composted, reducing the need for synthetic fertilizers and creating a nutrient-rich soil amendment. It's like a win-win situation, except for the fact that you'll still have to deal with the pesky peel itself.\\ \tt \\ \tt ### **Conclusion: The Slippery Truth**\\ \tt \\ \tt In conclusion, the *banana peel* is more than just a tasty snack or a pesky waste product. It's a fascinating example of evolutionary adaptation, a protective barrier, and a crucial component of fruit development. So next time you're tempted to throw away that banana peel, remember the amazing biology that lies beneath. And don't be disappointed – it's still a slippery mess. 
	  } 
	   } 
	   } 
	 & & \\ 
 

    \theutterance \stepcounter{utterance}  

    & & & \multicolumn{2}{p{0.3\linewidth}}{\cellcolor[rgb]{0.95,0.95,0.95}{%
	\makecell[{{p{\linewidth}}}]{% 
	  \tt {\tiny [GM$|$GM]}  
	 {'keywords:frequency': {'relation': 'at least', 'keyword': 'disappointed', 'frequency': 2}, 'detectable_format:number_highlighted_sections': {'num_highlights': 6}} 
	  } 
	   } 
	   } 
	 & & \\ 
 

    \theutterance \stepcounter{utterance}  

    & & & \multicolumn{2}{p{0.3\linewidth}}{\cellcolor[rgb]{0.95,0.95,0.95}{%
	\makecell[{{p{\linewidth}}}]{% 
	  \tt {\tiny [GM$|$GM]}  
	 game_result = WIN 
	  } 
	   } 
	   } 
	 & & \\ 
 

\end{supertabular}
}

\end{document}
