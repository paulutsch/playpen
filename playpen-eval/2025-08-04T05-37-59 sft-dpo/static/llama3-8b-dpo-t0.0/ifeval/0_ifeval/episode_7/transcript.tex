
\documentclass{article}
\usepackage{colortbl}
\usepackage{makecell}
\usepackage{multirow}
\usepackage{supertabular}

\begin{document}

\newcounter{utterance}

\twocolumn

{ \footnotesize  \setcounter{utterance}{1}
\setlength{\tabcolsep}{0pt}
\begin{supertabular}{c@{$\;$}|p{.15\linewidth}@{}p{.15\linewidth}p{.15\linewidth}p{.15\linewidth}p{.15\linewidth}p{.15\linewidth}}

    \# & $\;$A & \multicolumn{4}{c}{Game Master} & $\;\:$B\\
    \hline 

    \theutterance \stepcounter{utterance}  

    & & \multicolumn{4}{p{0.6\linewidth}}{\cellcolor[rgb]{0.9,0.9,0.9}{%
	\makecell[{{p{\linewidth}}}]{% 
	  \tt {\tiny [A$\langle$GM]}  
	 Take the text below as a starting point, and make it a complete article: "You may have to meet with a helper to work out a parenting plan. The first would be to estimate how many time you have everyday for parenting, and is that enough...."\\ \tt Avoid using the following keywords: sleep, cook, feed\\ \tt Mention the keyword 'schedule' for more than 5 times. 
	  } 
	   } 
	   } 
	 & \\ 
 

    \theutterance \stepcounter{utterance}  

    & \multicolumn{4}{p{0.6\linewidth}}{\cellcolor[rgb]{0.8,1,0.9}{%
	\makecell[{{p{\linewidth}}}]{% 
	  \tt {\tiny [A$\rangle$GM]}  
	 **Creating a Parenting Plan: Understanding Your Daily Schedule**\\ \tt \\ \tt When parents decide to co-parent, it's essential to establish a parenting plan that works for everyone involved. This plan outlines how you will share responsibilities, make decisions, and communicate with each other about your child's care. You may have to meet with a helper, such as a mediator or counselor, to work out the details of your parenting plan. The first step in this process is to estimate how many hours you have each day for parenting and whether that's enough to meet your child's needs.\\ \tt \\ \tt **Assessing Your Daily Schedule**\\ \tt \\ \tt To create a realistic parenting plan, you need to understand your daily schedule. This includes your work hours, personal time, and any other commitments you may have. You should also consider your child's schedule, including their school hours, extracurricular activities, and any other regular appointments. By mapping out your daily schedule, you can identify potential conflicts and areas where you may need to adjust your plan.\\ \tt \\ \tt **Estimating Parenting Time**\\ \tt \\ \tt Once you have a clear understanding of your daily schedule, you can estimate how many hours you have available for parenting. This will help you determine whether you have enough time to meet your child's needs and whether you need to make adjustments to your schedule. For example, if you work long hours, you may need to adjust your schedule to ensure that you have enough time for parenting responsibilities.\\ \tt \\ \tt **Creating a Parenting Schedule**\\ \tt \\ \tt A parenting schedule is a crucial component of your parenting plan. It outlines when each parent will have custody of the child and how you will share responsibilities. Your schedule should be flexible and take into account any changes that may occur in your daily routine. By creating a schedule that works for everyone, you can reduce conflict and ensure that your child receives the care they need.\\ \tt \\ \tt **Tips for Creating a Parenting Schedule**\\ \tt \\ \tt 1. **Be realistic**: When creating your parenting schedule, be realistic about your availability and your child's needs.\\ \tt 2. **Communicate with your co-parent**: Make sure to communicate with your co-parent about your schedule and any changes that may occur.\\ \tt 3. **Be flexible**: Your parenting schedule should be flexible and able to adapt to changes in your daily routine.\\ \tt 4. **Prioritize your child's needs**: Your parenting schedule should prioritize your child's needs and ensure that they receive the care they require.\\ \tt 5. **Review and revise**: Regularly review and revise your parenting schedule to ensure that it continues to meet your child's needs.\\ \tt \\ \tt **Benefits of a Parenting Schedule**\\ \tt \\ \tt A well-planned parenting schedule can have numerous benefits for your child and your co-parenting relationship. Some of the benefits include:\\ \tt \\ \tt * Reduced conflict: A clear parenting schedule can reduce conflict and ensure that both parents are on the same page.\\ \tt * Improved communication: A parenting schedule can improve communication between co-parents and ensure that both parents are aware of their responsibilities.\\ \tt * Increased stability: A parenting schedule can provide stability for your child and ensure that they receive the care they need.\\ \tt \\ \tt **Conclusion**\\ \tt \\ \tt Creating a parenting plan and schedule is an essential step in co-parenting. By understanding your daily schedule, estimating your parenting time, and creating a parenting schedule, you can ensure that your child receives the care they need and that you and your co-parent are on the same page. Remember to be realistic, communicate with your co-parent, and prioritize your child's needs when creating your parenting schedule. 
	  } 
	   } 
	   } 
	 & & \\ 
 

    \theutterance \stepcounter{utterance}  

    & & & \multicolumn{2}{p{0.3\linewidth}}{\cellcolor[rgb]{0.95,0.95,0.95}{%
	\makecell[{{p{\linewidth}}}]{% 
	  \tt {\tiny [GM$|$GM]}  
	 {'keywords:forbidden_words': {'forbidden_words': ['sleep', 'cook', 'feed']}, 'keywords:frequency': {'relation': 'at least', 'keyword': 'schedule', 'frequency': 6}} 
	  } 
	   } 
	   } 
	 & & \\ 
 

    \theutterance \stepcounter{utterance}  

    & & & \multicolumn{2}{p{0.3\linewidth}}{\cellcolor[rgb]{0.95,0.95,0.95}{%
	\makecell[{{p{\linewidth}}}]{% 
	  \tt {\tiny [GM$|$GM]}  
	 game_result = WIN 
	  } 
	   } 
	   } 
	 & & \\ 
 

\end{supertabular}
}

\end{document}
