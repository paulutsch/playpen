
\documentclass{article}
\usepackage{colortbl}
\usepackage{makecell}
\usepackage{multirow}
\usepackage{supertabular}

\begin{document}

\newcounter{utterance}

\twocolumn

{ \footnotesize  \setcounter{utterance}{1}
\setlength{\tabcolsep}{0pt}
\begin{supertabular}{c@{$\;$}|p{.15\linewidth}@{}p{.15\linewidth}p{.15\linewidth}p{.15\linewidth}p{.15\linewidth}p{.15\linewidth}}

    \# & $\;$A & \multicolumn{4}{c}{Game Master} & $\;\:$B\\
    \hline 

    \theutterance \stepcounter{utterance}  

    & & \multicolumn{4}{p{0.6\linewidth}}{\cellcolor[rgb]{0.9,0.9,0.9}{%
	\makecell[{{p{\linewidth}}}]{% 
	  \tt {\tiny [A$\langle$GM]}  
	 Write a casual blog post about how the outer solar system is different from the inner solar system, and what that means for the possibility of life. Wrap your entire response with double quotation marks. Your response should contain 17 or more sentences. 
	  } 
	   } 
	   } 
	 & \\ 
 

    \theutterance \stepcounter{utterance}  

    & \multicolumn{4}{p{0.6\linewidth}}{\cellcolor[rgb]{0.8,1,0.9}{%
	\makecell[{{p{\linewidth}}}]{% 
	  \tt {\tiny [A$\rangle$GM]}  
	 "Exploring the vastness of our solar system, we often find ourselves comparing the inner and outer regions. While the inner solar system is home to the rocky planets like Earth, Mercury, and Mars, the outer solar system is a realm of gas giants and icy worlds. The most notable difference between the two is the temperature and composition of the planets.\\ \tt \\ \tt In the inner solar system, the temperatures are relatively warm, and the planets are composed of rock and metal. This is due to the proximity to the sun, which provides the necessary heat and energy for these planets to form and sustain life. On the other hand, the outer solar system is a cold and dark place, with temperatures plummeting to -200 degrees Celsius. The planets in this region are primarily composed of gas and ice, with Jupiter and Saturn being the most prominent examples.\\ \tt \\ \tt The gas giants in the outer solar system are massive and have strong gravitational pull, which can trap heat and create conditions suitable for life. However, the lack of solid surface and the intense radiation from the sun make it challenging for life to thrive. The icy moons of these gas giants, such as Europa and Enceladus, are of particular interest to astrobiologists. These moons have liquid water beneath their icy crusts, which is a crucial ingredient for life.\\ \tt \\ \tt The possibility of life in the outer solar system is not entirely ruled out, but it's certainly more challenging than in the inner solar system. The extreme conditions and lack of sunlight make it difficult for life to emerge and sustain itself. However, the discovery of exoplanets and the study of our own solar system's moons have opened up new avenues for research and exploration.\\ \tt \\ \tt The outer solar system is also home to a variety of small, icy bodies known as Kuiper Belt Objects (KBOs) and Oort Cloud objects. These objects are remnants from the formation of the solar system and provide valuable insights into its early history. The study of these objects has also led to the discovery of new worlds and the possibility of life beyond our solar system.\\ \tt \\ \tt In conclusion, the outer solar system is a vastly different place from the inner solar system, with its own unique characteristics and challenges. While the possibility of life in the outer solar system is intriguing, it's still a topic of ongoing research and debate. As we continue to explore and study our solar system, we may uncover new evidence that sheds light on the possibility of life beyond Earth." 
	  } 
	   } 
	   } 
	 & & \\ 
 

    \theutterance \stepcounter{utterance}  

    & & & \multicolumn{2}{p{0.3\linewidth}}{\cellcolor[rgb]{0.95,0.95,0.95}{%
	\makecell[{{p{\linewidth}}}]{% 
	  \tt {\tiny [GM$|$GM]}  
	 {'startend:quotation': {}, 'length_constraints:number_sentences': {'relation': 'at least', 'num_sentences': 17}} 
	  } 
	   } 
	   } 
	 & & \\ 
 

    \theutterance \stepcounter{utterance}  

    & & & \multicolumn{2}{p{0.3\linewidth}}{\cellcolor[rgb]{0.95,0.95,0.95}{%
	\makecell[{{p{\linewidth}}}]{% 
	  \tt {\tiny [GM$|$GM]}  
	 game_result = WIN 
	  } 
	   } 
	   } 
	 & & \\ 
 

\end{supertabular}
}

\end{document}
