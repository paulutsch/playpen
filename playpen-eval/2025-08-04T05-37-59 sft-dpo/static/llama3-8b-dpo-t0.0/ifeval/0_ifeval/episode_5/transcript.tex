
\documentclass{article}
\usepackage{colortbl}
\usepackage{makecell}
\usepackage{multirow}
\usepackage{supertabular}

\begin{document}

\newcounter{utterance}

\twocolumn

{ \footnotesize  \setcounter{utterance}{1}
\setlength{\tabcolsep}{0pt}
\begin{supertabular}{c@{$\;$}|p{.15\linewidth}@{}p{.15\linewidth}p{.15\linewidth}p{.15\linewidth}p{.15\linewidth}p{.15\linewidth}}

    \# & $\;$A & \multicolumn{4}{c}{Game Master} & $\;\:$B\\
    \hline 

    \theutterance \stepcounter{utterance}  

    & & \multicolumn{4}{p{0.6\linewidth}}{\cellcolor[rgb]{0.9,0.9,0.9}{%
	\makecell[{{p{\linewidth}}}]{% 
	  \tt {\tiny [A$\langle$GM]}  
	 Write an essay about how aluminium cans are used in food storage. Don’t forget to include the keywords waste, material and meal. Have more than 30 sentences in your response. 
	  } 
	   } 
	   } 
	 & \\ 
 

    \theutterance \stepcounter{utterance}  

    & \multicolumn{4}{p{0.6\linewidth}}{\cellcolor[rgb]{0.8,1,0.9}{%
	\makecell[{{p{\linewidth}}}]{% 
	  \tt {\tiny [A$\rangle$GM]}  
	 The use of aluminium cans in food storage has become a ubiquitous practice in the modern world. These lightweight, corrosion-resistant containers have revolutionized the way we store and consume food, making it easier to preserve and transport a wide variety of products. From beverages to soups, and from fruits to meats, aluminium cans have become an essential component of the food industry.\\ \tt \\ \tt One of the primary advantages of aluminium cans is their ability to prevent spoilage and extend the shelf life of food products. By sealing the contents within a sterile environment, aluminium cans prevent the entry of air, moisture, and contaminants, thereby reducing the risk of spoilage and foodborne illnesses. This is particularly important for perishable items such as meats, dairy products, and fruits, which are highly susceptible to spoilage.\\ \tt \\ \tt Aluminium cans are also incredibly versatile, making them suitable for a wide range of food products. From soft drinks and juices to soups and stews, and from infant formula to pet food, aluminium cans can accommodate a vast array of products. Their ability to be easily sterilized and filled with a variety of contents makes them an ideal choice for food manufacturers.\\ \tt \\ \tt In addition to their practical uses, aluminium cans have also become a popular choice for consumers due to their convenience and portability. Aluminium cans are lightweight, making them easy to carry and transport, and their compact size allows for easy storage in pantries and cupboards. This convenience has led to a significant increase in the use of aluminium cans for meal preparation and consumption.\\ \tt \\ \tt However, the widespread use of aluminium cans has also raised concerns about waste and environmental impact. The production and disposal of aluminium cans contribute to greenhouse gas emissions, pollution, and waste management issues. The extraction and processing of aluminium require significant amounts of energy, water, and other resources, resulting in a substantial carbon footprint.\\ \tt \\ \tt Furthermore, the disposal of aluminium cans poses a significant challenge. While aluminium is a recyclable material, many cans are not recycled, resulting in a substantial amount of waste. In the United States alone, it is estimated that over 100 billion aluminium cans are discarded each year, with only a small percentage being recycled. This has led to concerns about the environmental impact of aluminium can production and disposal.\\ \tt \\ \tt To mitigate these concerns, many manufacturers and consumers are turning to sustainable practices and alternatives to traditional aluminium cans. Some companies are using biodegradable materials, such as plant-based bioplastics, to create compostable containers. Others are exploring the use of reusable containers and refill systems to reduce waste and minimize the need for single-use aluminium cans.\\ \tt \\ \tt In addition to these alternatives, many consumers are also taking steps to reduce their use of aluminium cans. By choosing products packaged in glass or cartons, consumers can reduce their reliance on single-use containers and minimize their contribution to waste. Others are opting for meal planning and preparation methods that reduce the need for pre-packaged foods and aluminium cans.\\ \tt \\ \tt In conclusion, aluminium cans have become an essential component of the food industry, providing a convenient and effective means of storing and transporting a wide variety of food products. However, the widespread use of aluminium cans has also raised concerns about waste and environmental impact. By exploring sustainable practices and alternatives to traditional aluminium cans, manufacturers and consumers can work together to reduce waste and minimize the environmental impact of food storage and consumption.\\ \tt \\ \tt The use of aluminium cans in food storage has also had a significant impact on the way we prepare and consume meals. With the convenience of pre-packaged foods and aluminium cans, consumers can easily prepare a meal on the go or in the comfort of their own homes. This has led to a shift towards more convenient and portable meal options, with many consumers opting for pre-packaged meals and snacks.\\ \tt \\ \tt However, this shift towards convenience has also led to concerns about the nutritional value and quality of pre-packaged foods. Many pre-packaged meals and snacks are high in salt, sugar, and unhealthy fats, contributing to a range of health problems, including obesity, diabetes, and heart disease. By choosing whole, unprocessed foods and preparing meals from scratch, consumers can reduce their reliance on pre-packaged foods and aluminium cans, and promote healthier eating habits.\\ \tt \\ \tt In addition to their impact on meal preparation and consumption, aluminium cans have also had a significant impact on the food industry as a whole. The widespread use of aluminium cans has led to a shift towards more efficient and cost-effective food production methods, with many manufacturers opting for automated filling and packaging systems. This has resulted in significant cost savings and increased productivity, allowing manufacturers to produce a wider range of products and meet growing demand.\\ \tt \\ \tt However, this shift towards automation and efficiency has also raised concerns about the impact on workers and communities. Many manufacturing facilities have been relocated to low-wage countries, resulting in job losses and economic disruption in local communities. By prioritizing sustainability and social responsibility, manufacturers can work to mitigate these impacts and promote more equitable and sustainable food production practices.\\ \tt \\ \tt In conclusion, the use of aluminium cans in food storage has had a profound impact on the way we prepare and consume meals, and the food industry as a whole. While aluminium cans provide a convenient and effective means of storing and transporting food products, their widespread use has also raised concerns about waste and environmental impact. By exploring sustainable practices and alternatives to traditional aluminium cans, manufacturers and consumers can work together to reduce waste and minimize the environmental impact of food storage and consumption. 
	  } 
	   } 
	   } 
	 & & \\ 
 

    \theutterance \stepcounter{utterance}  

    & & & \multicolumn{2}{p{0.3\linewidth}}{\cellcolor[rgb]{0.95,0.95,0.95}{%
	\makecell[{{p{\linewidth}}}]{% 
	  \tt {\tiny [GM$|$GM]}  
	 {'keywords:existence': {'keywords': ['waste', 'material', 'meal']}, 'length_constraints:number_sentences': {'relation': 'at least', 'num_sentences': 31}} 
	  } 
	   } 
	   } 
	 & & \\ 
 

    \theutterance \stepcounter{utterance}  

    & & & \multicolumn{2}{p{0.3\linewidth}}{\cellcolor[rgb]{0.95,0.95,0.95}{%
	\makecell[{{p{\linewidth}}}]{% 
	  \tt {\tiny [GM$|$GM]}  
	 game_result = WIN 
	  } 
	   } 
	   } 
	 & & \\ 
 

\end{supertabular}
}

\end{document}
