
\documentclass{article}
\usepackage{colortbl}
\usepackage{makecell}
\usepackage{multirow}
\usepackage{supertabular}

\begin{document}

\newcounter{utterance}

\twocolumn

{ \footnotesize  \setcounter{utterance}{1}
\setlength{\tabcolsep}{0pt}
\begin{supertabular}{c@{$\;$}|p{.15\linewidth}@{}p{.15\linewidth}p{.15\linewidth}p{.15\linewidth}p{.15\linewidth}p{.15\linewidth}}

    \# & $\;$A & \multicolumn{4}{c}{Game Master} & $\;\:$B\\
    \hline 

    \theutterance \stepcounter{utterance}  

    & & \multicolumn{4}{p{0.6\linewidth}}{\cellcolor[rgb]{0.9,0.9,0.9}{%
	\makecell[{{p{\linewidth}}}]{% 
	  \tt {\tiny [A$\langle$GM]}  
	 You are an expert in causal inference. The following question is not a typical commonsense query, but rather a meticulously designed question created by a professor specializing in causal inference, intended to assess the students' mastery of the course content.\\ \tt Imagine a self-contained, hypothetical world with only the following conditions, and without any unmentioned factors or causal relationships: Smoking has a direct effect on effort and college admission. Effort has a direct effect on college admission. For nonsmokers, the probability of college admission is 76%. For smokers, the probability of college admission is 22%. For smokers, would it be less likely to see college admission if the student had been a nonsmoker?\\ \tt Start your answer with 'Yes' or 'No', followed by additional reasoning or evidence to support your explanation. 
	  } 
	   } 
	   } 
	 & \\ 
 

    \theutterance \stepcounter{utterance}  

    & \multicolumn{4}{p{0.6\linewidth}}{\cellcolor[rgb]{0.8,1,0.9}{%
	\makecell[{{p{\linewidth}}}]{% 
	  \tt {\tiny [A$\rangle$GM]}  
	 Yes, it would be less likely to see college admission for a smoker if they had been a nonsmoker. This conclusion is based on the given causal relationships and probabilities:\\ \tt \\ \tt 1. **Direct Effect of Smoking on College Admission**: The probability of college admission is 22% for smokers, which is significantly lower than the 76% for nonsmokers. This indicates that smoking has a negative causal effect on college admission.\\ \tt \\ \tt 2. **Direct Effect of Effort on College Admission**: Effort has a direct positive effect on college admission, but since the question focuses on the direct effect of smoking, we primarily consider the smoking variable.\\ \tt \\ \tt 3. **Counterfactual Analysis**: To determine if a smoker would have a higher chance of admission if they were a nonsmoker, we perform a counterfactual analysis. Given the direct effect of smoking on college admission, we assume that the effort level remains constant (since effort is not affected by the smoking status in this hypothetical scenario).\\ \tt \\ \tt    - **Current Scenario (Smoker)**: Probability of admission = 22%\\ \tt    \\ \tt    - **Counterfactual Scenario (Same Effort, Nonsmoker)**: Probability of admission = 76%\\ \tt \\ \tt    Since the probability of college admission is higher for nonsmokers than for smokers, it is less likely to see college admission for a smoker if they had been a nonsmoker.\\ \tt \\ \tt This conclusion is based purely on the direct causal effects provided in the question and does not consider any potential confounding variables or 
	  } 
	   } 
	   } 
	 & & \\ 
 

    \theutterance \stepcounter{utterance}  

    & & & \multicolumn{2}{p{0.3\linewidth}}{\cellcolor[rgb]{0.95,0.95,0.95}{%
	\makecell[{{p{\linewidth}}}]{% 
	  \tt {\tiny [GM$|$GM]}  
	 Yes 
	  } 
	   } 
	   } 
	 & & \\ 
 

    \theutterance \stepcounter{utterance}  

    & & & \multicolumn{2}{p{0.3\linewidth}}{\cellcolor[rgb]{0.95,0.95,0.95}{%
	\makecell[{{p{\linewidth}}}]{% 
	  \tt {\tiny [GM$|$GM]}  
	 no 
	  } 
	   } 
	   } 
	 & & \\ 
 

    \theutterance \stepcounter{utterance}  

    & & & \multicolumn{2}{p{0.3\linewidth}}{\cellcolor[rgb]{0.95,0.95,0.95}{%
	\makecell[{{p{\linewidth}}}]{% 
	  \tt {\tiny [GM$|$GM]}  
	 game_result = LOSE 
	  } 
	   } 
	   } 
	 & & \\ 
 

\end{supertabular}
}

\end{document}
