
\documentclass{article}
\usepackage{colortbl}
\usepackage{makecell}
\usepackage{multirow}
\usepackage{supertabular}

\begin{document}

\newcounter{utterance}

\twocolumn

{ \footnotesize  \setcounter{utterance}{1}
\setlength{\tabcolsep}{0pt}
\begin{supertabular}{c@{$\;$}|p{.15\linewidth}@{}p{.15\linewidth}p{.15\linewidth}p{.15\linewidth}p{.15\linewidth}p{.15\linewidth}}

    \# & $\;$A & \multicolumn{4}{c}{Game Master} & $\;\:$B\\
    \hline 

    \theutterance \stepcounter{utterance}  

    & & \multicolumn{4}{p{0.6\linewidth}}{\cellcolor[rgb]{0.9,0.9,0.9}{%
	\makecell[{{p{\linewidth}}}]{% 
	  \tt {\tiny [A$\langle$GM]}  
	 You are an expert in causal inference. The following question is not a typical commonsense query, but rather a meticulously designed question created by a professor specializing in causal inference, intended to assess the students' mastery of the course content.\\ \tt Imagine a self-contained, hypothetical world with only the following conditions, and without any unmentioned factors or causal relationships: Gene has a direct effect on smoking and lung cancer. Pollution has a direct effect on smoking and lung cancer. Smoking has a direct effect on lung cancer. Pollution is unobserved. For people with nonsmoking genes and nonsmokers, the probability of lung cancer is 56%. For people with nonsmoking genes and smokers, the probability of lung cancer is 42%. For people with smoking genes and nonsmokers, the probability of lung cancer is 60%. For people with smoking genes and smokers, the probability of lung cancer is 40%. For people with nonsmoking genes and with low pollution, the probability of smoking is 76%. For people with nonsmoking genes and with high pollution, the probability of smoking is 48%. For people with smoking genes and with low pollution, the probability of smoking is 46%. For people with smoking genes and with high pollution, the probability of smoking is 21%. The overall probability of high pollution is 41%. Does gene positively affect lung cancer through smoking?\\ \tt Start your answer with 'Yes' or 'No', followed by additional reasoning or evidence to support your explanation. 
	  } 
	   } 
	   } 
	 & \\ 
 

    \theutterance \stepcounter{utterance}  

    & \multicolumn{4}{p{0.6\linewidth}}{\cellcolor[rgb]{0.8,1,0.9}{%
	\makecell[{{p{\linewidth}}}]{% 
	  \tt {\tiny [A$\rangle$GM]}  
	 No, gene does not positively affect lung cancer through smoking based on the given data.\\ \tt \\ \tt To analyze this, we need to consider the potential confounding effect of pollution, which is unobserved but has direct effects on both smoking and lung cancer. Here's a step-by-step breakdown:\\ \tt \\ \tt 1. **Direct Effects:**\\ \tt    - **Gene on Smoking:** The probability of smoking varies with gene type and pollution level.\\ \tt    - **Pollution on Smoking:** The probability of smoking also varies with pollution level, even after accounting for gene type.\\ \tt    - **Smoking on Lung Cancer:** The probability of lung cancer increases with smoking.\\ \tt \\ \tt 2. **Observed Probabilities:**\\ \tt    - **Nonsmoking Genes:**\\ \tt      - Nonsmokers (low pollution): 56% lung cancer\\ \tt      - Smokers (low pollution): 42% lung cancer\\ \tt      - Nonsmokers (high pollution): 48% lung cancer\\ \tt    - **Smoking Genes:**\\ \tt      - Nonsmokers (low pollution): 60% lung cancer\\ \tt      - Smokers (low pollution): 40% lung cancer\\ \tt      - Nonsmokers (high pollution): Not directly given, but implied by the overall probabilities.\\ \tt      - Smokers (high pollution): Not directly given, but implied by the overall probabilities.\\ \tt \\ \tt 3. **Implications:**\\ \tt    - **Nonsmoking Genes:** The increase in lung cancer risk from smoking (56% to 
	  } 
	   } 
	   } 
	 & & \\ 
 

    \theutterance \stepcounter{utterance}  

    & & & \multicolumn{2}{p{0.3\linewidth}}{\cellcolor[rgb]{0.95,0.95,0.95}{%
	\makecell[{{p{\linewidth}}}]{% 
	  \tt {\tiny [GM$|$GM]}  
	 No 
	  } 
	   } 
	   } 
	 & & \\ 
 

    \theutterance \stepcounter{utterance}  

    & & & \multicolumn{2}{p{0.3\linewidth}}{\cellcolor[rgb]{0.95,0.95,0.95}{%
	\makecell[{{p{\linewidth}}}]{% 
	  \tt {\tiny [GM$|$GM]}  
	 yes 
	  } 
	   } 
	   } 
	 & & \\ 
 

    \theutterance \stepcounter{utterance}  

    & & & \multicolumn{2}{p{0.3\linewidth}}{\cellcolor[rgb]{0.95,0.95,0.95}{%
	\makecell[{{p{\linewidth}}}]{% 
	  \tt {\tiny [GM$|$GM]}  
	 game_result = LOSE 
	  } 
	   } 
	   } 
	 & & \\ 
 

\end{supertabular}
}

\end{document}
