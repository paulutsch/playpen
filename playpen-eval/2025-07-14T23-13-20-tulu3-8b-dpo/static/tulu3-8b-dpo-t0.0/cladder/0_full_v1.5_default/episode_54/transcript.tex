
\documentclass{article}
\usepackage{colortbl}
\usepackage{makecell}
\usepackage{multirow}
\usepackage{supertabular}

\begin{document}

\newcounter{utterance}

\twocolumn

{ \footnotesize  \setcounter{utterance}{1}
\setlength{\tabcolsep}{0pt}
\begin{supertabular}{c@{$\;$}|p{.15\linewidth}@{}p{.15\linewidth}p{.15\linewidth}p{.15\linewidth}p{.15\linewidth}p{.15\linewidth}}

    \# & $\;$A & \multicolumn{4}{c}{Game Master} & $\;\:$B\\
    \hline 

    \theutterance \stepcounter{utterance}  

    & & \multicolumn{4}{p{0.6\linewidth}}{\cellcolor[rgb]{0.9,0.9,0.9}{%
	\makecell[{{p{\linewidth}}}]{% 
	  \tt {\tiny [A$\langle$GM]}  
	 You are an expert in causal inference. The following question is not a typical commonsense query, but rather a meticulously designed question created by a professor specializing in causal inference, intended to assess the students' mastery of the course content.\\ \tt Imagine a self-contained, hypothetical world with only the following conditions, and without any unmentioned factors or causal relationships: Ability to speak english has a direct effect on the forest. The smoker has a direct effect on the forest. The overall probability of speaking english is 93%. For people who do not speak english, the probability of the forest on fire is 79%. For people who speak english, the probability of the forest on fire is 65%. Is the forest on fire more likely than the forest not on fire overall?\\ \tt Start your answer with 'Yes' or 'No', followed by additional reasoning or evidence to support your explanation. 
	  } 
	   } 
	   } 
	 & \\ 
 

    \theutterance \stepcounter{utterance}  

    & \multicolumn{4}{p{0.6\linewidth}}{\cellcolor[rgb]{0.8,1,0.9}{%
	\makecell[{{p{\linewidth}}}]{% 
	  \tt {\tiny [A$\rangle$GM]}  
	 No, the forest is not more likely to be on fire overall. To determine this, we need to calculate the overall probability of the forest being on fire, taking into account both the probability of speaking English and the conditional probabilities given the language spoken.\\ \tt \\ \tt First, let's break down the probabilities:\\ \tt \\ \tt - Probability of speaking English (P(Speak English)) = 93% = 0.93\\ \tt - Probability of the forest on fire given not speaking English (P(Fire | Not Speak English)) = 79% = 0.79\\ \tt - Probability of the forest on fire given speaking English (P(Fire | Speak English)) = 65% = 0.65\\ \tt \\ \tt We can calculate the overall probability of the forest being on fire (P(Fire)) using the law of total probability:\\ \tt \\ \tt \[ P(Fire) = P(Fire | Speak English) \cdot P(Speak English) + P(Fire | Not Speak English) \cdot P(Not Speak English) \]\\ \tt \\ \tt Where \( P(Not Speak English) = 1 - P(Speak English) = 1 - 0.93 = 0.07 \).\\ \tt \\ \tt Substituting the values:\\ \tt \\ \tt \[ P(Fire) = (0.65 \cdot 0.93) + (0.79 \cdot 0.07) \]\\ \tt \[ P(Fire) = 0.6065 + 0.0543 \]\\ \tt \[ P(Fire) = 0.6608 
	  } 
	   } 
	   } 
	 & & \\ 
 

    \theutterance \stepcounter{utterance}  

    & & & \multicolumn{2}{p{0.3\linewidth}}{\cellcolor[rgb]{0.95,0.95,0.95}{%
	\makecell[{{p{\linewidth}}}]{% 
	  \tt {\tiny [GM$|$GM]}  
	 No 
	  } 
	   } 
	   } 
	 & & \\ 
 

    \theutterance \stepcounter{utterance}  

    & & & \multicolumn{2}{p{0.3\linewidth}}{\cellcolor[rgb]{0.95,0.95,0.95}{%
	\makecell[{{p{\linewidth}}}]{% 
	  \tt {\tiny [GM$|$GM]}  
	 yes 
	  } 
	   } 
	   } 
	 & & \\ 
 

    \theutterance \stepcounter{utterance}  

    & & & \multicolumn{2}{p{0.3\linewidth}}{\cellcolor[rgb]{0.95,0.95,0.95}{%
	\makecell[{{p{\linewidth}}}]{% 
	  \tt {\tiny [GM$|$GM]}  
	 game_result = LOSE 
	  } 
	   } 
	   } 
	 & & \\ 
 

\end{supertabular}
}

\end{document}
