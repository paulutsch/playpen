
\documentclass{article}
\usepackage{colortbl}
\usepackage{makecell}
\usepackage{multirow}
\usepackage{supertabular}

\begin{document}

\newcounter{utterance}

\twocolumn

{ \footnotesize  \setcounter{utterance}{1}
\setlength{\tabcolsep}{0pt}
\begin{supertabular}{c@{$\;$}|p{.15\linewidth}@{}p{.15\linewidth}p{.15\linewidth}p{.15\linewidth}p{.15\linewidth}p{.15\linewidth}}

    \# & $\;$A & \multicolumn{4}{c}{Game Master} & $\;\:$B\\
    \hline 

    \theutterance \stepcounter{utterance}  

    & & \multicolumn{4}{p{0.6\linewidth}}{\cellcolor[rgb]{0.9,0.9,0.9}{%
	\makecell[{{p{\linewidth}}}]{% 
	  \tt {\tiny [A$\langle$GM]}  
	 You are an expert in causal inference. The following question is not a typical commonsense query, but rather a meticulously designed question created by a professor specializing in causal inference, intended to assess the students' mastery of the course content.\\ \tt Imagine a self-contained, hypothetical world with only the following conditions, and without any unmentioned factors or causal relationships: Jyka has a direct effect on hwax. Hwax has a direct effect on lirg. The overall probability of jyka is 14%. The probability of not jyka and lirg is 72%. The probability of jyka and lirg is 12%. Is the chance of lirg smaller when observing jyka?\\ \tt Start your answer with 'Yes' or 'No', followed by additional reasoning or evidence to support your explanation. 
	  } 
	   } 
	   } 
	 & \\ 
 

    \theutterance \stepcounter{utterance}  

    & \multicolumn{4}{p{0.6\linewidth}}{\cellcolor[rgb]{0.8,1,0.9}{%
	\makecell[{{p{\linewidth}}}]{% 
	  \tt {\tiny [A$\rangle$GM]}  
	 No, the chance of lirg is not smaller when observing jyka. \\ \tt \\ \tt To determine this, we need to calculate the conditional probability of Lirg given Jyka (P(Lirg | Jyka)) and compare it to the unconditional probability of Lirg (P(Lirg)).\\ \tt \\ \tt First, let's define the given probabilities:\\ \tt - P(Jyka) = 0.14\\ \tt - P(Not Jyka and Lirg) = 0.72\\ \tt - P(Jyka and Lirg) = 0.12\\ \tt \\ \tt We can find the unconditional probability of Lirg (P(Lirg)) by using the law of total probability:\\ \tt \\ \tt P(Lirg) = P(Lirg and Jyka) + P(Lirg and Not Jyka)\\ \tt \\ \tt From the given information, P(Lirg and Jyka) = 0.12. To find P(Lirg and Not Jyka), we subtract the probability of Jyka and Lirg from the probability of Not Jyka and Lirg:\\ \tt \\ \tt P(Lirg and Not Jyka) = P(Not Jyka and Lirg) - P(Jyka and Lirg) = 0.72 - 0.12 = 0.60\\ \tt \\ \tt Now, calculate P(Lirg):\\ \tt \\ \tt P(Lirg) = 0.12 + 0.60 = 
	  } 
	   } 
	   } 
	 & & \\ 
 

    \theutterance \stepcounter{utterance}  

    & & & \multicolumn{2}{p{0.3\linewidth}}{\cellcolor[rgb]{0.95,0.95,0.95}{%
	\makecell[{{p{\linewidth}}}]{% 
	  \tt {\tiny [GM$|$GM]}  
	 No 
	  } 
	   } 
	   } 
	 & & \\ 
 

    \theutterance \stepcounter{utterance}  

    & & & \multicolumn{2}{p{0.3\linewidth}}{\cellcolor[rgb]{0.95,0.95,0.95}{%
	\makecell[{{p{\linewidth}}}]{% 
	  \tt {\tiny [GM$|$GM]}  
	 no 
	  } 
	   } 
	   } 
	 & & \\ 
 

    \theutterance \stepcounter{utterance}  

    & & & \multicolumn{2}{p{0.3\linewidth}}{\cellcolor[rgb]{0.95,0.95,0.95}{%
	\makecell[{{p{\linewidth}}}]{% 
	  \tt {\tiny [GM$|$GM]}  
	 game_result = WIN 
	  } 
	   } 
	   } 
	 & & \\ 
 

\end{supertabular}
}

\end{document}
