
\documentclass{article}
\usepackage{colortbl}
\usepackage{makecell}
\usepackage{multirow}
\usepackage{supertabular}

\begin{document}

\newcounter{utterance}

\twocolumn

{ \footnotesize  \setcounter{utterance}{1}
\setlength{\tabcolsep}{0pt}
\begin{supertabular}{c@{$\;$}|p{.15\linewidth}@{}p{.15\linewidth}p{.15\linewidth}p{.15\linewidth}p{.15\linewidth}p{.15\linewidth}}

    \# & $\;$A & \multicolumn{4}{c}{Game Master} & $\;\:$B\\
    \hline 

    \theutterance \stepcounter{utterance}  

    & & \multicolumn{4}{p{0.6\linewidth}}{\cellcolor[rgb]{0.9,0.9,0.9}{%
	\makecell[{{p{\linewidth}}}]{% 
	  \tt {\tiny [A$\langle$GM]}  
	 You are an expert in causal inference. The following question is not a typical commonsense query, but rather a meticulously designed question created by a professor specializing in causal inference, intended to assess the students' mastery of the course content.\\ \tt Imagine a self-contained, hypothetical world with only the following conditions, and without any unmentioned factors or causal relationships: CEO has a direct effect on director and having visited England. Having visited England has a direct effect on employee. Director has a direct effect on employee. For CEOs who fire employees and have not visited England, the probability of employee being fired is 21%. For CEOs who fire employees and have visited England, the probability of employee being fired is 51%. For CEOs who fire employees and have not visited England, the probability of employee being fired is 35%. For CEOs who fire employees and have visited England, the probability of employee being fired is 66%. The overall probability of CEO's decision to fire the employee is 28%. Will having visited England increase the chance of employee being fired?\\ \tt Start your answer with 'Yes' or 'No', followed by additional reasoning or evidence to support your explanation. 
	  } 
	   } 
	   } 
	 & \\ 
 

    \theutterance \stepcounter{utterance}  

    & \multicolumn{4}{p{0.6\linewidth}}{\cellcolor[rgb]{0.8,1,0.9}{%
	\makecell[{{p{\linewidth}}}]{% 
	  \tt {\tiny [A$\rangle$GM]}  
	 No, having visited England does not increase the chance of an employee being fired based solely on the information provided.\\ \tt \\ \tt To analyze this, we need to consider the potential confounding variable: the direct effect of the CEO on the employee's firing decision. The provided data suggests that the probability of an employee being fired is higher when a CEO who has visited England fires an employee compared to when they have not visited England. However, this comparison alone does not allow us to conclude a causal effect of visiting England on the firing decision.\\ \tt \\ \tt Here's why:\\ \tt \\ \tt 1. **Confounding Variable**: The direct effect of the CEO on the firing decision is not accounted for. The CEO's decision to fire an employee could be influenced by many factors, including but not limited to their personality, management style, company performance, and other contextual factors. These factors might also influence whether the CEO has visited England.\\ \tt \\ \tt 2. **Different Contexts**: The probabilities provided are for different contexts (fired employees), not for the overall population of employees. This makes it difficult to generalize a causal effect from these specific scenarios.\\ \tt \\ \tt 3. **Selection Bias**: The data might suffer from selection bias. For instance, CEOs who have visited England might be more likely to fire employees for reasons unrelated to the visit itself (e.g., they might be more aggressive managers).\\ \tt \\ \tt 4. **Counterfactual Analysis**: To establish causality, we would need to consider counterfactual scenarios: Would the probability of firing be the same if the CEO had not 
	  } 
	   } 
	   } 
	 & & \\ 
 

    \theutterance \stepcounter{utterance}  

    & & & \multicolumn{2}{p{0.3\linewidth}}{\cellcolor[rgb]{0.95,0.95,0.95}{%
	\makecell[{{p{\linewidth}}}]{% 
	  \tt {\tiny [GM$|$GM]}  
	 No 
	  } 
	   } 
	   } 
	 & & \\ 
 

    \theutterance \stepcounter{utterance}  

    & & & \multicolumn{2}{p{0.3\linewidth}}{\cellcolor[rgb]{0.95,0.95,0.95}{%
	\makecell[{{p{\linewidth}}}]{% 
	  \tt {\tiny [GM$|$GM]}  
	 yes 
	  } 
	   } 
	   } 
	 & & \\ 
 

    \theutterance \stepcounter{utterance}  

    & & & \multicolumn{2}{p{0.3\linewidth}}{\cellcolor[rgb]{0.95,0.95,0.95}{%
	\makecell[{{p{\linewidth}}}]{% 
	  \tt {\tiny [GM$|$GM]}  
	 game_result = LOSE 
	  } 
	   } 
	   } 
	 & & \\ 
 

\end{supertabular}
}

\end{document}
