
\documentclass{article}
\usepackage{colortbl}
\usepackage{makecell}
\usepackage{multirow}
\usepackage{supertabular}

\begin{document}

\newcounter{utterance}

\twocolumn

{ \footnotesize  \setcounter{utterance}{1}
\setlength{\tabcolsep}{0pt}
\begin{supertabular}{c@{$\;$}|p{.15\linewidth}@{}p{.15\linewidth}p{.15\linewidth}p{.15\linewidth}p{.15\linewidth}p{.15\linewidth}}

    \# & $\;$A & \multicolumn{4}{c}{Game Master} & $\;\:$B\\
    \hline 

    \theutterance \stepcounter{utterance}  

    & & \multicolumn{4}{p{0.6\linewidth}}{\cellcolor[rgb]{0.9,0.9,0.9}{%
	\makecell[{{p{\linewidth}}}]{% 
	  \tt {\tiny [A$\langle$GM]}  
	 You are an expert in causal inference. The following question is not a typical commonsense query, but rather a meticulously designed question created by a professor specializing in causal inference, intended to assess the students' mastery of the course content.\\ \tt Imagine a self-contained, hypothetical world with only the following conditions, and without any unmentioned factors or causal relationships: Encouragement level has a direct effect on studying habit and brown eyes. Studying habit has a direct effect on brown eyes. The overall probability of encouragement is 91%. The probability of discouragement and brown eyes is 7%. The probability of encouragement and brown eyes is 58%. Is the chance of brown eyes larger when observing encouragement?\\ \tt Start your answer with 'Yes' or 'No', followed by additional reasoning or evidence to support your explanation. 
	  } 
	   } 
	   } 
	 & \\ 
 

    \theutterance \stepcounter{utterance}  

    & \multicolumn{4}{p{0.6\linewidth}}{\cellcolor[rgb]{0.8,1,0.9}{%
	\makecell[{{p{\linewidth}}}]{% 
	  \tt {\tiny [A$\rangle$GM]}  
	 No, the chance of brown eyes is not necessarily larger when observing encouragement. \\ \tt \\ \tt To determine this, we need to calculate the conditional probability of having brown eyes given encouragement (P(Brown Eyes | Encouragement)) and compare it to the unconditional probability of having brown eyes (P(Brown Eyes)).\\ \tt \\ \tt First, let's calculate P(Brown Eyes | Encouragement):\\ \tt \\ \tt Given:\\ \tt - P(Encouragement) = 0.91\\ \tt - P(Brown Eyes | Encouragement) = P(Encouragement and Brown Eyes) / P(Encouragement)\\ \tt - P(Encouragement and Brown Eyes) = 0.58\\ \tt \\ \tt So,\\ \tt P(Brown Eyes | Encouragement) = 0.58 / 0.91 ≈ 0.6366\\ \tt \\ \tt Next, let's calculate P(Brown Eyes):\\ \tt \\ \tt Given:\\ \tt - P(Discouragement) = 1 - P(Encouragement) = 0.09\\ \tt - P(Brown Eyes) = P(Brown Eyes and Encouragement) + P(Brown Eyes and Discouragement)\\ \tt - P(Brown Eyes and Encouragement) = 0.58\\ \tt - P(Brown Eyes and Discouragement) = P(Discouragement) * P(Brown Eyes | Discouragement)\\ \tt \\ \tt Since we know P(Brown Eyes | Discouragement) = 0.07 (from the given probability of discouragement and brown eyes), we have:\\ \tt P(Brown 
	  } 
	   } 
	   } 
	 & & \\ 
 

    \theutterance \stepcounter{utterance}  

    & & & \multicolumn{2}{p{0.3\linewidth}}{\cellcolor[rgb]{0.95,0.95,0.95}{%
	\makecell[{{p{\linewidth}}}]{% 
	  \tt {\tiny [GM$|$GM]}  
	 No 
	  } 
	   } 
	   } 
	 & & \\ 
 

    \theutterance \stepcounter{utterance}  

    & & & \multicolumn{2}{p{0.3\linewidth}}{\cellcolor[rgb]{0.95,0.95,0.95}{%
	\makecell[{{p{\linewidth}}}]{% 
	  \tt {\tiny [GM$|$GM]}  
	 no 
	  } 
	   } 
	   } 
	 & & \\ 
 

    \theutterance \stepcounter{utterance}  

    & & & \multicolumn{2}{p{0.3\linewidth}}{\cellcolor[rgb]{0.95,0.95,0.95}{%
	\makecell[{{p{\linewidth}}}]{% 
	  \tt {\tiny [GM$|$GM]}  
	 game_result = WIN 
	  } 
	   } 
	   } 
	 & & \\ 
 

\end{supertabular}
}

\end{document}
