
\documentclass{article}
\usepackage{colortbl}
\usepackage{makecell}
\usepackage{multirow}
\usepackage{supertabular}

\begin{document}

\newcounter{utterance}

\twocolumn

{ \footnotesize  \setcounter{utterance}{1}
\setlength{\tabcolsep}{0pt}
\begin{supertabular}{c@{$\;$}|p{.15\linewidth}@{}p{.15\linewidth}p{.15\linewidth}p{.15\linewidth}p{.15\linewidth}p{.15\linewidth}}

    \# & $\;$A & \multicolumn{4}{c}{Game Master} & $\;\:$B\\
    \hline 

    \theutterance \stepcounter{utterance}  

    & & \multicolumn{4}{p{0.6\linewidth}}{\cellcolor[rgb]{0.9,0.9,0.9}{%
	\makecell[{{p{\linewidth}}}]{% 
	  \tt {\tiny [A$\langle$GM]}  
	 You are an expert in causal inference. The following question is not a typical commonsense query, but rather a meticulously designed question created by a professor specializing in causal inference, intended to assess the students' mastery of the course content.\\ \tt Imagine a self-contained, hypothetical world with only the following conditions, and without any unmentioned factors or causal relationships: Qwiu has a direct effect on jyka and kwox. Jyka has a direct effect on yupt. Yupt has a direct effect on kwox. Qwiu is unobserved. We know that qwiu causes jyka. jyka causes not yupt. qwiu or yupt causes kwox. We observed an individual is qwiu. Would an individual is kwox if jyka instead of not jyka?\\ \tt Start your answer with 'Yes' or 'No', followed by additional reasoning or evidence to support your explanation. 
	  } 
	   } 
	   } 
	 & \\ 
 

    \theutterance \stepcounter{utterance}  

    & \multicolumn{4}{p{0.6\linewidth}}{\cellcolor[rgb]{0.8,1,0.9}{%
	\makecell[{{p{\linewidth}}}]{% 
	  \tt {\tiny [A$\rangle$GM]}  
	 No, an individual would not necessarily be kwox if jyka instead of not jyka, based on the given information.\\ \tt \\ \tt Here's the reasoning:\\ \tt \\ \tt 1. **Direct Effects**: \\ \tt    - Qwiu directly affects jyka.\\ \tt    - Jyka directly affects yupt.\\ \tt    - Yupt directly affects kwox.\\ \tt \\ \tt 2. **Causal Relationships**:\\ \tt    - Qwiu causes jyka.\\ \tt    - Jyka causes not yupt (not yupt implies yupt is low or absent).\\ \tt    - Qwiu or yupt causes kwox.\\ \tt \\ \tt 3. **Observations**:\\ \tt    - We observe that the individual has qwiu.\\ \tt    - We want to know the effect of jyka (present) vs. not jyka (absent) on kwox.\\ \tt \\ \tt Given these conditions, let's analyze the scenarios:\\ \tt \\ \tt **Scenario 1: Jyka is present (jyka instead of not jyka)**\\ \tt \\ \tt - Since qwiu causes jyka, having qwiu means jyka will be present.\\ \tt - Jyka causes not yupt, which means yupt will be low or absent.\\ \tt - Since yupt is low or absent, and yupt directly affects kwox, kwox might be affected by the absence of yupt.\\ \tt \\ \tt **Scenario 2: Jyka is absent (not jyka)**\\ \tt \\ \tt - Without jyka, the 
	  } 
	   } 
	   } 
	 & & \\ 
 

    \theutterance \stepcounter{utterance}  

    & & & \multicolumn{2}{p{0.3\linewidth}}{\cellcolor[rgb]{0.95,0.95,0.95}{%
	\makecell[{{p{\linewidth}}}]{% 
	  \tt {\tiny [GM$|$GM]}  
	 No 
	  } 
	   } 
	   } 
	 & & \\ 
 

    \theutterance \stepcounter{utterance}  

    & & & \multicolumn{2}{p{0.3\linewidth}}{\cellcolor[rgb]{0.95,0.95,0.95}{%
	\makecell[{{p{\linewidth}}}]{% 
	  \tt {\tiny [GM$|$GM]}  
	 yes 
	  } 
	   } 
	   } 
	 & & \\ 
 

    \theutterance \stepcounter{utterance}  

    & & & \multicolumn{2}{p{0.3\linewidth}}{\cellcolor[rgb]{0.95,0.95,0.95}{%
	\makecell[{{p{\linewidth}}}]{% 
	  \tt {\tiny [GM$|$GM]}  
	 game_result = LOSE 
	  } 
	   } 
	   } 
	 & & \\ 
 

\end{supertabular}
}

\end{document}
