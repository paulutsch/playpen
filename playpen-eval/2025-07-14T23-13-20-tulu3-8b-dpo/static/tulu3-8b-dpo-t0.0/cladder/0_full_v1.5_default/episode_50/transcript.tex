
\documentclass{article}
\usepackage{colortbl}
\usepackage{makecell}
\usepackage{multirow}
\usepackage{supertabular}

\begin{document}

\newcounter{utterance}

\twocolumn

{ \footnotesize  \setcounter{utterance}{1}
\setlength{\tabcolsep}{0pt}
\begin{supertabular}{c@{$\;$}|p{.15\linewidth}@{}p{.15\linewidth}p{.15\linewidth}p{.15\linewidth}p{.15\linewidth}p{.15\linewidth}}

    \# & $\;$A & \multicolumn{4}{c}{Game Master} & $\;\:$B\\
    \hline 

    \theutterance \stepcounter{utterance}  

    & & \multicolumn{4}{p{0.6\linewidth}}{\cellcolor[rgb]{0.9,0.9,0.9}{%
	\makecell[{{p{\linewidth}}}]{% 
	  \tt {\tiny [A$\langle$GM]}  
	 You are an expert in causal inference. The following question is not a typical commonsense query, but rather a meticulously designed question created by a professor specializing in causal inference, intended to assess the students' mastery of the course content.\\ \tt Imagine a self-contained, hypothetical world with only the following conditions, and without any unmentioned factors or causal relationships: Having a brother has a direct effect on wife and alarm clock. Wife has a direct effect on alarm clock. The overall probability of having a brother is 1%. The probability of not having a brother and ringing alarm is 64%. The probability of having a brother and ringing alarm is 0%. Is the chance of ringing alarm larger when observing having a brother?\\ \tt Start your answer with 'Yes' or 'No', followed by additional reasoning or evidence to support your explanation. 
	  } 
	   } 
	   } 
	 & \\ 
 

    \theutterance \stepcounter{utterance}  

    & \multicolumn{4}{p{0.6\linewidth}}{\cellcolor[rgb]{0.8,1,0.9}{%
	\makecell[{{p{\linewidth}}}]{% 
	  \tt {\tiny [A$\rangle$GM]}  
	 No, the chance of ringing alarm is not larger when observing having a brother. \\ \tt \\ \tt To analyze this, we need to consider the probabilities given in the problem:\\ \tt \\ \tt 1. **Probability of not having a brother and ringing alarm:** This is given as 64%, or 0.64.\\ \tt \\ \tt 2. **Probability of having a brother:** This is given as 1%, or 0.01.\\ \tt \\ \tt 3. **Probability of having a brother and ringing alarm:** This is given as 0%.\\ \tt \\ \tt Given these probabilities, we can calculate the probability of ringing the alarm under two scenarios: having a brother and not having a brother.\\ \tt \\ \tt **Probability of ringing alarm given no brother (P(Ring Alarm | No Brother)):**\\ \tt \\ \tt \[ P(Ring Alarm | No Brother) = P(Ring Alarm and No Brother) / P(No Brother) \]\\ \tt \\ \tt \[ P(Ring Alarm and No Brother) = 0.64 \] (since this is directly given)\\ \tt \\ \tt \[ P(No Brother) = 1 - P(Having a Brother) = 1 - 0.01 = 0.99 \]\\ \tt \\ \tt \[ P(Ring Alarm | No Brother) = 0.64 / 0.99 \approx 0.6465 \]\\ \tt \\ \tt **Probability of ringing alarm given having a brother (P(Ring Alarm | Brother)):**\\ \tt \\ \tt \[ P(Ring Alarm | Brother) = P(Ring Alarm and Brother) / P(Brother) \]\\ \tt \\ \tt \[ P(Ring Alarm and Brother 
	  } 
	   } 
	   } 
	 & & \\ 
 

    \theutterance \stepcounter{utterance}  

    & & & \multicolumn{2}{p{0.3\linewidth}}{\cellcolor[rgb]{0.95,0.95,0.95}{%
	\makecell[{{p{\linewidth}}}]{% 
	  \tt {\tiny [GM$|$GM]}  
	 No 
	  } 
	   } 
	   } 
	 & & \\ 
 

    \theutterance \stepcounter{utterance}  

    & & & \multicolumn{2}{p{0.3\linewidth}}{\cellcolor[rgb]{0.95,0.95,0.95}{%
	\makecell[{{p{\linewidth}}}]{% 
	  \tt {\tiny [GM$|$GM]}  
	 no 
	  } 
	   } 
	   } 
	 & & \\ 
 

    \theutterance \stepcounter{utterance}  

    & & & \multicolumn{2}{p{0.3\linewidth}}{\cellcolor[rgb]{0.95,0.95,0.95}{%
	\makecell[{{p{\linewidth}}}]{% 
	  \tt {\tiny [GM$|$GM]}  
	 game_result = WIN 
	  } 
	   } 
	   } 
	 & & \\ 
 

\end{supertabular}
}

\end{document}
