
\documentclass{article}
\usepackage{colortbl}
\usepackage{makecell}
\usepackage{multirow}
\usepackage{supertabular}

\begin{document}

\newcounter{utterance}

\twocolumn

{ \footnotesize  \setcounter{utterance}{1}
\setlength{\tabcolsep}{0pt}
\begin{supertabular}{c@{$\;$}|p{.15\linewidth}@{}p{.15\linewidth}p{.15\linewidth}p{.15\linewidth}p{.15\linewidth}p{.15\linewidth}}

    \# & $\;$A & \multicolumn{4}{c}{Game Master} & $\;\:$B\\
    \hline 

    \theutterance \stepcounter{utterance}  

    & & \multicolumn{4}{p{0.6\linewidth}}{\cellcolor[rgb]{0.9,0.9,0.9}{%
	\makecell[{{p{\linewidth}}}]{% 
	  \tt {\tiny [A$\langle$GM]}  
	 You are an expert in causal inference. The following question is not a typical commonsense query, but rather a meticulously designed question created by a professor specializing in causal inference, intended to assess the students' mastery of the course content.\\ \tt Imagine a self-contained, hypothetical world with only the following conditions, and without any unmentioned factors or causal relationships: Education level has a direct effect on skill. Skill has a direct effect on salary. Method 1: We look at how education level correlates with salary case by case according to skill. Method 2: We look directly at how education level correlates with salary in general. To understand how education level affects salary, is it more correct to use the Method 1 than Method 2?\\ \tt Start your answer with 'Yes' or 'No', followed by additional reasoning or evidence to support your explanation. 
	  } 
	   } 
	   } 
	 & \\ 
 

    \theutterance \stepcounter{utterance}  

    & \multicolumn{4}{p{0.6\linewidth}}{\cellcolor[rgb]{0.8,1,0.9}{%
	\makecell[{{p{\linewidth}}}]{% 
	  \tt {\tiny [A$\rangle$GM]}  
	 No, it is not more correct to use Method 1 than Method 2 to understand how education level affects salary.\\ \tt \\ \tt The reason is that Method 1 introduces a confounding variable (skill) into the analysis, which can lead to biased estimates of the effect of education on salary. In Method 1, we are essentially trying to understand the effect of education on salary by controlling for skill, which is itself a consequence of education. This creates a situation where the correlation between education and salary is being influenced by skill, leading to an overestimation of the direct effect of education on salary.\\ \tt \\ \tt In contrast, Method 2 directly examines the correlation between education level and salary without controlling for skill. While this method does not account for the mediating effect of skill, it provides a more straightforward and unbiased estimate of the direct effect of education on salary, assuming that skill is not the only mechanism through which education influences salary.\\ \tt \\ \tt However, it is important to note that neither method alone is sufficient for a complete causal analysis. To truly understand the causal relationship between education and salary, one would need to employ more sophisticated methods such as:\\ \tt \\ \tt 1. **Causal Diagrams (Directed Acyclic Graphs - DAGs)**: These can help identify confounders and guide the choice of appropriate statistical methods to estimate causal effects.\\ \tt \\ \tt 2. **Instrumental Variable (IV) Analysis**: This method can be used to control for unobserved confounders when a valid instrumental variable is available.\\ \tt \\ \tt 3. **Difference 
	  } 
	   } 
	   } 
	 & & \\ 
 

    \theutterance \stepcounter{utterance}  

    & & & \multicolumn{2}{p{0.3\linewidth}}{\cellcolor[rgb]{0.95,0.95,0.95}{%
	\makecell[{{p{\linewidth}}}]{% 
	  \tt {\tiny [GM$|$GM]}  
	 No 
	  } 
	   } 
	   } 
	 & & \\ 
 

    \theutterance \stepcounter{utterance}  

    & & & \multicolumn{2}{p{0.3\linewidth}}{\cellcolor[rgb]{0.95,0.95,0.95}{%
	\makecell[{{p{\linewidth}}}]{% 
	  \tt {\tiny [GM$|$GM]}  
	 no 
	  } 
	   } 
	   } 
	 & & \\ 
 

    \theutterance \stepcounter{utterance}  

    & & & \multicolumn{2}{p{0.3\linewidth}}{\cellcolor[rgb]{0.95,0.95,0.95}{%
	\makecell[{{p{\linewidth}}}]{% 
	  \tt {\tiny [GM$|$GM]}  
	 game_result = WIN 
	  } 
	   } 
	   } 
	 & & \\ 
 

\end{supertabular}
}

\end{document}
