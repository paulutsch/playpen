
\documentclass{article}
\usepackage{colortbl}
\usepackage{makecell}
\usepackage{multirow}
\usepackage{supertabular}

\begin{document}

\newcounter{utterance}

\twocolumn

{ \footnotesize  \setcounter{utterance}{1}
\setlength{\tabcolsep}{0pt}
\begin{supertabular}{c@{$\;$}|p{.15\linewidth}@{}p{.15\linewidth}p{.15\linewidth}p{.15\linewidth}p{.15\linewidth}p{.15\linewidth}}

    \# & $\;$A & \multicolumn{4}{c}{Game Master} & $\;\:$B\\
    \hline 

    \theutterance \stepcounter{utterance}  

    & & \multicolumn{4}{p{0.6\linewidth}}{\cellcolor[rgb]{0.9,0.9,0.9}{%
	\makecell[{{p{\linewidth}}}]{% 
	  \tt {\tiny [A$\langle$GM]}  
	 You are an expert in causal inference. The following question is not a typical commonsense query, but rather a meticulously designed question created by a professor specializing in causal inference, intended to assess the students' mastery of the course content.\\ \tt Imagine a self-contained, hypothetical world with only the following conditions, and without any unmentioned factors or causal relationships: Alice waking up has a direct effect on Alice arriving to school. Traffic has a direct effect on Alice arriving to school. The overall probability of waking up late is 5%. For days when Alice wakes up on time, the probability of arriving to school on time is 52%. For days when Alice wakes up late, the probability of arriving to school on time is 87%. Is arriving to school on time more likely than arriving to school late overall?\\ \tt Start your answer with 'Yes' or 'No', followed by additional reasoning or evidence to support your explanation. 
	  } 
	   } 
	   } 
	 & \\ 
 

    \theutterance \stepcounter{utterance}  

    & \multicolumn{4}{p{0.6\linewidth}}{\cellcolor[rgb]{0.8,1,0.9}{%
	\makecell[{{p{\linewidth}}}]{% 
	  \tt {\tiny [A$\rangle$GM]}  
	 No, arriving to school on time is not more likely than arriving to school late overall. \\ \tt \\ \tt To determine this, we need to calculate the overall probability of Alice arriving to school on time versus arriving late. We can do this by considering the probabilities given:\\ \tt \\ \tt 1. **Probability of waking up on time:** 95% (since the probability of waking up late is 5%).\\ \tt \\ \tt 2. **Probability of arriving on time given waking up on time:** 52%.\\ \tt \\ \tt 3. **Probability of arriving on time given waking up late:** 87%.\\ \tt \\ \tt The overall probability of arriving on time can be calculated using the law of total probability:\\ \tt \\ \tt \[ P(\text{On Time}) = P(\text{On Time} \mid \text{On Time}) \cdot P(\text{On Time}) + P(\text{On Time} \mid \text{Late}) \cdot P(\text{Late}) \]\\ \tt \\ \tt Substituting the given probabilities:\\ \tt \\ \tt \[ P(\text{On Time}) = (0.52 \cdot 0.95) + (0.87 \cdot 0.05) \]\\ \tt \\ \tt \[ P(\text{On Time}) = (0.494) + (0.0435) \]\\ \tt \\ \tt \[ P(\text{On Time}) = 0.5375 \]\\ \tt \\ \tt The overall probability of arriving on time is 53.75%. \\ \tt \\ \tt The probability of arriving late is the complement of arriving on time:\\ \tt \\ \tt \[ P(\text{Late}) = 
	  } 
	   } 
	   } 
	 & & \\ 
 

    \theutterance \stepcounter{utterance}  

    & & & \multicolumn{2}{p{0.3\linewidth}}{\cellcolor[rgb]{0.95,0.95,0.95}{%
	\makecell[{{p{\linewidth}}}]{% 
	  \tt {\tiny [GM$|$GM]}  
	 No 
	  } 
	   } 
	   } 
	 & & \\ 
 

    \theutterance \stepcounter{utterance}  

    & & & \multicolumn{2}{p{0.3\linewidth}}{\cellcolor[rgb]{0.95,0.95,0.95}{%
	\makecell[{{p{\linewidth}}}]{% 
	  \tt {\tiny [GM$|$GM]}  
	 yes 
	  } 
	   } 
	   } 
	 & & \\ 
 

    \theutterance \stepcounter{utterance}  

    & & & \multicolumn{2}{p{0.3\linewidth}}{\cellcolor[rgb]{0.95,0.95,0.95}{%
	\makecell[{{p{\linewidth}}}]{% 
	  \tt {\tiny [GM$|$GM]}  
	 game_result = LOSE 
	  } 
	   } 
	   } 
	 & & \\ 
 

\end{supertabular}
}

\end{document}
