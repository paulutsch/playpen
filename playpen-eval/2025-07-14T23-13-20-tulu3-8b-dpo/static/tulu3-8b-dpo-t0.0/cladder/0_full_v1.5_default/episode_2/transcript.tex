
\documentclass{article}
\usepackage{colortbl}
\usepackage{makecell}
\usepackage{multirow}
\usepackage{supertabular}

\begin{document}

\newcounter{utterance}

\twocolumn

{ \footnotesize  \setcounter{utterance}{1}
\setlength{\tabcolsep}{0pt}
\begin{supertabular}{c@{$\;$}|p{.15\linewidth}@{}p{.15\linewidth}p{.15\linewidth}p{.15\linewidth}p{.15\linewidth}p{.15\linewidth}}

    \# & $\;$A & \multicolumn{4}{c}{Game Master} & $\;\:$B\\
    \hline 

    \theutterance \stepcounter{utterance}  

    & & \multicolumn{4}{p{0.6\linewidth}}{\cellcolor[rgb]{0.9,0.9,0.9}{%
	\makecell[{{p{\linewidth}}}]{% 
	  \tt {\tiny [A$\langle$GM]}  
	 You are an expert in causal inference. The following question is not a typical commonsense query, but rather a meticulously designed question created by a professor specializing in causal inference, intended to assess the students' mastery of the course content.\\ \tt Imagine a self-contained, hypothetical world with only the following conditions, and without any unmentioned factors or causal relationships: CEO has a direct effect on director and manager. Manager has a direct effect on employee. Director has a direct effect on employee. For CEOs who fire employees and managers who don't sign termination letters, the probability of employee being fired is 24%. For CEOs who fire employees and managers who sign termination letters, the probability of employee being fired is 60%. For CEOs who fire employees and managers who don't sign termination letters, the probability of employee being fired is 39%. For CEOs who fire employees and managers who sign termination letters, the probability of employee being fired is 74%. The overall probability of CEO's decision to fire the employee is 18%. Will manager signing the termination letter increase the chance of employee being fired?\\ \tt Start your answer with 'Yes' or 'No', followed by additional reasoning or evidence to support your explanation. 
	  } 
	   } 
	   } 
	 & \\ 
 

    \theutterance \stepcounter{utterance}  

    & \multicolumn{4}{p{0.6\linewidth}}{\cellcolor[rgb]{0.8,1,0.9}{%
	\makecell[{{p{\linewidth}}}]{% 
	  \tt {\tiny [A$\rangle$GM]}  
	 No, the manager signing the termination letter does not increase the chance of the employee being fired, based on the given data.\\ \tt \\ \tt To analyze this, we need to consider the probabilities provided and understand the causal relationships:\\ \tt \\ \tt 1. **Direct Effects:**\\ \tt    - CEO has a direct effect on the employee (firing decision).\\ \tt    - Manager has a direct effect on the employee (through signing termination letters).\\ \tt    - Director has a direct effect on the employee.\\ \tt \\ \tt 2. **Probabilities:**\\ \tt    - For CEOs who fire employees and managers who don't sign termination letters, the probability of employee being fired is 24%.\\ \tt    - For CEOs who fire employees and managers who sign termination letters, the probability of employee being fired is 60%.\\ \tt \\ \tt 3. **Overall Probability:**\\ \tt    - The overall probability of a CEO's decision to fire the employee is 18%.\\ \tt \\ \tt **Analysis:**\\ \tt \\ \tt The overall probability of 18% suggests that there are various factors at play, including but not limited to the CEO's decision to fire and the manager's role in signing termination letters. However, when we look at the specific scenarios provided:\\ \tt \\ \tt - In the scenario where the manager does not sign the termination letter, the probability of the employee being fired is 24% when the CEO decides to fire.\\ \tt - In the scenario where the manager signs the termination letter, the probability of the employee being fired increases to 60%.\\ \tt \\ \tt This increase from 24% to 60% when the manager signs the termination letter might suggest 
	  } 
	   } 
	   } 
	 & & \\ 
 

    \theutterance \stepcounter{utterance}  

    & & & \multicolumn{2}{p{0.3\linewidth}}{\cellcolor[rgb]{0.95,0.95,0.95}{%
	\makecell[{{p{\linewidth}}}]{% 
	  \tt {\tiny [GM$|$GM]}  
	 No 
	  } 
	   } 
	   } 
	 & & \\ 
 

    \theutterance \stepcounter{utterance}  

    & & & \multicolumn{2}{p{0.3\linewidth}}{\cellcolor[rgb]{0.95,0.95,0.95}{%
	\makecell[{{p{\linewidth}}}]{% 
	  \tt {\tiny [GM$|$GM]}  
	 yes 
	  } 
	   } 
	   } 
	 & & \\ 
 

    \theutterance \stepcounter{utterance}  

    & & & \multicolumn{2}{p{0.3\linewidth}}{\cellcolor[rgb]{0.95,0.95,0.95}{%
	\makecell[{{p{\linewidth}}}]{% 
	  \tt {\tiny [GM$|$GM]}  
	 game_result = LOSE 
	  } 
	   } 
	   } 
	 & & \\ 
 

\end{supertabular}
}

\end{document}
