
\documentclass{article}
\usepackage{colortbl}
\usepackage{makecell}
\usepackage{multirow}
\usepackage{supertabular}

\begin{document}

\newcounter{utterance}

\twocolumn

{ \footnotesize  \setcounter{utterance}{1}
\setlength{\tabcolsep}{0pt}
\begin{supertabular}{c@{$\;$}|p{.15\linewidth}@{}p{.15\linewidth}p{.15\linewidth}p{.15\linewidth}p{.15\linewidth}p{.15\linewidth}}

    \# & $\;$A & \multicolumn{4}{c}{Game Master} & $\;\:$B\\
    \hline 

    \theutterance \stepcounter{utterance}  

    & & \multicolumn{4}{p{0.6\linewidth}}{\cellcolor[rgb]{0.9,0.9,0.9}{%
	\makecell[{{p{\linewidth}}}]{% 
	  \tt {\tiny [A$\langle$GM]}  
	 You are an expert in causal inference. The following question is not a typical commonsense query, but rather a meticulously designed question created by a professor specializing in causal inference, intended to assess the students' mastery of the course content.\\ \tt Imagine a self-contained, hypothetical world with only the following conditions, and without any unmentioned factors or causal relationships: Drinking coffee has a direct effect on hospitalization status. Broken bones has a direct effect on hospitalization status. Method 1: We look directly at how drinking coffee correlates with broken bones in general. Method 2: We look at this correlation case by case according to hospitalization status. To understand how drinking coffee affects broken bones, is it more correct to use the Method 1 than Method 2?\\ \tt Start your answer with 'Yes' or 'No', followed by additional reasoning or evidence to support your explanation. 
	  } 
	   } 
	   } 
	 & \\ 
 

    \theutterance \stepcounter{utterance}  

    & \multicolumn{4}{p{0.6\linewidth}}{\cellcolor[rgb]{0.8,1,0.9}{%
	\makecell[{{p{\linewidth}}}]{% 
	  \tt {\tiny [A$\rangle$GM]}  
	 No, it is not more correct to use Method 1 than Method 2 to understand how drinking coffee affects broken bones.\\ \tt \\ \tt The reason is that Method 1, which looks directly at the correlation between drinking coffee and broken bones in general, does not account for the confounding variable of hospitalization status. Hospitalization status is a critical factor that could affect both coffee consumption and the likelihood of having broken bones. For instance, individuals who are hospitalized might be less likely to drink coffee due to restrictions or personal preference, and they might be more likely to have broken bones due to the circumstances leading to hospitalization (e.g., accidents).\\ \tt \\ \tt Method 2, on the other hand, considers the hospitalization status as a potential mediator or confounder. By examining the relationship between coffee consumption and broken bones on a case-by-case basis according to hospitalization status, we can control for this confounding variable. This method allows us to isolate the direct effect of coffee on broken bones, as opposed to any indirect effects mediated through hospitalization.\\ \tt \\ \tt In causal inference, it is crucial to account for confounding variables to avoid drawing incorrect conclusions. Therefore, Method 2, which controls for hospitalization status, is more appropriate for understanding the causal relationship between drinking coffee and broken bones. This approach aligns with principles of causal inference, such as the need to consider potential confounders and to use appropriate statistical methods to adjust for them, such as matching, stratification, or regression adjustment.<|end_of_text|> 
	  } 
	   } 
	   } 
	 & & \\ 
 

    \theutterance \stepcounter{utterance}  

    & & & \multicolumn{2}{p{0.3\linewidth}}{\cellcolor[rgb]{0.95,0.95,0.95}{%
	\makecell[{{p{\linewidth}}}]{% 
	  \tt {\tiny [GM$|$GM]}  
	 No 
	  } 
	   } 
	   } 
	 & & \\ 
 

    \theutterance \stepcounter{utterance}  

    & & & \multicolumn{2}{p{0.3\linewidth}}{\cellcolor[rgb]{0.95,0.95,0.95}{%
	\makecell[{{p{\linewidth}}}]{% 
	  \tt {\tiny [GM$|$GM]}  
	 yes 
	  } 
	   } 
	   } 
	 & & \\ 
 

    \theutterance \stepcounter{utterance}  

    & & & \multicolumn{2}{p{0.3\linewidth}}{\cellcolor[rgb]{0.95,0.95,0.95}{%
	\makecell[{{p{\linewidth}}}]{% 
	  \tt {\tiny [GM$|$GM]}  
	 game_result = LOSE 
	  } 
	   } 
	   } 
	 & & \\ 
 

\end{supertabular}
}

\end{document}
