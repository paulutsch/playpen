
\documentclass{article}
\usepackage{colortbl}
\usepackage{makecell}
\usepackage{multirow}
\usepackage{supertabular}

\begin{document}

\newcounter{utterance}

\twocolumn

{ \footnotesize  \setcounter{utterance}{1}
\setlength{\tabcolsep}{0pt}
\begin{supertabular}{c@{$\;$}|p{.15\linewidth}@{}p{.15\linewidth}p{.15\linewidth}p{.15\linewidth}p{.15\linewidth}p{.15\linewidth}}

    \# & $\;$A & \multicolumn{4}{c}{Game Master} & $\;\:$B\\
    \hline 

    \theutterance \stepcounter{utterance}  

    & & \multicolumn{4}{p{0.6\linewidth}}{\cellcolor[rgb]{0.9,0.9,0.9}{%
	\makecell[{{p{\linewidth}}}]{% 
	  \tt {\tiny [A$\langle$GM]}  
	 You are an expert in causal inference. The following question is not a typical commonsense query, but rather a meticulously designed question created by a professor specializing in causal inference, intended to assess the students' mastery of the course content.\\ \tt Imagine a self-contained, hypothetical world with only the following conditions, and without any unmentioned factors or causal relationships: My decision has a direct effect on penguin mood and curly hair. Penguin mood has a direct effect on curly hair. For those who choose to take the stairs, the probability of curly hair is 70%. For those who choose to take the elevator, the probability of curly hair is 17%. For those who choose to take the elevator, would it be less likely to see curly hair if I had taken the stairs?\\ \tt Start your answer with 'Yes' or 'No', followed by additional reasoning or evidence to support your explanation. 
	  } 
	   } 
	   } 
	 & \\ 
 

    \theutterance \stepcounter{utterance}  

    & \multicolumn{4}{p{0.6\linewidth}}{\cellcolor[rgb]{0.8,1,0.9}{%
	\makecell[{{p{\linewidth}}}]{% 
	  \tt {\tiny [A$\rangle$GM]}  
	 No, it would not be less likely to see curly hair if I had taken the stairs instead of the elevator. \\ \tt \\ \tt The reason is that the causal pathway from decision (taking the stairs vs. taking the elevator) to curly hair is not direct but mediated by penguin mood. Given the conditions provided:\\ \tt \\ \tt 1. **Direct Effect of Decision on Curly Hair**: There is a direct effect of choosing to take the stairs (70% probability of curly hair) versus taking the elevator (17% probability of curly hair).\\ \tt \\ \tt 2. **Direct Effect of Penguin Mood on Curly Hair**: Penguin mood has a direct effect on curly hair, but the question does not specify how decisions (taking stairs vs. elevator) affect penguin mood. \\ \tt \\ \tt 3. **Assumption of No Direct Effect of Decision on Penguin Mood**: To answer this question, we must assume that the decision to take the stairs or the elevator does not directly affect penguin mood. If this assumption holds, then the effect of taking the stairs or the elevator on curly hair is solely through the direct probabilities provided.\\ \tt \\ \tt Given this assumption, the probability of curly hair is determined solely by the decision to take the stairs or the elevator, and not by any changes in penguin mood that might occur if one had made a different choice. Since the probability of curly hair is 17% when taking the elevator, this probability remains unchanged regardless of whether the stairs were taken instead.\\ \tt \\ \tt Therefore, under the assumption that decisions do not affect penguin mood, 
	  } 
	   } 
	   } 
	 & & \\ 
 

    \theutterance \stepcounter{utterance}  

    & & & \multicolumn{2}{p{0.3\linewidth}}{\cellcolor[rgb]{0.95,0.95,0.95}{%
	\makecell[{{p{\linewidth}}}]{% 
	  \tt {\tiny [GM$|$GM]}  
	 No 
	  } 
	   } 
	   } 
	 & & \\ 
 

    \theutterance \stepcounter{utterance}  

    & & & \multicolumn{2}{p{0.3\linewidth}}{\cellcolor[rgb]{0.95,0.95,0.95}{%
	\makecell[{{p{\linewidth}}}]{% 
	  \tt {\tiny [GM$|$GM]}  
	 no 
	  } 
	   } 
	   } 
	 & & \\ 
 

    \theutterance \stepcounter{utterance}  

    & & & \multicolumn{2}{p{0.3\linewidth}}{\cellcolor[rgb]{0.95,0.95,0.95}{%
	\makecell[{{p{\linewidth}}}]{% 
	  \tt {\tiny [GM$|$GM]}  
	 game_result = WIN 
	  } 
	   } 
	   } 
	 & & \\ 
 

\end{supertabular}
}

\end{document}
