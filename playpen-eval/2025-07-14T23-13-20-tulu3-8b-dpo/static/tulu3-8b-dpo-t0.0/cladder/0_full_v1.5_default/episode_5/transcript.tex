
\documentclass{article}
\usepackage{colortbl}
\usepackage{makecell}
\usepackage{multirow}
\usepackage{supertabular}

\begin{document}

\newcounter{utterance}

\twocolumn

{ \footnotesize  \setcounter{utterance}{1}
\setlength{\tabcolsep}{0pt}
\begin{supertabular}{c@{$\;$}|p{.15\linewidth}@{}p{.15\linewidth}p{.15\linewidth}p{.15\linewidth}p{.15\linewidth}p{.15\linewidth}}

    \# & $\;$A & \multicolumn{4}{c}{Game Master} & $\;\:$B\\
    \hline 

    \theutterance \stepcounter{utterance}  

    & & \multicolumn{4}{p{0.6\linewidth}}{\cellcolor[rgb]{0.9,0.9,0.9}{%
	\makecell[{{p{\linewidth}}}]{% 
	  \tt {\tiny [A$\langle$GM]}  
	 You are an expert in causal inference. The following question is not a typical commonsense query, but rather a meticulously designed question created by a professor specializing in causal inference, intended to assess the students' mastery of the course content.\\ \tt Imagine a self-contained, hypothetical world with only the following conditions, and without any unmentioned factors or causal relationships: Vaccination status has a direct effect on vaccination reaction and getting smallpox. Getting smallpox has a direct effect on smallpox survival. Vaccination reaction has a direct effect on smallpox survival. Method 1: We look directly at how vaccination status correlates with smallpox survival in general. Method 2: We look at this correlation case by case according to vaccination reaction. To understand how vaccination status affects smallpox survival, is it more correct to use the Method 1 than Method 2?\\ \tt Start your answer with 'Yes' or 'No', followed by additional reasoning or evidence to support your explanation. 
	  } 
	   } 
	   } 
	 & \\ 
 

    \theutterance \stepcounter{utterance}  

    & \multicolumn{4}{p{0.6\linewidth}}{\cellcolor[rgb]{0.8,1,0.9}{%
	\makecell[{{p{\linewidth}}}]{% 
	  \tt {\tiny [A$\rangle$GM]}  
	 No, it is not more correct to use Method 1 than Method 2 to understand how vaccination status affects smallpox survival.\\ \tt \\ \tt The reason is that Method 1, which looks directly at how vaccination status correlates with smallpox survival in general, may not fully capture the nuanced effects of vaccination on smallpox survival. This method might overlook the mediating role of vaccination reaction, which has a direct effect on smallpox survival.\\ \tt \\ \tt Method 2, on the other hand, considers the case-by-case correlation according to vaccination reaction. This approach allows for a more detailed analysis by accounting for the variability in individual responses to vaccination. By considering the different vaccination reactions, Method 2 can provide a more comprehensive understanding of how vaccination status influences smallpox survival, as it takes into account the direct effect of vaccination reaction on survival.\\ \tt \\ \tt In causal inference, it is often crucial to consider potential mediators and confounders to avoid biased estimates. In this hypothetical scenario, vaccination reaction serves as a mediator between vaccination status and smallpox survival. Ignoring this mediator could lead to an incorrect understanding of the causal relationship.\\ \tt \\ \tt Therefore, Method 2, which accounts for the mediating effect of vaccination reaction, is more likely to provide a more accurate and nuanced understanding of how vaccination status affects smallpox survival compared to Method 1.<|end_of_text|> 
	  } 
	   } 
	   } 
	 & & \\ 
 

    \theutterance \stepcounter{utterance}  

    & & & \multicolumn{2}{p{0.3\linewidth}}{\cellcolor[rgb]{0.95,0.95,0.95}{%
	\makecell[{{p{\linewidth}}}]{% 
	  \tt {\tiny [GM$|$GM]}  
	 No 
	  } 
	   } 
	   } 
	 & & \\ 
 

    \theutterance \stepcounter{utterance}  

    & & & \multicolumn{2}{p{0.3\linewidth}}{\cellcolor[rgb]{0.95,0.95,0.95}{%
	\makecell[{{p{\linewidth}}}]{% 
	  \tt {\tiny [GM$|$GM]}  
	 yes 
	  } 
	   } 
	   } 
	 & & \\ 
 

    \theutterance \stepcounter{utterance}  

    & & & \multicolumn{2}{p{0.3\linewidth}}{\cellcolor[rgb]{0.95,0.95,0.95}{%
	\makecell[{{p{\linewidth}}}]{% 
	  \tt {\tiny [GM$|$GM]}  
	 game_result = LOSE 
	  } 
	   } 
	   } 
	 & & \\ 
 

\end{supertabular}
}

\end{document}
