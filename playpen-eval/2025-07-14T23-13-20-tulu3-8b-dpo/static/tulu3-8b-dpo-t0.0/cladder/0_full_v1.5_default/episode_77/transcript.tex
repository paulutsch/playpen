
\documentclass{article}
\usepackage{colortbl}
\usepackage{makecell}
\usepackage{multirow}
\usepackage{supertabular}

\begin{document}

\newcounter{utterance}

\twocolumn

{ \footnotesize  \setcounter{utterance}{1}
\setlength{\tabcolsep}{0pt}
\begin{supertabular}{c@{$\;$}|p{.15\linewidth}@{}p{.15\linewidth}p{.15\linewidth}p{.15\linewidth}p{.15\linewidth}p{.15\linewidth}}

    \# & $\;$A & \multicolumn{4}{c}{Game Master} & $\;\:$B\\
    \hline 

    \theutterance \stepcounter{utterance}  

    & & \multicolumn{4}{p{0.6\linewidth}}{\cellcolor[rgb]{0.9,0.9,0.9}{%
	\makecell[{{p{\linewidth}}}]{% 
	  \tt {\tiny [A$\langle$GM]}  
	 You are an expert in causal inference. The following question is not a typical commonsense query, but rather a meticulously designed question created by a professor specializing in causal inference, intended to assess the students' mastery of the course content.\\ \tt Imagine a self-contained, hypothetical world with only the following conditions, and without any unmentioned factors or causal relationships: Wibl has a direct effect on zuph and uvzi. Zuph has a direct effect on uvzi. Method 1: We look at how zuph correlates with uvzi case by case according to wibl. Method 2: We look directly at how zuph correlates with uvzi in general. To understand how zuph affects uvzi, is it more correct to use the Method 1 than Method 2?\\ \tt Start your answer with 'Yes' or 'No', followed by additional reasoning or evidence to support your explanation. 
	  } 
	   } 
	   } 
	 & \\ 
 

    \theutterance \stepcounter{utterance}  

    & \multicolumn{4}{p{0.6\linewidth}}{\cellcolor[rgb]{0.8,1,0.9}{%
	\makecell[{{p{\linewidth}}}]{% 
	  \tt {\tiny [A$\rangle$GM]}  
	 No, it is not more correct to use Method 1 than Method 2 to understand how zuph affects uvzi.\\ \tt \\ \tt The reason is that Method 1, which involves looking at how zuph correlates with uvzi case by case according to wibl, introduces a confounding variable (wibl) into the analysis. This can lead to biased estimates of the effect of zuph on uvzi because wibl might itself have an effect on uvzi, either directly or indirectly through zuph. This would make it difficult to isolate the true causal effect of zuph on uvzi.\\ \tt \\ \tt In contrast, Method 2 directly examines the correlation between zuph and uvzi without introducing an additional variable (wibl) into the analysis. This approach is more straightforward and less likely to introduce bias, assuming that wibl does not have a confounding effect on the relationship between zuph and uvzi.\\ \tt \\ \tt However, it's important to note that Method 2 assumes that there are no unobserved confounders that might affect both zuph and uvzi. In real-world scenarios, it is often necessary to control for potential confounders to accurately estimate causal effects. But in this hypothetical world with only the specified conditions, Method 2 is the more appropriate choice for understanding the direct effect of zuph on uvzi.\\ \tt \\ \tt Therefore, while Method 1 might seem more detailed and nuanced, it does not provide a more accurate measure 
	  } 
	   } 
	   } 
	 & & \\ 
 

    \theutterance \stepcounter{utterance}  

    & & & \multicolumn{2}{p{0.3\linewidth}}{\cellcolor[rgb]{0.95,0.95,0.95}{%
	\makecell[{{p{\linewidth}}}]{% 
	  \tt {\tiny [GM$|$GM]}  
	 No 
	  } 
	   } 
	   } 
	 & & \\ 
 

    \theutterance \stepcounter{utterance}  

    & & & \multicolumn{2}{p{0.3\linewidth}}{\cellcolor[rgb]{0.95,0.95,0.95}{%
	\makecell[{{p{\linewidth}}}]{% 
	  \tt {\tiny [GM$|$GM]}  
	 yes 
	  } 
	   } 
	   } 
	 & & \\ 
 

    \theutterance \stepcounter{utterance}  

    & & & \multicolumn{2}{p{0.3\linewidth}}{\cellcolor[rgb]{0.95,0.95,0.95}{%
	\makecell[{{p{\linewidth}}}]{% 
	  \tt {\tiny [GM$|$GM]}  
	 game_result = LOSE 
	  } 
	   } 
	   } 
	 & & \\ 
 

\end{supertabular}
}

\end{document}
