
\documentclass{article}
\usepackage{colortbl}
\usepackage{makecell}
\usepackage{multirow}
\usepackage{supertabular}

\begin{document}

\newcounter{utterance}

\twocolumn

{ \footnotesize  \setcounter{utterance}{1}
\setlength{\tabcolsep}{0pt}
\begin{supertabular}{c@{$\;$}|p{.15\linewidth}@{}p{.15\linewidth}p{.15\linewidth}p{.15\linewidth}p{.15\linewidth}p{.15\linewidth}}

    \# & $\;$A & \multicolumn{4}{c}{Game Master} & $\;\:$B\\
    \hline 

    \theutterance \stepcounter{utterance}  

    & & \multicolumn{4}{p{0.6\linewidth}}{\cellcolor[rgb]{0.9,0.9,0.9}{%
	\makecell[{{p{\linewidth}}}]{% 
	  \tt {\tiny [A$\langle$GM]}  
	 You are an expert in causal inference. The following question is not a typical commonsense query, but rather a meticulously designed question created by a professor specializing in causal inference, intended to assess the students' mastery of the course content.\\ \tt Imagine a self-contained, hypothetical world with only the following conditions, and without any unmentioned factors or causal relationships: Husband has a direct effect on wife and alarm clock. Wife has a direct effect on alarm clock. For husbands that don't set the alarm and wives that don't set the alarm, the probability of ringing alarm is 11%. For husbands that don't set the alarm and wives that set the alarm, the probability of ringing alarm is 60%. For husbands that set the alarm and wives that don't set the alarm, the probability of ringing alarm is 46%. For husbands that set the alarm and wives that set the alarm, the probability of ringing alarm is 92%. For husbands that don't set the alarm, the probability of alarm set by wife is 61%. For husbands that set the alarm, the probability of alarm set by wife is 1%. Does husband positively affect alarm clock through wife?\\ \tt Start your answer with 'Yes' or 'No', followed by additional reasoning or evidence to support your explanation. 
	  } 
	   } 
	   } 
	 & \\ 
 

    \theutterance \stepcounter{utterance}  

    & \multicolumn{4}{p{0.6\linewidth}}{\cellcolor[rgb]{0.8,1,0.9}{%
	\makecell[{{p{\linewidth}}}]{% 
	  \tt {\tiny [A$\rangle$GM]}  
	 No, husband does not positively affect the alarm clock through the wife.\\ \tt \\ \tt To determine whether the husband positively affects the alarm clock through the wife, we need to consider the causal pathways and the probabilities provided. We are looking for evidence that the husband's action (setting or not setting the alarm) has a direct effect on the alarm ringing, mediated by the wife's action.\\ \tt \\ \tt Let's break down the given probabilities:\\ \tt \\ \tt 1. **Husband doesn't set the alarm, Wife doesn't set the alarm:** Probability of alarm ringing = 11%\\ \tt 2. **Husband doesn't set the alarm, Wife sets the alarm:** Probability of alarm ringing = 60%\\ \tt 3. **Husband sets the alarm, Wife doesn't set the alarm:** Probability of alarm ringing = 46%\\ \tt 4. **Husband sets the alarm, Wife sets the alarm:** Probability of alarm ringing = 92%\\ \tt \\ \tt From these probabilities, we can infer the following:\\ \tt \\ \tt - **Direct Effect of Husband on Alarm:** \\ \tt   - When the husband doesn't set the alarm, the probability of the alarm ringing is 11%, regardless of whether the wife sets the alarm (11% in both cases).\\ \tt   - When the husband sets the alarm, the probability of the alarm ringing increases significantly to 92% if the wife also sets the alarm, and to 46% if the wife doesn't set the alarm.\\ \tt \\ \tt - **Mediating Effect of Wife:**\\ \tt   - The wife's action (setting or not setting the alarm) 
	  } 
	   } 
	   } 
	 & & \\ 
 

    \theutterance \stepcounter{utterance}  

    & & & \multicolumn{2}{p{0.3\linewidth}}{\cellcolor[rgb]{0.95,0.95,0.95}{%
	\makecell[{{p{\linewidth}}}]{% 
	  \tt {\tiny [GM$|$GM]}  
	 No 
	  } 
	   } 
	   } 
	 & & \\ 
 

    \theutterance \stepcounter{utterance}  

    & & & \multicolumn{2}{p{0.3\linewidth}}{\cellcolor[rgb]{0.95,0.95,0.95}{%
	\makecell[{{p{\linewidth}}}]{% 
	  \tt {\tiny [GM$|$GM]}  
	 no 
	  } 
	   } 
	   } 
	 & & \\ 
 

    \theutterance \stepcounter{utterance}  

    & & & \multicolumn{2}{p{0.3\linewidth}}{\cellcolor[rgb]{0.95,0.95,0.95}{%
	\makecell[{{p{\linewidth}}}]{% 
	  \tt {\tiny [GM$|$GM]}  
	 game_result = WIN 
	  } 
	   } 
	   } 
	 & & \\ 
 

\end{supertabular}
}

\end{document}
