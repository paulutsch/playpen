
\documentclass{article}
\usepackage{colortbl}
\usepackage{makecell}
\usepackage{multirow}
\usepackage{supertabular}

\begin{document}

\newcounter{utterance}

\twocolumn

{ \footnotesize  \setcounter{utterance}{1}
\setlength{\tabcolsep}{0pt}
\begin{supertabular}{c@{$\;$}|p{.15\linewidth}@{}p{.15\linewidth}p{.15\linewidth}p{.15\linewidth}p{.15\linewidth}p{.15\linewidth}}

    \# & $\;$A & \multicolumn{4}{c}{Game Master} & $\;\:$B\\
    \hline 

    \theutterance \stepcounter{utterance}  

    & & \multicolumn{4}{p{0.6\linewidth}}{\cellcolor[rgb]{0.9,0.9,0.9}{%
	\makecell[{{p{\linewidth}}}]{% 
	  \tt {\tiny [A$\langle$GM]}  
	 You are an expert in causal inference. The following question is not a typical commonsense query, but rather a meticulously designed question created by a professor specializing in causal inference, intended to assess the students' mastery of the course content.\\ \tt Imagine a self-contained, hypothetical world with only the following conditions, and without any unmentioned factors or causal relationships: Husband has a direct effect on wife and alarm clock. Wife has a direct effect on alarm clock. For husbands that don't set the alarm, the probability of ringing alarm is 78%. For husbands that set the alarm, the probability of ringing alarm is 32%. For husbands that set the alarm, would it be less likely to see ringing alarm if the husband had not set the alarm?\\ \tt Start your answer with 'Yes' or 'No', followed by additional reasoning or evidence to support your explanation. 
	  } 
	   } 
	   } 
	 & \\ 
 

    \theutterance \stepcounter{utterance}  

    & \multicolumn{4}{p{0.6\linewidth}}{\cellcolor[rgb]{0.8,1,0.9}{%
	\makecell[{{p{\linewidth}}}]{% 
	  \tt {\tiny [A$\rangle$GM]}  
	 No, it would not be less likely to see a ringing alarm if the husband had not set the alarm. \\ \tt \\ \tt The question is asking whether the probability of the alarm ringing is lower for husbands who set the alarm compared to those who do not set the alarm. However, we need to carefully consider the given probabilities:\\ \tt \\ \tt - Probability of alarm ringing when not set by husband = 78%\\ \tt - Probability of alarm ringing when set by husband = 32%\\ \tt \\ \tt The key here is to understand the causal relationships and the conditional probabilities provided. The probability of the alarm ringing given that the husband did not set it (78%) is higher than the probability of the alarm ringing given that the husband did set it (32%). \\ \tt \\ \tt This suggests that the act of setting the alarm by the husband does not increase the likelihood of the alarm ringing; instead, it decreases it. The reason for this is that setting the alarm involves some action (like adjusting the time or ensuring it's working properly), which could introduce human error or malfunction, thereby reducing the probability of the alarm ringing.\\ \tt \\ \tt Therefore, if we consider the scenario where the husband did not set the alarm, the probability of the alarm ringing is higher (78%) compared to when he did set it (32%). Hence, it would not be less likely to see the alarm ringing if the husband had not set it; it would actually be more likely.\\ \tt \\ \tt This conclusion is based on the given probabilities and the direct causal effect of the husband's action (setting the alarm) on 
	  } 
	   } 
	   } 
	 & & \\ 
 

    \theutterance \stepcounter{utterance}  

    & & & \multicolumn{2}{p{0.3\linewidth}}{\cellcolor[rgb]{0.95,0.95,0.95}{%
	\makecell[{{p{\linewidth}}}]{% 
	  \tt {\tiny [GM$|$GM]}  
	 No 
	  } 
	   } 
	   } 
	 & & \\ 
 

    \theutterance \stepcounter{utterance}  

    & & & \multicolumn{2}{p{0.3\linewidth}}{\cellcolor[rgb]{0.95,0.95,0.95}{%
	\makecell[{{p{\linewidth}}}]{% 
	  \tt {\tiny [GM$|$GM]}  
	 no 
	  } 
	   } 
	   } 
	 & & \\ 
 

    \theutterance \stepcounter{utterance}  

    & & & \multicolumn{2}{p{0.3\linewidth}}{\cellcolor[rgb]{0.95,0.95,0.95}{%
	\makecell[{{p{\linewidth}}}]{% 
	  \tt {\tiny [GM$|$GM]}  
	 game_result = WIN 
	  } 
	   } 
	   } 
	 & & \\ 
 

\end{supertabular}
}

\end{document}
