
\documentclass{article}
\usepackage{colortbl}
\usepackage{makecell}
\usepackage{multirow}
\usepackage{supertabular}

\begin{document}

\newcounter{utterance}

\twocolumn

{ \footnotesize  \setcounter{utterance}{1}
\setlength{\tabcolsep}{0pt}
\begin{supertabular}{c@{$\;$}|p{.15\linewidth}@{}p{.15\linewidth}p{.15\linewidth}p{.15\linewidth}p{.15\linewidth}p{.15\linewidth}}

    \# & $\;$A & \multicolumn{4}{c}{Game Master} & $\;\:$B\\
    \hline 

    \theutterance \stepcounter{utterance}  

    & & \multicolumn{4}{p{0.6\linewidth}}{\cellcolor[rgb]{0.9,0.9,0.9}{%
	\makecell[{{p{\linewidth}}}]{% 
	  \tt {\tiny [A$\langle$GM]}  
	 You are an expert in causal inference. The following question is not a typical commonsense query, but rather a meticulously designed question created by a professor specializing in causal inference, intended to assess the students' mastery of the course content.\\ \tt Imagine a self-contained, hypothetical world with only the following conditions, and without any unmentioned factors or causal relationships: Jyka has a direct effect on lirg. Hwax has a direct effect on lirg. The overall probability of jyka is 14%. For those who are not jyka, the probability of lirg is 71%. For those who are jyka, the probability of lirg is 66%. Is lirg less likely than not lirg overall?\\ \tt Start your answer with 'Yes' or 'No', followed by additional reasoning or evidence to support your explanation. 
	  } 
	   } 
	   } 
	 & \\ 
 

    \theutterance \stepcounter{utterance}  

    & \multicolumn{4}{p{0.6\linewidth}}{\cellcolor[rgb]{0.8,1,0.9}{%
	\makecell[{{p{\linewidth}}}]{% 
	  \tt {\tiny [A$\rangle$GM]}  
	 No, lirg is not less likely than not lirg overall. To determine this, we need to calculate the overall probability of lirg, taking into account both the probability of being jyka and the conditional probabilities given jyka status.\\ \tt \\ \tt Let's denote:\\ \tt - \( P(J) \) as the probability of being jyka, which is 0.14.\\ \tt - \( P(L|J) \) as the probability of lirg given jyka, which is 0.66.\\ \tt - \( P(L|not J) \) as the probability of lirg given not jyka, which is 0.71.\\ \tt \\ \tt The overall probability of lirg, \( P(L) \), can be calculated using the law of total probability:\\ \tt \\ \tt \[ P(L) = P(L|J) \cdot P(J) + P(L|not J) \cdot P(not J) \]\\ \tt \\ \tt Where \( P(not J) = 1 - P(J) = 1 - 0.14 = 0.86 \).\\ \tt \\ \tt Substituting the given values:\\ \tt \\ \tt \[ P(L) = (0.66 \cdot 0.14) + (0.71 \cdot 0.86) \]\\ \tt \[ P(L) = 0.0924 + 0.6106 \]\\ \tt \[ P(L) = 0.703 \]\\ \tt \\ \tt Since \( P(L) = 0.703 \), which is 
	  } 
	   } 
	   } 
	 & & \\ 
 

    \theutterance \stepcounter{utterance}  

    & & & \multicolumn{2}{p{0.3\linewidth}}{\cellcolor[rgb]{0.95,0.95,0.95}{%
	\makecell[{{p{\linewidth}}}]{% 
	  \tt {\tiny [GM$|$GM]}  
	 No 
	  } 
	   } 
	   } 
	 & & \\ 
 

    \theutterance \stepcounter{utterance}  

    & & & \multicolumn{2}{p{0.3\linewidth}}{\cellcolor[rgb]{0.95,0.95,0.95}{%
	\makecell[{{p{\linewidth}}}]{% 
	  \tt {\tiny [GM$|$GM]}  
	 no 
	  } 
	   } 
	   } 
	 & & \\ 
 

    \theutterance \stepcounter{utterance}  

    & & & \multicolumn{2}{p{0.3\linewidth}}{\cellcolor[rgb]{0.95,0.95,0.95}{%
	\makecell[{{p{\linewidth}}}]{% 
	  \tt {\tiny [GM$|$GM]}  
	 game_result = WIN 
	  } 
	   } 
	   } 
	 & & \\ 
 

\end{supertabular}
}

\end{document}
