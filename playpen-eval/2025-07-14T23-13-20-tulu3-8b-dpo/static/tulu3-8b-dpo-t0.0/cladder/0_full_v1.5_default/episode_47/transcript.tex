
\documentclass{article}
\usepackage{colortbl}
\usepackage{makecell}
\usepackage{multirow}
\usepackage{supertabular}

\begin{document}

\newcounter{utterance}

\twocolumn

{ \footnotesize  \setcounter{utterance}{1}
\setlength{\tabcolsep}{0pt}
\begin{supertabular}{c@{$\;$}|p{.15\linewidth}@{}p{.15\linewidth}p{.15\linewidth}p{.15\linewidth}p{.15\linewidth}p{.15\linewidth}}

    \# & $\;$A & \multicolumn{4}{c}{Game Master} & $\;\:$B\\
    \hline 

    \theutterance \stepcounter{utterance}  

    & & \multicolumn{4}{p{0.6\linewidth}}{\cellcolor[rgb]{0.9,0.9,0.9}{%
	\makecell[{{p{\linewidth}}}]{% 
	  \tt {\tiny [A$\langle$GM]}  
	 You are an expert in causal inference. The following question is not a typical commonsense query, but rather a meticulously designed question created by a professor specializing in causal inference, intended to assess the students' mastery of the course content.\\ \tt Imagine a self-contained, hypothetical world with only the following conditions, and without any unmentioned factors or causal relationships: Maternal smoking status has a direct effect on infant's birth weight and freckles. Health condition has a direct effect on infant's birth weight and freckles. Infant's birth weight has a direct effect on freckles. Health condition is unobserved. Method 1: We look directly at how maternal smoking status correlates with freckles in general. Method 2: We look at this correlation case by case according to infant's birth weight. To understand how maternal smoking status affects freckles, is it more correct to use the Method 1 than Method 2?\\ \tt Start your answer with 'Yes' or 'No', followed by additional reasoning or evidence to support your explanation. 
	  } 
	   } 
	   } 
	 & \\ 
 

    \theutterance \stepcounter{utterance}  

    & \multicolumn{4}{p{0.6\linewidth}}{\cellcolor[rgb]{0.8,1,0.9}{%
	\makecell[{{p{\linewidth}}}]{% 
	  \tt {\tiny [A$\rangle$GM]}  
	 No, it is not more correct to use Method 1 than Method 2 to understand how maternal smoking status affects freckles.\\ \tt \\ \tt The reason is that Method 1, which looks directly at the correlation between maternal smoking status and freckles in general, does not account for the potential confounding variable, health condition, which has a direct effect on both birth weight and freckles. Since health condition is unobserved, using Method 1 could lead to a spurious correlation where the observed association between maternal smoking and freckles is actually due to the unmeasured health condition, not the smoking itself.\\ \tt \\ \tt Method 2, on the other hand, attempts to control for the effect of birth weight, which is directly influenced by both maternal smoking status and health condition. By examining the correlation case by case according to infant's birth weight, Method 2 can potentially adjust for the confounding effect of birth weight, which in turn might be influenced by the unobserved health condition. This method is more likely to provide a more accurate estimate of the causal effect of maternal smoking on freckles, as it attempts to account for the mediating variable (birth weight) that lies on the causal pathway.\\ \tt \\ \tt However, it's important to note that even Method 2 might not fully eliminate the bias due to the unobserved health condition, as it is still a confounder. A more rigorous approach would involve using statistical methods designed to handle unobserved confounding, such as instrumental 
	  } 
	   } 
	   } 
	 & & \\ 
 

    \theutterance \stepcounter{utterance}  

    & & & \multicolumn{2}{p{0.3\linewidth}}{\cellcolor[rgb]{0.95,0.95,0.95}{%
	\makecell[{{p{\linewidth}}}]{% 
	  \tt {\tiny [GM$|$GM]}  
	 No 
	  } 
	   } 
	   } 
	 & & \\ 
 

    \theutterance \stepcounter{utterance}  

    & & & \multicolumn{2}{p{0.3\linewidth}}{\cellcolor[rgb]{0.95,0.95,0.95}{%
	\makecell[{{p{\linewidth}}}]{% 
	  \tt {\tiny [GM$|$GM]}  
	 yes 
	  } 
	   } 
	   } 
	 & & \\ 
 

    \theutterance \stepcounter{utterance}  

    & & & \multicolumn{2}{p{0.3\linewidth}}{\cellcolor[rgb]{0.95,0.95,0.95}{%
	\makecell[{{p{\linewidth}}}]{% 
	  \tt {\tiny [GM$|$GM]}  
	 game_result = LOSE 
	  } 
	   } 
	   } 
	 & & \\ 
 

\end{supertabular}
}

\end{document}
