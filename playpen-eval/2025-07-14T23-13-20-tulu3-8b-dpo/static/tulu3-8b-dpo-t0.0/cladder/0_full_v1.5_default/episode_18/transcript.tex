
\documentclass{article}
\usepackage{colortbl}
\usepackage{makecell}
\usepackage{multirow}
\usepackage{supertabular}

\begin{document}

\newcounter{utterance}

\twocolumn

{ \footnotesize  \setcounter{utterance}{1}
\setlength{\tabcolsep}{0pt}
\begin{supertabular}{c@{$\;$}|p{.15\linewidth}@{}p{.15\linewidth}p{.15\linewidth}p{.15\linewidth}p{.15\linewidth}p{.15\linewidth}}

    \# & $\;$A & \multicolumn{4}{c}{Game Master} & $\;\:$B\\
    \hline 

    \theutterance \stepcounter{utterance}  

    & & \multicolumn{4}{p{0.6\linewidth}}{\cellcolor[rgb]{0.9,0.9,0.9}{%
	\makecell[{{p{\linewidth}}}]{% 
	  \tt {\tiny [A$\langle$GM]}  
	 You are an expert in causal inference. The following question is not a typical commonsense query, but rather a meticulously designed question created by a professor specializing in causal inference, intended to assess the students' mastery of the course content.\\ \tt Imagine a self-contained, hypothetical world with only the following conditions, and without any unmentioned factors or causal relationships: CEO has a direct effect on director and manager. Manager has a direct effect on employee. Director has a direct effect on employee. The overall probability of manager signing the termination letter is 35%. For managers who don't sign termination letters, the probability of employee being fired is 76%. For managers who sign termination letters, the probability of employee being fired is 42%. Is employee being fired more likely than employee not being fired overall?\\ \tt Start your answer with 'Yes' or 'No', followed by additional reasoning or evidence to support your explanation. 
	  } 
	   } 
	   } 
	 & \\ 
 

    \theutterance \stepcounter{utterance}  

    & \multicolumn{4}{p{0.6\linewidth}}{\cellcolor[rgb]{0.8,1,0.9}{%
	\makecell[{{p{\linewidth}}}]{% 
	  \tt {\tiny [A$\rangle$GM]}  
	 No, the employee is not more likely to be fired overall. To determine the overall probability of an employee being fired, we need to consider both pathways through which an employee can be fired: either directly by the director or indirectly through the manager signing a termination letter.\\ \tt \\ \tt Given:\\ \tt - Probability of manager signing the termination letter (P(S)) = 35% = 0.35\\ \tt - Probability of employee being fired given the manager signs (P(F|S)) = 42% = 0.42\\ \tt - Probability of employee being fired given the manager does not sign (P(F|¬S)) = 76% = 0.76\\ \tt \\ \tt The probability of the manager not signing the termination letter is:\\ \tt P(¬S) = 1 - P(S) = 1 - 0.35 = 0.65\\ \tt \\ \tt Now, we calculate the overall probability of the employee being fired (P(F)):\\ \tt P(F) = P(F|S) * P(S) + P(F|¬S) * P(¬S)\\ \tt \\ \tt Substituting the given probabilities:\\ \tt P(F) = (0.42 * 0.35) + (0.76 * 0.65)\\ \tt \\ \tt Calculating each term:\\ \tt P(F|S) * P(S) = 0.42 * 0.35 = 0.147\\ \tt P(F|¬S) * P(¬S) = 0.76 * 0.65 = 0. 
	  } 
	   } 
	   } 
	 & & \\ 
 

    \theutterance \stepcounter{utterance}  

    & & & \multicolumn{2}{p{0.3\linewidth}}{\cellcolor[rgb]{0.95,0.95,0.95}{%
	\makecell[{{p{\linewidth}}}]{% 
	  \tt {\tiny [GM$|$GM]}  
	 No 
	  } 
	   } 
	   } 
	 & & \\ 
 

    \theutterance \stepcounter{utterance}  

    & & & \multicolumn{2}{p{0.3\linewidth}}{\cellcolor[rgb]{0.95,0.95,0.95}{%
	\makecell[{{p{\linewidth}}}]{% 
	  \tt {\tiny [GM$|$GM]}  
	 yes 
	  } 
	   } 
	   } 
	 & & \\ 
 

    \theutterance \stepcounter{utterance}  

    & & & \multicolumn{2}{p{0.3\linewidth}}{\cellcolor[rgb]{0.95,0.95,0.95}{%
	\makecell[{{p{\linewidth}}}]{% 
	  \tt {\tiny [GM$|$GM]}  
	 game_result = LOSE 
	  } 
	   } 
	   } 
	 & & \\ 
 

\end{supertabular}
}

\end{document}
