
\documentclass{article}
\usepackage{colortbl}
\usepackage{makecell}
\usepackage{multirow}
\usepackage{supertabular}

\begin{document}

\newcounter{utterance}

\twocolumn

{ \footnotesize  \setcounter{utterance}{1}
\setlength{\tabcolsep}{0pt}
\begin{supertabular}{c@{$\;$}|p{.15\linewidth}@{}p{.15\linewidth}p{.15\linewidth}p{.15\linewidth}p{.15\linewidth}p{.15\linewidth}}

    \# & $\;$A & \multicolumn{4}{c}{Game Master} & $\;\:$B\\
    \hline 

    \theutterance \stepcounter{utterance}  

    & & \multicolumn{4}{p{0.6\linewidth}}{\cellcolor[rgb]{0.9,0.9,0.9}{%
	\makecell[{{p{\linewidth}}}]{% 
	  \tt {\tiny [A$\langle$GM]}  
	 You are an expert in causal inference. The following question is not a typical commonsense query, but rather a meticulously designed question created by a professor specializing in causal inference, intended to assess the students' mastery of the course content.\\ \tt Imagine a self-contained, hypothetical world with only the following conditions, and without any unmentioned factors or causal relationships: Ability to speak english has a direct effect on diabetes and lifespan. Smoking has a direct effect on diabetes and lifespan. Diabetes has a direct effect on lifespan. Smoking is unobserved. For people who do not speak english, the probability of long lifespan is 70%. For people who speak english, the probability of long lifespan is 38%. For people who speak english, would it be less likely to see long lifespan if the person did not speak english?\\ \tt Start your answer with 'Yes' or 'No', followed by additional reasoning or evidence to support your explanation. 
	  } 
	   } 
	   } 
	 & \\ 
 

    \theutterance \stepcounter{utterance}  

    & \multicolumn{4}{p{0.6\linewidth}}{\cellcolor[rgb]{0.8,1,0.9}{%
	\makecell[{{p{\linewidth}}}]{% 
	  \tt {\tiny [A$\rangle$GM]}  
	 No, it would not be less likely to see a long lifespan for people who speak English if they did not speak English, given the conditions provided.\\ \tt \\ \tt The problem presents a scenario where speaking English and smoking are both directly related to diabetes and lifespan, but smoking is unobserved. The key to understanding this question lies in the concept of confounding variables and the principle of counterfactual reasoning in causal inference.\\ \tt \\ \tt Here's the reasoning:\\ \tt \\ \tt 1. **Direct Effects**: Speaking English and smoking both have direct effects on diabetes and lifespan. However, we are not given the specific nature of these effects (positive or negative), only their existence.\\ \tt \\ \tt 2. **Unobserved Variable**: Smoking is unobserved, which means we cannot directly account for its effect in our analysis. This is a classic confounding variable problem.\\ \tt \\ \tt 3. **Observed Data**: We know:\\ \tt    - Probability of long lifespan for non-English speakers = 70%\\ \tt    - Probability of long lifespan for English speakers = 38%\\ \tt \\ \tt 4. **Counterfactual Reasoning**: To answer the question, we need to consider the counterfactual: what would happen if English speakers did not speak English (i.e., their language ability changed from speaking English to not speaking English).\\ \tt \\ \tt 5. **Assumption of No Spurious Relationship**: We are assuming that the difference in lifespan probability between English speakers and non-English speakers is due to the language ability itself, and not confounded by other variables like smoking.\\ \tt \\ \tt 6. **C 
	  } 
	   } 
	   } 
	 & & \\ 
 

    \theutterance \stepcounter{utterance}  

    & & & \multicolumn{2}{p{0.3\linewidth}}{\cellcolor[rgb]{0.95,0.95,0.95}{%
	\makecell[{{p{\linewidth}}}]{% 
	  \tt {\tiny [GM$|$GM]}  
	 No 
	  } 
	   } 
	   } 
	 & & \\ 
 

    \theutterance \stepcounter{utterance}  

    & & & \multicolumn{2}{p{0.3\linewidth}}{\cellcolor[rgb]{0.95,0.95,0.95}{%
	\makecell[{{p{\linewidth}}}]{% 
	  \tt {\tiny [GM$|$GM]}  
	 no 
	  } 
	   } 
	   } 
	 & & \\ 
 

    \theutterance \stepcounter{utterance}  

    & & & \multicolumn{2}{p{0.3\linewidth}}{\cellcolor[rgb]{0.95,0.95,0.95}{%
	\makecell[{{p{\linewidth}}}]{% 
	  \tt {\tiny [GM$|$GM]}  
	 game_result = WIN 
	  } 
	   } 
	   } 
	 & & \\ 
 

\end{supertabular}
}

\end{document}
