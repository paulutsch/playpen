
\documentclass{article}
\usepackage{colortbl}
\usepackage{makecell}
\usepackage{multirow}
\usepackage{supertabular}

\begin{document}

\newcounter{utterance}

\twocolumn

{ \footnotesize  \setcounter{utterance}{1}
\setlength{\tabcolsep}{0pt}
\begin{supertabular}{c@{$\;$}|p{.15\linewidth}@{}p{.15\linewidth}p{.15\linewidth}p{.15\linewidth}p{.15\linewidth}p{.15\linewidth}}

    \# & $\;$A & \multicolumn{4}{c}{Game Master} & $\;\:$B\\
    \hline 

    \theutterance \stepcounter{utterance}  

    & & \multicolumn{4}{p{0.6\linewidth}}{\cellcolor[rgb]{0.9,0.9,0.9}{%
	\makecell[{{p{\linewidth}}}]{% 
	  \tt {\tiny [A$\langle$GM]}  
	 You are an expert in causal inference. The following question is not a typical commonsense query, but rather a meticulously designed question created by a professor specializing in causal inference, intended to assess the students' mastery of the course content.\\ \tt Imagine a self-contained, hypothetical world with only the following conditions, and without any unmentioned factors or causal relationships: Gender has a direct effect on department competitiveness and admission status. Residency status has a direct effect on department competitiveness and admission status. Department competitiveness has a direct effect on admission status. Residency status is unobserved. For individuals who are not male, the probability of admission acceptance is 40%. For individuals who are male, the probability of admission acceptance is 32%. Will male gender increase the chance of admission acceptance?\\ \tt Start your answer with 'Yes' or 'No', followed by additional reasoning or evidence to support your explanation. 
	  } 
	   } 
	   } 
	 & \\ 
 

    \theutterance \stepcounter{utterance}  

    & \multicolumn{4}{p{0.6\linewidth}}{\cellcolor[rgb]{0.8,1,0.9}{%
	\makecell[{{p{\linewidth}}}]{% 
	  \tt {\tiny [A$\rangle$GM]}  
	 No, male gender does not necessarily increase the chance of admission acceptance, even given the direct effects specified in the problem.\\ \tt \\ \tt The problem states that gender has a direct effect on department competitiveness and admission status, and department competitiveness has a direct effect on admission status. However, it does not provide a direct causal relationship between gender and admission status. \\ \tt \\ \tt Here's the breakdown:\\ \tt \\ \tt 1. **Gender and Department Competitiveness**: Gender affects department competitiveness. This means that departments might be more or less competitive depending on the gender composition of their applicants. However, this does not directly tell us how gender affects admission status.\\ \tt \\ \tt 2. **Department Competitiveness and Admission Status**: Department competitiveness affects admission status. A more competitive department might have lower admission rates because it can be more selective.\\ \tt \\ \tt 3. **Gender and Admission Status**: The problem states that for individuals who are not male, the probability of admission acceptance is 40%, and for males, it is 32%. This suggests a direct effect of gender on admission status. However, this effect could be confounded by the effect of gender on department competitiveness.\\ \tt \\ \tt To determine whether male gender increases the chance of admission acceptance, we need to consider the indirect effect of gender through department competitiveness:\\ \tt \\ \tt - **Indirect Effect**: If male applicants tend to come from less competitive departments (due to the direct effect of gender on department competitiveness), this might actually increase their chances of admission because less competitive departments might have higher acceptance rates.\\ \tt \\ \tt - **Direct Effect**: The direct effect of gender 
	  } 
	   } 
	   } 
	 & & \\ 
 

    \theutterance \stepcounter{utterance}  

    & & & \multicolumn{2}{p{0.3\linewidth}}{\cellcolor[rgb]{0.95,0.95,0.95}{%
	\makecell[{{p{\linewidth}}}]{% 
	  \tt {\tiny [GM$|$GM]}  
	 No 
	  } 
	   } 
	   } 
	 & & \\ 
 

    \theutterance \stepcounter{utterance}  

    & & & \multicolumn{2}{p{0.3\linewidth}}{\cellcolor[rgb]{0.95,0.95,0.95}{%
	\makecell[{{p{\linewidth}}}]{% 
	  \tt {\tiny [GM$|$GM]}  
	 no 
	  } 
	   } 
	   } 
	 & & \\ 
 

    \theutterance \stepcounter{utterance}  

    & & & \multicolumn{2}{p{0.3\linewidth}}{\cellcolor[rgb]{0.95,0.95,0.95}{%
	\makecell[{{p{\linewidth}}}]{% 
	  \tt {\tiny [GM$|$GM]}  
	 game_result = WIN 
	  } 
	   } 
	   } 
	 & & \\ 
 

\end{supertabular}
}

\end{document}
