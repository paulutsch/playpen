
\documentclass{article}
\usepackage{colortbl}
\usepackage{makecell}
\usepackage{multirow}
\usepackage{supertabular}

\begin{document}

\newcounter{utterance}

\twocolumn

{ \footnotesize  \setcounter{utterance}{1}
\setlength{\tabcolsep}{0pt}
\begin{supertabular}{c@{$\;$}|p{.15\linewidth}@{}p{.15\linewidth}p{.15\linewidth}p{.15\linewidth}p{.15\linewidth}p{.15\linewidth}}

    \# & $\;$A & \multicolumn{4}{c}{Game Master} & $\;\:$B\\
    \hline 

    \theutterance \stepcounter{utterance}  

    & & \multicolumn{4}{p{0.6\linewidth}}{\cellcolor[rgb]{0.9,0.9,0.9}{%
	\makecell[{{p{\linewidth}}}]{% 
	  \tt {\tiny [A$\langle$GM]}  
	 You are an expert in causal inference. The following question is not a typical commonsense query, but rather a meticulously designed question created by a professor specializing in causal inference, intended to assess the students' mastery of the course content.\\ \tt Imagine a self-contained, hypothetical world with only the following conditions, and without any unmentioned factors or causal relationships: Unobserved confounders has a direct effect on education level and salary. Proximity to a college has a direct effect on education level. Education level has a direct effect on salary. Unobserved confounders is unobserved. For people living far from a college, the probability of high salary is 39%. For people living close to a college, the probability of high salary is 49%. For people living far from a college, the probability of college degree or higher is 57%. For people living close to a college, the probability of college degree or higher is 30%. Will college degree or higher increase the chance of high salary?\\ \tt Start your answer with 'Yes' or 'No', followed by additional reasoning or evidence to support your explanation. 
	  } 
	   } 
	   } 
	 & \\ 
 

    \theutterance \stepcounter{utterance}  

    & \multicolumn{4}{p{0.6\linewidth}}{\cellcolor[rgb]{0.8,1,0.9}{%
	\makecell[{{p{\linewidth}}}]{% 
	  \tt {\tiny [A$\rangle$GM]}  
	 No, a college degree or higher does not necessarily increase the chance of a high salary, given the conditions provided.\\ \tt \\ \tt The key to understanding this lies in the presence of unobserved confounders that affect both education level and salary, and the fact that proximity to a college also affects education level. Here's a breakdown of the reasoning:\\ \tt \\ \tt 1. **Unobserved Confounders**: These are factors that affect both education level and salary but are not observed in the data. They could include innate abilities, family background, or other environmental factors that influence both one's educational attainment and their earning potential.\\ \tt \\ \tt 2. **Proximity to College**: People living close to a college have a higher probability of obtaining a college degree (30%) compared to those living far away (57%). This suggests that geographical proximity to a college is a significant factor in educational attainment.\\ \tt \\ \tt 3. **Education Level and Salary**: Education level has a direct effect on salary. However, the presence of unobserved confounders means that the relationship between education and salary might be confounded. People with higher innate abilities or better family backgrounds might both be more likely to attend college and earn higher salaries, regardless of the college proximity.\\ \tt \\ \tt 4. **Observed Probabilities**: \\ \tt    - For those living far from a college, the probability of a high salary is 39%, but the probability of having a college degree or higher is 57%. This suggests that having a college degree does not guarantee a high salary, especially when 
	  } 
	   } 
	   } 
	 & & \\ 
 

    \theutterance \stepcounter{utterance}  

    & & & \multicolumn{2}{p{0.3\linewidth}}{\cellcolor[rgb]{0.95,0.95,0.95}{%
	\makecell[{{p{\linewidth}}}]{% 
	  \tt {\tiny [GM$|$GM]}  
	 No 
	  } 
	   } 
	   } 
	 & & \\ 
 

    \theutterance \stepcounter{utterance}  

    & & & \multicolumn{2}{p{0.3\linewidth}}{\cellcolor[rgb]{0.95,0.95,0.95}{%
	\makecell[{{p{\linewidth}}}]{% 
	  \tt {\tiny [GM$|$GM]}  
	 no 
	  } 
	   } 
	   } 
	 & & \\ 
 

    \theutterance \stepcounter{utterance}  

    & & & \multicolumn{2}{p{0.3\linewidth}}{\cellcolor[rgb]{0.95,0.95,0.95}{%
	\makecell[{{p{\linewidth}}}]{% 
	  \tt {\tiny [GM$|$GM]}  
	 game_result = WIN 
	  } 
	   } 
	   } 
	 & & \\ 
 

\end{supertabular}
}

\end{document}
