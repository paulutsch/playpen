
\documentclass{article}
\usepackage{colortbl}
\usepackage{makecell}
\usepackage{multirow}
\usepackage{supertabular}

\begin{document}

\newcounter{utterance}

\twocolumn

{ \footnotesize  \setcounter{utterance}{1}
\setlength{\tabcolsep}{0pt}
\begin{supertabular}{c@{$\;$}|p{.15\linewidth}@{}p{.15\linewidth}p{.15\linewidth}p{.15\linewidth}p{.15\linewidth}p{.15\linewidth}}

    \# & $\;$A & \multicolumn{4}{c}{Game Master} & $\;\:$B\\
    \hline 

    \theutterance \stepcounter{utterance}  

    & & \multicolumn{4}{p{0.6\linewidth}}{\cellcolor[rgb]{0.9,0.9,0.9}{%
	\makecell[{{p{\linewidth}}}]{% 
	  \tt {\tiny [A$\langle$GM]}  
	 You are an expert in causal inference. The following question is not a typical commonsense query, but rather a meticulously designed question created by a professor specializing in causal inference, intended to assess the students' mastery of the course content.\\ \tt Imagine a self-contained, hypothetical world with only the following conditions, and without any unmentioned factors or causal relationships: Husband has a direct effect on wife and alarm clock. Wife has a direct effect on alarm clock. The overall probability of alarm set by husband is 17%. The probability of alarm not set by husband and ringing alarm is 69%. The probability of alarm set by husband and ringing alarm is 3%. Is the chance of ringing alarm smaller when observing alarm set by husband?\\ \tt Start your answer with 'Yes' or 'No', followed by additional reasoning or evidence to support your explanation. 
	  } 
	   } 
	   } 
	 & \\ 
 

    \theutterance \stepcounter{utterance}  

    & \multicolumn{4}{p{0.6\linewidth}}{\cellcolor[rgb]{0.8,1,0.9}{%
	\makecell[{{p{\linewidth}}}]{% 
	  \tt {\tiny [A$\rangle$GM]}  
	 No, the chance of the alarm ringing is not smaller when observing that it was set by the husband.\\ \tt \\ \tt To solve this problem, we need to consider the probabilities given and understand the relationships between the variables.\\ \tt \\ \tt Let's define the events:\\ \tt - \( A \): Alarm is set by the husband.\\ \tt - \( R \): Alarm is ringing.\\ \tt \\ \tt We are given the following probabilities:\\ \tt - \( P(A) = 0.17 \) (Probability that the alarm is set by the husband)\\ \tt - \( P(R|A^c) = 0.69 \) (Probability that the alarm rings given it was not set by the husband)\\ \tt - \( P(R \cap A) = 0.03 \) (Probability that the alarm rings and it was set by the husband)\\ \tt \\ \tt We want to determine if \( P(R|A) < P(R) \), i.e., whether the probability of the alarm ringing given that it was set by the husband is less than the overall probability of the alarm ringing.\\ \tt \\ \tt First, calculate the overall probability of the alarm ringing, \( P(R) \):\\ \tt \\ \tt \[ P(R) = P(R \cap A) + P(R|A^c) \cdot P(A^c) \]\\ \tt \\ \tt Where \( P(A^c) = 1 - P(A) = 1 - 0.17 = 0.83 \).\\ \tt \\ \tt Substitute the known values:\\ \tt \\ \tt \[ P(R) = 0.03 + 0.69 \times 
	  } 
	   } 
	   } 
	 & & \\ 
 

    \theutterance \stepcounter{utterance}  

    & & & \multicolumn{2}{p{0.3\linewidth}}{\cellcolor[rgb]{0.95,0.95,0.95}{%
	\makecell[{{p{\linewidth}}}]{% 
	  \tt {\tiny [GM$|$GM]}  
	 No 
	  } 
	   } 
	   } 
	 & & \\ 
 

    \theutterance \stepcounter{utterance}  

    & & & \multicolumn{2}{p{0.3\linewidth}}{\cellcolor[rgb]{0.95,0.95,0.95}{%
	\makecell[{{p{\linewidth}}}]{% 
	  \tt {\tiny [GM$|$GM]}  
	 yes 
	  } 
	   } 
	   } 
	 & & \\ 
 

    \theutterance \stepcounter{utterance}  

    & & & \multicolumn{2}{p{0.3\linewidth}}{\cellcolor[rgb]{0.95,0.95,0.95}{%
	\makecell[{{p{\linewidth}}}]{% 
	  \tt {\tiny [GM$|$GM]}  
	 game_result = LOSE 
	  } 
	   } 
	   } 
	 & & \\ 
 

\end{supertabular}
}

\end{document}
