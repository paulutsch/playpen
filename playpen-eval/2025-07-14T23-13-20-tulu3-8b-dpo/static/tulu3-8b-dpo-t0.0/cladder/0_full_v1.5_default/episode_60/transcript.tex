
\documentclass{article}
\usepackage{colortbl}
\usepackage{makecell}
\usepackage{multirow}
\usepackage{supertabular}

\begin{document}

\newcounter{utterance}

\twocolumn

{ \footnotesize  \setcounter{utterance}{1}
\setlength{\tabcolsep}{0pt}
\begin{supertabular}{c@{$\;$}|p{.15\linewidth}@{}p{.15\linewidth}p{.15\linewidth}p{.15\linewidth}p{.15\linewidth}p{.15\linewidth}}

    \# & $\;$A & \multicolumn{4}{c}{Game Master} & $\;\:$B\\
    \hline 

    \theutterance \stepcounter{utterance}  

    & & \multicolumn{4}{p{0.6\linewidth}}{\cellcolor[rgb]{0.9,0.9,0.9}{%
	\makecell[{{p{\linewidth}}}]{% 
	  \tt {\tiny [A$\langle$GM]}  
	 You are an expert in causal inference. The following question is not a typical commonsense query, but rather a meticulously designed question created by a professor specializing in causal inference, intended to assess the students' mastery of the course content.\\ \tt Imagine a self-contained, hypothetical world with only the following conditions, and without any unmentioned factors or causal relationships: Unobserved confounders has a direct effect on education level and foot size. Proximity to a college has a direct effect on education level. Education level has a direct effect on foot size. Unobserved confounders is unobserved. The overall probability of college degree or higher is 47%. The probability of high school degree or lower and large feet is 32%. The probability of college degree or higher and large feet is 10%. Is the chance of large feet smaller when observing college degree or higher?\\ \tt Start your answer with 'Yes' or 'No', followed by additional reasoning or evidence to support your explanation. 
	  } 
	   } 
	   } 
	 & \\ 
 

    \theutterance \stepcounter{utterance}  

    & \multicolumn{4}{p{0.6\linewidth}}{\cellcolor[rgb]{0.8,1,0.9}{%
	\makecell[{{p{\linewidth}}}]{% 
	  \tt {\tiny [A$\rangle$GM]}  
	 No, the chance of having large feet is not smaller when observing a college degree or higher. \\ \tt \\ \tt To understand why, let's break down the given causal relationships and probabilities:\\ \tt \\ \tt 1. **Direct Effects:**\\ \tt    - Unobserved confounders affect both education level and foot size.\\ \tt    - Proximity to a college affects education level.\\ \tt    - Education level affects foot size.\\ \tt \\ \tt 2. **Probabilities:**\\ \tt    - Probability of having a college degree or higher (P(C)) = 47%.\\ \tt    - Probability of having a high school degree or lower and large feet (P(HS ∧ L)) = 32%.\\ \tt    - Probability of having a college degree or higher and large feet (P(C ∧ L)) = 10%.\\ \tt \\ \tt 3. **Calculation of P(L | C):**\\ \tt    To find the probability of having large feet given a college degree or higher (P(L | C)), we use Bayes' Theorem:\\ \tt \\ \tt    \[\\ \tt    P(L | C) = \frac{P(C ∧ L)}{P(C)}\\ \tt    \]\\ \tt \\ \tt    Substituting the given probabilities:\\ \tt \\ \tt    \[\\ \tt    P(L | C) = \frac{0.10}{0.47} \approx 0.213\\ \tt    \]\\ \tt \\ \tt    This means the probability of having large feet given a college degree or higher is approximately 21.3%.\\ \tt \\ \tt 4. **Calculation of P(L):**\\ \tt    To find the overall probability of having large feet (P(L 
	  } 
	   } 
	   } 
	 & & \\ 
 

    \theutterance \stepcounter{utterance}  

    & & & \multicolumn{2}{p{0.3\linewidth}}{\cellcolor[rgb]{0.95,0.95,0.95}{%
	\makecell[{{p{\linewidth}}}]{% 
	  \tt {\tiny [GM$|$GM]}  
	 No 
	  } 
	   } 
	   } 
	 & & \\ 
 

    \theutterance \stepcounter{utterance}  

    & & & \multicolumn{2}{p{0.3\linewidth}}{\cellcolor[rgb]{0.95,0.95,0.95}{%
	\makecell[{{p{\linewidth}}}]{% 
	  \tt {\tiny [GM$|$GM]}  
	 yes 
	  } 
	   } 
	   } 
	 & & \\ 
 

    \theutterance \stepcounter{utterance}  

    & & & \multicolumn{2}{p{0.3\linewidth}}{\cellcolor[rgb]{0.95,0.95,0.95}{%
	\makecell[{{p{\linewidth}}}]{% 
	  \tt {\tiny [GM$|$GM]}  
	 game_result = LOSE 
	  } 
	   } 
	   } 
	 & & \\ 
 

\end{supertabular}
}

\end{document}
