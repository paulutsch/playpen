
\documentclass{article}
\usepackage{colortbl}
\usepackage{makecell}
\usepackage{multirow}
\usepackage{supertabular}

\begin{document}

\newcounter{utterance}

\twocolumn

{ \footnotesize  \setcounter{utterance}{1}
\setlength{\tabcolsep}{0pt}
\begin{supertabular}{c@{$\;$}|p{.15\linewidth}@{}p{.15\linewidth}p{.15\linewidth}p{.15\linewidth}p{.15\linewidth}p{.15\linewidth}}

    \# & $\;$A & \multicolumn{4}{c}{Game Master} & $\;\:$B\\
    \hline 

    \theutterance \stepcounter{utterance}  

    & & \multicolumn{4}{p{0.6\linewidth}}{\cellcolor[rgb]{0.9,0.9,0.9}{%
	\makecell[{{p{\linewidth}}}]{% 
	  \tt {\tiny [A$\langle$GM]}  
	 You are an expert in causal inference. The following question is not a typical commonsense query, but rather a meticulously designed question created by a professor specializing in causal inference, intended to assess the students' mastery of the course content.\\ \tt Imagine a self-contained, hypothetical world with only the following conditions, and without any unmentioned factors or causal relationships: Gender has a direct effect on smoking and lung cancer. Smoking has a direct effect on tar deposit. Tar deposit has a direct effect on lung cancer. Gender is unobserved. The overall probability of smoking is 12%. For nonsmokers, the probability of lung cancer is 58%. For smokers, the probability of lung cancer is 61%. Is lung cancer more likely than absence of lung cancer overall?\\ \tt Start your answer with 'Yes' or 'No', followed by additional reasoning or evidence to support your explanation. 
	  } 
	   } 
	   } 
	 & \\ 
 

    \theutterance \stepcounter{utterance}  

    & \multicolumn{4}{p{0.6\linewidth}}{\cellcolor[rgb]{0.8,1,0.9}{%
	\makecell[{{p{\linewidth}}}]{% 
	  \tt {\tiny [A$\rangle$GM]}  
	 No, lung cancer is not more likely than the absence of lung cancer overall. \\ \tt \\ \tt To understand this, we need to consider the probabilities and the relationships between the variables:\\ \tt \\ \tt 1. **Gender and Smoking**: Gender has a direct effect on smoking, but since gender is unobserved, we cannot use it to make predictions.\\ \tt \\ \tt 2. **Smoking and Lung Cancer**: The probability of lung cancer for smokers is 61%, and for nonsmokers, it is 58%. This indicates that smoking increases the risk of lung cancer, but the increase is relatively small (3%).\\ \tt \\ \tt 3. **Tar Deposit and Lung Cancer**: Tar deposit has a direct effect on lung cancer, but since we are not given the specific probabilities related to tar deposits, we cannot incorporate this into our direct calculation.\\ \tt \\ \tt Given the probabilities:\\ \tt - Probability of being a smoker = 12% = 0.12\\ \tt - Probability of lung cancer given smoking (P(Lung Cancer | Smoking)) = 61% = 0.61\\ \tt - Probability of lung cancer given not smoking (P(Lung Cancer | Not Smoking)) = 58% = 0.58\\ \tt \\ \tt We can calculate the overall probability of lung cancer (P(Lung Cancer)) using the law of total probability:\\ \tt \\ \tt \[ P(Lung Cancer) = P(Lung Cancer | Smoking) \cdot P(Smoking) + P(Lung Cancer | Not Smoking) \cdot P(Not Smoking) \]\\ \tt \\ \tt Where \( P(Not Smoking) = 
	  } 
	   } 
	   } 
	 & & \\ 
 

    \theutterance \stepcounter{utterance}  

    & & & \multicolumn{2}{p{0.3\linewidth}}{\cellcolor[rgb]{0.95,0.95,0.95}{%
	\makecell[{{p{\linewidth}}}]{% 
	  \tt {\tiny [GM$|$GM]}  
	 No 
	  } 
	   } 
	   } 
	 & & \\ 
 

    \theutterance \stepcounter{utterance}  

    & & & \multicolumn{2}{p{0.3\linewidth}}{\cellcolor[rgb]{0.95,0.95,0.95}{%
	\makecell[{{p{\linewidth}}}]{% 
	  \tt {\tiny [GM$|$GM]}  
	 yes 
	  } 
	   } 
	   } 
	 & & \\ 
 

    \theutterance \stepcounter{utterance}  

    & & & \multicolumn{2}{p{0.3\linewidth}}{\cellcolor[rgb]{0.95,0.95,0.95}{%
	\makecell[{{p{\linewidth}}}]{% 
	  \tt {\tiny [GM$|$GM]}  
	 game_result = LOSE 
	  } 
	   } 
	   } 
	 & & \\ 
 

\end{supertabular}
}

\end{document}
