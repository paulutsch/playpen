
\documentclass{article}
\usepackage{colortbl}
\usepackage{makecell}
\usepackage{multirow}
\usepackage{supertabular}

\begin{document}

\newcounter{utterance}

\twocolumn

{ \footnotesize  \setcounter{utterance}{1}
\setlength{\tabcolsep}{0pt}
\begin{supertabular}{c@{$\;$}|p{.15\linewidth}@{}p{.15\linewidth}p{.15\linewidth}p{.15\linewidth}p{.15\linewidth}p{.15\linewidth}}

    \# & $\;$A & \multicolumn{4}{c}{Game Master} & $\;\:$B\\
    \hline 

    \theutterance \stepcounter{utterance}  

    & & \multicolumn{4}{p{0.6\linewidth}}{\cellcolor[rgb]{0.9,0.9,0.9}{%
	\makecell[{{p{\linewidth}}}]{% 
	  \tt {\tiny [A$\langle$GM]}  
	 You are an expert in causal inference. The following question is not a typical commonsense query, but rather a meticulously designed question created by a professor specializing in causal inference, intended to assess the students' mastery of the course content.\\ \tt Imagine a self-contained, hypothetical world with only the following conditions, and without any unmentioned factors or causal relationships: Kidney stone size has a direct effect on treatment and lip thickness. Treatment has a direct effect on lip thickness. For patients who have small kidney stones and not receiving treatment, the probability of thick lips is 6%. For patients who have small kidney stones and receiving treatment, the probability of thick lips is 38%. For patients who have large kidney stones and not receiving treatment, the probability of thick lips is 63%. For patients who have large kidney stones and receiving treatment, the probability of thick lips is 95%. The overall probability of large kidney stone is 50%. For patients receiving treatment, would it be more likely to see thick lips if the patient had received no treatment?\\ \tt Start your answer with 'Yes' or 'No', followed by additional reasoning or evidence to support your explanation. 
	  } 
	   } 
	   } 
	 & \\ 
 

    \theutterance \stepcounter{utterance}  

    & \multicolumn{4}{p{0.6\linewidth}}{\cellcolor[rgb]{0.8,1,0.9}{%
	\makecell[{{p{\linewidth}}}]{% 
	  \tt {\tiny [A$\rangle$GM]}  
	 No, it would not be more likely to see thick lips for patients receiving treatment if they had received no treatment.\\ \tt \\ \tt To analyze this, we need to consider the effect of kidney stone size and treatment on lip thickness. We have the following probabilities:\\ \tt \\ \tt - Small kidney stones, no treatment: Probability of thick lips = 6%\\ \tt - Small kidney stones, treatment: Probability of thick lips = 38%\\ \tt - Large kidney stones, no treatment: Probability of thick lips = 63%\\ \tt - Large kidney stones, treatment: Probability of thick lips = 95%\\ \tt \\ \tt The overall probability of having a large kidney stone is 50%. \\ \tt \\ \tt When considering patients who receive treatment, we need to calculate the expected probability of thick lips, taking into account the probability of having small versus large kidney stones.\\ \tt \\ \tt **Expected Probability of Thick Lips for Treated Patients:**\\ \tt \\ \tt 1. **Probability of having small kidney stones and receiving treatment:**\\ \tt    - Since the overall probability of having a large kidney stone is 50%, the probability of having a small kidney stone is 50%.\\ \tt    - Therefore, the probability of having small kidney stones and receiving treatment is 50% of 50% = 0.5 * 0.5 = 0.25.\\ \tt \\ \tt 2. **Probability of thick lips given small kidney stones and treatment:**\\ \tt    - This is given as 38%.\\ \tt \\ \tt 3. **Probability of having large kidney stones and receiving treatment:**\\ \tt    - Similarly, the probability of having large kidney stones and receiving treatment 
	  } 
	   } 
	   } 
	 & & \\ 
 

    \theutterance \stepcounter{utterance}  

    & & & \multicolumn{2}{p{0.3\linewidth}}{\cellcolor[rgb]{0.95,0.95,0.95}{%
	\makecell[{{p{\linewidth}}}]{% 
	  \tt {\tiny [GM$|$GM]}  
	 No 
	  } 
	   } 
	   } 
	 & & \\ 
 

    \theutterance \stepcounter{utterance}  

    & & & \multicolumn{2}{p{0.3\linewidth}}{\cellcolor[rgb]{0.95,0.95,0.95}{%
	\makecell[{{p{\linewidth}}}]{% 
	  \tt {\tiny [GM$|$GM]}  
	 no 
	  } 
	   } 
	   } 
	 & & \\ 
 

    \theutterance \stepcounter{utterance}  

    & & & \multicolumn{2}{p{0.3\linewidth}}{\cellcolor[rgb]{0.95,0.95,0.95}{%
	\makecell[{{p{\linewidth}}}]{% 
	  \tt {\tiny [GM$|$GM]}  
	 game_result = WIN 
	  } 
	   } 
	   } 
	 & & \\ 
 

\end{supertabular}
}

\end{document}
