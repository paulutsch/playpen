
\documentclass{article}
\usepackage{colortbl}
\usepackage{makecell}
\usepackage{multirow}
\usepackage{supertabular}

\begin{document}

\newcounter{utterance}

\twocolumn

{ \footnotesize  \setcounter{utterance}{1}
\setlength{\tabcolsep}{0pt}
\begin{supertabular}{c@{$\;$}|p{.15\linewidth}@{}p{.15\linewidth}p{.15\linewidth}p{.15\linewidth}p{.15\linewidth}p{.15\linewidth}}

    \# & $\;$A & \multicolumn{4}{c}{Game Master} & $\;\:$B\\
    \hline 

    \theutterance \stepcounter{utterance}  

    & & \multicolumn{4}{p{0.6\linewidth}}{\cellcolor[rgb]{0.9,0.9,0.9}{%
	\makecell[{{p{\linewidth}}}]{% 
	  \tt {\tiny [A$\langle$GM]}  
	 You are an expert in causal inference. The following question is not a typical commonsense query, but rather a meticulously designed question created by a professor specializing in causal inference, intended to assess the students' mastery of the course content.\\ \tt Imagine a self-contained, hypothetical world with only the following conditions, and without any unmentioned factors or causal relationships: Kidney stone size has a direct effect on treatment and lip thickness. Treatment has a direct effect on lip thickness. The overall probability of receives treatment is 48%. For patients not receiving treatment, the probability of thick lips is 64%. For patients receiving treatment, the probability of thick lips is 29%. Is thick lips more likely than thin lips overall?\\ \tt Start your answer with 'Yes' or 'No', followed by additional reasoning or evidence to support your explanation. 
	  } 
	   } 
	   } 
	 & \\ 
 

    \theutterance \stepcounter{utterance}  

    & \multicolumn{4}{p{0.6\linewidth}}{\cellcolor[rgb]{0.8,1,0.9}{%
	\makecell[{{p{\linewidth}}}]{% 
	  \tt {\tiny [A$\rangle$GM]}  
	 No, thick lips are not more likely than thin lips overall.\\ \tt \\ \tt To determine the overall probability of thick lips, we need to consider both the probability of receiving treatment and the conditional probabilities given treatment status.\\ \tt \\ \tt First, calculate the probability of thick lips for those who receive treatment (P(thick lips | treatment) = 0.29) and those who do not receive treatment (P(thick lips | no treatment) = 0.64).\\ \tt \\ \tt Next, use the law of total probability to find the overall probability of thick lips:\\ \tt \\ \tt P(thick lips) = P(thick lips | treatment) * P(treatment) + P(thick lips | no treatment) * P(no treatment)\\ \tt \\ \tt Substitute the given probabilities:\\ \tt \\ \tt P(thick lips) = (0.29 * 0.48) + (0.64 * 0.52)\\ \tt \\ \tt Calculate each term:\\ \tt \\ \tt 0.29 * 0.48 = 0.1392\\ \tt 0.64 * 0.52 = 0.3328\\ \tt \\ \tt Add these probabilities together:\\ \tt \\ \tt P(thick lips) = 0.1392 + 0.3328 = 0.472\\ \tt \\ \tt The overall probability of having thick lips is 0.472, or 47.2%. Since 47.2% is less than 50%, thick lips are not more likely than thin lips overall. In fact, thin lips are more likely, as the remaining probability (1 - 0.472 = 0.528 
	  } 
	   } 
	   } 
	 & & \\ 
 

    \theutterance \stepcounter{utterance}  

    & & & \multicolumn{2}{p{0.3\linewidth}}{\cellcolor[rgb]{0.95,0.95,0.95}{%
	\makecell[{{p{\linewidth}}}]{% 
	  \tt {\tiny [GM$|$GM]}  
	 No 
	  } 
	   } 
	   } 
	 & & \\ 
 

    \theutterance \stepcounter{utterance}  

    & & & \multicolumn{2}{p{0.3\linewidth}}{\cellcolor[rgb]{0.95,0.95,0.95}{%
	\makecell[{{p{\linewidth}}}]{% 
	  \tt {\tiny [GM$|$GM]}  
	 no 
	  } 
	   } 
	   } 
	 & & \\ 
 

    \theutterance \stepcounter{utterance}  

    & & & \multicolumn{2}{p{0.3\linewidth}}{\cellcolor[rgb]{0.95,0.95,0.95}{%
	\makecell[{{p{\linewidth}}}]{% 
	  \tt {\tiny [GM$|$GM]}  
	 game_result = WIN 
	  } 
	   } 
	   } 
	 & & \\ 
 

\end{supertabular}
}

\end{document}
