
\documentclass{article}
\usepackage{colortbl}
\usepackage{makecell}
\usepackage{multirow}
\usepackage{supertabular}

\begin{document}

\newcounter{utterance}

\twocolumn

{ \footnotesize  \setcounter{utterance}{1}
\setlength{\tabcolsep}{0pt}
\begin{supertabular}{c@{$\;$}|p{.15\linewidth}@{}p{.15\linewidth}p{.15\linewidth}p{.15\linewidth}p{.15\linewidth}p{.15\linewidth}}

    \# & $\;$A & \multicolumn{4}{c}{Game Master} & $\;\:$B\\
    \hline 

    \theutterance \stepcounter{utterance}  

    & & \multicolumn{4}{p{0.6\linewidth}}{\cellcolor[rgb]{0.9,0.9,0.9}{%
	\makecell[{{p{\linewidth}}}]{% 
	  \tt {\tiny [A$\langle$GM]}  
	 You are an expert in causal inference. The following question is not a typical commonsense query, but rather a meticulously designed question created by a professor specializing in causal inference, intended to assess the students' mastery of the course content.\\ \tt Imagine a self-contained, hypothetical world with only the following conditions, and without any unmentioned factors or causal relationships: Obesity has a direct effect on diabetes and lifespan. Smoking has a direct effect on diabetes and lifespan. Diabetes has a direct effect on lifespan. Smoking is unobserved. For normal weight people and without diabetes, the probability of long lifespan is 91%. For normal weight people and with diabetes, the probability of long lifespan is 71%. For obese people and without diabetes, the probability of long lifespan is 69%. For obese people and with diabetes, the probability of long lifespan is 43%. For normal weight people and nonsmokers, the probability of having diabetes is 28%. For normal weight people and smokers, the probability of having diabetes is 20%. For obese people and nonsmokers, the probability of having diabetes is 57%. For obese people and smokers, the probability of having diabetes is 70%. The overall probability of smoker is 8%. If we disregard the mediation effect through diabetes, would obesity negatively affect lifespan?\\ \tt Start your answer with 'Yes' or 'No', followed by additional reasoning or evidence to support your explanation. 
	  } 
	   } 
	   } 
	 & \\ 
 

    \theutterance \stepcounter{utterance}  

    & \multicolumn{4}{p{0.6\linewidth}}{\cellcolor[rgb]{0.8,1,0.9}{%
	\makecell[{{p{\linewidth}}}]{% 
	  \tt {\tiny [A$\rangle$GM]}  
	 Yes, obesity negatively affects lifespan when disregarding the mediation effect through diabetes. This conclusion is based on the provided probabilities and the direct effects specified in the hypothetical scenario.\\ \tt \\ \tt Firstly, we observe that for normal weight individuals, the probability of a long lifespan is higher than for obese individuals, regardless of whether they have diabetes or not. Specifically:\\ \tt - Normal weight without diabetes: 91% chance of long lifespan.\\ \tt - Obese without diabetes: 69% chance of long lifespan.\\ \tt \\ \tt This difference (91% - 69% = 22%) suggests that obesity, on its own, reduces the probability of a long lifespan by 22 percentage points, even before considering the effect of diabetes.\\ \tt \\ \tt Secondly, the presence of diabetes further reduces the probability of a long lifespan, but the effect of obesity remains significant:\\ \tt - Normal weight with diabetes: 71% chance of long lifespan.\\ \tt - Obese with diabetes: 43% chance of long lifespan.\\ \tt \\ \tt The difference between obese individuals with diabetes (43%) and normal weight individuals with diabetes (71%) is 28 percentage points. However, this difference does not isolate the effect of obesity from diabetes. To isolate obesity's effect, we compare the obese without diabetes (69%) to the normal weight without diabetes (91%), which still shows a significant negative impact of obesity on lifespan (22 percentage points).\\ \tt \\ \tt Given these comparisons, even when disregarding the mediation effect through diabetes, obesity has a direct negative effect on lifespan. The probability of a long lifespan is lower 
	  } 
	   } 
	   } 
	 & & \\ 
 

    \theutterance \stepcounter{utterance}  

    & & & \multicolumn{2}{p{0.3\linewidth}}{\cellcolor[rgb]{0.95,0.95,0.95}{%
	\makecell[{{p{\linewidth}}}]{% 
	  \tt {\tiny [GM$|$GM]}  
	 Yes 
	  } 
	   } 
	   } 
	 & & \\ 
 

    \theutterance \stepcounter{utterance}  

    & & & \multicolumn{2}{p{0.3\linewidth}}{\cellcolor[rgb]{0.95,0.95,0.95}{%
	\makecell[{{p{\linewidth}}}]{% 
	  \tt {\tiny [GM$|$GM]}  
	 yes 
	  } 
	   } 
	   } 
	 & & \\ 
 

    \theutterance \stepcounter{utterance}  

    & & & \multicolumn{2}{p{0.3\linewidth}}{\cellcolor[rgb]{0.95,0.95,0.95}{%
	\makecell[{{p{\linewidth}}}]{% 
	  \tt {\tiny [GM$|$GM]}  
	 game_result = WIN 
	  } 
	   } 
	   } 
	 & & \\ 
 

\end{supertabular}
}

\end{document}
