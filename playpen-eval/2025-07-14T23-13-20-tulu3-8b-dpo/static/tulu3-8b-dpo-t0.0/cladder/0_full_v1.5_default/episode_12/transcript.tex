
\documentclass{article}
\usepackage{colortbl}
\usepackage{makecell}
\usepackage{multirow}
\usepackage{supertabular}

\begin{document}

\newcounter{utterance}

\twocolumn

{ \footnotesize  \setcounter{utterance}{1}
\setlength{\tabcolsep}{0pt}
\begin{supertabular}{c@{$\;$}|p{.15\linewidth}@{}p{.15\linewidth}p{.15\linewidth}p{.15\linewidth}p{.15\linewidth}p{.15\linewidth}}

    \# & $\;$A & \multicolumn{4}{c}{Game Master} & $\;\:$B\\
    \hline 

    \theutterance \stepcounter{utterance}  

    & & \multicolumn{4}{p{0.6\linewidth}}{\cellcolor[rgb]{0.9,0.9,0.9}{%
	\makecell[{{p{\linewidth}}}]{% 
	  \tt {\tiny [A$\langle$GM]}  
	 You are an expert in causal inference. The following question is not a typical commonsense query, but rather a meticulously designed question created by a professor specializing in causal inference, intended to assess the students' mastery of the course content.\\ \tt Imagine a self-contained, hypothetical world with only the following conditions, and without any unmentioned factors or causal relationships: Eating citrus has a direct effect on vitmain C. Vitmain C has a direct effect on scurvy. Method 1: We look at how eating citrus correlates with scurvy case by case according to vitmain C. Method 2: We look directly at how eating citrus correlates with scurvy in general. To understand how eating citrus affects scurvy, is it more correct to use the Method 1 than Method 2?\\ \tt Start your answer with 'Yes' or 'No', followed by additional reasoning or evidence to support your explanation. 
	  } 
	   } 
	   } 
	 & \\ 
 

    \theutterance \stepcounter{utterance}  

    & \multicolumn{4}{p{0.6\linewidth}}{\cellcolor[rgb]{0.8,1,0.9}{%
	\makecell[{{p{\linewidth}}}]{% 
	  \tt {\tiny [A$\rangle$GM]}  
	 No, it is not more correct to use Method 1 than Method 2 to understand how eating citrus affects scurvy.\\ \tt \\ \tt The reason is that Method 1 attempts to understand the causal relationship by looking at the mediating variable (vitamin C) between eating citrus and scurvy. This approach is known as mediation analysis and is a powerful tool in causal inference. However, it requires strong assumptions, such as no unmeasured confounding and the absence of other pathways between citrus consumption and scurvy that do not go through vitamin C.\\ \tt \\ \tt Method 2, on the other hand, directly assesses the correlation between eating citrus and scurvy, which is a more straightforward approach. In this method, we assume that any observed correlation is due to the causal effect of citrus on scurvy, mediated by vitamin C. This approach is simpler and does not require the same level of assumptions as Method 1.\\ \tt \\ \tt In the context of this hypothetical world with only the specified conditions, Method 2 is more appropriate because:\\ \tt \\ \tt 1. There are no unmentioned factors or confounders that could affect the relationship between citrus consumption and scurvy.\\ \tt 2. Vitamin C is the known mediator between citrus consumption and the prevention of scurvy.\\ \tt 3. The direct correlation between citrus consumption and scurvy can be assumed to be causal, given the direct effect of vitamin C on scurvy and the known mechanism by which citrus fruits provide vitamin C.\\ \tt \\ \tt Therefore, in this simplified scenario, 
	  } 
	   } 
	   } 
	 & & \\ 
 

    \theutterance \stepcounter{utterance}  

    & & & \multicolumn{2}{p{0.3\linewidth}}{\cellcolor[rgb]{0.95,0.95,0.95}{%
	\makecell[{{p{\linewidth}}}]{% 
	  \tt {\tiny [GM$|$GM]}  
	 No 
	  } 
	   } 
	   } 
	 & & \\ 
 

    \theutterance \stepcounter{utterance}  

    & & & \multicolumn{2}{p{0.3\linewidth}}{\cellcolor[rgb]{0.95,0.95,0.95}{%
	\makecell[{{p{\linewidth}}}]{% 
	  \tt {\tiny [GM$|$GM]}  
	 no 
	  } 
	   } 
	   } 
	 & & \\ 
 

    \theutterance \stepcounter{utterance}  

    & & & \multicolumn{2}{p{0.3\linewidth}}{\cellcolor[rgb]{0.95,0.95,0.95}{%
	\makecell[{{p{\linewidth}}}]{% 
	  \tt {\tiny [GM$|$GM]}  
	 game_result = WIN 
	  } 
	   } 
	   } 
	 & & \\ 
 

\end{supertabular}
}

\end{document}
