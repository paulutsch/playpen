
\documentclass{article}
\usepackage{colortbl}
\usepackage{makecell}
\usepackage{multirow}
\usepackage{supertabular}

\begin{document}

\newcounter{utterance}

\twocolumn

{ \footnotesize  \setcounter{utterance}{1}
\setlength{\tabcolsep}{0pt}
\begin{supertabular}{c@{$\;$}|p{.15\linewidth}@{}p{.15\linewidth}p{.15\linewidth}p{.15\linewidth}p{.15\linewidth}p{.15\linewidth}}

    \# & $\;$A & \multicolumn{4}{c}{Game Master} & $\;\:$B\\
    \hline 

    \theutterance \stepcounter{utterance}  

    & & \multicolumn{4}{p{0.6\linewidth}}{\cellcolor[rgb]{0.9,0.9,0.9}{%
	\makecell[{{p{\linewidth}}}]{% 
	  \tt {\tiny [A$\langle$GM]}  
	 You are an expert in causal inference. The following question is not a typical commonsense query, but rather a meticulously designed question created by a professor specializing in causal inference, intended to assess the students' mastery of the course content.\\ \tt Imagine a self-contained, hypothetical world with only the following conditions, and without any unmentioned factors or causal relationships: Personality has a direct effect on relationship status. Appearance has a direct effect on relationship status. The overall probability of kindness is 85%. For people with mean personalities and single people, the probability of attractive appearance is 38%. For people with mean personalities and in a relationship, the probability of attractive appearance is 71%. For people with kind personalities and single people, the probability of attractive appearance is 32%. For people with kind personalities and in a relationship, the probability of attractive appearance is 57%. If we look at people in a relationship, does the chance of attractive appearance decrease when kindness?\\ \tt Start your answer with 'Yes' or 'No', followed by additional reasoning or evidence to support your explanation. 
	  } 
	   } 
	   } 
	 & \\ 
 

    \theutterance \stepcounter{utterance}  

    & \multicolumn{4}{p{0.6\linewidth}}{\cellcolor[rgb]{0.8,1,0.9}{%
	\makecell[{{p{\linewidth}}}]{% 
	  \tt {\tiny [A$\rangle$GM]}  
	 No, the chance of attractive appearance does not necessarily decrease when kindness is considered among people in a relationship.\\ \tt \\ \tt To analyze this, we need to consider the given probabilities and how they relate to the conditions of being in a relationship and having a kind personality.\\ \tt \\ \tt First, let's summarize the given probabilities:\\ \tt \\ \tt - For people with mean personalities and single, the probability of attractive appearance is 38%.\\ \tt - For people with mean personalities and in a relationship, the probability of attractive appearance is 71%.\\ \tt - For people with kind personalities and single, the probability of attractive appearance is 32%.\\ \tt - For people with kind personalities and in a relationship, the probability of attractive appearance is 57%.\\ \tt \\ \tt When we focus on people who are in a relationship, we compare the probabilities of attractive appearance between those with mean personalities and those with kind personalities:\\ \tt \\ \tt - Mean personality in a relationship: 71% attractive appearance.\\ \tt - Kind personality in a relationship: 57% attractive appearance.\\ \tt \\ \tt At first glance, it might seem that the chance of attractive appearance decreases when kindness is considered among people in a relationship, because 57% (kind personality) is less than 71% (mean personality).\\ \tt \\ \tt However, this conclusion is not definitive for several reasons:\\ \tt \\ \tt 1. **Confounding Variables**: The given data does not control for other potential confounding variables that could affect both relationship status and attractiveness. For instance, it's possible that people with kind personalities are more likely to be in relationships with less emphasis on physical appearance, or that the definition 
	  } 
	   } 
	   } 
	 & & \\ 
 

    \theutterance \stepcounter{utterance}  

    & & & \multicolumn{2}{p{0.3\linewidth}}{\cellcolor[rgb]{0.95,0.95,0.95}{%
	\makecell[{{p{\linewidth}}}]{% 
	  \tt {\tiny [GM$|$GM]}  
	 No 
	  } 
	   } 
	   } 
	 & & \\ 
 

    \theutterance \stepcounter{utterance}  

    & & & \multicolumn{2}{p{0.3\linewidth}}{\cellcolor[rgb]{0.95,0.95,0.95}{%
	\makecell[{{p{\linewidth}}}]{% 
	  \tt {\tiny [GM$|$GM]}  
	 yes 
	  } 
	   } 
	   } 
	 & & \\ 
 

    \theutterance \stepcounter{utterance}  

    & & & \multicolumn{2}{p{0.3\linewidth}}{\cellcolor[rgb]{0.95,0.95,0.95}{%
	\makecell[{{p{\linewidth}}}]{% 
	  \tt {\tiny [GM$|$GM]}  
	 game_result = LOSE 
	  } 
	   } 
	   } 
	 & & \\ 
 

\end{supertabular}
}

\end{document}
