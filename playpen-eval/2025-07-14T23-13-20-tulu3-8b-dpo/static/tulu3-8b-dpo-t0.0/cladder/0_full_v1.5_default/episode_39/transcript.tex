
\documentclass{article}
\usepackage{colortbl}
\usepackage{makecell}
\usepackage{multirow}
\usepackage{supertabular}

\begin{document}

\newcounter{utterance}

\twocolumn

{ \footnotesize  \setcounter{utterance}{1}
\setlength{\tabcolsep}{0pt}
\begin{supertabular}{c@{$\;$}|p{.15\linewidth}@{}p{.15\linewidth}p{.15\linewidth}p{.15\linewidth}p{.15\linewidth}p{.15\linewidth}}

    \# & $\;$A & \multicolumn{4}{c}{Game Master} & $\;\:$B\\
    \hline 

    \theutterance \stepcounter{utterance}  

    & & \multicolumn{4}{p{0.6\linewidth}}{\cellcolor[rgb]{0.9,0.9,0.9}{%
	\makecell[{{p{\linewidth}}}]{% 
	  \tt {\tiny [A$\langle$GM]}  
	 You are an expert in causal inference. The following question is not a typical commonsense query, but rather a meticulously designed question created by a professor specializing in causal inference, intended to assess the students' mastery of the course content.\\ \tt Imagine a self-contained, hypothetical world with only the following conditions, and without any unmentioned factors or causal relationships: Tanning salon treatment has a direct effect on foot size. Going to the beach has a direct effect on foot size. The overall probability of tanning salon treatment is 10%. For people not using tanning salon treatments, the probability of large feet is 13%. For people who went to tanning salons, the probability of large feet is 70%. Is large feet more likely than small feet overall?\\ \tt Start your answer with 'Yes' or 'No', followed by additional reasoning or evidence to support your explanation. 
	  } 
	   } 
	   } 
	 & \\ 
 

    \theutterance \stepcounter{utterance}  

    & \multicolumn{4}{p{0.6\linewidth}}{\cellcolor[rgb]{0.8,1,0.9}{%
	\makecell[{{p{\linewidth}}}]{% 
	  \tt {\tiny [A$\rangle$GM]}  
	 No, large feet are not more likely than small feet overall.\\ \tt \\ \tt To determine the overall probability of having large feet, we need to consider both groups: those who use tanning salon treatments and those who do not.\\ \tt \\ \tt 1. **Probability of Large Feet for Non-users of Tanning Salons:**\\ \tt    - Probability of not using a tanning salon = 1 - 0.10 = 0.90\\ \tt    - Probability of large feet given not using a tanning salon = 0.13\\ \tt \\ \tt    Therefore, the probability of having large feet for non-users is:\\ \tt    \[\\ \tt    P(\text{Large Feet} \mid \text{Not using tanning salon}) = 0.90 \times 0.13 = 0.117\\ \tt    \]\\ \tt \\ \tt 2. **Probability of Large Feet for Users of Tanning Salons:**\\ \tt    - Probability of using a tanning salon = 0.10\\ \tt    - Probability of large feet given using a tanning salon = 0.70\\ \tt \\ \tt    Therefore, the probability of having large feet for users is:\\ \tt    \[\\ \tt    P(\text{Large Feet} \mid \text{Using tanning salon}) = 0.10 \times 0.70 = 0.07\\ \tt    \]\\ \tt \\ \tt 3. **Overall Probability of Large Feet:**\\ \tt    - The overall probability of large feet is a weighted average of the probabilities for the two groups:\\ \tt    \[\\ \tt    P(\text{ 
	  } 
	   } 
	   } 
	 & & \\ 
 

    \theutterance \stepcounter{utterance}  

    & & & \multicolumn{2}{p{0.3\linewidth}}{\cellcolor[rgb]{0.95,0.95,0.95}{%
	\makecell[{{p{\linewidth}}}]{% 
	  \tt {\tiny [GM$|$GM]}  
	 No 
	  } 
	   } 
	   } 
	 & & \\ 
 

    \theutterance \stepcounter{utterance}  

    & & & \multicolumn{2}{p{0.3\linewidth}}{\cellcolor[rgb]{0.95,0.95,0.95}{%
	\makecell[{{p{\linewidth}}}]{% 
	  \tt {\tiny [GM$|$GM]}  
	 no 
	  } 
	   } 
	   } 
	 & & \\ 
 

    \theutterance \stepcounter{utterance}  

    & & & \multicolumn{2}{p{0.3\linewidth}}{\cellcolor[rgb]{0.95,0.95,0.95}{%
	\makecell[{{p{\linewidth}}}]{% 
	  \tt {\tiny [GM$|$GM]}  
	 game_result = WIN 
	  } 
	   } 
	   } 
	 & & \\ 
 

\end{supertabular}
}

\end{document}
