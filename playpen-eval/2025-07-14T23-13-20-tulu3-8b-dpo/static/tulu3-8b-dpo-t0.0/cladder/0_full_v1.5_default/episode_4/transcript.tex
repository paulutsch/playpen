
\documentclass{article}
\usepackage{colortbl}
\usepackage{makecell}
\usepackage{multirow}
\usepackage{supertabular}

\begin{document}

\newcounter{utterance}

\twocolumn

{ \footnotesize  \setcounter{utterance}{1}
\setlength{\tabcolsep}{0pt}
\begin{supertabular}{c@{$\;$}|p{.15\linewidth}@{}p{.15\linewidth}p{.15\linewidth}p{.15\linewidth}p{.15\linewidth}p{.15\linewidth}}

    \# & $\;$A & \multicolumn{4}{c}{Game Master} & $\;\:$B\\
    \hline 

    \theutterance \stepcounter{utterance}  

    & & \multicolumn{4}{p{0.6\linewidth}}{\cellcolor[rgb]{0.9,0.9,0.9}{%
	\makecell[{{p{\linewidth}}}]{% 
	  \tt {\tiny [A$\langle$GM]}  
	 You are an expert in causal inference. The following question is not a typical commonsense query, but rather a meticulously designed question created by a professor specializing in causal inference, intended to assess the students' mastery of the course content.\\ \tt Imagine a self-contained, hypothetical world with only the following conditions, and without any unmentioned factors or causal relationships: Vaccination status has a direct effect on vaccination reaction and getting smallpox. Getting smallpox has a direct effect on smallpox survival. Vaccination reaction has a direct effect on smallpox survival. For unvaccinated individuals, the probability of smallpox survival is 39%. For vaccinated individuals, the probability of smallpox survival is 46%. For vaccinated individuals, would it be more likely to see smallpox survival if the person had been vaccinated?\\ \tt Start your answer with 'Yes' or 'No', followed by additional reasoning or evidence to support your explanation. 
	  } 
	   } 
	   } 
	 & \\ 
 

    \theutterance \stepcounter{utterance}  

    & \multicolumn{4}{p{0.6\linewidth}}{\cellcolor[rgb]{0.8,1,0.9}{%
	\makecell[{{p{\linewidth}}}]{% 
	  \tt {\tiny [A$\rangle$GM]}  
	 Yes, for vaccinated individuals, it would be more likely to see smallpox survival compared to unvaccinated individuals. This conclusion is based on the given data:\\ \tt \\ \tt 1. **Direct Effect of Vaccination Status on Smallpox Survival**: The problem states that for unvaccinated individuals, the probability of smallpox survival is 39%, while for vaccinated individuals, it is 46%. This directly implies that being vaccinated increases the likelihood of survival from smallpox.\\ \tt \\ \tt 2. **No Confounding Variables**: The problem specifies a self-contained, hypothetical world with only the mentioned causal relationships. This means there are no other factors (confounders) that could influence the relationship between vaccination status and smallpox survival.\\ \tt \\ \tt 3. **Direct Effect of Vaccination Reaction on Smallpox Survival**: The problem also mentions that vaccination reaction has a direct effect on smallpox survival. However, without additional information on how this reaction influences survival (positive or negative), we cannot quantify its impact. However, assuming the vaccination reaction is generally beneficial (as is often the case in real-world scenarios), it would further support the increased survival rate among vaccinated individuals.\\ \tt \\ \tt 4. **Statistical Evidence**: The provided probabilities (39% for unvaccinated and 46% for vaccinated) are direct statistical evidence that being vaccinated increases the odds of surviving smallpox.\\ \tt \\ \tt Therefore, based on the given data and assuming no other unmentioned factors, it is more likely to see smallpox survival among 
	  } 
	   } 
	   } 
	 & & \\ 
 

    \theutterance \stepcounter{utterance}  

    & & & \multicolumn{2}{p{0.3\linewidth}}{\cellcolor[rgb]{0.95,0.95,0.95}{%
	\makecell[{{p{\linewidth}}}]{% 
	  \tt {\tiny [GM$|$GM]}  
	 Yes 
	  } 
	   } 
	   } 
	 & & \\ 
 

    \theutterance \stepcounter{utterance}  

    & & & \multicolumn{2}{p{0.3\linewidth}}{\cellcolor[rgb]{0.95,0.95,0.95}{%
	\makecell[{{p{\linewidth}}}]{% 
	  \tt {\tiny [GM$|$GM]}  
	 no 
	  } 
	   } 
	   } 
	 & & \\ 
 

    \theutterance \stepcounter{utterance}  

    & & & \multicolumn{2}{p{0.3\linewidth}}{\cellcolor[rgb]{0.95,0.95,0.95}{%
	\makecell[{{p{\linewidth}}}]{% 
	  \tt {\tiny [GM$|$GM]}  
	 game_result = LOSE 
	  } 
	   } 
	   } 
	 & & \\ 
 

\end{supertabular}
}

\end{document}
