
\documentclass{article}
\usepackage{colortbl}
\usepackage{makecell}
\usepackage{multirow}
\usepackage{supertabular}

\begin{document}

\newcounter{utterance}

\twocolumn

{ \footnotesize  \setcounter{utterance}{1}
\setlength{\tabcolsep}{0pt}
\begin{supertabular}{c@{$\;$}|p{.15\linewidth}@{}p{.15\linewidth}p{.15\linewidth}p{.15\linewidth}p{.15\linewidth}p{.15\linewidth}}

    \# & $\;$A & \multicolumn{4}{c}{Game Master} & $\;\:$B\\
    \hline 

    \theutterance \stepcounter{utterance}  

    & & \multicolumn{4}{p{0.6\linewidth}}{\cellcolor[rgb]{0.9,0.9,0.9}{%
	\makecell[{{p{\linewidth}}}]{% 
	  \tt {\tiny [A$\langle$GM]}  
	 You are an expert in causal inference. The following question is not a typical commonsense query, but rather a meticulously designed question created by a professor specializing in causal inference, intended to assess the students' mastery of the course content.\\ \tt Imagine a self-contained, hypothetical world with only the following conditions, and without any unmentioned factors or causal relationships: Ability to speak english has a direct effect on diabetes and lifespan. Smoking has a direct effect on diabetes and lifespan. Diabetes has a direct effect on lifespan. Smoking is unobserved. For people who do not speak english, the probability of long lifespan is 77%. For people who speak english, the probability of long lifespan is 45%. Will speaking english decrease the chance of long lifespan?\\ \tt Start your answer with 'Yes' or 'No', followed by additional reasoning or evidence to support your explanation. 
	  } 
	   } 
	   } 
	 & \\ 
 

    \theutterance \stepcounter{utterance}  

    & \multicolumn{4}{p{0.6\linewidth}}{\cellcolor[rgb]{0.8,1,0.9}{%
	\makecell[{{p{\linewidth}}}]{% 
	  \tt {\tiny [A$\rangle$GM]}  
	 No, speaking English does not decrease the chance of a long lifespan based solely on the information provided.\\ \tt \\ \tt The given conditions state that:\\ \tt \\ \tt 1. Speaking English has a direct effect on diabetes.\\ \tt 2. Smoking has a direct effect on diabetes and lifespan.\\ \tt 3. Diabetes has a direct effect on lifespan.\\ \tt 4. Smoking is unobserved.\\ \tt 5. The probability of a long lifespan is 77% for those who do not speak English and 45% for those who do speak English.\\ \tt \\ \tt To determine whether speaking English decreases the chance of a long lifespan, we need to consider the direct and indirect effects of speaking English on lifespan.\\ \tt \\ \tt **Direct Effect of Speaking English:**\\ \tt The problem states that speaking English has a direct effect on diabetes. However, it does not specify the nature of this effect. It could be that speaking English is associated with certain behaviors or lifestyles that affect diabetes risk. For instance, speaking English might correlate with higher education levels, which could lead to better health knowledge and behaviors, or it might correlate with living in an environment where certain unhealthy behaviors (like smoking) are more prevalent.\\ \tt \\ \tt **Indirect Effect through Diabetes:**\\ \tt Diabetes has a direct effect on lifespan. If speaking English increases the risk of diabetes (which is not explicitly stated but could be inferred from the direct effect on diabetes), then this would indirectly decrease the chance of a long lifespan.\\ \tt \\ \tt **Effect of Smoking:**\\ \tt Smoking is unobserved, but it has a direct effect on both diabetes and lifespan. If speaking 
	  } 
	   } 
	   } 
	 & & \\ 
 

    \theutterance \stepcounter{utterance}  

    & & & \multicolumn{2}{p{0.3\linewidth}}{\cellcolor[rgb]{0.95,0.95,0.95}{%
	\makecell[{{p{\linewidth}}}]{% 
	  \tt {\tiny [GM$|$GM]}  
	 No 
	  } 
	   } 
	   } 
	 & & \\ 
 

    \theutterance \stepcounter{utterance}  

    & & & \multicolumn{2}{p{0.3\linewidth}}{\cellcolor[rgb]{0.95,0.95,0.95}{%
	\makecell[{{p{\linewidth}}}]{% 
	  \tt {\tiny [GM$|$GM]}  
	 yes 
	  } 
	   } 
	   } 
	 & & \\ 
 

    \theutterance \stepcounter{utterance}  

    & & & \multicolumn{2}{p{0.3\linewidth}}{\cellcolor[rgb]{0.95,0.95,0.95}{%
	\makecell[{{p{\linewidth}}}]{% 
	  \tt {\tiny [GM$|$GM]}  
	 game_result = LOSE 
	  } 
	   } 
	   } 
	 & & \\ 
 

\end{supertabular}
}

\end{document}
