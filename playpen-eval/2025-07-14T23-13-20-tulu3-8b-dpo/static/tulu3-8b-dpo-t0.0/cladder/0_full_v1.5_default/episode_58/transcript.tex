
\documentclass{article}
\usepackage{colortbl}
\usepackage{makecell}
\usepackage{multirow}
\usepackage{supertabular}

\begin{document}

\newcounter{utterance}

\twocolumn

{ \footnotesize  \setcounter{utterance}{1}
\setlength{\tabcolsep}{0pt}
\begin{supertabular}{c@{$\;$}|p{.15\linewidth}@{}p{.15\linewidth}p{.15\linewidth}p{.15\linewidth}p{.15\linewidth}p{.15\linewidth}}

    \# & $\;$A & \multicolumn{4}{c}{Game Master} & $\;\:$B\\
    \hline 

    \theutterance \stepcounter{utterance}  

    & & \multicolumn{4}{p{0.6\linewidth}}{\cellcolor[rgb]{0.9,0.9,0.9}{%
	\makecell[{{p{\linewidth}}}]{% 
	  \tt {\tiny [A$\langle$GM]}  
	 You are an expert in causal inference. The following question is not a typical commonsense query, but rather a meticulously designed question created by a professor specializing in causal inference, intended to assess the students' mastery of the course content.\\ \tt Imagine a self-contained, hypothetical world with only the following conditions, and without any unmentioned factors or causal relationships: Vaccination status has a direct effect on vaccination reaction and getting smallpox. Getting smallpox has a direct effect on black hair. Vaccination reaction has a direct effect on black hair. For unvaccinated individuals, the probability of black hair is 27%. For vaccinated individuals, the probability of black hair is 37%. Will vaccination decrease the chance of black hair?\\ \tt Start your answer with 'Yes' or 'No', followed by additional reasoning or evidence to support your explanation. 
	  } 
	   } 
	   } 
	 & \\ 
 

    \theutterance \stepcounter{utterance}  

    & \multicolumn{4}{p{0.6\linewidth}}{\cellcolor[rgb]{0.8,1,0.9}{%
	\makecell[{{p{\linewidth}}}]{% 
	  \tt {\tiny [A$\rangle$GM]}  
	 No, vaccination will not decrease the chance of black hair based on the given information. \\ \tt \\ \tt The problem states that there are direct effects as follows:\\ \tt \\ \tt 1. Vaccination status affects vaccination reaction.\\ \tt 2. Getting smallpox affects black hair.\\ \tt 3. Vaccination reaction affects black hair.\\ \tt \\ \tt Given these direct effects, we can infer the following:\\ \tt \\ \tt - Unvaccinated individuals have a 27% chance of having black hair.\\ \tt - Vaccinated individuals have a 37% chance of having black hair.\\ \tt \\ \tt The key to understanding this question is recognizing that the effect of vaccination on black hair is mediated through its effect on vaccination reaction and its effect on the risk of getting smallpox. \\ \tt \\ \tt Here's the reasoning:\\ \tt \\ \tt - **Vaccination Reaction**: The problem does not specify how vaccination reaction affects black hair, but it does imply that there is a direct effect. This means that the way an individual reacts to the vaccination could influence the likelihood of black hair. However, without further information on the nature of this relationship, we cannot determine whether the reaction is positive or negative towards black hair.\\ \tt \\ \tt - **Getting Smallpox**: The probability of getting smallpox is likely lower for vaccinated individuals compared to unvaccinated ones, given the purpose of vaccination is to prevent disease. However, the problem does not specify the exact probabilities of getting smallpox for vaccinated and unvaccinated individuals, nor does it specify how getting smallpox directly affects black hair.\\ \tt \\ \tt Given these points, we cannot conclude 
	  } 
	   } 
	   } 
	 & & \\ 
 

    \theutterance \stepcounter{utterance}  

    & & & \multicolumn{2}{p{0.3\linewidth}}{\cellcolor[rgb]{0.95,0.95,0.95}{%
	\makecell[{{p{\linewidth}}}]{% 
	  \tt {\tiny [GM$|$GM]}  
	 No 
	  } 
	   } 
	   } 
	 & & \\ 
 

    \theutterance \stepcounter{utterance}  

    & & & \multicolumn{2}{p{0.3\linewidth}}{\cellcolor[rgb]{0.95,0.95,0.95}{%
	\makecell[{{p{\linewidth}}}]{% 
	  \tt {\tiny [GM$|$GM]}  
	 no 
	  } 
	   } 
	   } 
	 & & \\ 
 

    \theutterance \stepcounter{utterance}  

    & & & \multicolumn{2}{p{0.3\linewidth}}{\cellcolor[rgb]{0.95,0.95,0.95}{%
	\makecell[{{p{\linewidth}}}]{% 
	  \tt {\tiny [GM$|$GM]}  
	 game_result = WIN 
	  } 
	   } 
	   } 
	 & & \\ 
 

\end{supertabular}
}

\end{document}
