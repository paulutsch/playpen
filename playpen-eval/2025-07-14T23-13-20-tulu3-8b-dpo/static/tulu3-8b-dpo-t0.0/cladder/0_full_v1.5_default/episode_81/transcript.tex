
\documentclass{article}
\usepackage{colortbl}
\usepackage{makecell}
\usepackage{multirow}
\usepackage{supertabular}

\begin{document}

\newcounter{utterance}

\twocolumn

{ \footnotesize  \setcounter{utterance}{1}
\setlength{\tabcolsep}{0pt}
\begin{supertabular}{c@{$\;$}|p{.15\linewidth}@{}p{.15\linewidth}p{.15\linewidth}p{.15\linewidth}p{.15\linewidth}p{.15\linewidth}}

    \# & $\;$A & \multicolumn{4}{c}{Game Master} & $\;\:$B\\
    \hline 

    \theutterance \stepcounter{utterance}  

    & & \multicolumn{4}{p{0.6\linewidth}}{\cellcolor[rgb]{0.9,0.9,0.9}{%
	\makecell[{{p{\linewidth}}}]{% 
	  \tt {\tiny [A$\langle$GM]}  
	 You are an expert in causal inference. The following question is not a typical commonsense query, but rather a meticulously designed question created by a professor specializing in causal inference, intended to assess the students' mastery of the course content.\\ \tt Imagine a self-contained, hypothetical world with only the following conditions, and without any unmentioned factors or causal relationships: Rixq has a direct effect on zuph and xevu. Swoy has a direct effect on zuph and xevu. Zuph has a direct effect on xevu. Swoy is unobserved. The overall probability of rixq is 3%. The probability of not rixq and xevu is 78%. The probability of rixq and xevu is 2%. Is the chance of xevu larger when observing rixq?\\ \tt Start your answer with 'Yes' or 'No', followed by additional reasoning or evidence to support your explanation. 
	  } 
	   } 
	   } 
	 & \\ 
 

    \theutterance \stepcounter{utterance}  

    & \multicolumn{4}{p{0.6\linewidth}}{\cellcolor[rgb]{0.8,1,0.9}{%
	\makecell[{{p{\linewidth}}}]{% 
	  \tt {\tiny [A$\rangle$GM]}  
	 No, the chance of xevu is not larger when observing rixq.\\ \tt \\ \tt To determine whether the chance of xevu is larger when observing rixq, we need to calculate the conditional probability \( P(Xevu | Rixq) \) and compare it to \( P(Xevu) \).\\ \tt \\ \tt Given:\\ \tt - \( P(Rixq) = 0.03 \)\\ \tt - \( P(\neg Rixq \land Xevu) = 0.78 \)\\ \tt - \( P(Rixq \land Xevu) = 0.02 \)\\ \tt \\ \tt First, calculate \( P(Xevu) \):\\ \tt \\ \tt \[ P(Xevu) = P(Xevu \land Rixq) + P(Xevu \land \neg Rixq) \]\\ \tt \[ P(Xevu) = P(Rixq \land Xevu) + P(\neg Rixq \land Xevu) \]\\ \tt \[ P(Xevu) = 0.02 + 0.78 = 0.80 \]\\ \tt \\ \tt Next, calculate \( P(Xevu | Rixq) \):\\ \tt \\ \tt \[ P(Xevu | Rixq) = \frac{P(Rixq \land Xevu)}{P(Rixq)} \]\\ \tt \[ P(Xevu | Rixq) = \frac{0.02}{0.03} \]\\ \tt \[ P(Xevu | Rixq) 
	  } 
	   } 
	   } 
	 & & \\ 
 

    \theutterance \stepcounter{utterance}  

    & & & \multicolumn{2}{p{0.3\linewidth}}{\cellcolor[rgb]{0.95,0.95,0.95}{%
	\makecell[{{p{\linewidth}}}]{% 
	  \tt {\tiny [GM$|$GM]}  
	 No 
	  } 
	   } 
	   } 
	 & & \\ 
 

    \theutterance \stepcounter{utterance}  

    & & & \multicolumn{2}{p{0.3\linewidth}}{\cellcolor[rgb]{0.95,0.95,0.95}{%
	\makecell[{{p{\linewidth}}}]{% 
	  \tt {\tiny [GM$|$GM]}  
	 no 
	  } 
	   } 
	   } 
	 & & \\ 
 

    \theutterance \stepcounter{utterance}  

    & & & \multicolumn{2}{p{0.3\linewidth}}{\cellcolor[rgb]{0.95,0.95,0.95}{%
	\makecell[{{p{\linewidth}}}]{% 
	  \tt {\tiny [GM$|$GM]}  
	 game_result = WIN 
	  } 
	   } 
	   } 
	 & & \\ 
 

\end{supertabular}
}

\end{document}
