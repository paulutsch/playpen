
\documentclass{article}
\usepackage{colortbl}
\usepackage{makecell}
\usepackage{multirow}
\usepackage{supertabular}

\begin{document}

\newcounter{utterance}

\twocolumn

{ \footnotesize  \setcounter{utterance}{1}
\setlength{\tabcolsep}{0pt}
\begin{supertabular}{c@{$\;$}|p{.15\linewidth}@{}p{.15\linewidth}p{.15\linewidth}p{.15\linewidth}p{.15\linewidth}p{.15\linewidth}}

    \# & $\;$A & \multicolumn{4}{c}{Game Master} & $\;\:$B\\
    \hline 

    \theutterance \stepcounter{utterance}  

    & & \multicolumn{4}{p{0.6\linewidth}}{\cellcolor[rgb]{0.9,0.9,0.9}{%
	\makecell[{{p{\linewidth}}}]{% 
	  \tt {\tiny [A$\langle$GM]}  
	 You are an expert in causal inference. The following question is not a typical commonsense query, but rather a meticulously designed question created by a professor specializing in causal inference, intended to assess the students' mastery of the course content.\\ \tt Imagine a self-contained, hypothetical world with only the following conditions, and without any unmentioned factors or causal relationships: Having a brother has a direct effect on wife and alarm clock. Wife has a direct effect on alarm clock. For people who do not have a brother and wives that don't set the alarm, the probability of ringing alarm is 88%. For people who do not have a brother and wives that set the alarm, the probability of ringing alarm is 45%. For people who have a brother and wives that don't set the alarm, the probability of ringing alarm is 58%. For people who have a brother and wives that set the alarm, the probability of ringing alarm is 13%. For people who do not have a brother, the probability of alarm set by wife is 9%. For people who have a brother, the probability of alarm set by wife is 64%. Does having a brother negatively affect alarm clock through wife?\\ \tt Start your answer with 'Yes' or 'No', followed by additional reasoning or evidence to support your explanation. 
	  } 
	   } 
	   } 
	 & \\ 
 

    \theutterance \stepcounter{utterance}  

    & \multicolumn{4}{p{0.6\linewidth}}{\cellcolor[rgb]{0.8,1,0.9}{%
	\makecell[{{p{\linewidth}}}]{% 
	  \tt {\tiny [A$\rangle$GM]}  
	 No, having a brother does not negatively affect the alarm clock through the wife based on the given data.\\ \tt \\ \tt To analyze this, we need to consider the effect of having a brother on the probability that a wife sets the alarm, and then how this affects the probability of the alarm ringing.\\ \tt \\ \tt First, let's summarize the probabilities:\\ \tt \\ \tt - Probability that a wife sets the alarm given no brother: 9% (P(S|NB))\\ \tt - Probability that the alarm rings given no brother and wife sets alarm: 45% (P(R|NB, S))\\ \tt - Probability that the alarm rings given no brother and wife doesn't set alarm: 88% (P(R|NB, ¬S))\\ \tt \\ \tt - Probability that a wife sets the alarm given brother: 64% (P(S|B))\\ \tt - Probability that the alarm rings given brother and wife sets alarm: 13% (P(R|B, S))\\ \tt - Probability that the alarm rings given brother and wife doesn't set alarm: 58% (P(R|B, ¬S))\\ \tt \\ \tt To determine if having a brother negatively affects the alarm clock through the wife, we need to look at the combined probabilities:\\ \tt \\ \tt 1. **Without a brother (NB):**\\ \tt    - Probability that the alarm rings: \\ \tt      - If the wife sets the alarm: 0.45 (45%)\\ \tt      - If the wife doesn't set the alarm: 0.88 (88%)\\ \tt      - Combined probability: P(R|NB) = P(R 
	  } 
	   } 
	   } 
	 & & \\ 
 

    \theutterance \stepcounter{utterance}  

    & & & \multicolumn{2}{p{0.3\linewidth}}{\cellcolor[rgb]{0.95,0.95,0.95}{%
	\makecell[{{p{\linewidth}}}]{% 
	  \tt {\tiny [GM$|$GM]}  
	 No 
	  } 
	   } 
	   } 
	 & & \\ 
 

    \theutterance \stepcounter{utterance}  

    & & & \multicolumn{2}{p{0.3\linewidth}}{\cellcolor[rgb]{0.95,0.95,0.95}{%
	\makecell[{{p{\linewidth}}}]{% 
	  \tt {\tiny [GM$|$GM]}  
	 yes 
	  } 
	   } 
	   } 
	 & & \\ 
 

    \theutterance \stepcounter{utterance}  

    & & & \multicolumn{2}{p{0.3\linewidth}}{\cellcolor[rgb]{0.95,0.95,0.95}{%
	\makecell[{{p{\linewidth}}}]{% 
	  \tt {\tiny [GM$|$GM]}  
	 game_result = LOSE 
	  } 
	   } 
	   } 
	 & & \\ 
 

\end{supertabular}
}

\end{document}
