
\documentclass{article}
\usepackage{colortbl}
\usepackage{makecell}
\usepackage{multirow}
\usepackage{supertabular}

\begin{document}

\newcounter{utterance}

\twocolumn

{ \footnotesize  \setcounter{utterance}{1}
\setlength{\tabcolsep}{0pt}
\begin{supertabular}{c@{$\;$}|p{.15\linewidth}@{}p{.15\linewidth}p{.15\linewidth}p{.15\linewidth}p{.15\linewidth}p{.15\linewidth}}

    \# & $\;$A & \multicolumn{4}{c}{Game Master} & $\;\:$B\\
    \hline 

    \theutterance \stepcounter{utterance}  

    & & \multicolumn{4}{p{0.6\linewidth}}{\cellcolor[rgb]{0.9,0.9,0.9}{%
	\makecell[{{p{\linewidth}}}]{% 
	  \tt {\tiny [A$\langle$GM]}  
	 You are an expert in causal inference. The following question is not a typical commonsense query, but rather a meticulously designed question created by a professor specializing in causal inference, intended to assess the students' mastery of the course content.\\ \tt Imagine a self-contained, hypothetical world with only the following conditions, and without any unmentioned factors or causal relationships: Appearance has a direct effect on fame. Lip thickness has a direct effect on fame. Method 1: We look at how appearance correlates with lip thickness case by case according to fame. Method 2: We look directly at how appearance correlates with lip thickness in general. To understand how appearance affects lip thickness, is it more correct to use the Method 1 than Method 2?\\ \tt Start your answer with 'Yes' or 'No', followed by additional reasoning or evidence to support your explanation. 
	  } 
	   } 
	   } 
	 & \\ 
 

    \theutterance \stepcounter{utterance}  

    & \multicolumn{4}{p{0.6\linewidth}}{\cellcolor[rgb]{0.8,1,0.9}{%
	\makecell[{{p{\linewidth}}}]{% 
	  \tt {\tiny [A$\rangle$GM]}  
	 No, it is not more correct to use Method 1 than Method 2 to understand how appearance affects lip thickness.\\ \tt \\ \tt The reason is that Method 1 involves a case-by-case analysis, which introduces a high level of noise and potential confounding variables. When we look at how appearance correlates with lip thickness according to fame, we are mixing two variables (appearance and lip thickness) with another variable (fame), which could introduce spurious correlations or obscure the true relationship between appearance and lip thickness.\\ \tt \\ \tt On the other hand, Method 2 directly examines the correlation between appearance and lip thickness in general, without introducing the additional variable of fame. This approach is more straightforward and less likely to introduce bias or confounding factors. It allows us to observe the direct relationship between the two characteristics of interest, assuming that the hypothetical world contains no other unmentioned factors that could affect either appearance or lip thickness.\\ \tt \\ \tt In causal inference, it is generally better to isolate the variables of interest and examine their relationship directly, rather than through a third variable, unless there is a clear reason to believe that the third variable is a causal intermediary or a confounder that needs to be controlled for.\\ \tt \\ \tt Therefore, Method 2 is the more appropriate method for understanding the direct causal relationship between appearance and lip thickness, assuming that fame does not mediate or confound the relationship between these two variables.<|end_of_text|> 
	  } 
	   } 
	   } 
	 & & \\ 
 

    \theutterance \stepcounter{utterance}  

    & & & \multicolumn{2}{p{0.3\linewidth}}{\cellcolor[rgb]{0.95,0.95,0.95}{%
	\makecell[{{p{\linewidth}}}]{% 
	  \tt {\tiny [GM$|$GM]}  
	 No 
	  } 
	   } 
	   } 
	 & & \\ 
 

    \theutterance \stepcounter{utterance}  

    & & & \multicolumn{2}{p{0.3\linewidth}}{\cellcolor[rgb]{0.95,0.95,0.95}{%
	\makecell[{{p{\linewidth}}}]{% 
	  \tt {\tiny [GM$|$GM]}  
	 no 
	  } 
	   } 
	   } 
	 & & \\ 
 

    \theutterance \stepcounter{utterance}  

    & & & \multicolumn{2}{p{0.3\linewidth}}{\cellcolor[rgb]{0.95,0.95,0.95}{%
	\makecell[{{p{\linewidth}}}]{% 
	  \tt {\tiny [GM$|$GM]}  
	 game_result = WIN 
	  } 
	   } 
	   } 
	 & & \\ 
 

\end{supertabular}
}

\end{document}
