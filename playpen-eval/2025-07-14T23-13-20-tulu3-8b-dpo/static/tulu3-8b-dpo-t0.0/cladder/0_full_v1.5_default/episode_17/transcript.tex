
\documentclass{article}
\usepackage{colortbl}
\usepackage{makecell}
\usepackage{multirow}
\usepackage{supertabular}

\begin{document}

\newcounter{utterance}

\twocolumn

{ \footnotesize  \setcounter{utterance}{1}
\setlength{\tabcolsep}{0pt}
\begin{supertabular}{c@{$\;$}|p{.15\linewidth}@{}p{.15\linewidth}p{.15\linewidth}p{.15\linewidth}p{.15\linewidth}p{.15\linewidth}}

    \# & $\;$A & \multicolumn{4}{c}{Game Master} & $\;\:$B\\
    \hline 

    \theutterance \stepcounter{utterance}  

    & & \multicolumn{4}{p{0.6\linewidth}}{\cellcolor[rgb]{0.9,0.9,0.9}{%
	\makecell[{{p{\linewidth}}}]{% 
	  \tt {\tiny [A$\langle$GM]}  
	 You are an expert in causal inference. The following question is not a typical commonsense query, but rather a meticulously designed question created by a professor specializing in causal inference, intended to assess the students' mastery of the course content.\\ \tt Imagine a self-contained, hypothetical world with only the following conditions, and without any unmentioned factors or causal relationships: Husband has a direct effect on wife and alarm clock. Wife has a direct effect on alarm clock. The overall probability of alarm set by husband is 8%. For husbands that don't set the alarm, the probability of ringing alarm is 75%. For husbands that set the alarm, the probability of ringing alarm is 16%. Is ringing alarm more likely than silent alarm overall?\\ \tt Start your answer with 'Yes' or 'No', followed by additional reasoning or evidence to support your explanation. 
	  } 
	   } 
	   } 
	 & \\ 
 

    \theutterance \stepcounter{utterance}  

    & \multicolumn{4}{p{0.6\linewidth}}{\cellcolor[rgb]{0.8,1,0.9}{%
	\makecell[{{p{\linewidth}}}]{% 
	  \tt {\tiny [A$\rangle$GM]}  
	 No, ringing alarm is not more likely than silent alarm overall.\\ \tt \\ \tt To solve this problem, we need to calculate the overall probability of a ringing alarm and compare it to the probability of a silent alarm.\\ \tt \\ \tt First, let's define the probabilities given in the problem:\\ \tt \\ \tt - Probability that the husband sets the alarm (P(Husband sets alarm)) = 8% = 0.08\\ \tt - Probability that the alarm rings given the husband sets it (P(Ringing | Husband sets)) = 16% = 0.16\\ \tt - Probability that the alarm rings given the husband does not set it (P(Ringing | Husband does not set)) = 75% = 0.75\\ \tt \\ \tt We can calculate the overall probability of the alarm ringing (P(Ringing)) by considering both scenarios: when the husband sets the alarm and when he does not.\\ \tt \\ \tt **Step 1: Calculate P(Ringing | Husband sets) * P(Husband sets)**\\ \tt \[ P(\text{Ringing} | \text{Husband sets}) \times P(\text{Husband sets}) = 0.16 \times 0.08 = 0.0128 \]\\ \tt \\ \tt **Step 2: Calculate P(Ringing | Husband does not set) * P(Husband does not set)**\\ \tt Since the probability that the husband does not set the alarm is 1 minus the probability that he sets it:\\ \tt \[ P(\text{Husband does not set}) = 1 - 0. 
	  } 
	   } 
	   } 
	 & & \\ 
 

    \theutterance \stepcounter{utterance}  

    & & & \multicolumn{2}{p{0.3\linewidth}}{\cellcolor[rgb]{0.95,0.95,0.95}{%
	\makecell[{{p{\linewidth}}}]{% 
	  \tt {\tiny [GM$|$GM]}  
	 No 
	  } 
	   } 
	   } 
	 & & \\ 
 

    \theutterance \stepcounter{utterance}  

    & & & \multicolumn{2}{p{0.3\linewidth}}{\cellcolor[rgb]{0.95,0.95,0.95}{%
	\makecell[{{p{\linewidth}}}]{% 
	  \tt {\tiny [GM$|$GM]}  
	 yes 
	  } 
	   } 
	   } 
	 & & \\ 
 

    \theutterance \stepcounter{utterance}  

    & & & \multicolumn{2}{p{0.3\linewidth}}{\cellcolor[rgb]{0.95,0.95,0.95}{%
	\makecell[{{p{\linewidth}}}]{% 
	  \tt {\tiny [GM$|$GM]}  
	 game_result = LOSE 
	  } 
	   } 
	   } 
	 & & \\ 
 

\end{supertabular}
}

\end{document}
