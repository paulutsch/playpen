
\documentclass{article}
\usepackage{colortbl}
\usepackage{makecell}
\usepackage{multirow}
\usepackage{supertabular}

\begin{document}

\newcounter{utterance}

\twocolumn

{ \footnotesize  \setcounter{utterance}{1}
\setlength{\tabcolsep}{0pt}
\begin{supertabular}{c@{$\;$}|p{.15\linewidth}@{}p{.15\linewidth}p{.15\linewidth}p{.15\linewidth}p{.15\linewidth}p{.15\linewidth}}

    \# & $\;$A & \multicolumn{4}{c}{Game Master} & $\;\:$B\\
    \hline 

    \theutterance \stepcounter{utterance}  

    & & \multicolumn{4}{p{0.6\linewidth}}{\cellcolor[rgb]{0.9,0.9,0.9}{%
	\makecell[{{p{\linewidth}}}]{% 
	  \tt {\tiny [A$\langle$GM]}  
	 You are an expert in causal inference. The following question is not a typical commonsense query, but rather a meticulously designed question created by a professor specializing in causal inference, intended to assess the students' mastery of the course content.\\ \tt Imagine a self-contained, hypothetical world with only the following conditions, and without any unmentioned factors or causal relationships: Xevo has a direct effect on gyzp. Tijv has a direct effect on gyzp. We know that xevo or tijv causes gyzp. We observed an individual is not tijv. Would an individual is gyzp if not xevo instead of xevo?\\ \tt Start your answer with 'Yes' or 'No', followed by additional reasoning or evidence to support your explanation. 
	  } 
	   } 
	   } 
	 & \\ 
 

    \theutterance \stepcounter{utterance}  

    & \multicolumn{4}{p{0.6\linewidth}}{\cellcolor[rgb]{0.8,1,0.9}{%
	\makecell[{{p{\linewidth}}}]{% 
	  \tt {\tiny [A$\rangle$GM]}  
	 No, an individual is not necessarily gyzp if not xevo instead of xevo, even if we know the individual is not tijv. \\ \tt \\ \tt The reason is that we are given that both Xevo and Tijv have direct effects on Gyzp, and either can cause Gyzp. The fact that an individual is not Tijv does not provide information about whether Xevo is the cause of Gyzp in that individual. \\ \tt \\ \tt To determine if an individual would be Gyzp if not Xevo, we would need additional information about the nature of the causal relationships between Xevo, Tijv, and Gyzp. Specifically, we would need to know:\\ \tt \\ \tt 1. The relative strength of the causal effects of Xevo and Tijv on Gyzp. If Xevo has a stronger effect than Tijv, then the absence of Xevo might be more likely to result in the absence of Gyzp, even if Tijv is not present.\\ \tt \\ \tt 2. The possibility of interaction effects between Xevo and Tijv on Gyzp. If the presence of both Xevo and Tijv is required to cause Gyzp, then the absence of Tijv does not imply that Gyzp would be absent if Xevo were also absent.\\ \tt \\ \tt 3. The possibility of other confounding variables or unobserved factors that could affect Gyzp. Without controlling for these 
	  } 
	   } 
	   } 
	 & & \\ 
 

    \theutterance \stepcounter{utterance}  

    & & & \multicolumn{2}{p{0.3\linewidth}}{\cellcolor[rgb]{0.95,0.95,0.95}{%
	\makecell[{{p{\linewidth}}}]{% 
	  \tt {\tiny [GM$|$GM]}  
	 No 
	  } 
	   } 
	   } 
	 & & \\ 
 

    \theutterance \stepcounter{utterance}  

    & & & \multicolumn{2}{p{0.3\linewidth}}{\cellcolor[rgb]{0.95,0.95,0.95}{%
	\makecell[{{p{\linewidth}}}]{% 
	  \tt {\tiny [GM$|$GM]}  
	 no 
	  } 
	   } 
	   } 
	 & & \\ 
 

    \theutterance \stepcounter{utterance}  

    & & & \multicolumn{2}{p{0.3\linewidth}}{\cellcolor[rgb]{0.95,0.95,0.95}{%
	\makecell[{{p{\linewidth}}}]{% 
	  \tt {\tiny [GM$|$GM]}  
	 game_result = WIN 
	  } 
	   } 
	   } 
	 & & \\ 
 

\end{supertabular}
}

\end{document}
