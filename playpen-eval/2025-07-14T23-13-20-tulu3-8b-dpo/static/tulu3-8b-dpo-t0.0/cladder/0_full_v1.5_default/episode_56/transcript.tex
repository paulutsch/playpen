
\documentclass{article}
\usepackage{colortbl}
\usepackage{makecell}
\usepackage{multirow}
\usepackage{supertabular}

\begin{document}

\newcounter{utterance}

\twocolumn

{ \footnotesize  \setcounter{utterance}{1}
\setlength{\tabcolsep}{0pt}
\begin{supertabular}{c@{$\;$}|p{.15\linewidth}@{}p{.15\linewidth}p{.15\linewidth}p{.15\linewidth}p{.15\linewidth}p{.15\linewidth}}

    \# & $\;$A & \multicolumn{4}{c}{Game Master} & $\;\:$B\\
    \hline 

    \theutterance \stepcounter{utterance}  

    & & \multicolumn{4}{p{0.6\linewidth}}{\cellcolor[rgb]{0.9,0.9,0.9}{%
	\makecell[{{p{\linewidth}}}]{% 
	  \tt {\tiny [A$\langle$GM]}  
	 You are an expert in causal inference. The following question is not a typical commonsense query, but rather a meticulously designed question created by a professor specializing in causal inference, intended to assess the students' mastery of the course content.\\ \tt Imagine a self-contained, hypothetical world with only the following conditions, and without any unmentioned factors or causal relationships: Tanning salon treatment has a direct effect on black hair. Going to the beach has a direct effect on black hair. The overall probability of tanning salon treatment is 99%. For people not using tanning salon treatments, the probability of black hair is 66%. For people who went to tanning salons, the probability of black hair is 24%. Is black hair less likely than blond hair overall?\\ \tt Start your answer with 'Yes' or 'No', followed by additional reasoning or evidence to support your explanation. 
	  } 
	   } 
	   } 
	 & \\ 
 

    \theutterance \stepcounter{utterance}  

    & \multicolumn{4}{p{0.6\linewidth}}{\cellcolor[rgb]{0.8,1,0.9}{%
	\makecell[{{p{\linewidth}}}]{% 
	  \tt {\tiny [A$\rangle$GM]}  
	 No, black hair is not less likely than blond hair overall. \\ \tt \\ \tt To determine the overall probability of black hair, we need to consider both groups: those who use tanning salon treatments and those who do not. Given the conditions:\\ \tt \\ \tt 1. The probability of using a tanning salon treatment is 99%, so the probability of not using a tanning salon treatment is 1% (or 0.01).\\ \tt \\ \tt 2. For people not using tanning salon treatments, the probability of having black hair is 66% (or 0.66).\\ \tt \\ \tt 3. For people who use tanning salon treatments, the probability of having black hair is 24% (or 0.24).\\ \tt \\ \tt We calculate the overall probability of black hair (P(B)) as follows:\\ \tt \\ \tt P(B) = P(B|not salon) * P(not salon) + P(B|salon) * P(salon)\\ \tt \\ \tt Substituting the given probabilities:\\ \tt \\ \tt P(B) = (0.66 * 0.01) + (0.24 * 0.99)\\ \tt \\ \tt P(B) = 0.0066 + 0.2376\\ \tt \\ \tt P(B) = 0.2442\\ \tt \\ \tt The overall probability of black hair is 24.42%, which is higher than the probability of blond hair, assuming blond hair is the default hair color in this hypothetical world (since no specific probability for blond hair is given, we infer it to be the most common hair color, and thus its probability 
	  } 
	   } 
	   } 
	 & & \\ 
 

    \theutterance \stepcounter{utterance}  

    & & & \multicolumn{2}{p{0.3\linewidth}}{\cellcolor[rgb]{0.95,0.95,0.95}{%
	\makecell[{{p{\linewidth}}}]{% 
	  \tt {\tiny [GM$|$GM]}  
	 No 
	  } 
	   } 
	   } 
	 & & \\ 
 

    \theutterance \stepcounter{utterance}  

    & & & \multicolumn{2}{p{0.3\linewidth}}{\cellcolor[rgb]{0.95,0.95,0.95}{%
	\makecell[{{p{\linewidth}}}]{% 
	  \tt {\tiny [GM$|$GM]}  
	 yes 
	  } 
	   } 
	   } 
	 & & \\ 
 

    \theutterance \stepcounter{utterance}  

    & & & \multicolumn{2}{p{0.3\linewidth}}{\cellcolor[rgb]{0.95,0.95,0.95}{%
	\makecell[{{p{\linewidth}}}]{% 
	  \tt {\tiny [GM$|$GM]}  
	 game_result = LOSE 
	  } 
	   } 
	   } 
	 & & \\ 
 

\end{supertabular}
}

\end{document}
