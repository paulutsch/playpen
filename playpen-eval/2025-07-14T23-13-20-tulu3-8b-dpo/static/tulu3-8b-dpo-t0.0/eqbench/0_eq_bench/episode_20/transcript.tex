
\documentclass{article}
\usepackage{colortbl}
\usepackage{makecell}
\usepackage{multirow}
\usepackage{supertabular}

\begin{document}

\newcounter{utterance}

\twocolumn

{ \footnotesize  \setcounter{utterance}{1}
\setlength{\tabcolsep}{0pt}
\begin{supertabular}{c@{$\;$}|p{.15\linewidth}@{}p{.15\linewidth}p{.15\linewidth}p{.15\linewidth}p{.15\linewidth}p{.15\linewidth}}

    \# & $\;$A & \multicolumn{4}{c}{Game Master} & $\;\:$B\\
    \hline 

    \theutterance \stepcounter{utterance}  

    & & \multicolumn{4}{p{0.6\linewidth}}{\cellcolor[rgb]{0.9,0.9,0.9}{%
	\makecell[{{p{\linewidth}}}]{% 
	  \tt {\tiny [A$\langle$GM]}  
	 \\ \tt Your task is to predict the likely emotional responses of a character in this dialogue:\\ \tt \\ \tt Arthur: You remember that time we built that treehouse in the back garden, right?\\ \tt Sam: What's that got to do with anything?\\ \tt Arthur: I'm just saying, we were a team, you and me. We built something out of nothing. \\ \tt Sam: Yeah, and then you decided it would be fun to paint it neon green without asking me. \\ \tt Arthur: It was a creative decision, Sammy. \\ \tt Sam: It was a stupid decision, Arthur. Just like your speech at my wedding.\\ \tt Arthur: That's not fair, Sam. I messed up, I know. But I'm trying to apologize here.\\ \tt Sam: And I'm supposed to just accept it because you've decided to say sorry? \\ \tt Arthur: No, not because I've decided to. Because I mean it, Sam. I'm sorry.\\ \tt Sam: You're always sorry, Arthur. After every stupid, thoughtless thing you do. \\ \tt Arthur: I know, and I'm trying to change that. I'm trying to be better.\\ \tt Sam: Better? You think saying sorry makes you better? You ruined the most important day of my life, Arthur.\\ \tt Arthur: I know, and I can't take it back. But I can try to make it up to you, Sam.\\ \tt Sam: How, Arthur? By bringing up old memories? By reminding me of when we were kids?\\ \tt Arthur: Maybe, yeah. Maybe by reminding you that we're brothers, Sam. And brothers mess up sometimes.\\ \tt Sam: Not like this, Arthur. Not like this.\\ \tt Arthur: I know, Sam. And I'm sorry.\\ \tt Sam: Stop saying you're sorry. It doesn't change anything.\\ \tt [End dialogue]\\ \tt \\ \tt At the end of this dialogue, At this point, Arthur would feel...\\ \tt Rebellious\\ \tt Defeated\\ \tt Vengeful\\ \tt Remorseful\\ \tt \\ \tt Give each of these possible emotions a score from 0-10 for the relative intensity that they are likely to be feeling each.\\ \tt \\ \tt You must output in the following format, including headings (of course, you should give your own scores), with no additional commentary:\\ \tt \\ \tt Rebellious: <score>\\ \tt Defeated: <score>\\ \tt Vengeful: <score>\\ \tt Remorseful: <score>\\ \tt \\ \tt \\ \tt [End of answer]\\ \tt \\ \tt Remember: zero is a valid score, meaning they are likely not feeling that emotion. You must score at least one emotion > 0.\\ \tt \\ \tt Your answer:\\ \tt  
	  } 
	   } 
	   } 
	 & \\ 
 

    \theutterance \stepcounter{utterance}  

    & \multicolumn{4}{p{0.6\linewidth}}{\cellcolor[rgb]{0.8,1,0.9}{%
	\makecell[{{p{\linewidth}}}]{% 
	  \tt {\tiny [A$\rangle$GM]}  
	 Rebellious: 0\\ \tt Defeated: 8\\ \tt Vengeful: 2\\ \tt Remorseful: 9<|end_of_text|> 
	  } 
	   } 
	   } 
	 & & \\ 
 

    \theutterance \stepcounter{utterance}  

    & & & \multicolumn{2}{p{0.3\linewidth}}{\cellcolor[rgb]{0.95,0.95,0.95}{%
	\makecell[{{p{\linewidth}}}]{% 
	  \tt {\tiny [GM$|$GM]}  
	 Rebellious: 0\\ \tt Defeated: 8\\ \tt Vengeful: 2\\ \tt Remorseful: 9 
	  } 
	   } 
	   } 
	 & & \\ 
 

    \theutterance \stepcounter{utterance}  

    & & & \multicolumn{2}{p{0.3\linewidth}}{\cellcolor[rgb]{0.95,0.95,0.95}{%
	\makecell[{{p{\linewidth}}}]{% 
	  \tt {\tiny [GM$|$GM]}  
	 Rebellious: 0\\ \tt Defeated: 7\\ \tt Vengeful: 0\\ \tt Remorseful: 9 
	  } 
	   } 
	   } 
	 & & \\ 
 

    \theutterance \stepcounter{utterance}  

    & & & \multicolumn{2}{p{0.3\linewidth}}{\cellcolor[rgb]{0.95,0.95,0.95}{%
	\makecell[{{p{\linewidth}}}]{% 
	  \tt {\tiny [GM$|$GM]}  
	 game_result = LOSE 
	  } 
	   } 
	   } 
	 & & \\ 
 

\end{supertabular}
}

\end{document}
