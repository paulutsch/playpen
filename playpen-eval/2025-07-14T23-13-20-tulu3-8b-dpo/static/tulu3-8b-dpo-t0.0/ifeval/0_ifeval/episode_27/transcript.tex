
\documentclass{article}
\usepackage{colortbl}
\usepackage{makecell}
\usepackage{multirow}
\usepackage{supertabular}

\begin{document}

\newcounter{utterance}

\twocolumn

{ \footnotesize  \setcounter{utterance}{1}
\setlength{\tabcolsep}{0pt}
\begin{supertabular}{c@{$\;$}|p{.15\linewidth}@{}p{.15\linewidth}p{.15\linewidth}p{.15\linewidth}p{.15\linewidth}p{.15\linewidth}}

    \# & $\;$A & \multicolumn{4}{c}{Game Master} & $\;\:$B\\
    \hline 

    \theutterance \stepcounter{utterance}  

    & & \multicolumn{4}{p{0.6\linewidth}}{\cellcolor[rgb]{0.9,0.9,0.9}{%
	\makecell[{{p{\linewidth}}}]{% 
	  \tt {\tiny [A$\langle$GM]}  
	 Write a casual blog post about how the outer solar system is different from the inner solar system, and what that means for the possibility of life. Wrap your entire response with double quotation marks. Your response should contain 17 or more sentences. 
	  } 
	   } 
	   } 
	 & \\ 
 

    \theutterance \stepcounter{utterance}  

    & \multicolumn{4}{p{0.6\linewidth}}{\cellcolor[rgb]{0.8,1,0.9}{%
	\makecell[{{p{\linewidth}}}]{% 
	  \tt {\tiny [A$\rangle$GM]}  
	 "Exploring the vast expanse of our solar system reveals striking differences between its inner and outer regions, each with implications for the potential harboring of life. The inner solar system, comprising the orbits of Mercury, Venus, Earth, and Mars, is characterized by its proximity to the Sun. This closeness means these planets experience higher surface temperatures and stronger solar winds, which can strip away atmospheres and make surface conditions harsh. In contrast, the outer solar system, home to Jupiter, Saturn, Uranus, and Neptune, is much colder and farther from the Sun's warming rays. Here, the temperatures are low enough to support the existence of ices and potentially liquid oceans beneath the crusts of moons like Europa and Enceladus.\\ \tt \\ \tt The gas giants Jupiter and Saturn dominate the outer solar system, with their massive gravitational fields influencing the orbits of smaller bodies. These planets have extensive atmospheres rich in hydrogen and helium, with hints of water, ammonia, and methane. Their moons, however, offer more promising environments for life. Europa, one of Jupiter's moons, is thought to harbor a subsurface ocean covered by a thick ice sheet. This ocean could potentially support life, given the right chemical conditions and energy sources. Similarly, Enceladus, a moon of Saturn, has a subsurface ocean discovered through its geysers, which spew ice and liquid water into space.\\ \tt \\ \tt The Kuiper Belt and the Oort Cloud, reservoirs of icy bodies beyond Neptune, further expand the mysteries of the outer solar system. These regions are thought to be the source of many comets that visit the inner solar system. The conditions in these distant realms are extremely cold, but they may have preserved organic molecules that could be the building blocks of life. The presence of these organic compounds, along with water ice, suggests that the outer solar system could be a crucible for prebiotic chemistry.\\ \tt \\ \tt The differences in temperature, atmospheric composition, and presence of liquid water significantly affect the potential for life. In the inner solar system, the harsh conditions make it challenging for life as we know it to thrive on the surfaces of planets. However, the outer solar system's colder, more stable environments, particularly the subsurface oceans of moons, offer more hospitable conditions. These hidden oceans could provide the necessary conditions for life, such as stable temperatures and access to chemical energy sources.\\ \tt \\ \tt The search for life in the outer solar system is a compelling area of astrobiology. Missions like NASA's Europa Clipper and the European Space Agency's JUICE (Jupiter Icy Moons Explorer) aim to explore these moons in greater detail. These missions could provide crucial insights into the potential habitability of these environments and the possibility of life beyond Earth. The outer solar system's unique conditions not only challenge our understanding of planetary science but also inspire us to consider the vast diversity of life that could exist in the universe.\\ \tt \\ \tt In conclusion, while the inner solar system presents challenges for life, the outer solar system offers intriguing possibilities. The presence of subsurface oceans on moons like Europa and Enceladus suggests that life could exist in environments very different from Earth. As we continue to explore these distant worlds, we expand our understanding of the universe and our place within it. The quest for life beyond Earth remains one of the most exciting frontiers in science, driven by the hope and curiosity to discover if we are alone in the cosmos."<|end_of_text|> 
	  } 
	   } 
	   } 
	 & & \\ 
 

    \theutterance \stepcounter{utterance}  

    & & & \multicolumn{2}{p{0.3\linewidth}}{\cellcolor[rgb]{0.95,0.95,0.95}{%
	\makecell[{{p{\linewidth}}}]{% 
	  \tt {\tiny [GM$|$GM]}  
	 {'startend:quotation': {}, 'length_constraints:number_sentences': {'relation': 'at least', 'num_sentences': 17}} 
	  } 
	   } 
	   } 
	 & & \\ 
 

    \theutterance \stepcounter{utterance}  

    & & & \multicolumn{2}{p{0.3\linewidth}}{\cellcolor[rgb]{0.95,0.95,0.95}{%
	\makecell[{{p{\linewidth}}}]{% 
	  \tt {\tiny [GM$|$GM]}  
	 game_result = LOSE 
	  } 
	   } 
	   } 
	 & & \\ 
 

\end{supertabular}
}

\end{document}
