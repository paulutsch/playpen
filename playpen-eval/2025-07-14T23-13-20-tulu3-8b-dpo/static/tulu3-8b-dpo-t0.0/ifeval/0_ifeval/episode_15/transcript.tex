
\documentclass{article}
\usepackage{colortbl}
\usepackage{makecell}
\usepackage{multirow}
\usepackage{supertabular}

\begin{document}

\newcounter{utterance}

\twocolumn

{ \footnotesize  \setcounter{utterance}{1}
\setlength{\tabcolsep}{0pt}
\begin{supertabular}{c@{$\;$}|p{.15\linewidth}@{}p{.15\linewidth}p{.15\linewidth}p{.15\linewidth}p{.15\linewidth}p{.15\linewidth}}

    \# & $\;$A & \multicolumn{4}{c}{Game Master} & $\;\:$B\\
    \hline 

    \theutterance \stepcounter{utterance}  

    & & \multicolumn{4}{p{0.6\linewidth}}{\cellcolor[rgb]{0.9,0.9,0.9}{%
	\makecell[{{p{\linewidth}}}]{% 
	  \tt {\tiny [A$\langle$GM]}  
	 Write an interesting and funny article about the biology of a banana peel. In your response, the word disappointed should appear at least 2 times, and at least six section should be highlighted with markdown,  i.e *banana peel*. 
	  } 
	   } 
	   } 
	 & \\ 
 

    \theutterance \stepcounter{utterance}  

    & \multicolumn{4}{p{0.6\linewidth}}{\cellcolor[rgb]{0.8,1,0.9}{%
	\makecell[{{p{\linewidth}}}]{% 
	  \tt {\tiny [A$\rangle$GM]}  
	 **The Unbelievable World of the *Banana Peel***\\ \tt \\ \tt Banana peels are not just the neglected leftovers of our fruity indulgence; they are the unsung heroes of the botanical world. Let's embark on a journey to explore the fascinating biology of this often *disappointed* piece of nature.\\ \tt \\ \tt **1. The *Banana Peel* as a Superfood**\\ \tt \\ \tt You might be *disappointed* to hear that banana peels aren't typically considered a culinary delight, but they are a treasure trove of nutrients. Packed with vitamin B6, potassium, and fiber, these peels can be a sustainable and eco-friendly addition to your diet. Next time you peel a banana, don't toss it aside; consider it a potential superfood waiting to be appreciated.\\ \tt \\ \tt **2. The *Banana Peel* and Its Slippery Reputation**\\ \tt \\ \tt Ah, the infamous banana peel slip. It's a classic comedy trope for a reason. The *banana peel* is nature's way of reminding us to watch our step. Its slick surface, when wet, transforms it into a slippery hazard. But don't be *disappointed*; this trait also makes it an excellent material for eco-friendly floor polish and even slip-resistant footwear soles.\\ \tt \\ \tt **3. *Banana Peel* as a Green Solution**\\ \tt \\ \tt Did you know that *banana peels* can be composted to enrich your garden soil? They're a natural source of nitrogen, potassium, and phosphorus, making them a gardener's best friend. So, instead of feeling *disappointed* about throwing them away, think of them as a free, organic way to nurture your plants.\\ \tt \\ \tt **4. The *Banana Peel* and Animal Friends**\\ \tt \\ \tt Many animals, from chickens to monkeys, find *banana peels* to be a delightful treat. Monkeys, in particular, have been observed peeling bananas with remarkable precision, leaving only the peel behind. This behavior might leave you *disappointed* if you were hoping for a cleaner environment, but it's a testament to the adaptability and intelligence of our fellow creatures.\\ \tt \\ \tt **5. *Banana Peel* Art and Innovation**\\ \tt \\ \tt Forget about paper and paint; some artists are using *banana peels* as their canvas. By applying dyes and inks to the peels, they create stunning works of art. This innovative use of the *banana peel* not only reduces waste but also challenges our perceptions of what can be considered art. It's a refreshing twist that might leave you *disappointed* if you thought art materials were limited to traditional mediums.\\ \tt \\ \tt **6. The *Banana Peel* in Fashion**\\ \tt \\ \tt Yes, you read that right. *Banana peels* are making their way into the fashion world. Designers are experimenting with natural dyes and fibers, incorporating *banana peel* elements into clothing and accessories. This eco-friendly approach to fashion might leave you *disappointed* if you're not a fan of unconventional styles, but it's a bold step towards sustainable fashion.\\ \tt \\ \tt In conclusion, the *banana peel* is more than just a discarded piece of fruit. It's a versatile, nutrient-rich, and surprisingly innovative material. So, the next time you're about to toss one aside, think twice. You might just be *disappointed* if you miss out on all the wonders it holds.<|end_of_text|> 
	  } 
	   } 
	   } 
	 & & \\ 
 

    \theutterance \stepcounter{utterance}  

    & & & \multicolumn{2}{p{0.3\linewidth}}{\cellcolor[rgb]{0.95,0.95,0.95}{%
	\makecell[{{p{\linewidth}}}]{% 
	  \tt {\tiny [GM$|$GM]}  
	 {'keywords:frequency': {'relation': 'at least', 'keyword': 'disappointed', 'frequency': 2}, 'detectable_format:number_highlighted_sections': {'num_highlights': 6}} 
	  } 
	   } 
	   } 
	 & & \\ 
 

    \theutterance \stepcounter{utterance}  

    & & & \multicolumn{2}{p{0.3\linewidth}}{\cellcolor[rgb]{0.95,0.95,0.95}{%
	\makecell[{{p{\linewidth}}}]{% 
	  \tt {\tiny [GM$|$GM]}  
	 game_result = WIN 
	  } 
	   } 
	   } 
	 & & \\ 
 

\end{supertabular}
}

\end{document}
