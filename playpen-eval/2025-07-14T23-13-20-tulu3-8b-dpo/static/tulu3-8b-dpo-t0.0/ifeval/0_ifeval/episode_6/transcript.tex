
\documentclass{article}
\usepackage{colortbl}
\usepackage{makecell}
\usepackage{multirow}
\usepackage{supertabular}

\begin{document}

\newcounter{utterance}

\twocolumn

{ \footnotesize  \setcounter{utterance}{1}
\setlength{\tabcolsep}{0pt}
\begin{supertabular}{c@{$\;$}|p{.15\linewidth}@{}p{.15\linewidth}p{.15\linewidth}p{.15\linewidth}p{.15\linewidth}p{.15\linewidth}}

    \# & $\;$A & \multicolumn{4}{c}{Game Master} & $\;\:$B\\
    \hline 

    \theutterance \stepcounter{utterance}  

    & & \multicolumn{4}{p{0.6\linewidth}}{\cellcolor[rgb]{0.9,0.9,0.9}{%
	\makecell[{{p{\linewidth}}}]{% 
	  \tt {\tiny [A$\langle$GM]}  
	 Can you give me an example for a journal entry about stress management? Tell me how you come up with the example. Your entire response should contain less than 6 sentences. 
	  } 
	   } 
	   } 
	 & \\ 
 

    \theutterance \stepcounter{utterance}  

    & \multicolumn{4}{p{0.6\linewidth}}{\cellcolor[rgb]{0.8,1,0.9}{%
	\makecell[{{p{\linewidth}}}]{% 
	  \tt {\tiny [A$\rangle$GM]}  
	 Certainly! Here's an example of a journal entry about stress management:\\ \tt \\ \tt **Journal Entry:**\\ \tt \\ \tt *Date: March 15, 2023*\\ \tt \\ \tt *Today, I focused on stress management by starting my day with 20 minutes of meditation. I then made a to-do list, prioritizing tasks to avoid feeling overwhelmed. I took a short walk during lunch, which helped clear my mind. I practiced deep breathing exercises whenever I felt tension rising. In the evening, I engaged in a hobby I enjoy, which is painting, to relax and unwind. Reflecting on these activities, I feel more in control and less stressed.*\\ \tt \\ \tt **How I Came Up With It:**\\ \tt \\ \tt I imagined a day where the writer actively seeks out stress-relief techniques throughout the day. Starting with a morning routine of meditation sets a calm tone. A to-do list helps organize tasks, reducing anxiety. A walk and breathing exercises are common, effective methods for stress relief. Engaging in a hobby in the evening provides a relaxing end to the day. The reflection at the end summarizes the effectiveness of these strategies.<|end_of_text|> 
	  } 
	   } 
	   } 
	 & & \\ 
 

    \theutterance \stepcounter{utterance}  

    & & & \multicolumn{2}{p{0.3\linewidth}}{\cellcolor[rgb]{0.95,0.95,0.95}{%
	\makecell[{{p{\linewidth}}}]{% 
	  \tt {\tiny [GM$|$GM]}  
	 {'length_constraints:number_sentences': {'relation': 'less than', 'num_sentences': 6}} 
	  } 
	   } 
	   } 
	 & & \\ 
 

    \theutterance \stepcounter{utterance}  

    & & & \multicolumn{2}{p{0.3\linewidth}}{\cellcolor[rgb]{0.95,0.95,0.95}{%
	\makecell[{{p{\linewidth}}}]{% 
	  \tt {\tiny [GM$|$GM]}  
	 game_result = LOSE 
	  } 
	   } 
	   } 
	 & & \\ 
 

\end{supertabular}
}

\end{document}
