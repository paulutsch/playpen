
\documentclass{article}
\usepackage{colortbl}
\usepackage{makecell}
\usepackage{multirow}
\usepackage{supertabular}

\begin{document}

\newcounter{utterance}

\twocolumn

{ \footnotesize  \setcounter{utterance}{1}
\setlength{\tabcolsep}{0pt}
\begin{supertabular}{c@{$\;$}|p{.15\linewidth}@{}p{.15\linewidth}p{.15\linewidth}p{.15\linewidth}p{.15\linewidth}p{.15\linewidth}}

    \# & $\;$A & \multicolumn{4}{c}{Game Master} & $\;\:$B\\
    \hline 

    \theutterance \stepcounter{utterance}  

    & & \multicolumn{4}{p{0.6\linewidth}}{\cellcolor[rgb]{0.9,0.9,0.9}{%
	\makecell[{{p{\linewidth}}}]{% 
	  \tt {\tiny [A$\langle$GM]}  
	 Write a story about the importance of understanding the truths that are not obvious. Add stress words which are capitalized. Limit those stress words for less than 20 times. 
	  } 
	   } 
	   } 
	 & \\ 
 

    \theutterance \stepcounter{utterance}  

    & \multicolumn{4}{p{0.6\linewidth}}{\cellcolor[rgb]{0.8,1,0.9}{%
	\makecell[{{p{\linewidth}}}]{% 
	  \tt {\tiny [A$\rangle$GM]}  
	 In the quaint village of Eldergrove, nestled between rolling hills and whispering forests, lived a young woman named Elara. She was known for her keen intellect and insatiable curiosity, traits that often set her apart from her peers. Yet, it was her ability to see beyond the obvious that truly made her unique.\\ \tt \\ \tt Elara's life took a dramatic turn one fateful autumn when the village was struck by a mysterious illness. People fell sick with a fever that burned hotter than the midday sun, and a fatigue that seemed to drain the very life from their bones. The village elders, wise and experienced, were at a loss. They called for healers, but their remedies did little to ease the suffering.\\ \tt \\ \tt Desperate, the villagers turned to superstition and fear, blaming the sickness on the moon's ill omen or the spirits' wrath. Elara, however, saw through the panic. She knew that the truth often lay hidden, veiled by the noise of fear and the clamor of unreasoned assumptions.\\ \tt \\ \tt Determined to uncover the root cause, Elara spent her days and nights in the village's small library, poring over ancient tomes and dusty scrolls. She discovered that the illness bore a striking resemblance to one described in a forgotten text, a sickness caused not by spirits or celestial bodies, but by a particular type of mold that thrived in damp, neglected conditions.\\ \tt \\ \tt Armed with this knowledge, Elara set out to investigate the village's neglected wells, long forgotten in the wake of a recent drought. To the villagers' horror, she found the wells contaminated, the mold thriving in the damp, dark environment. The realization hit them like a thunderclap: the source of their suffering was not supernatural, but entirely preventable.\\ \tt \\ \tt With Elara's guidance, the villagers worked tirelessly to clean and sanitize the wells. They also improved the drainage around their homes, ensuring the land would no longer provide a breeding ground for the insidious mold. As the weeks passed, the sickness began to wane, and hope returned to Eldergrove.\\ \tt \\ \tt Elara's story spread far and wide, a testament to the power of understanding and the importance of seeking truths that are not immediately obvious. Her actions saved countless lives, not just in Eldergrove, but in neighboring villages as well.\\ \tt \\ \tt In the end, Elara taught the villagers a valuable lesson: to look beyond the surface, to question assumptions, and to seek knowledge with an open mind. It was this understanding that allowed them to confront their greatest challenge and emerge victorious.\\ \tt \\ \tt Through her journey, Elara proved that the most profound truths often lie hidden, waiting to be discovered by those brave enough to seek them. And in doing so, she illuminated a path for others to follow, a path paved with the courage to look beyond the obvious and embrace the unknown.<|end_of_text|> 
	  } 
	   } 
	   } 
	 & & \\ 
 

    \theutterance \stepcounter{utterance}  

    & & & \multicolumn{2}{p{0.3\linewidth}}{\cellcolor[rgb]{0.95,0.95,0.95}{%
	\makecell[{{p{\linewidth}}}]{% 
	  \tt {\tiny [GM$|$GM]}  
	 {'change_case:capital_word_frequency': {'capital_relation': 'less than', 'capital_frequency': 20}} 
	  } 
	   } 
	   } 
	 & & \\ 
 

    \theutterance \stepcounter{utterance}  

    & & & \multicolumn{2}{p{0.3\linewidth}}{\cellcolor[rgb]{0.95,0.95,0.95}{%
	\makecell[{{p{\linewidth}}}]{% 
	  \tt {\tiny [GM$|$GM]}  
	 game_result = WIN 
	  } 
	   } 
	   } 
	 & & \\ 
 

\end{supertabular}
}

\end{document}
