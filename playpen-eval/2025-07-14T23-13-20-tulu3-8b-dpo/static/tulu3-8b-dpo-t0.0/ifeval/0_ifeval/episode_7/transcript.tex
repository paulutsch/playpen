
\documentclass{article}
\usepackage{colortbl}
\usepackage{makecell}
\usepackage{multirow}
\usepackage{supertabular}

\begin{document}

\newcounter{utterance}

\twocolumn

{ \footnotesize  \setcounter{utterance}{1}
\setlength{\tabcolsep}{0pt}
\begin{supertabular}{c@{$\;$}|p{.15\linewidth}@{}p{.15\linewidth}p{.15\linewidth}p{.15\linewidth}p{.15\linewidth}p{.15\linewidth}}

    \# & $\;$A & \multicolumn{4}{c}{Game Master} & $\;\:$B\\
    \hline 

    \theutterance \stepcounter{utterance}  

    & & \multicolumn{4}{p{0.6\linewidth}}{\cellcolor[rgb]{0.9,0.9,0.9}{%
	\makecell[{{p{\linewidth}}}]{% 
	  \tt {\tiny [A$\langle$GM]}  
	 Take the text below as a starting point, and make it a complete article: "You may have to meet with a helper to work out a parenting plan. The first would be to estimate how many time you have everyday for parenting, and is that enough...."\\ \tt Avoid using the following keywords: sleep, cook, feed\\ \tt Mention the keyword 'schedule' for more than 5 times. 
	  } 
	   } 
	   } 
	 & \\ 
 

    \theutterance \stepcounter{utterance}  

    & \multicolumn{4}{p{0.6\linewidth}}{\cellcolor[rgb]{0.8,1,0.9}{%
	\makecell[{{p{\linewidth}}}]{% 
	  \tt {\tiny [A$\rangle$GM]}  
	 Title: Crafting a Parenting Schedule: A Comprehensive Guide\\ \tt \\ \tt When embarking on the journey of parenthood, one of the most crucial steps is establishing a solid parenting schedule. This plan is the backbone of a harmonious family life, ensuring that both parents and children know what to expect and when. You may have to meet with a helper to work out a parenting schedule that suits your family's unique needs. Here's a comprehensive guide to help you through the process.\\ \tt \\ \tt **1. Assessing Your Daily Time for Parenting**\\ \tt \\ \tt The first step in creating a parenting schedule is to estimate how much time you have every day dedicated to parenting. This involves considering all the activities that contribute to raising your child, such as playtime, educational activities, and quality time. It's essential to be honest with yourself about whether the available time is enough to meet your child's needs.\\ \tt \\ \tt **2. Prioritizing Activities**\\ \tt \\ \tt Once you have a rough estimate of your daily parenting time, the next step is to prioritize activities. Not all activities are created equal, and some may be more important than others. A well-structured schedule will allocate more time to essential activities like play, learning, and bonding. Your schedule should reflect these priorities, ensuring that your child's developmental needs are met.\\ \tt \\ \tt **3. Creating a Flexible Schedule**\\ \tt \\ \tt A good parenting schedule is flexible and adaptable. Life with children is unpredictable, and there will be days when plans need to change. However, having a flexible schedule doesn't mean being unprepared. It means having a framework that can be adjusted as needed. For instance, if a spontaneous trip to the park becomes a possibility, your schedule should allow for this without disrupting the day's structure.\\ \tt \\ \tt **4. Involving Your Child in the Schedule**\\ \tt \\ \tt Involving your child in the schedule creation process can be incredibly beneficial. Children thrive on routine and understanding what to expect. By letting them know about the schedule, you're helping them feel more secure and in control. This can also be an opportunity to teach them about responsibility and time management.\\ \tt \\ \tt **5. Balancing Work and Parenting**\\ \tt \\ \tt For many parents, balancing work and parenting is a significant challenge. Your schedule must accommodate both your professional commitments and your parenting duties. This might mean setting specific times for work and parenting activities, ensuring that you're present and engaged during those parenting moments. A well-organized schedule can help you manage your time effectively, reducing stress and increasing productivity.\\ \tt \\ \tt **6. Regularly Reviewing and Adjusting the Schedule**\\ \tt \\ \tt A parenting schedule is not a set-it-and-forget-it document. It should be reviewed and adjusted regularly to accommodate changes in your child's needs, your family dynamics, or your personal circumstances. Quarterly reviews can be a good practice to ensure that the schedule remains relevant and effective.\\ \tt \\ \tt **7. Communicating the Schedule**\\ \tt \\ \tt Clear communication is key to a successful parenting schedule. Make sure both parents are on the same page and that your child understands the expectations. This can be achieved through shared calendars, daily check-ins, and open discussions about the schedule's effectiveness.\\ \tt \\ \tt **8. Embracing the Benefits of a Structured Schedule**\\ \tt \\ \tt A well-crafted parenting schedule offers numerous benefits, including reduced stress, improved family harmony, and better time management. It provides a sense of predictability and security for your child and helps you make the most of your parenting time.\\ \tt \\ \tt In conclusion, creating a parenting schedule is a vital step in ensuring a balanced and fulfilling family life. By estimating your daily time for parenting, prioritizing activities, and maintaining a flexible yet structured approach, you can create a schedule that works for your family. Regularly reviewing and adjusting the schedule, involving your child, and clear communication will ensure its success. Remember, a good parenting schedule is not a rigid rulebook but a living, breathing document that evolves with your family.<|end_of_text|> 
	  } 
	   } 
	   } 
	 & & \\ 
 

    \theutterance \stepcounter{utterance}  

    & & & \multicolumn{2}{p{0.3\linewidth}}{\cellcolor[rgb]{0.95,0.95,0.95}{%
	\makecell[{{p{\linewidth}}}]{% 
	  \tt {\tiny [GM$|$GM]}  
	 {'keywords:forbidden_words': {'forbidden_words': ['sleep', 'cook', 'feed']}, 'keywords:frequency': {'relation': 'at least', 'keyword': 'schedule', 'frequency': 6}} 
	  } 
	   } 
	   } 
	 & & \\ 
 

    \theutterance \stepcounter{utterance}  

    & & & \multicolumn{2}{p{0.3\linewidth}}{\cellcolor[rgb]{0.95,0.95,0.95}{%
	\makecell[{{p{\linewidth}}}]{% 
	  \tt {\tiny [GM$|$GM]}  
	 game_result = WIN 
	  } 
	   } 
	   } 
	 & & \\ 
 

\end{supertabular}
}

\end{document}
