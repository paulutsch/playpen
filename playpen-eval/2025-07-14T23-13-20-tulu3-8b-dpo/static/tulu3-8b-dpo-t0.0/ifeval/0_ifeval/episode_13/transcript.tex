
\documentclass{article}
\usepackage{colortbl}
\usepackage{makecell}
\usepackage{multirow}
\usepackage{supertabular}

\begin{document}

\newcounter{utterance}

\twocolumn

{ \footnotesize  \setcounter{utterance}{1}
\setlength{\tabcolsep}{0pt}
\begin{supertabular}{c@{$\;$}|p{.15\linewidth}@{}p{.15\linewidth}p{.15\linewidth}p{.15\linewidth}p{.15\linewidth}p{.15\linewidth}}

    \# & $\;$A & \multicolumn{4}{c}{Game Master} & $\;\:$B\\
    \hline 

    \theutterance \stepcounter{utterance}  

    & & \multicolumn{4}{p{0.6\linewidth}}{\cellcolor[rgb]{0.9,0.9,0.9}{%
	\makecell[{{p{\linewidth}}}]{% 
	  \tt {\tiny [A$\langle$GM]}  
	 What's a good way to ask Sonia out? Please reply with exactly 4 paragraphs and separate each paragraph with two new lines. Put double quotation marks around your entire response. The very first paragraph must start with the word "weekend". 
	  } 
	   } 
	   } 
	 & \\ 
 

    \theutterance \stepcounter{utterance}  

    & \multicolumn{4}{p{0.6\linewidth}}{\cellcolor[rgb]{0.8,1,0.9}{%
	\makecell[{{p{\linewidth}}}]{% 
	  \tt {\tiny [A$\rangle$GM]}  
	 "Weekend plans are always a great opportunity to connect with someone special, and I've been thinking about how to make the most of it with Sonia. She's someone I truly admire and enjoy spending time with, so I want to make sure my approach is thoughtful and considerate. I've been brainstorming ideas that could create a relaxed and enjoyable atmosphere where we can both feel comfortable and at ease. Perhaps a casual picnic in the park or a movie night at my place could set a perfect tone for getting to know each other better.\\ \tt \\ \tt Before I take the plunge, I want to ensure that my invitation feels personal and genuine. I've been reflecting on our shared interests and memories, trying to find a common ground that could spark a meaningful conversation. Maybe I could send her a text, something like, 'Hey Sonia, I was thinking this weekend could be perfect for a little adventure. How about we grab some takeout and find a cozy spot to watch the stars? Let me know if you're up for it!' This way, I'm suggesting an activity that's low-pressure but still exciting, giving her a clear and inviting option to say yes.\\ \tt \\ \tt Preparation is key, so I've been thinking about how to make the actual date memorable. I want to create an environment where we can both relax and enjoy each other's company without any distractions. I might prepare some of her favorite snacks and have a few good movies lined up, just in case we need a break from the stars. Additionally, I've considered bringing along a board game or two to keep the evening lively and engaging. The goal is to make her feel welcomed and valued, ensuring she has a great time.\\ \tt \\ \tt Finally, I remind myself to stay open and observant during the date. It's important to listen actively and respond thoughtfully to whatever she shares. By focusing on the present moment and her experience, I can ensure the evening is as enjoyable for her as I hope it will be for me. If things go well, I might casually mention the possibility of doing this again, leaving the door open for future plans. Ultimately, the key is to be myself and let the natural chemistry we share guide the evening."<|end_of_text|> 
	  } 
	   } 
	   } 
	 & & \\ 
 

    \theutterance \stepcounter{utterance}  

    & & & \multicolumn{2}{p{0.3\linewidth}}{\cellcolor[rgb]{0.95,0.95,0.95}{%
	\makecell[{{p{\linewidth}}}]{% 
	  \tt {\tiny [GM$|$GM]}  
	 {'length_constraints:nth_paragraph_first_word': {'num_paragraphs': 4, 'first_word': 'weekend', 'nth_paragraph': 1}, 'startend:quotation': {}} 
	  } 
	   } 
	   } 
	 & & \\ 
 

    \theutterance \stepcounter{utterance}  

    & & & \multicolumn{2}{p{0.3\linewidth}}{\cellcolor[rgb]{0.95,0.95,0.95}{%
	\makecell[{{p{\linewidth}}}]{% 
	  \tt {\tiny [GM$|$GM]}  
	 game_result = LOSE 
	  } 
	   } 
	   } 
	 & & \\ 
 

\end{supertabular}
}

\end{document}
