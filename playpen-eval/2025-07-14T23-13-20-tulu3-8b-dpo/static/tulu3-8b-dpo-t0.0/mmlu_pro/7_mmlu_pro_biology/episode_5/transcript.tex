
\documentclass{article}
\usepackage{colortbl}
\usepackage{makecell}
\usepackage{multirow}
\usepackage{supertabular}

\begin{document}

\newcounter{utterance}

\twocolumn

{ \footnotesize  \setcounter{utterance}{1}
\setlength{\tabcolsep}{0pt}
\begin{supertabular}{c@{$\;$}|p{.15\linewidth}@{}p{.15\linewidth}p{.15\linewidth}p{.15\linewidth}p{.15\linewidth}p{.15\linewidth}}

    \# & $\;$A & \multicolumn{4}{c}{Game Master} & $\;\:$B\\
    \hline 

    \theutterance \stepcounter{utterance}  

    & & \multicolumn{4}{p{0.6\linewidth}}{\cellcolor[rgb]{0.9,0.9,0.9}{%
	\makecell[{{p{\linewidth}}}]{% 
	  \tt {\tiny [A$\langle$GM]}  
	 The following are multiple choice questions (with answers) about biology. Answer only with "The answer is (X)" where X is the correct letter choice.\\ \tt \\ \tt Question:\\ \tt Which of the following represents an accurate statement concerning arthropods?\\ \tt Options:\\ \tt A. They possess an exoskeleton composed primarily of peptidoglycan.\\ \tt B. They possess an open circulatory system with a dorsal heart.\\ \tt C. They are members of a biologically unsuccessful phylum incapable of exploiting diverse habitats and nutrition sources.\\ \tt D. They lack paired, jointed appendages.\\ \tt E. N/A\\ \tt F. N/A\\ \tt G. N/A\\ \tt H. N/A\\ \tt I. N/A\\ \tt J. N/A\\ \tt The answer is (B)\\ \tt \\ \tt Question:\\ \tt In a given population, 1 out of every 400 people has a cancer caused by a completely recessive allele, b. Assuming the population is in Hardy-Weinberg equilibrium, which of the following is the expected proportion of individuals who carry the b allele but are not expected to develop the cancer?\\ \tt Options:\\ \tt A. 19/400\\ \tt B. 1/400\\ \tt C. 40/400\\ \tt D. 38/400\\ \tt E. 2/400\\ \tt F. 1/200\\ \tt G. 20/400\\ \tt H. 50/400\\ \tt I. N/A\\ \tt J. N/A\\ \tt The answer is (D)\\ \tt \\ \tt Question:\\ \tt A mutation in a bacterial enzyme changed a previously polar amino acid into a nonpolar amino acid. This amino acid was located at a site distant from the enzyme’s active site. How might this mutation alter the enzyme’s substrate specificity?\\ \tt Options:\\ \tt A. By changing the enzyme’s pH optimum\\ \tt B. By changing the enzyme's molecular weight\\ \tt C. An amino acid change away from the active site increases the enzyme's substrate specificity.\\ \tt D. By changing the shape of the protein\\ \tt E. By changing the enzyme's temperature optimum\\ \tt F. By altering the enzyme's ability to be denatured\\ \tt G. By changing the enzyme’s location in the cell\\ \tt H. By changing the enzyme's color\\ \tt I. An amino acid change away from the active site cannot alter the enzyme’s substrate specificity.\\ \tt J. By altering the enzyme's rate of reaction\\ \tt The answer is (D)\\ \tt \\ \tt Question:\\ \tt Which of the following is not a way to form recombinant DNA?\\ \tt Options:\\ \tt A. Translation\\ \tt B. Conjugation\\ \tt C. Specialized transduction\\ \tt D. Transformation\\ \tt E. N/A\\ \tt F. N/A\\ \tt G. N/A\\ \tt H. N/A\\ \tt I. N/A\\ \tt J. N/A\\ \tt The answer is (A)\\ \tt \\ \tt Question:\\ \tt Which of the following is not known to be involved in the control of cell division?\\ \tt Options:\\ \tt A. Microtubules\\ \tt B. Checkpoints\\ \tt C. DNA polymerase\\ \tt D. Centrosomes\\ \tt E. Cyclins\\ \tt F. Mitochondria\\ \tt G. Protein kinases\\ \tt H. Fibroblast cells\\ \tt I. N/A\\ \tt J. N/A\\ \tt The answer is (H)\\ \tt \\ \tt Question:\\ \tt In human, the M, MN, and N blood groups are determined by twocodominantalleles: L^m andL^n. In a population of 800 college students the following phenotypes were identified: Blood Group Number M 392 MN 336 N 72 800 Determine the frequency of the Lm and I" alleles.\\ \tt Options:\\ \tt A. 0.65 for L^m and 0.35 for L^n\\ \tt B. 0.60 for L^m and 0.40 for L^n\\ \tt C. 0.5 for both L^m and L^n\\ \tt D. 0.8 for L^m and 0.2 for L^n\\ \tt E. 0.75 for L^m and 0.25 for L^n\\ \tt F. 0.9 for L^m and 0.1 for L^n\\ \tt G. 0.7 for L^m and 0.3 for L^n\\ \tt H. 0.85 for L^m and 0.15 for L^n\\ \tt I. 0.55 for L^m and 0.45 for L^n\\ \tt J. 0.6 for L^m and 0.4 for L^n\\ \tt  
	  } 
	   } 
	   } 
	 & \\ 
 

    \theutterance \stepcounter{utterance}  

    & \multicolumn{4}{p{0.6\linewidth}}{\cellcolor[rgb]{0.8,1,0.9}{%
	\makecell[{{p{\linewidth}}}]{% 
	  \tt {\tiny [A$\rangle$GM]}  
	 The answer is (B)\\ \tt \\ \tt Explanation:\\ \tt To determine the allele frequencies, we start by calculating the expected numbers based on the codominant nature of the alleles L^m and L^n.\\ \tt \\ \tt Given phenotypes:\\ \tt - M: 392 (L^mL^m)\\ \tt - MN: 336 (L^mL^n)\\ \tt - N: 72 (L^nL^n)\\ \tt \\ \tt First, calculate the total number of L^m alleles and L^n alleles:\\ \tt - L^m alleles: 392 (from M) + 2 * 336 (from MN) = 392 + 672 = 1064\\ \tt - L^n alleles: 336 (from MN) + 72 (from N) = 408\\ \tt \\ \tt Since there are 800 students, the total number of alleles (2 * 800) is 1600. The frequency of each allele is the number of that allele divided by the total number of alleles.\\ \tt \\ \tt Frequency of L^m:\\ \tt L^m frequency = (Number of L^m alleles) / (Total number of alleles) = 1064 / 1600 = 0.66\\ \tt \\ \tt Frequency of L^n:\\ \tt L^n frequency = (Number of L^n alleles) / (Total number of alleles) = 408 / 1600 = 0.255\\ \tt \\ \tt Since the alleles are codominant, the sum of their frequencies should be 1:\\ \tt 0.66 + 0.255 = 0.915, which 
	  } 
	   } 
	   } 
	 & & \\ 
 

    \theutterance \stepcounter{utterance}  

    & & & \multicolumn{2}{p{0.3\linewidth}}{\cellcolor[rgb]{0.95,0.95,0.95}{%
	\makecell[{{p{\linewidth}}}]{% 
	  \tt {\tiny [GM$|$GM]}  
	 B 
	  } 
	   } 
	   } 
	 & & \\ 
 

    \theutterance \stepcounter{utterance}  

    & & & \multicolumn{2}{p{0.3\linewidth}}{\cellcolor[rgb]{0.95,0.95,0.95}{%
	\makecell[{{p{\linewidth}}}]{% 
	  \tt {\tiny [GM$|$GM]}  
	 G 
	  } 
	   } 
	   } 
	 & & \\ 
 

    \theutterance \stepcounter{utterance}  

    & & & \multicolumn{2}{p{0.3\linewidth}}{\cellcolor[rgb]{0.95,0.95,0.95}{%
	\makecell[{{p{\linewidth}}}]{% 
	  \tt {\tiny [GM$|$GM]}  
	 game_result = LOSE 
	  } 
	   } 
	   } 
	 & & \\ 
 

\end{supertabular}
}

\end{document}
