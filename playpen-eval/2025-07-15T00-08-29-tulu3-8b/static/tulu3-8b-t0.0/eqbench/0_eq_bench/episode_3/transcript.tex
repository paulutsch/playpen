
\documentclass{article}
\usepackage{colortbl}
\usepackage{makecell}
\usepackage{multirow}
\usepackage{supertabular}

\begin{document}

\newcounter{utterance}

\twocolumn

{ \footnotesize  \setcounter{utterance}{1}
\setlength{\tabcolsep}{0pt}
\begin{supertabular}{c@{$\;$}|p{.15\linewidth}@{}p{.15\linewidth}p{.15\linewidth}p{.15\linewidth}p{.15\linewidth}p{.15\linewidth}}

    \# & $\;$A & \multicolumn{4}{c}{Game Master} & $\;\:$B\\
    \hline 

    \theutterance \stepcounter{utterance}  

    & & \multicolumn{4}{p{0.6\linewidth}}{\cellcolor[rgb]{0.9,0.9,0.9}{%
	\makecell[{{p{\linewidth}}}]{% 
	  \tt {\tiny [A$\langle$GM]}  
	 \\ \tt Your task is to predict the likely emotional responses of a character in this dialogue:\\ \tt \\ \tt Sophia: Jamal, I've been looking at your design drafts. They're... unconventional.\\ \tt Jamal: That's one way to put it, Soph. I prefer to call it 'innovative'.\\ \tt Sophia: Innovative or not, we need to stick to the project guidelines. We can't afford to take risks.\\ \tt Jamal: Risks? That's what creativity is all about, Sophia. We need to break the mold.\\ \tt Sophia: This isn't about breaking molds, Jamal. This is about delivering a project on time and within budget.\\ \tt Jamal: And what if we deliver a perfectly timed, perfectly budgeted, perfectly boring project? What then?\\ \tt Sophia: We're not here to entertain, Jamal. We're here to work.\\ \tt Jamal: And who said work can't be entertaining? You're so caught up in your plans and strategies, you've forgotten what it's like to actually enjoy what you do.\\ \tt Sophia: This isn't about enjoyment. This is about responsibility.\\ \tt Jamal: And who's to say we can't be responsible and creative at the same time?\\ \tt Sophia: I'm not saying that, Jamal. I'm saying we need to prioritize.\\ \tt Jamal: And I'm saying we need to think outside the box.\\ \tt Sophia: We don't have time for this. I'm going to revise the project plan.\\ \tt Jamal: And I'm going to continue working on my designs.\\ \tt Sophia: Fine. But remember, Jamal, if this project fails, it's on you.\\ \tt Jamal: And if it succeeds, it's on me too, right?\\ \tt Sophia: We'll see about that.\\ \tt [End dialogue]\\ \tt \\ \tt At the end of this dialogue, Jamal would feel...\\ \tt Defiant\\ \tt Resentful\\ \tt Confident\\ \tt Worried\\ \tt \\ \tt Give each of these possible emotions a score from 0-10 for the relative intensity that they are likely to be feeling each.\\ \tt \\ \tt You must output in the following format, including headings (of course, you should give your own scores), with no additional commentary:\\ \tt \\ \tt Defiant: <score>\\ \tt Resentful: <score>\\ \tt Confident: <score>\\ \tt Worried: <score>\\ \tt \\ \tt \\ \tt [End of answer]\\ \tt \\ \tt Remember: zero is a valid score, meaning they are likely not feeling that emotion. You must score at least one emotion > 0.\\ \tt \\ \tt Your answer:\\ \tt  
	  } 
	   } 
	   } 
	 & \\ 
 

    \theutterance \stepcounter{utterance}  

    & \multicolumn{4}{p{0.6\linewidth}}{\cellcolor[rgb]{0.8,1,0.9}{%
	\makecell[{{p{\linewidth}}}]{% 
	  \tt {\tiny [A$\rangle$GM]}  
	 Defiant: 8\\ \tt Resentful: 5\\ \tt Confident: 7\\ \tt Worried: 3<|end_of_text|> 
	  } 
	   } 
	   } 
	 & & \\ 
 

    \theutterance \stepcounter{utterance}  

    & & & \multicolumn{2}{p{0.3\linewidth}}{\cellcolor[rgb]{0.95,0.95,0.95}{%
	\makecell[{{p{\linewidth}}}]{% 
	  \tt {\tiny [GM$|$GM]}  
	 Defiant: 8\\ \tt Resentful: 5\\ \tt Confident: 7\\ \tt Worried: 3 
	  } 
	   } 
	   } 
	 & & \\ 
 

    \theutterance \stepcounter{utterance}  

    & & & \multicolumn{2}{p{0.3\linewidth}}{\cellcolor[rgb]{0.95,0.95,0.95}{%
	\makecell[{{p{\linewidth}}}]{% 
	  \tt {\tiny [GM$|$GM]}  
	 Defiant: 8\\ \tt Resentful: 3\\ \tt Confident: 7\\ \tt Worried: 0 
	  } 
	   } 
	   } 
	 & & \\ 
 

    \theutterance \stepcounter{utterance}  

    & & & \multicolumn{2}{p{0.3\linewidth}}{\cellcolor[rgb]{0.95,0.95,0.95}{%
	\makecell[{{p{\linewidth}}}]{% 
	  \tt {\tiny [GM$|$GM]}  
	 game_result = LOSE 
	  } 
	   } 
	   } 
	 & & \\ 
 

\end{supertabular}
}

\end{document}
