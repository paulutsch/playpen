
\documentclass{article}
\usepackage{colortbl}
\usepackage{makecell}
\usepackage{multirow}
\usepackage{supertabular}

\begin{document}

\newcounter{utterance}

\twocolumn

{ \footnotesize  \setcounter{utterance}{1}
\setlength{\tabcolsep}{0pt}
\begin{supertabular}{c@{$\;$}|p{.15\linewidth}@{}p{.15\linewidth}p{.15\linewidth}p{.15\linewidth}p{.15\linewidth}p{.15\linewidth}}

    \# & $\;$A & \multicolumn{4}{c}{Game Master} & $\;\:$B\\
    \hline 

    \theutterance \stepcounter{utterance}  

    & & \multicolumn{4}{p{0.6\linewidth}}{\cellcolor[rgb]{0.9,0.9,0.9}{%
	\makecell[{{p{\linewidth}}}]{% 
	  \tt {\tiny [A$\langle$GM]}  
	 You are an expert in causal inference. The following question is not a typical commonsense query, but rather a meticulously designed question created by a professor specializing in causal inference, intended to assess the students' mastery of the course content.\\ \tt Imagine a self-contained, hypothetical world with only the following conditions, and without any unmentioned factors or causal relationships: Rixq has a direct effect on zuph and xevu. Swoy has a direct effect on zuph and xevu. Zuph has a direct effect on xevu. Swoy is unobserved. The overall probability of rixq is 70%. The probability of not rixq and xevu is 6%. The probability of rixq and xevu is 10%. Is the chance of xevu smaller when observing rixq?\\ \tt Start your answer with 'Yes' or 'No', followed by additional reasoning or evidence to support your explanation. 
	  } 
	   } 
	   } 
	 & \\ 
 

    \theutterance \stepcounter{utterance}  

    & \multicolumn{4}{p{0.6\linewidth}}{\cellcolor[rgb]{0.8,1,0.9}{%
	\makecell[{{p{\linewidth}}}]{% 
	  \tt {\tiny [A$\rangle$GM]}  
	 No, the chance of xevu is not smaller when observing rixq. \\ \tt \\ \tt To understand this, we need to use the principles of conditional probability and the given information about the relationships and probabilities in the hypothetical world.\\ \tt \\ \tt First, let's define the events:\\ \tt - \( R \): Rixq occurs\\ \tt - \( \neg R \): Rixq does not occur\\ \tt - \( X \): Xevu occurs\\ \tt - \( S \): Swoy occurs (though unobserved)\\ \tt \\ \tt Given probabilities:\\ \tt - \( P(R) = 0.70 \)\\ \tt - \( P(\neg R \cap X) = 0.06 \)\\ \tt - \( P(R \cap X) = 0.10 \)\\ \tt \\ \tt We want to determine if \( P(X|R) < P(X) \).\\ \tt \\ \tt **Step 1: Calculate \( P(X) \)**\\ \tt \\ \tt Since \( P(R) = 0.70 \) and \( P(R \cap X) = 0.10 \), we can calculate \( P(X) \) using the law of total probability:\\ \tt \\ \tt \[ P(X) = P(X|R)P(R) + P(X|\neg R)P(\neg R) \]\\ \tt \\ \tt We know \( P(X|R) = \frac{P(R \cap X)}{P(R)} = \frac{0.10}{0.70} = \frac{1}{7} \).\\ \tt \\ \tt We also know \( P(\neg R) = 
	  } 
	   } 
	   } 
	 & & \\ 
 

    \theutterance \stepcounter{utterance}  

    & & & \multicolumn{2}{p{0.3\linewidth}}{\cellcolor[rgb]{0.95,0.95,0.95}{%
	\makecell[{{p{\linewidth}}}]{% 
	  \tt {\tiny [GM$|$GM]}  
	 No 
	  } 
	   } 
	   } 
	 & & \\ 
 

    \theutterance \stepcounter{utterance}  

    & & & \multicolumn{2}{p{0.3\linewidth}}{\cellcolor[rgb]{0.95,0.95,0.95}{%
	\makecell[{{p{\linewidth}}}]{% 
	  \tt {\tiny [GM$|$GM]}  
	 yes 
	  } 
	   } 
	   } 
	 & & \\ 
 

    \theutterance \stepcounter{utterance}  

    & & & \multicolumn{2}{p{0.3\linewidth}}{\cellcolor[rgb]{0.95,0.95,0.95}{%
	\makecell[{{p{\linewidth}}}]{% 
	  \tt {\tiny [GM$|$GM]}  
	 game_result = LOSE 
	  } 
	   } 
	   } 
	 & & \\ 
 

\end{supertabular}
}

\end{document}
