
\documentclass{article}
\usepackage{colortbl}
\usepackage{makecell}
\usepackage{multirow}
\usepackage{supertabular}

\begin{document}

\newcounter{utterance}

\twocolumn

{ \footnotesize  \setcounter{utterance}{1}
\setlength{\tabcolsep}{0pt}
\begin{supertabular}{c@{$\;$}|p{.15\linewidth}@{}p{.15\linewidth}p{.15\linewidth}p{.15\linewidth}p{.15\linewidth}p{.15\linewidth}}

    \# & $\;$A & \multicolumn{4}{c}{Game Master} & $\;\:$B\\
    \hline 

    \theutterance \stepcounter{utterance}  

    & & \multicolumn{4}{p{0.6\linewidth}}{\cellcolor[rgb]{0.9,0.9,0.9}{%
	\makecell[{{p{\linewidth}}}]{% 
	  \tt {\tiny [A$\langle$GM]}  
	 You are an expert in causal inference. The following question is not a typical commonsense query, but rather a meticulously designed question created by a professor specializing in causal inference, intended to assess the students' mastery of the course content.\\ \tt Imagine a self-contained, hypothetical world with only the following conditions, and without any unmentioned factors or causal relationships: Poverty has a direct effect on liking spicy food and cholera. Water company has a direct effect on liking spicy food. Liking spicy food has a direct effect on cholera. Poverty is unobserved. The overall probability of liking spicy food is 41%. For people who do not like spicy food, the probability of cholera contraction is 59%. For people who like spicy food, the probability of cholera contraction is 33%. Is cholera contraction more likely than cholera prevention overall?\\ \tt Start your answer with 'Yes' or 'No', followed by additional reasoning or evidence to support your explanation. 
	  } 
	   } 
	   } 
	 & \\ 
 

    \theutterance \stepcounter{utterance}  

    & \multicolumn{4}{p{0.6\linewidth}}{\cellcolor[rgb]{0.8,1,0.9}{%
	\makecell[{{p{\linewidth}}}]{% 
	  \tt {\tiny [A$\rangle$GM]}  
	 No, cholera contraction is not more likely than cholera prevention overall, based on the given data.\\ \tt \\ \tt First, let's break down the information provided:\\ \tt \\ \tt 1. **Direct Effects:**\\ \tt    - Poverty → Liking Spicy Food\\ \tt    - Water Company → Liking Spicy Food\\ \tt    - Liking Spicy Food → Cholera\\ \tt \\ \tt 2. **Probabilities:**\\ \tt    - Probability of liking spicy food = 41%\\ \tt    - Probability of cholera given not liking spicy food = 59%\\ \tt    - Probability of cholera given liking spicy food = 33%\\ \tt \\ \tt To determine whether cholera contraction is more likely than cholera prevention overall, we need to calculate the overall probability of contracting cholera and compare it to the complement, which would be the probability of not contracting cholera (i.e., cholera prevention).\\ \tt \\ \tt **Step 1: Calculate the probability of contracting cholera (P(Cholera)):**\\ \tt \\ \tt - For those who do not like spicy food (probability = 1 - 0.41 = 0.59):\\ \tt   - Probability of cholera = 0.59\\ \tt \\ \tt - For those who like spicy food (probability = 0.41):\\ \tt   - Probability of cholera = 0.33\\ \tt \\ \tt **Step 2: Calculate the overall probability of contracting cholera:**\\ \tt \\ \tt \[ P(\text{Cholera}) = (0.59 \times 0.59) + (0.33 \times 0.41 
	  } 
	   } 
	   } 
	 & & \\ 
 

    \theutterance \stepcounter{utterance}  

    & & & \multicolumn{2}{p{0.3\linewidth}}{\cellcolor[rgb]{0.95,0.95,0.95}{%
	\makecell[{{p{\linewidth}}}]{% 
	  \tt {\tiny [GM$|$GM]}  
	 No 
	  } 
	   } 
	   } 
	 & & \\ 
 

    \theutterance \stepcounter{utterance}  

    & & & \multicolumn{2}{p{0.3\linewidth}}{\cellcolor[rgb]{0.95,0.95,0.95}{%
	\makecell[{{p{\linewidth}}}]{% 
	  \tt {\tiny [GM$|$GM]}  
	 no 
	  } 
	   } 
	   } 
	 & & \\ 
 

    \theutterance \stepcounter{utterance}  

    & & & \multicolumn{2}{p{0.3\linewidth}}{\cellcolor[rgb]{0.95,0.95,0.95}{%
	\makecell[{{p{\linewidth}}}]{% 
	  \tt {\tiny [GM$|$GM]}  
	 game_result = WIN 
	  } 
	   } 
	   } 
	 & & \\ 
 

\end{supertabular}
}

\end{document}
