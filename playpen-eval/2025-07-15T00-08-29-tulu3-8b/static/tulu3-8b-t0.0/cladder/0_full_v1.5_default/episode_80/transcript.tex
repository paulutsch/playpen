
\documentclass{article}
\usepackage{colortbl}
\usepackage{makecell}
\usepackage{multirow}
\usepackage{supertabular}

\begin{document}

\newcounter{utterance}

\twocolumn

{ \footnotesize  \setcounter{utterance}{1}
\setlength{\tabcolsep}{0pt}
\begin{supertabular}{c@{$\;$}|p{.15\linewidth}@{}p{.15\linewidth}p{.15\linewidth}p{.15\linewidth}p{.15\linewidth}p{.15\linewidth}}

    \# & $\;$A & \multicolumn{4}{c}{Game Master} & $\;\:$B\\
    \hline 

    \theutterance \stepcounter{utterance}  

    & & \multicolumn{4}{p{0.6\linewidth}}{\cellcolor[rgb]{0.9,0.9,0.9}{%
	\makecell[{{p{\linewidth}}}]{% 
	  \tt {\tiny [A$\langle$GM]}  
	 You are an expert in causal inference. The following question is not a typical commonsense query, but rather a meticulously designed question created by a professor specializing in causal inference, intended to assess the students' mastery of the course content.\\ \tt Imagine a self-contained, hypothetical world with only the following conditions, and without any unmentioned factors or causal relationships: Zory has a direct effect on zuph and glimx. Zuph has a direct effect on jyka. Jyka has a direct effect on glimx. Zory is unobserved. The overall probability of zuph is 8%. The probability of not zuph and glimx is 58%. The probability of zuph and glimx is 1%. Is the chance of glimx smaller when observing zuph?\\ \tt Start your answer with 'Yes' or 'No', followed by additional reasoning or evidence to support your explanation. 
	  } 
	   } 
	   } 
	 & \\ 
 

    \theutterance \stepcounter{utterance}  

    & \multicolumn{4}{p{0.6\linewidth}}{\cellcolor[rgb]{0.8,1,0.9}{%
	\makecell[{{p{\linewidth}}}]{% 
	  \tt {\tiny [A$\rangle$GM]}  
	 No, the chance of glimx is not smaller when observing zuph.\\ \tt \\ \tt To understand this, let's break down the given information:\\ \tt \\ \tt 1. **Direct Effects:**\\ \tt    - Zory affects zuph and glimx.\\ \tt    - Zuph affects jyka.\\ \tt    - Jyka affects glimx.\\ \tt \\ \tt 2. **Given Probabilities:**\\ \tt    - Probability of zuph (P(Zuph)) = 8% = 0.08\\ \tt    - Probability of not zuph and glimx (P(¬Zuph ∧ Glimx)) = 58% = 0.58\\ \tt    - Probability of zuph and glimx (P(Zuph ∧ Glimx)) = 1% = 0.01\\ \tt \\ \tt 3. **Calculation:**\\ \tt    - From the probability of not zuph and glimx (0.58), we can calculate the probability of glimx given not zuph (P(Glimx | ¬Zuph)):\\ \tt \\ \tt      P(¬Zuph ∧ Glimx) = P(Glimx | ¬Zuph) * P(¬Zuph)\\ \tt \\ \tt      Solving for P(Glimx | ¬Zuph):\\ \tt \\ \tt      0.58 = P(Glimx | ¬Zuph) * (1 - P(Zuph))\\ \tt \\ \tt      0.58 = P(Glimx | ¬Z 
	  } 
	   } 
	   } 
	 & & \\ 
 

    \theutterance \stepcounter{utterance}  

    & & & \multicolumn{2}{p{0.3\linewidth}}{\cellcolor[rgb]{0.95,0.95,0.95}{%
	\makecell[{{p{\linewidth}}}]{% 
	  \tt {\tiny [GM$|$GM]}  
	 No 
	  } 
	   } 
	   } 
	 & & \\ 
 

    \theutterance \stepcounter{utterance}  

    & & & \multicolumn{2}{p{0.3\linewidth}}{\cellcolor[rgb]{0.95,0.95,0.95}{%
	\makecell[{{p{\linewidth}}}]{% 
	  \tt {\tiny [GM$|$GM]}  
	 yes 
	  } 
	   } 
	   } 
	 & & \\ 
 

    \theutterance \stepcounter{utterance}  

    & & & \multicolumn{2}{p{0.3\linewidth}}{\cellcolor[rgb]{0.95,0.95,0.95}{%
	\makecell[{{p{\linewidth}}}]{% 
	  \tt {\tiny [GM$|$GM]}  
	 game_result = LOSE 
	  } 
	   } 
	   } 
	 & & \\ 
 

\end{supertabular}
}

\end{document}
