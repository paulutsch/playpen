
\documentclass{article}
\usepackage{colortbl}
\usepackage{makecell}
\usepackage{multirow}
\usepackage{supertabular}

\begin{document}

\newcounter{utterance}

\twocolumn

{ \footnotesize  \setcounter{utterance}{1}
\setlength{\tabcolsep}{0pt}
\begin{supertabular}{c@{$\;$}|p{.15\linewidth}@{}p{.15\linewidth}p{.15\linewidth}p{.15\linewidth}p{.15\linewidth}p{.15\linewidth}}

    \# & $\;$A & \multicolumn{4}{c}{Game Master} & $\;\:$B\\
    \hline 

    \theutterance \stepcounter{utterance}  

    & & \multicolumn{4}{p{0.6\linewidth}}{\cellcolor[rgb]{0.9,0.9,0.9}{%
	\makecell[{{p{\linewidth}}}]{% 
	  \tt {\tiny [A$\langle$GM]}  
	 You are an expert in causal inference. The following question is not a typical commonsense query, but rather a meticulously designed question created by a professor specializing in causal inference, intended to assess the students' mastery of the course content.\\ \tt Imagine a self-contained, hypothetical world with only the following conditions, and without any unmentioned factors or causal relationships: Xevo has a direct effect on tijw and tijv. Tijv has a direct effect on gyzp. Tijw has a direct effect on gyzp. The overall probability of xevo is 48%. For those who are not xevo, the probability of gyzp is 39%. For those who are xevo, the probability of gyzp is 78%. Is gyzp more likely than not gyzp overall?\\ \tt Start your answer with 'Yes' or 'No', followed by additional reasoning or evidence to support your explanation. 
	  } 
	   } 
	   } 
	 & \\ 
 

    \theutterance \stepcounter{utterance}  

    & \multicolumn{4}{p{0.6\linewidth}}{\cellcolor[rgb]{0.8,1,0.9}{%
	\makecell[{{p{\linewidth}}}]{% 
	  \tt {\tiny [A$\rangle$GM]}  
	 No, gyzp is not more likely than not gyzp overall. To determine the overall probability of gyzp, we need to consider the probability of gyzp for both those who are xevo and those who are not xevo, weighted by their respective probabilities.\\ \tt \\ \tt Given:\\ \tt - Probability of Xevo (P(Xevo)) = 0.48\\ \tt - Probability of Gyzp given not Xevo (P(Gyzp|~Xevo)) = 0.39\\ \tt - Probability of Gyzp given Xevo (P(Gyzp|Xevo)) = 0.78\\ \tt \\ \tt The overall probability of Gyzp (P(Gyzp)) can be calculated using the law of total probability:\\ \tt \\ \tt \[ P(Gyzp) = P(Gyzp|Xevo) \cdot P(Xevo) + P(Gyzp|~Xevo) \cdot P(~Xevo) \]\\ \tt \\ \tt Where \( P(~Xevo) = 1 - P(Xevo) = 1 - 0.48 = 0.52 \).\\ \tt \\ \tt Substituting the given values:\\ \tt \\ \tt \[ P(Gyzp) = (0.78 \cdot 0.48) + (0.39 \cdot 0.52) \]\\ \tt \\ \tt \[ P(Gyzp) = 0.3744 + 0.2028 \]\\ \tt \\ \tt \[ P(Gyzp) = 0.577 
	  } 
	   } 
	   } 
	 & & \\ 
 

    \theutterance \stepcounter{utterance}  

    & & & \multicolumn{2}{p{0.3\linewidth}}{\cellcolor[rgb]{0.95,0.95,0.95}{%
	\makecell[{{p{\linewidth}}}]{% 
	  \tt {\tiny [GM$|$GM]}  
	 No 
	  } 
	   } 
	   } 
	 & & \\ 
 

    \theutterance \stepcounter{utterance}  

    & & & \multicolumn{2}{p{0.3\linewidth}}{\cellcolor[rgb]{0.95,0.95,0.95}{%
	\makecell[{{p{\linewidth}}}]{% 
	  \tt {\tiny [GM$|$GM]}  
	 yes 
	  } 
	   } 
	   } 
	 & & \\ 
 

    \theutterance \stepcounter{utterance}  

    & & & \multicolumn{2}{p{0.3\linewidth}}{\cellcolor[rgb]{0.95,0.95,0.95}{%
	\makecell[{{p{\linewidth}}}]{% 
	  \tt {\tiny [GM$|$GM]}  
	 game_result = LOSE 
	  } 
	   } 
	   } 
	 & & \\ 
 

\end{supertabular}
}

\end{document}
