
\documentclass{article}
\usepackage{colortbl}
\usepackage{makecell}
\usepackage{multirow}
\usepackage{supertabular}

\begin{document}

\newcounter{utterance}

\twocolumn

{ \footnotesize  \setcounter{utterance}{1}
\setlength{\tabcolsep}{0pt}
\begin{supertabular}{c@{$\;$}|p{.15\linewidth}@{}p{.15\linewidth}p{.15\linewidth}p{.15\linewidth}p{.15\linewidth}p{.15\linewidth}}

    \# & $\;$A & \multicolumn{4}{c}{Game Master} & $\;\:$B\\
    \hline 

    \theutterance \stepcounter{utterance}  

    & & \multicolumn{4}{p{0.6\linewidth}}{\cellcolor[rgb]{0.9,0.9,0.9}{%
	\makecell[{{p{\linewidth}}}]{% 
	  \tt {\tiny [A$\langle$GM]}  
	 You are an expert in causal inference. The following question is not a typical commonsense query, but rather a meticulously designed question created by a professor specializing in causal inference, intended to assess the students' mastery of the course content.\\ \tt Imagine a self-contained, hypothetical world with only the following conditions, and without any unmentioned factors or causal relationships: Husband has a direct effect on wife and alarm clock. Wife has a direct effect on alarm clock. For husbands that don't set the alarm, the probability of ringing alarm is 78%. For husbands that set the alarm, the probability of ringing alarm is 32%. For husbands that set the alarm, would it be less likely to see ringing alarm if the husband had not set the alarm?\\ \tt Start your answer with 'Yes' or 'No', followed by additional reasoning or evidence to support your explanation. 
	  } 
	   } 
	   } 
	 & \\ 
 

    \theutterance \stepcounter{utterance}  

    & \multicolumn{4}{p{0.6\linewidth}}{\cellcolor[rgb]{0.8,1,0.9}{%
	\makecell[{{p{\linewidth}}}]{% 
	  \tt {\tiny [A$\rangle$GM]}  
	 No, it would not be less likely to see a ringing alarm if the husband had not set the alarm. \\ \tt \\ \tt The question is asking whether the probability of the alarm ringing is lower for husbands who set the alarm compared to those who do not set the alarm. However, we need to carefully consider the given probabilities:\\ \tt \\ \tt - Probability of alarm ringing when the husband does not set it: 78%\\ \tt - Probability of alarm ringing when the husband sets it: 32%\\ \tt \\ \tt The key here is to understand the causal relationships and the conditional probabilities provided:\\ \tt \\ \tt 1. **Direct Effect of Husband on Alarm**: The husband's action (setting or not setting the alarm) directly affects the probability of the alarm ringing. When he sets the alarm, it reduces the probability of the alarm ringing from 78% to 32%.\\ \tt \\ \tt 2. **Direct Effect of Wife on Alarm**: The wife also has a direct effect on the alarm, but this information is not directly relevant to the question at hand, which focuses on the husband's action.\\ \tt \\ \tt 3. **Causal Inference**: The question asks whether it would be less likely to see the alarm ringing if the husband had not set it, given that he did set it. The answer is 'No' because the provided probabilities already tell us that the alarm is more likely to ring when the husband does not set it (78% chance) compared to when he does set it (32% chance).\\ \tt \\ \tt Therefore, the answer is 'No', it would not be less likely to 
	  } 
	   } 
	   } 
	 & & \\ 
 

    \theutterance \stepcounter{utterance}  

    & & & \multicolumn{2}{p{0.3\linewidth}}{\cellcolor[rgb]{0.95,0.95,0.95}{%
	\makecell[{{p{\linewidth}}}]{% 
	  \tt {\tiny [GM$|$GM]}  
	 No 
	  } 
	   } 
	   } 
	 & & \\ 
 

    \theutterance \stepcounter{utterance}  

    & & & \multicolumn{2}{p{0.3\linewidth}}{\cellcolor[rgb]{0.95,0.95,0.95}{%
	\makecell[{{p{\linewidth}}}]{% 
	  \tt {\tiny [GM$|$GM]}  
	 no 
	  } 
	   } 
	   } 
	 & & \\ 
 

    \theutterance \stepcounter{utterance}  

    & & & \multicolumn{2}{p{0.3\linewidth}}{\cellcolor[rgb]{0.95,0.95,0.95}{%
	\makecell[{{p{\linewidth}}}]{% 
	  \tt {\tiny [GM$|$GM]}  
	 game_result = WIN 
	  } 
	   } 
	   } 
	 & & \\ 
 

\end{supertabular}
}

\end{document}
