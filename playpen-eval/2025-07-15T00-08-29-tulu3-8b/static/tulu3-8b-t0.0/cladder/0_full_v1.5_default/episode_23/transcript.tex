
\documentclass{article}
\usepackage{colortbl}
\usepackage{makecell}
\usepackage{multirow}
\usepackage{supertabular}

\begin{document}

\newcounter{utterance}

\twocolumn

{ \footnotesize  \setcounter{utterance}{1}
\setlength{\tabcolsep}{0pt}
\begin{supertabular}{c@{$\;$}|p{.15\linewidth}@{}p{.15\linewidth}p{.15\linewidth}p{.15\linewidth}p{.15\linewidth}p{.15\linewidth}}

    \# & $\;$A & \multicolumn{4}{c}{Game Master} & $\;\:$B\\
    \hline 

    \theutterance \stepcounter{utterance}  

    & & \multicolumn{4}{p{0.6\linewidth}}{\cellcolor[rgb]{0.9,0.9,0.9}{%
	\makecell[{{p{\linewidth}}}]{% 
	  \tt {\tiny [A$\langle$GM]}  
	 You are an expert in causal inference. The following question is not a typical commonsense query, but rather a meticulously designed question created by a professor specializing in causal inference, intended to assess the students' mastery of the course content.\\ \tt Imagine a self-contained, hypothetical world with only the following conditions, and without any unmentioned factors or causal relationships: Education level has a direct effect on skill. Skill has a direct effect on salary. Method 1: We look at how education level correlates with salary case by case according to skill. Method 2: We look directly at how education level correlates with salary in general. To understand how education level affects salary, is it more correct to use the Method 1 than Method 2?\\ \tt Start your answer with 'Yes' or 'No', followed by additional reasoning or evidence to support your explanation. 
	  } 
	   } 
	   } 
	 & \\ 
 

    \theutterance \stepcounter{utterance}  

    & \multicolumn{4}{p{0.6\linewidth}}{\cellcolor[rgb]{0.8,1,0.9}{%
	\makecell[{{p{\linewidth}}}]{% 
	  \tt {\tiny [A$\rangle$GM]}  
	 No, it is not more correct to use Method 1 than Method 2 to understand how education level affects salary.\\ \tt \\ \tt The reason is that Method 1 introduces a confounding variable (skill) into the analysis, which can lead to biased estimates of the effect of education on salary. In Method 1, we are essentially trying to understand the effect of education on salary by controlling for skill, which is itself a consequence of education. This creates a circular relationship where education affects skill, and skill affects salary, but by controlling for skill, we are essentially adjusting for a variable that is on the causal pathway from education to salary.\\ \tt \\ \tt In contrast, Method 2 directly examines the correlation between education level and salary without adjusting for skill. While this approach does not control for the potential mediating effect of skill, it provides a more straightforward and unbiased estimate of the direct effect of education on salary, assuming that skill does not confound the relationship.\\ \tt \\ \tt To clarify, the direct effect of education on salary is the increase in salary that can be attributed to having a higher level of education, holding all other factors constant. Method 2 attempts to capture this direct effect, while Method 1 risks over-adjusting by controlling for a variable that is on the causal pathway.\\ \tt \\ \tt In causal inference, it is crucial to distinguish between direct effects and indirect effects, and to avoid over-adjustment for variables that are on the causal pathway. Therefore, while Method 2 may not capture the full complexity of the relationship (as it 
	  } 
	   } 
	   } 
	 & & \\ 
 

    \theutterance \stepcounter{utterance}  

    & & & \multicolumn{2}{p{0.3\linewidth}}{\cellcolor[rgb]{0.95,0.95,0.95}{%
	\makecell[{{p{\linewidth}}}]{% 
	  \tt {\tiny [GM$|$GM]}  
	 No 
	  } 
	   } 
	   } 
	 & & \\ 
 

    \theutterance \stepcounter{utterance}  

    & & & \multicolumn{2}{p{0.3\linewidth}}{\cellcolor[rgb]{0.95,0.95,0.95}{%
	\makecell[{{p{\linewidth}}}]{% 
	  \tt {\tiny [GM$|$GM]}  
	 no 
	  } 
	   } 
	   } 
	 & & \\ 
 

    \theutterance \stepcounter{utterance}  

    & & & \multicolumn{2}{p{0.3\linewidth}}{\cellcolor[rgb]{0.95,0.95,0.95}{%
	\makecell[{{p{\linewidth}}}]{% 
	  \tt {\tiny [GM$|$GM]}  
	 game_result = WIN 
	  } 
	   } 
	   } 
	 & & \\ 
 

\end{supertabular}
}

\end{document}
