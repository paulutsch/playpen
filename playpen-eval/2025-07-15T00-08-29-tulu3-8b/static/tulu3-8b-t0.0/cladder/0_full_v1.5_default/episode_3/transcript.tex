
\documentclass{article}
\usepackage{colortbl}
\usepackage{makecell}
\usepackage{multirow}
\usepackage{supertabular}

\begin{document}

\newcounter{utterance}

\twocolumn

{ \footnotesize  \setcounter{utterance}{1}
\setlength{\tabcolsep}{0pt}
\begin{supertabular}{c@{$\;$}|p{.15\linewidth}@{}p{.15\linewidth}p{.15\linewidth}p{.15\linewidth}p{.15\linewidth}p{.15\linewidth}}

    \# & $\;$A & \multicolumn{4}{c}{Game Master} & $\;\:$B\\
    \hline 

    \theutterance \stepcounter{utterance}  

    & & \multicolumn{4}{p{0.6\linewidth}}{\cellcolor[rgb]{0.9,0.9,0.9}{%
	\makecell[{{p{\linewidth}}}]{% 
	  \tt {\tiny [A$\langle$GM]}  
	 You are an expert in causal inference. The following question is not a typical commonsense query, but rather a meticulously designed question created by a professor specializing in causal inference, intended to assess the students' mastery of the course content.\\ \tt Imagine a self-contained, hypothetical world with only the following conditions, and without any unmentioned factors or causal relationships: CEO has a direct effect on director and manager. Manager has a direct effect on employee. Director has a direct effect on employee. Method 1: We look at how manager correlates with employee case by case according to director. Method 2: We look directly at how manager correlates with employee in general. To understand how manager affects employee, is it more correct to use the Method 1 than Method 2?\\ \tt Start your answer with 'Yes' or 'No', followed by additional reasoning or evidence to support your explanation. 
	  } 
	   } 
	   } 
	 & \\ 
 

    \theutterance \stepcounter{utterance}  

    & \multicolumn{4}{p{0.6\linewidth}}{\cellcolor[rgb]{0.8,1,0.9}{%
	\makecell[{{p{\linewidth}}}]{% 
	  \tt {\tiny [A$\rangle$GM]}  
	 No, it is not more correct to use Method 1 than Method 2 to understand how manager affects employee.\\ \tt \\ \tt The reason is that Method 1 introduces an additional layer of complexity by considering the director's influence on both the manager and the employee. This can potentially confound the direct effect of the manager on the employee, leading to an overestimation or underestimation of the true causal relationship.\\ \tt \\ \tt In contrast, Method 2 directly examines the relationship between the manager and the employee, controlling for any other variables that might affect both (such as the director). This approach is more straightforward and aligns with the principles of causal inference, which emphasize the importance of isolating the effect of a single cause (in this case, the manager) on an outcome (the employee).\\ \tt \\ \tt While the director's influence is acknowledged in both methods, Method 2 is more focused on the direct causal pathway from manager to employee, which is the primary question at hand. Including the director in the analysis (Method 1) might introduce spurious correlations or confounding factors that do not directly relate to the manager's influence on the employee.\\ \tt \\ \tt In causal inference, it is crucial to avoid overadjustment bias, which can occur when adjusting for too many variables, some of which may not be on the causal pathway of interest. Therefore, Method 2, which directly assesses the manager-employee relationship, is generally more appropriate for understanding the direct effect of the manager on the employee.\\ \tt \\ \tt To further support this, consider 
	  } 
	   } 
	   } 
	 & & \\ 
 

    \theutterance \stepcounter{utterance}  

    & & & \multicolumn{2}{p{0.3\linewidth}}{\cellcolor[rgb]{0.95,0.95,0.95}{%
	\makecell[{{p{\linewidth}}}]{% 
	  \tt {\tiny [GM$|$GM]}  
	 No 
	  } 
	   } 
	   } 
	 & & \\ 
 

    \theutterance \stepcounter{utterance}  

    & & & \multicolumn{2}{p{0.3\linewidth}}{\cellcolor[rgb]{0.95,0.95,0.95}{%
	\makecell[{{p{\linewidth}}}]{% 
	  \tt {\tiny [GM$|$GM]}  
	 yes 
	  } 
	   } 
	   } 
	 & & \\ 
 

    \theutterance \stepcounter{utterance}  

    & & & \multicolumn{2}{p{0.3\linewidth}}{\cellcolor[rgb]{0.95,0.95,0.95}{%
	\makecell[{{p{\linewidth}}}]{% 
	  \tt {\tiny [GM$|$GM]}  
	 game_result = LOSE 
	  } 
	   } 
	   } 
	 & & \\ 
 

\end{supertabular}
}

\end{document}
