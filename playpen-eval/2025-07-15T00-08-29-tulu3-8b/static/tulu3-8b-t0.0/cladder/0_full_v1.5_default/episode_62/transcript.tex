
\documentclass{article}
\usepackage{colortbl}
\usepackage{makecell}
\usepackage{multirow}
\usepackage{supertabular}

\begin{document}

\newcounter{utterance}

\twocolumn

{ \footnotesize  \setcounter{utterance}{1}
\setlength{\tabcolsep}{0pt}
\begin{supertabular}{c@{$\;$}|p{.15\linewidth}@{}p{.15\linewidth}p{.15\linewidth}p{.15\linewidth}p{.15\linewidth}p{.15\linewidth}}

    \# & $\;$A & \multicolumn{4}{c}{Game Master} & $\;\:$B\\
    \hline 

    \theutterance \stepcounter{utterance}  

    & & \multicolumn{4}{p{0.6\linewidth}}{\cellcolor[rgb]{0.9,0.9,0.9}{%
	\makecell[{{p{\linewidth}}}]{% 
	  \tt {\tiny [A$\langle$GM]}  
	 You are an expert in causal inference. The following question is not a typical commonsense query, but rather a meticulously designed question created by a professor specializing in causal inference, intended to assess the students' mastery of the course content.\\ \tt Imagine a self-contained, hypothetical world with only the following conditions, and without any unmentioned factors or causal relationships: Drinking coffee has a direct effect on hospitalization status. Broken bones has a direct effect on hospitalization status. Method 1: We look directly at how drinking coffee correlates with broken bones in general. Method 2: We look at this correlation case by case according to hospitalization status. To understand how drinking coffee affects broken bones, is it more correct to use the Method 1 than Method 2?\\ \tt Start your answer with 'Yes' or 'No', followed by additional reasoning or evidence to support your explanation. 
	  } 
	   } 
	   } 
	 & \\ 
 

    \theutterance \stepcounter{utterance}  

    & \multicolumn{4}{p{0.6\linewidth}}{\cellcolor[rgb]{0.8,1,0.9}{%
	\makecell[{{p{\linewidth}}}]{% 
	  \tt {\tiny [A$\rangle$GM]}  
	 No, it is not more correct to use Method 1 than Method 2 to understand how drinking coffee affects broken bones.\\ \tt \\ \tt The reason is that Method 1, which looks directly at the correlation between drinking coffee and broken bones in general, does not account for the potential confounding variables, such as age, physical activity, or lifestyle factors, which could influence both coffee consumption and the likelihood of broken bones. This approach can lead to spurious correlations and incorrect causal inferences.\\ \tt \\ \tt On the other hand, Method 2, which examines the correlation case by case according to hospitalization status, allows for the consideration of potential confounders and the possibility of different causal relationships in different subgroups. By stratifying the analysis by hospitalization status, we can control for the effect of broken bones on hospitalization and thereby isolate the effect of coffee consumption on the risk of broken bones.\\ \tt \\ \tt In causal inference, it is crucial to consider the potential confounding variables and to use methods that allow for the control of these variables. Therefore, Method 2, which provides a more nuanced and controlled analysis, is generally more appropriate for understanding the causal relationship between drinking coffee and broken bones.\\ \tt \\ \tt To further support this, consider the potential for confounding variables such as:\\ \tt \\ \tt - Age: Older individuals may drink more coffee and have a higher risk of broken bones due to age-related factors.\\ \tt - Physical activity: Individuals who drink more coffee might also engage in more physical activities, increasing their risk of broken bones.\\ \tt - Medication: 
	  } 
	   } 
	   } 
	 & & \\ 
 

    \theutterance \stepcounter{utterance}  

    & & & \multicolumn{2}{p{0.3\linewidth}}{\cellcolor[rgb]{0.95,0.95,0.95}{%
	\makecell[{{p{\linewidth}}}]{% 
	  \tt {\tiny [GM$|$GM]}  
	 No 
	  } 
	   } 
	   } 
	 & & \\ 
 

    \theutterance \stepcounter{utterance}  

    & & & \multicolumn{2}{p{0.3\linewidth}}{\cellcolor[rgb]{0.95,0.95,0.95}{%
	\makecell[{{p{\linewidth}}}]{% 
	  \tt {\tiny [GM$|$GM]}  
	 yes 
	  } 
	   } 
	   } 
	 & & \\ 
 

    \theutterance \stepcounter{utterance}  

    & & & \multicolumn{2}{p{0.3\linewidth}}{\cellcolor[rgb]{0.95,0.95,0.95}{%
	\makecell[{{p{\linewidth}}}]{% 
	  \tt {\tiny [GM$|$GM]}  
	 game_result = LOSE 
	  } 
	   } 
	   } 
	 & & \\ 
 

\end{supertabular}
}

\end{document}
