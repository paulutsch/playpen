
\documentclass{article}
\usepackage{colortbl}
\usepackage{makecell}
\usepackage{multirow}
\usepackage{supertabular}

\begin{document}

\newcounter{utterance}

\twocolumn

{ \footnotesize  \setcounter{utterance}{1}
\setlength{\tabcolsep}{0pt}
\begin{supertabular}{c@{$\;$}|p{.15\linewidth}@{}p{.15\linewidth}p{.15\linewidth}p{.15\linewidth}p{.15\linewidth}p{.15\linewidth}}

    \# & $\;$A & \multicolumn{4}{c}{Game Master} & $\;\:$B\\
    \hline 

    \theutterance \stepcounter{utterance}  

    & & \multicolumn{4}{p{0.6\linewidth}}{\cellcolor[rgb]{0.9,0.9,0.9}{%
	\makecell[{{p{\linewidth}}}]{% 
	  \tt {\tiny [A$\langle$GM]}  
	 You are an expert in causal inference. The following question is not a typical commonsense query, but rather a meticulously designed question created by a professor specializing in causal inference, intended to assess the students' mastery of the course content.\\ \tt Imagine a self-contained, hypothetical world with only the following conditions, and without any unmentioned factors or causal relationships: Tanning salon treatment has a direct effect on foot size. Going to the beach has a direct effect on foot size. The overall probability of tanning salon treatment is 10%. For people not using tanning salon treatments, the probability of large feet is 13%. For people who went to tanning salons, the probability of large feet is 70%. Is large feet more likely than small feet overall?\\ \tt Start your answer with 'Yes' or 'No', followed by additional reasoning or evidence to support your explanation. 
	  } 
	   } 
	   } 
	 & \\ 
 

    \theutterance \stepcounter{utterance}  

    & \multicolumn{4}{p{0.6\linewidth}}{\cellcolor[rgb]{0.8,1,0.9}{%
	\makecell[{{p{\linewidth}}}]{% 
	  \tt {\tiny [A$\rangle$GM]}  
	 No, large feet are not more likely than small feet overall. To determine the overall probability of having large feet, we need to consider both groups: those who use tanning salon treatments and those who do not.\\ \tt \\ \tt First, calculate the probability of having large feet for the entire population:\\ \tt \\ \tt 1. **Probability of large feet for those who do not use tanning salon treatments:** 13%\\ \tt 2. **Probability of not using tanning salon treatments:** 100% - 10% = 90%\\ \tt 3. **Expected probability of large feet for those who do not use tanning salons (weighted by their probability):** 13% * 90% = 11.7%\\ \tt \\ \tt Next, calculate the probability of having large feet for those who use tanning salon treatments:\\ \tt \\ \tt 1. **Probability of large feet for those who use tanning salon treatments:** 70%\\ \tt 2. **Probability of using tanning salon treatments:** 10%\\ \tt 3. **Expected probability of large feet for those who use tanning salons (weighted by their probability):** 70% * 10% = 7%\\ \tt \\ \tt Now, combine these two probabilities to find the overall probability of having large feet:\\ \tt \\ \tt **Overall probability of large feet = (Probability of large feet for non-users * Probability of not being a user) + (Probability of large feet for users * Probability of being a user)**\\ \tt **Overall probability of large feet = (11.7% + 7%) = 18.7 
	  } 
	   } 
	   } 
	 & & \\ 
 

    \theutterance \stepcounter{utterance}  

    & & & \multicolumn{2}{p{0.3\linewidth}}{\cellcolor[rgb]{0.95,0.95,0.95}{%
	\makecell[{{p{\linewidth}}}]{% 
	  \tt {\tiny [GM$|$GM]}  
	 No 
	  } 
	   } 
	   } 
	 & & \\ 
 

    \theutterance \stepcounter{utterance}  

    & & & \multicolumn{2}{p{0.3\linewidth}}{\cellcolor[rgb]{0.95,0.95,0.95}{%
	\makecell[{{p{\linewidth}}}]{% 
	  \tt {\tiny [GM$|$GM]}  
	 no 
	  } 
	   } 
	   } 
	 & & \\ 
 

    \theutterance \stepcounter{utterance}  

    & & & \multicolumn{2}{p{0.3\linewidth}}{\cellcolor[rgb]{0.95,0.95,0.95}{%
	\makecell[{{p{\linewidth}}}]{% 
	  \tt {\tiny [GM$|$GM]}  
	 game_result = WIN 
	  } 
	   } 
	   } 
	 & & \\ 
 

\end{supertabular}
}

\end{document}
