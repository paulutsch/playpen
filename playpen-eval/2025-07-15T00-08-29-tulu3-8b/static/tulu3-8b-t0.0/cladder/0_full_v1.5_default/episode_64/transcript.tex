
\documentclass{article}
\usepackage{colortbl}
\usepackage{makecell}
\usepackage{multirow}
\usepackage{supertabular}

\begin{document}

\newcounter{utterance}

\twocolumn

{ \footnotesize  \setcounter{utterance}{1}
\setlength{\tabcolsep}{0pt}
\begin{supertabular}{c@{$\;$}|p{.15\linewidth}@{}p{.15\linewidth}p{.15\linewidth}p{.15\linewidth}p{.15\linewidth}p{.15\linewidth}}

    \# & $\;$A & \multicolumn{4}{c}{Game Master} & $\;\:$B\\
    \hline 

    \theutterance \stepcounter{utterance}  

    & & \multicolumn{4}{p{0.6\linewidth}}{\cellcolor[rgb]{0.9,0.9,0.9}{%
	\makecell[{{p{\linewidth}}}]{% 
	  \tt {\tiny [A$\langle$GM]}  
	 You are an expert in causal inference. The following question is not a typical commonsense query, but rather a meticulously designed question created by a professor specializing in causal inference, intended to assess the students' mastery of the course content.\\ \tt Imagine a self-contained, hypothetical world with only the following conditions, and without any unmentioned factors or causal relationships: Zuph has a direct effect on rixq and xevu. Rixq has a direct effect on xevu. For those who are not zuph and are not rixq, the probability of xevu is 15%. For those who are not zuph and are rixq, the probability of xevu is 50%. For those who are zuph and are not rixq, the probability of xevu is 32%. For those who are zuph and are rixq, the probability of xevu is 85%. The overall probability of zuph is 84%. Will rixq increase the chance of xevu?\\ \tt Start your answer with 'Yes' or 'No', followed by additional reasoning or evidence to support your explanation. 
	  } 
	   } 
	   } 
	 & \\ 
 

    \theutterance \stepcounter{utterance}  

    & \multicolumn{4}{p{0.6\linewidth}}{\cellcolor[rgb]{0.8,1,0.9}{%
	\makecell[{{p{\linewidth}}}]{% 
	  \tt {\tiny [A$\rangle$GM]}  
	 No, rixq does not increase the chance of xevu. \\ \tt \\ \tt To determine whether rixq increases the probability of xevu, we need to compare the conditional probabilities of xevu given rixq and not rixq, while considering the effect of zuph.\\ \tt \\ \tt First, let's break down the given probabilities:\\ \tt \\ \tt - Probability of xevu for those who are not zuph and are not rixq: 15%\\ \tt - Probability of xevu for those who are not zuph and are rixq: 50%\\ \tt - Probability of xevu for those who are zuph and are not rixq: 32%\\ \tt - Probability of xevu for those who are zuph and are rixq: 85%\\ \tt \\ \tt Next, we calculate the overall probability of xevu by considering the probability of zuph and the conditional probabilities:\\ \tt \\ \tt 1. **Probability of xevu for those who are not zuph:**\\ \tt    - Probability of not being zuph = 1 - 0.84 = 0.16\\ \tt    - Probability of xevu given not zuph and not rixq = 0.15\\ \tt    - Probability of xevu given not zuph and rixq = 0.50\\ \tt \\ \tt    The overall probability of xevu for those who are not zuph can be calculated as a weighted average:\\ \tt    \[ 
	  } 
	   } 
	   } 
	 & & \\ 
 

    \theutterance \stepcounter{utterance}  

    & & & \multicolumn{2}{p{0.3\linewidth}}{\cellcolor[rgb]{0.95,0.95,0.95}{%
	\makecell[{{p{\linewidth}}}]{% 
	  \tt {\tiny [GM$|$GM]}  
	 No 
	  } 
	   } 
	   } 
	 & & \\ 
 

    \theutterance \stepcounter{utterance}  

    & & & \multicolumn{2}{p{0.3\linewidth}}{\cellcolor[rgb]{0.95,0.95,0.95}{%
	\makecell[{{p{\linewidth}}}]{% 
	  \tt {\tiny [GM$|$GM]}  
	 yes 
	  } 
	   } 
	   } 
	 & & \\ 
 

    \theutterance \stepcounter{utterance}  

    & & & \multicolumn{2}{p{0.3\linewidth}}{\cellcolor[rgb]{0.95,0.95,0.95}{%
	\makecell[{{p{\linewidth}}}]{% 
	  \tt {\tiny [GM$|$GM]}  
	 game_result = LOSE 
	  } 
	   } 
	   } 
	 & & \\ 
 

\end{supertabular}
}

\end{document}
