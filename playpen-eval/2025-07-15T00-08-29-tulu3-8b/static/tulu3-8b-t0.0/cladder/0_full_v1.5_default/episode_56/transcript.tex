
\documentclass{article}
\usepackage{colortbl}
\usepackage{makecell}
\usepackage{multirow}
\usepackage{supertabular}

\begin{document}

\newcounter{utterance}

\twocolumn

{ \footnotesize  \setcounter{utterance}{1}
\setlength{\tabcolsep}{0pt}
\begin{supertabular}{c@{$\;$}|p{.15\linewidth}@{}p{.15\linewidth}p{.15\linewidth}p{.15\linewidth}p{.15\linewidth}p{.15\linewidth}}

    \# & $\;$A & \multicolumn{4}{c}{Game Master} & $\;\:$B\\
    \hline 

    \theutterance \stepcounter{utterance}  

    & & \multicolumn{4}{p{0.6\linewidth}}{\cellcolor[rgb]{0.9,0.9,0.9}{%
	\makecell[{{p{\linewidth}}}]{% 
	  \tt {\tiny [A$\langle$GM]}  
	 You are an expert in causal inference. The following question is not a typical commonsense query, but rather a meticulously designed question created by a professor specializing in causal inference, intended to assess the students' mastery of the course content.\\ \tt Imagine a self-contained, hypothetical world with only the following conditions, and without any unmentioned factors or causal relationships: Tanning salon treatment has a direct effect on black hair. Going to the beach has a direct effect on black hair. The overall probability of tanning salon treatment is 99%. For people not using tanning salon treatments, the probability of black hair is 66%. For people who went to tanning salons, the probability of black hair is 24%. Is black hair less likely than blond hair overall?\\ \tt Start your answer with 'Yes' or 'No', followed by additional reasoning or evidence to support your explanation. 
	  } 
	   } 
	   } 
	 & \\ 
 

    \theutterance \stepcounter{utterance}  

    & \multicolumn{4}{p{0.6\linewidth}}{\cellcolor[rgb]{0.8,1,0.9}{%
	\makecell[{{p{\linewidth}}}]{% 
	  \tt {\tiny [A$\rangle$GM]}  
	 No, black hair is not less likely than blond hair overall. To determine the overall probability of black hair, we need to consider both groups: those who use tanning salon treatments and those who do not.\\ \tt \\ \tt Given:\\ \tt - Probability of tanning salon treatment (P(T)) = 99%\\ \tt - Probability of black hair for those not using tanning salons (P(B|¬T)) = 66%\\ \tt - Probability of black hair for those using tanning salons (P(B|T)) = 24%\\ \tt \\ \tt We can calculate the overall probability of black hair (P(B)) using the law of total probability:\\ \tt \\ \tt P(B) = P(B|T) * P(T) + P(B|¬T) * P(¬T)\\ \tt \\ \tt Where P(¬T) is the probability of not using a tanning salon treatment, which is 1 - P(T) = 1 - 0.99 = 0.01.\\ \tt \\ \tt Substituting the given values:\\ \tt \\ \tt P(B) = (0.24 * 0.99) + (0.66 * 0.01)\\ \tt P(B) = 0.2376 + 0.0066\\ \tt P(B) = 0.2442\\ \tt \\ \tt The overall probability of black hair (P(B)) is 0.2442, or 24.42%. \\ \tt \\ \tt To compare this with the likelihood of blond hair, we would need additional information about the probability of blond hair under both conditions (using t 
	  } 
	   } 
	   } 
	 & & \\ 
 

    \theutterance \stepcounter{utterance}  

    & & & \multicolumn{2}{p{0.3\linewidth}}{\cellcolor[rgb]{0.95,0.95,0.95}{%
	\makecell[{{p{\linewidth}}}]{% 
	  \tt {\tiny [GM$|$GM]}  
	 No 
	  } 
	   } 
	   } 
	 & & \\ 
 

    \theutterance \stepcounter{utterance}  

    & & & \multicolumn{2}{p{0.3\linewidth}}{\cellcolor[rgb]{0.95,0.95,0.95}{%
	\makecell[{{p{\linewidth}}}]{% 
	  \tt {\tiny [GM$|$GM]}  
	 yes 
	  } 
	   } 
	   } 
	 & & \\ 
 

    \theutterance \stepcounter{utterance}  

    & & & \multicolumn{2}{p{0.3\linewidth}}{\cellcolor[rgb]{0.95,0.95,0.95}{%
	\makecell[{{p{\linewidth}}}]{% 
	  \tt {\tiny [GM$|$GM]}  
	 game_result = LOSE 
	  } 
	   } 
	   } 
	 & & \\ 
 

\end{supertabular}
}

\end{document}
