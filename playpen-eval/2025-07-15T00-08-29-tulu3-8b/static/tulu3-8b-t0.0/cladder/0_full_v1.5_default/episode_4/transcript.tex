
\documentclass{article}
\usepackage{colortbl}
\usepackage{makecell}
\usepackage{multirow}
\usepackage{supertabular}

\begin{document}

\newcounter{utterance}

\twocolumn

{ \footnotesize  \setcounter{utterance}{1}
\setlength{\tabcolsep}{0pt}
\begin{supertabular}{c@{$\;$}|p{.15\linewidth}@{}p{.15\linewidth}p{.15\linewidth}p{.15\linewidth}p{.15\linewidth}p{.15\linewidth}}

    \# & $\;$A & \multicolumn{4}{c}{Game Master} & $\;\:$B\\
    \hline 

    \theutterance \stepcounter{utterance}  

    & & \multicolumn{4}{p{0.6\linewidth}}{\cellcolor[rgb]{0.9,0.9,0.9}{%
	\makecell[{{p{\linewidth}}}]{% 
	  \tt {\tiny [A$\langle$GM]}  
	 You are an expert in causal inference. The following question is not a typical commonsense query, but rather a meticulously designed question created by a professor specializing in causal inference, intended to assess the students' mastery of the course content.\\ \tt Imagine a self-contained, hypothetical world with only the following conditions, and without any unmentioned factors or causal relationships: Vaccination status has a direct effect on vaccination reaction and getting smallpox. Getting smallpox has a direct effect on smallpox survival. Vaccination reaction has a direct effect on smallpox survival. For unvaccinated individuals, the probability of smallpox survival is 39%. For vaccinated individuals, the probability of smallpox survival is 46%. For vaccinated individuals, would it be more likely to see smallpox survival if the person had been vaccinated?\\ \tt Start your answer with 'Yes' or 'No', followed by additional reasoning or evidence to support your explanation. 
	  } 
	   } 
	   } 
	 & \\ 
 

    \theutterance \stepcounter{utterance}  

    & \multicolumn{4}{p{0.6\linewidth}}{\cellcolor[rgb]{0.8,1,0.9}{%
	\makecell[{{p{\linewidth}}}]{% 
	  \tt {\tiny [A$\rangle$GM]}  
	 Yes, for vaccinated individuals, it would be more likely to see smallpox survival compared to unvaccinated individuals. This conclusion is based on the given data:\\ \tt \\ \tt - Probability of smallpox survival for unvaccinated individuals: 39%\\ \tt - Probability of smallpox survival for vaccinated individuals: 46%\\ \tt \\ \tt The data provided indicates that the probability of surviving smallpox is higher (46%) for vaccinated individuals compared to unvaccinated individuals (39%). This difference suggests that vaccination has a positive effect on smallpox survival. \\ \tt \\ \tt The direct effects as described in the problem are:\\ \tt 1. Vaccination status directly affects vaccination reaction.\\ \tt 2. Getting smallpox directly affects smallpox survival.\\ \tt 3. Vaccination reaction directly affects smallpox survival.\\ \tt \\ \tt Given these direct effects and the provided probabilities, we can infer that being vaccinated has a beneficial impact on smallpox survival. This inference aligns with the principles of causal inference, where observed differences in outcomes (in this case, survival rates) between different groups (vaccinated vs. unvaccinated) can be used to infer causal relationships, assuming that all other factors are held constant.\\ \tt \\ \tt It is important to note that this conclusion is based solely on the provided data and does not account for any potential confounding variables or other unmentioned factors that might exist in a real-world scenario. However, within the confines of the hypothetical world described, the data strongly supports the statement that vaccinated individuals are more likely to survive 
	  } 
	   } 
	   } 
	 & & \\ 
 

    \theutterance \stepcounter{utterance}  

    & & & \multicolumn{2}{p{0.3\linewidth}}{\cellcolor[rgb]{0.95,0.95,0.95}{%
	\makecell[{{p{\linewidth}}}]{% 
	  \tt {\tiny [GM$|$GM]}  
	 Yes 
	  } 
	   } 
	   } 
	 & & \\ 
 

    \theutterance \stepcounter{utterance}  

    & & & \multicolumn{2}{p{0.3\linewidth}}{\cellcolor[rgb]{0.95,0.95,0.95}{%
	\makecell[{{p{\linewidth}}}]{% 
	  \tt {\tiny [GM$|$GM]}  
	 no 
	  } 
	   } 
	   } 
	 & & \\ 
 

    \theutterance \stepcounter{utterance}  

    & & & \multicolumn{2}{p{0.3\linewidth}}{\cellcolor[rgb]{0.95,0.95,0.95}{%
	\makecell[{{p{\linewidth}}}]{% 
	  \tt {\tiny [GM$|$GM]}  
	 game_result = LOSE 
	  } 
	   } 
	   } 
	 & & \\ 
 

\end{supertabular}
}

\end{document}
