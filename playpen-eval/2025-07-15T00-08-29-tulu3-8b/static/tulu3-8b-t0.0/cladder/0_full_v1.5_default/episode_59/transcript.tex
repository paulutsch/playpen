
\documentclass{article}
\usepackage{colortbl}
\usepackage{makecell}
\usepackage{multirow}
\usepackage{supertabular}

\begin{document}

\newcounter{utterance}

\twocolumn

{ \footnotesize  \setcounter{utterance}{1}
\setlength{\tabcolsep}{0pt}
\begin{supertabular}{c@{$\;$}|p{.15\linewidth}@{}p{.15\linewidth}p{.15\linewidth}p{.15\linewidth}p{.15\linewidth}p{.15\linewidth}}

    \# & $\;$A & \multicolumn{4}{c}{Game Master} & $\;\:$B\\
    \hline 

    \theutterance \stepcounter{utterance}  

    & & \multicolumn{4}{p{0.6\linewidth}}{\cellcolor[rgb]{0.9,0.9,0.9}{%
	\makecell[{{p{\linewidth}}}]{% 
	  \tt {\tiny [A$\langle$GM]}  
	 You are an expert in causal inference. The following question is not a typical commonsense query, but rather a meticulously designed question created by a professor specializing in causal inference, intended to assess the students' mastery of the course content.\\ \tt Imagine a self-contained, hypothetical world with only the following conditions, and without any unmentioned factors or causal relationships: Unobserved confounders has a direct effect on having a sister and cholesterol level. Treatment assignment has a direct effect on having a sister. Having a sister has a direct effect on cholesterol level. Unobserved confounders is unobserved. The overall probability of having a sister is 20%. The probability of not having a sister and low cholesterol is 65%. The probability of having a sister and low cholesterol is 14%. Is the chance of low cholesterol larger when observing having a sister?\\ \tt Start your answer with 'Yes' or 'No', followed by additional reasoning or evidence to support your explanation. 
	  } 
	   } 
	   } 
	 & \\ 
 

    \theutterance \stepcounter{utterance}  

    & \multicolumn{4}{p{0.6\linewidth}}{\cellcolor[rgb]{0.8,1,0.9}{%
	\makecell[{{p{\linewidth}}}]{% 
	  \tt {\tiny [A$\rangle$GM]}  
	 No, the chance of having low cholesterol is not necessarily larger when observing having a sister. \\ \tt \\ \tt To understand this, we need to carefully consider the given causal relationships and probabilities:\\ \tt \\ \tt 1. **Direct Effects:**\\ \tt    - Unobserved confounders (U) affect both having a sister (S) and cholesterol level (C).\\ \tt    - Treatment assignment (T) affects having a sister (S).\\ \tt    - Having a sister (S) affects cholesterol level (C).\\ \tt \\ \tt 2. **Probabilities:**\\ \tt    - Probability of having a sister (P(S=1)) = 20% = 0.2\\ \tt    - Probability of not having a sister and low cholesterol (P(S=0, C=low)) = 65% = 0.65\\ \tt    - Probability of having a sister and low cholesterol (P(S=1, C=low)) = 14% = 0.14\\ \tt \\ \tt 3. **Calculation of Conditional Probabilities:**\\ \tt    - **Probability of low cholesterol given no sister (P(C=low|S=0)):**\\ \tt      \[\\ \tt      P(C=low|S=0) = \frac{P(S=0, C=low)}{P(S=0)} = \frac{0.65}{1 - 0.2} = \frac{0.65}{0.8} = 0.8125\\ \tt      \]\\ \tt    - **Probability of low cholesterol given having a sister (P 
	  } 
	   } 
	   } 
	 & & \\ 
 

    \theutterance \stepcounter{utterance}  

    & & & \multicolumn{2}{p{0.3\linewidth}}{\cellcolor[rgb]{0.95,0.95,0.95}{%
	\makecell[{{p{\linewidth}}}]{% 
	  \tt {\tiny [GM$|$GM]}  
	 No 
	  } 
	   } 
	   } 
	 & & \\ 
 

    \theutterance \stepcounter{utterance}  

    & & & \multicolumn{2}{p{0.3\linewidth}}{\cellcolor[rgb]{0.95,0.95,0.95}{%
	\makecell[{{p{\linewidth}}}]{% 
	  \tt {\tiny [GM$|$GM]}  
	 no 
	  } 
	   } 
	   } 
	 & & \\ 
 

    \theutterance \stepcounter{utterance}  

    & & & \multicolumn{2}{p{0.3\linewidth}}{\cellcolor[rgb]{0.95,0.95,0.95}{%
	\makecell[{{p{\linewidth}}}]{% 
	  \tt {\tiny [GM$|$GM]}  
	 game_result = WIN 
	  } 
	   } 
	   } 
	 & & \\ 
 

\end{supertabular}
}

\end{document}
