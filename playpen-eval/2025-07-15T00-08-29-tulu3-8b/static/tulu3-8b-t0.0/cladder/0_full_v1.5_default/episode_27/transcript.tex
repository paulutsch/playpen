
\documentclass{article}
\usepackage{colortbl}
\usepackage{makecell}
\usepackage{multirow}
\usepackage{supertabular}

\begin{document}

\newcounter{utterance}

\twocolumn

{ \footnotesize  \setcounter{utterance}{1}
\setlength{\tabcolsep}{0pt}
\begin{supertabular}{c@{$\;$}|p{.15\linewidth}@{}p{.15\linewidth}p{.15\linewidth}p{.15\linewidth}p{.15\linewidth}p{.15\linewidth}}

    \# & $\;$A & \multicolumn{4}{c}{Game Master} & $\;\:$B\\
    \hline 

    \theutterance \stepcounter{utterance}  

    & & \multicolumn{4}{p{0.6\linewidth}}{\cellcolor[rgb]{0.9,0.9,0.9}{%
	\makecell[{{p{\linewidth}}}]{% 
	  \tt {\tiny [A$\langle$GM]}  
	 You are an expert in causal inference. The following question is not a typical commonsense query, but rather a meticulously designed question created by a professor specializing in causal inference, intended to assess the students' mastery of the course content.\\ \tt Imagine a self-contained, hypothetical world with only the following conditions, and without any unmentioned factors or causal relationships: Parents' intelligence has a direct effect on parents' social status and child's intelligence. Other unobserved factors has a direct effect on parents' social status and child's intelligence. Parents' social status has a direct effect on child's intelligence. Other unobserved factors is unobserved. The overall probability of intelligent parents is 15%. The probability of unintelligent parents and intelligent child is 54%. The probability of intelligent parents and intelligent child is 4%. Is the chance of intelligent child smaller when observing intelligent parents?\\ \tt Start your answer with 'Yes' or 'No', followed by additional reasoning or evidence to support your explanation. 
	  } 
	   } 
	   } 
	 & \\ 
 

    \theutterance \stepcounter{utterance}  

    & \multicolumn{4}{p{0.6\linewidth}}{\cellcolor[rgb]{0.8,1,0.9}{%
	\makecell[{{p{\linewidth}}}]{% 
	  \tt {\tiny [A$\rangle$GM]}  
	 No, the chance of an intelligent child is not necessarily smaller when observing intelligent parents, based on the given probabilities.\\ \tt \\ \tt First, let's break down the provided probabilities:\\ \tt \\ \tt 1. Probability of intelligent parents (IP) = 15% = 0.15\\ \tt 2. Probability of unintelligent parents (UP) = 1 - 0.15 = 0.85\\ \tt 3. Probability of an intelligent child given unintelligent parents (IC|UP) = 54% = 0.54\\ \tt 4. Probability of an intelligent child given intelligent parents (IC|IP) = 4% = 0.04\\ \tt \\ \tt We are asked whether the chance of an intelligent child (IC) is smaller when observing intelligent parents (IP). To answer this, we need to compare P(IC|IP) and P(IC|~IP), where ~IP denotes not intelligent parents.\\ \tt \\ \tt From the given data:\\ \tt - P(IC|IP) = 0.04\\ \tt - P(IC|UP) = 0.54\\ \tt \\ \tt Since P(IC|IP) < P(IC|UP), it might seem that the chance of an intelligent child is smaller when observing intelligent parents. However, we must consider the overall probabilities and the confounding factors.\\ \tt \\ \tt The overall probability of an intelligent child (IC) can be calculated using the law of total probability:\\ \tt \\ \tt P(IC) = P(IC|IP) * P(IP) + P(IC|UP) * P(UP)\\ \tt \\ \tt Substituting the given probabilities: 
	  } 
	   } 
	   } 
	 & & \\ 
 

    \theutterance \stepcounter{utterance}  

    & & & \multicolumn{2}{p{0.3\linewidth}}{\cellcolor[rgb]{0.95,0.95,0.95}{%
	\makecell[{{p{\linewidth}}}]{% 
	  \tt {\tiny [GM$|$GM]}  
	 No 
	  } 
	   } 
	   } 
	 & & \\ 
 

    \theutterance \stepcounter{utterance}  

    & & & \multicolumn{2}{p{0.3\linewidth}}{\cellcolor[rgb]{0.95,0.95,0.95}{%
	\makecell[{{p{\linewidth}}}]{% 
	  \tt {\tiny [GM$|$GM]}  
	 yes 
	  } 
	   } 
	   } 
	 & & \\ 
 

    \theutterance \stepcounter{utterance}  

    & & & \multicolumn{2}{p{0.3\linewidth}}{\cellcolor[rgb]{0.95,0.95,0.95}{%
	\makecell[{{p{\linewidth}}}]{% 
	  \tt {\tiny [GM$|$GM]}  
	 game_result = LOSE 
	  } 
	   } 
	   } 
	 & & \\ 
 

\end{supertabular}
}

\end{document}
