
\documentclass{article}
\usepackage{colortbl}
\usepackage{makecell}
\usepackage{multirow}
\usepackage{supertabular}

\begin{document}

\newcounter{utterance}

\twocolumn

{ \footnotesize  \setcounter{utterance}{1}
\setlength{\tabcolsep}{0pt}
\begin{supertabular}{c@{$\;$}|p{.15\linewidth}@{}p{.15\linewidth}p{.15\linewidth}p{.15\linewidth}p{.15\linewidth}p{.15\linewidth}}

    \# & $\;$A & \multicolumn{4}{c}{Game Master} & $\;\:$B\\
    \hline 

    \theutterance \stepcounter{utterance}  

    & & \multicolumn{4}{p{0.6\linewidth}}{\cellcolor[rgb]{0.9,0.9,0.9}{%
	\makecell[{{p{\linewidth}}}]{% 
	  \tt {\tiny [A$\langle$GM]}  
	 You are an expert in causal inference. The following question is not a typical commonsense query, but rather a meticulously designed question created by a professor specializing in causal inference, intended to assess the students' mastery of the course content.\\ \tt Imagine a self-contained, hypothetical world with only the following conditions, and without any unmentioned factors or causal relationships: CEO has a direct effect on director and having visited England. Having visited England has a direct effect on employee. Director has a direct effect on employee. For CEOs who fire employees and have not visited England, the probability of employee being fired is 21%. For CEOs who fire employees and have visited England, the probability of employee being fired is 51%. For CEOs who fire employees and have not visited England, the probability of employee being fired is 35%. For CEOs who fire employees and have visited England, the probability of employee being fired is 66%. The overall probability of CEO's decision to fire the employee is 28%. Will having visited England increase the chance of employee being fired?\\ \tt Start your answer with 'Yes' or 'No', followed by additional reasoning or evidence to support your explanation. 
	  } 
	   } 
	   } 
	 & \\ 
 

    \theutterance \stepcounter{utterance}  

    & \multicolumn{4}{p{0.6\linewidth}}{\cellcolor[rgb]{0.8,1,0.9}{%
	\makecell[{{p{\linewidth}}}]{% 
	  \tt {\tiny [A$\rangle$GM]}  
	 No, having visited England does not increase the chance of an employee being fired based solely on the provided data.\\ \tt \\ \tt To analyze this, we need to consider the conditional probabilities and the overall probability of firing an employee. Here's a breakdown:\\ \tt \\ \tt 1. **Conditional Probabilities:**\\ \tt    - Probability of firing given the CEO has not visited England: \\ \tt      - Without visiting England: \( P(Fire | \neg England) = \frac{21\% + 35\%}{28\%} \)\\ \tt      - Visiting England: \( P(Fire | England) = \frac{51\% + 66\%}{28\%} \)\\ \tt \\ \tt 2. **Calculations:**\\ \tt    - Without visiting England: \( \frac{21\% + 35\%}{28\%} = \frac{56\%}{28\%} = 2 \)\\ \tt    - Visiting England: \( \frac{51\% + 66\%}{28\%} = \frac{117\%}{28\%} \approx 4.18 \)\\ \tt \\ \tt 3. **Interpretation:**\\ \tt    - The ratio \( \frac{P(Fire | England)}{P(Fire | \neg England)} \approx 4.18 \) suggests that the probability of firing an employee is higher when the CEO has visited England compared to when they have not. However, this ratio alone does not conclusively determine causation. 
	  } 
	   } 
	   } 
	 & & \\ 
 

    \theutterance \stepcounter{utterance}  

    & & & \multicolumn{2}{p{0.3\linewidth}}{\cellcolor[rgb]{0.95,0.95,0.95}{%
	\makecell[{{p{\linewidth}}}]{% 
	  \tt {\tiny [GM$|$GM]}  
	 No 
	  } 
	   } 
	   } 
	 & & \\ 
 

    \theutterance \stepcounter{utterance}  

    & & & \multicolumn{2}{p{0.3\linewidth}}{\cellcolor[rgb]{0.95,0.95,0.95}{%
	\makecell[{{p{\linewidth}}}]{% 
	  \tt {\tiny [GM$|$GM]}  
	 yes 
	  } 
	   } 
	   } 
	 & & \\ 
 

    \theutterance \stepcounter{utterance}  

    & & & \multicolumn{2}{p{0.3\linewidth}}{\cellcolor[rgb]{0.95,0.95,0.95}{%
	\makecell[{{p{\linewidth}}}]{% 
	  \tt {\tiny [GM$|$GM]}  
	 game_result = LOSE 
	  } 
	   } 
	   } 
	 & & \\ 
 

\end{supertabular}
}

\end{document}
