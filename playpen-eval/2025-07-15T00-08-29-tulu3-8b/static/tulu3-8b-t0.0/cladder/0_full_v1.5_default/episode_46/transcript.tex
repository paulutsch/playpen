
\documentclass{article}
\usepackage{colortbl}
\usepackage{makecell}
\usepackage{multirow}
\usepackage{supertabular}

\begin{document}

\newcounter{utterance}

\twocolumn

{ \footnotesize  \setcounter{utterance}{1}
\setlength{\tabcolsep}{0pt}
\begin{supertabular}{c@{$\;$}|p{.15\linewidth}@{}p{.15\linewidth}p{.15\linewidth}p{.15\linewidth}p{.15\linewidth}p{.15\linewidth}}

    \# & $\;$A & \multicolumn{4}{c}{Game Master} & $\;\:$B\\
    \hline 

    \theutterance \stepcounter{utterance}  

    & & \multicolumn{4}{p{0.6\linewidth}}{\cellcolor[rgb]{0.9,0.9,0.9}{%
	\makecell[{{p{\linewidth}}}]{% 
	  \tt {\tiny [A$\langle$GM]}  
	 You are an expert in causal inference. The following question is not a typical commonsense query, but rather a meticulously designed question created by a professor specializing in causal inference, intended to assess the students' mastery of the course content.\\ \tt Imagine a self-contained, hypothetical world with only the following conditions, and without any unmentioned factors or causal relationships: Kidney stone size has a direct effect on treatment and lip thickness. Treatment has a direct effect on lip thickness. The overall probability of receives treatment is 48%. For patients not receiving treatment, the probability of thick lips is 64%. For patients receiving treatment, the probability of thick lips is 29%. Is thick lips more likely than thin lips overall?\\ \tt Start your answer with 'Yes' or 'No', followed by additional reasoning or evidence to support your explanation. 
	  } 
	   } 
	   } 
	 & \\ 
 

    \theutterance \stepcounter{utterance}  

    & \multicolumn{4}{p{0.6\linewidth}}{\cellcolor[rgb]{0.8,1,0.9}{%
	\makecell[{{p{\linewidth}}}]{% 
	  \tt {\tiny [A$\rangle$GM]}  
	 No, thick lips are not more likely than thin lips overall. To determine the overall probability of having thick lips, we need to consider both groups: those who receive treatment and those who do not.\\ \tt \\ \tt First, calculate the probability of having thick lips for each group:\\ \tt \\ \tt - Probability of thick lips for those not receiving treatment: 64%\\ \tt - Probability of thick lips for those receiving treatment: 29%\\ \tt \\ \tt Next, calculate the probability of each group:\\ \tt \\ \tt - Probability of receiving treatment: 48%\\ \tt - Probability of not receiving treatment: 52% (since 100% - 48% = 52%)\\ \tt \\ \tt Now, calculate the overall probability of having thick lips by considering both groups:\\ \tt \\ \tt Overall probability of thick lips = (Probability of thick lips for those not receiving treatment * Probability of not receiving treatment) + (Probability of thick lips for those receiving treatment * Probability of receiving treatment)\\ \tt \\ \tt Overall probability of thick lips = (64% * 52%) + (29% * 48%)\\ \tt \\ \tt Overall probability of thick lips = (0.64 * 0.52) + (0.29 * 0.48)\\ \tt \\ \tt Overall probability of thick lips = 0.3328 + 0.1392\\ \tt \\ \tt Overall probability of thick lips = 0.472\\ \tt \\ \tt Converting to a percentage, the overall probability of having thick lips is 47.2%.\\ \tt \\ \tt Since 47.2% is less than 50%, thick lips are not more likely than thin lips overall in this hypothetical world. The majority 
	  } 
	   } 
	   } 
	 & & \\ 
 

    \theutterance \stepcounter{utterance}  

    & & & \multicolumn{2}{p{0.3\linewidth}}{\cellcolor[rgb]{0.95,0.95,0.95}{%
	\makecell[{{p{\linewidth}}}]{% 
	  \tt {\tiny [GM$|$GM]}  
	 No 
	  } 
	   } 
	   } 
	 & & \\ 
 

    \theutterance \stepcounter{utterance}  

    & & & \multicolumn{2}{p{0.3\linewidth}}{\cellcolor[rgb]{0.95,0.95,0.95}{%
	\makecell[{{p{\linewidth}}}]{% 
	  \tt {\tiny [GM$|$GM]}  
	 no 
	  } 
	   } 
	   } 
	 & & \\ 
 

    \theutterance \stepcounter{utterance}  

    & & & \multicolumn{2}{p{0.3\linewidth}}{\cellcolor[rgb]{0.95,0.95,0.95}{%
	\makecell[{{p{\linewidth}}}]{% 
	  \tt {\tiny [GM$|$GM]}  
	 game_result = WIN 
	  } 
	   } 
	   } 
	 & & \\ 
 

\end{supertabular}
}

\end{document}
