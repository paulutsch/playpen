
\documentclass{article}
\usepackage{colortbl}
\usepackage{makecell}
\usepackage{multirow}
\usepackage{supertabular}

\begin{document}

\newcounter{utterance}

\twocolumn

{ \footnotesize  \setcounter{utterance}{1}
\setlength{\tabcolsep}{0pt}
\begin{supertabular}{c@{$\;$}|p{.15\linewidth}@{}p{.15\linewidth}p{.15\linewidth}p{.15\linewidth}p{.15\linewidth}p{.15\linewidth}}

    \# & $\;$A & \multicolumn{4}{c}{Game Master} & $\;\:$B\\
    \hline 

    \theutterance \stepcounter{utterance}  

    & & \multicolumn{4}{p{0.6\linewidth}}{\cellcolor[rgb]{0.9,0.9,0.9}{%
	\makecell[{{p{\linewidth}}}]{% 
	  \tt {\tiny [A$\langle$GM]}  
	 You are an expert in causal inference. The following question is not a typical commonsense query, but rather a meticulously designed question created by a professor specializing in causal inference, intended to assess the students' mastery of the course content.\\ \tt Imagine a self-contained, hypothetical world with only the following conditions, and without any unmentioned factors or causal relationships: Ability to speak english has a direct effect on the forest. The smoker has a direct effect on the forest. The overall probability of speaking english is 93%. For people who do not speak english, the probability of the forest on fire is 79%. For people who speak english, the probability of the forest on fire is 65%. Is the forest on fire more likely than the forest not on fire overall?\\ \tt Start your answer with 'Yes' or 'No', followed by additional reasoning or evidence to support your explanation. 
	  } 
	   } 
	   } 
	 & \\ 
 

    \theutterance \stepcounter{utterance}  

    & \multicolumn{4}{p{0.6\linewidth}}{\cellcolor[rgb]{0.8,1,0.9}{%
	\makecell[{{p{\linewidth}}}]{% 
	  \tt {\tiny [A$\rangle$GM]}  
	 No, the forest is not more likely to be on fire overall. To determine this, we need to calculate the overall probability of the forest being on fire, taking into account both groups: those who speak English and those who do not.\\ \tt \\ \tt First, calculate the probability of the forest being on fire for each group:\\ \tt \\ \tt - For those who do not speak English: Probability = 79% = 0.79\\ \tt - For those who speak English: Probability = 65% = 0.65\\ \tt \\ \tt Next, calculate the probability of each group in the population:\\ \tt \\ \tt - Probability of not speaking English = 1 - Probability of speaking English = 1 - 0.93 = 0.07\\ \tt - Probability of speaking English = 0.93\\ \tt \\ \tt Now, calculate the overall probability of the forest being on fire by considering the weighted average of the probabilities for each group:\\ \tt \\ \tt Overall Probability of Forest on Fire = (Probability of Forest on Fire for Non-English Speakers * Probability of Being Non-English Speaker) + (Probability of Forest on Fire for English Speakers * Probability of Being English Speaker)\\ \tt \\ \tt Overall Probability = (0.79 * 0.07) + (0.65 * 0.93)\\ \tt \\ \tt Calculate each term:\\ \tt \\ \tt - (0.79 * 0.07) = 0.0547\\ \tt - (0.65 * 0.93) = 0.603\\ \tt \\ \tt Add these together to get the overall probability:\\ \tt \\ \tt Overall Probability = 0.0547 
	  } 
	   } 
	   } 
	 & & \\ 
 

    \theutterance \stepcounter{utterance}  

    & & & \multicolumn{2}{p{0.3\linewidth}}{\cellcolor[rgb]{0.95,0.95,0.95}{%
	\makecell[{{p{\linewidth}}}]{% 
	  \tt {\tiny [GM$|$GM]}  
	 No 
	  } 
	   } 
	   } 
	 & & \\ 
 

    \theutterance \stepcounter{utterance}  

    & & & \multicolumn{2}{p{0.3\linewidth}}{\cellcolor[rgb]{0.95,0.95,0.95}{%
	\makecell[{{p{\linewidth}}}]{% 
	  \tt {\tiny [GM$|$GM]}  
	 yes 
	  } 
	   } 
	   } 
	 & & \\ 
 

    \theutterance \stepcounter{utterance}  

    & & & \multicolumn{2}{p{0.3\linewidth}}{\cellcolor[rgb]{0.95,0.95,0.95}{%
	\makecell[{{p{\linewidth}}}]{% 
	  \tt {\tiny [GM$|$GM]}  
	 game_result = LOSE 
	  } 
	   } 
	   } 
	 & & \\ 
 

\end{supertabular}
}

\end{document}
