
\documentclass{article}
\usepackage{colortbl}
\usepackage{makecell}
\usepackage{multirow}
\usepackage{supertabular}

\begin{document}

\newcounter{utterance}

\twocolumn

{ \footnotesize  \setcounter{utterance}{1}
\setlength{\tabcolsep}{0pt}
\begin{supertabular}{c@{$\;$}|p{.15\linewidth}@{}p{.15\linewidth}p{.15\linewidth}p{.15\linewidth}p{.15\linewidth}p{.15\linewidth}}

    \# & $\;$A & \multicolumn{4}{c}{Game Master} & $\;\:$B\\
    \hline 

    \theutterance \stepcounter{utterance}  

    & & \multicolumn{4}{p{0.6\linewidth}}{\cellcolor[rgb]{0.9,0.9,0.9}{%
	\makecell[{{p{\linewidth}}}]{% 
	  \tt {\tiny [A$\langle$GM]}  
	 You are an expert in causal inference. The following question is not a typical commonsense query, but rather a meticulously designed question created by a professor specializing in causal inference, intended to assess the students' mastery of the course content.\\ \tt Imagine a self-contained, hypothetical world with only the following conditions, and without any unmentioned factors or causal relationships: Parents' intelligence has a direct effect on parents' social status and child's intelligence. Other unobserved factors has a direct effect on parents' social status and child's intelligence. Parents' social status has a direct effect on child's intelligence. Other unobserved factors is unobserved. We know that intelligent parents and confounder active causes high parental social status. intelligent parents and confounder active and high parental social status causes intelligent child. We observed confounder active. Would the child is intelligent if intelligent parents instead of unintelligent parents?\\ \tt Start your answer with 'Yes' or 'No', followed by additional reasoning or evidence to support your explanation. 
	  } 
	   } 
	   } 
	 & \\ 
 

    \theutterance \stepcounter{utterance}  

    & \multicolumn{4}{p{0.6\linewidth}}{\cellcolor[rgb]{0.8,1,0.9}{%
	\makecell[{{p{\linewidth}}}]{% 
	  \tt {\tiny [A$\rangle$GM]}  
	 Yes, the child would likely be intelligent if the parents were intelligent instead of unintelligent, given the conditions specified.\\ \tt \\ \tt Here's the reasoning:\\ \tt \\ \tt 1. **Direct Effect of Parents' Intelligence on Child's Intelligence**: According to the problem statement, parents' intelligence has a direct effect on the child's intelligence. Intelligent parents are more likely to pass on their intelligence genes, provide a stimulating home environment, and engage in activities that foster cognitive development in their children.\\ \tt \\ \tt 2. **Direct Effect of Parents' Social Status on Child's Intelligence**: Parents' social status also has a direct effect on the child's intelligence. However, this effect is not the primary driver of the child's intelligence in this scenario, as the problem specifies that parents' intelligence has a direct effect on the child's intelligence.\\ \tt \\ \tt 3. **Role of Confounder**: The confounder (actively present) is another factor that affects both parents' social status and the child's intelligence. Intelligent parents are more likely to have a higher social status due to their intelligence and the confounder's influence. However, the confounder does not negate the direct effect of parents' intelligence on the child's intelligence.\\ \tt \\ \tt 4. **Given Conditions**: Given that intelligent parents are present instead of unintelligent parents, and considering the direct effect of parents' intelligence on the child's intelligence, it is reasonable to conclude that the child would be more likely to be intelligent.\\ \tt \\ \tt 5. **Neglecting Unobserved Factors**: The problem states that unobserved factors have 
	  } 
	   } 
	   } 
	 & & \\ 
 

    \theutterance \stepcounter{utterance}  

    & & & \multicolumn{2}{p{0.3\linewidth}}{\cellcolor[rgb]{0.95,0.95,0.95}{%
	\makecell[{{p{\linewidth}}}]{% 
	  \tt {\tiny [GM$|$GM]}  
	 Yes 
	  } 
	   } 
	   } 
	 & & \\ 
 

    \theutterance \stepcounter{utterance}  

    & & & \multicolumn{2}{p{0.3\linewidth}}{\cellcolor[rgb]{0.95,0.95,0.95}{%
	\makecell[{{p{\linewidth}}}]{% 
	  \tt {\tiny [GM$|$GM]}  
	 yes 
	  } 
	   } 
	   } 
	 & & \\ 
 

    \theutterance \stepcounter{utterance}  

    & & & \multicolumn{2}{p{0.3\linewidth}}{\cellcolor[rgb]{0.95,0.95,0.95}{%
	\makecell[{{p{\linewidth}}}]{% 
	  \tt {\tiny [GM$|$GM]}  
	 game_result = WIN 
	  } 
	   } 
	   } 
	 & & \\ 
 

\end{supertabular}
}

\end{document}
