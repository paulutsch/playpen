
\documentclass{article}
\usepackage{colortbl}
\usepackage{makecell}
\usepackage{multirow}
\usepackage{supertabular}

\begin{document}

\newcounter{utterance}

\twocolumn

{ \footnotesize  \setcounter{utterance}{1}
\setlength{\tabcolsep}{0pt}
\begin{supertabular}{c@{$\;$}|p{.15\linewidth}@{}p{.15\linewidth}p{.15\linewidth}p{.15\linewidth}p{.15\linewidth}p{.15\linewidth}}

    \# & $\;$A & \multicolumn{4}{c}{Game Master} & $\;\:$B\\
    \hline 

    \theutterance \stepcounter{utterance}  

    & & \multicolumn{4}{p{0.6\linewidth}}{\cellcolor[rgb]{0.9,0.9,0.9}{%
	\makecell[{{p{\linewidth}}}]{% 
	  \tt {\tiny [A$\langle$GM]}  
	 You are an expert in causal inference. The following question is not a typical commonsense query, but rather a meticulously designed question created by a professor specializing in causal inference, intended to assess the students' mastery of the course content.\\ \tt Imagine a self-contained, hypothetical world with only the following conditions, and without any unmentioned factors or causal relationships: Rixq has a direct effect on zuph and xevu. Swoy has a direct effect on zuph and xevu. Zuph has a direct effect on xevu. Swoy is unobserved. The overall probability of rixq is 70%. The probability of not rixq and xevu is 6%. The probability of rixq and xevu is 10%. Is the chance of xevu larger when observing rixq?\\ \tt Start your answer with 'Yes' or 'No', followed by additional reasoning or evidence to support your explanation. 
	  } 
	   } 
	   } 
	 & \\ 
 

    \theutterance \stepcounter{utterance}  

    & \multicolumn{4}{p{0.6\linewidth}}{\cellcolor[rgb]{0.8,1,0.9}{%
	\makecell[{{p{\linewidth}}}]{% 
	  \tt {\tiny [A$\rangle$GM]}  
	 No, the chance of xevu is not larger when observing rixq. \\ \tt \\ \tt To determine this, we need to calculate the conditional probability \( P(Xevu | Rixq) \) and compare it to \( P(Xevu) \).\\ \tt \\ \tt First, let's define the given probabilities:\\ \tt \\ \tt - \( P(Rixq) = 0.70 \)\\ \tt - \( P(\neg Rixq \land Xevu) = 0.06 \)\\ \tt - \( P(Rixq \land Xevu) = 0.10 \)\\ \tt \\ \tt From these, we can calculate \( P(Xevu) \):\\ \tt \\ \tt \[ P(Xevu) = P(\neg Rixq \land Xevu) + P(Rixq \land Xevu) \]\\ \tt \[ P(Xevu) = 0.06 + 0.10 = 0.16 \]\\ \tt \\ \tt Next, we calculate \( P(Xevu | Rixq) \) using Bayes' theorem:\\ \tt \\ \tt \[ P(Xevu | Rixq) = \frac{P(Rixq \land Xevu)}{P(Rixq)} \]\\ \tt \[ P(Xevu | Rixq) = \frac{0.10}{0.70} \]\\ \tt \[ P(Xevu | Rixq) = \frac{1}{7} \approx 0.1429 \]\\ \tt \\ \tt Now, compare \( P(Xevu | R 
	  } 
	   } 
	   } 
	 & & \\ 
 

    \theutterance \stepcounter{utterance}  

    & & & \multicolumn{2}{p{0.3\linewidth}}{\cellcolor[rgb]{0.95,0.95,0.95}{%
	\makecell[{{p{\linewidth}}}]{% 
	  \tt {\tiny [GM$|$GM]}  
	 No 
	  } 
	   } 
	   } 
	 & & \\ 
 

    \theutterance \stepcounter{utterance}  

    & & & \multicolumn{2}{p{0.3\linewidth}}{\cellcolor[rgb]{0.95,0.95,0.95}{%
	\makecell[{{p{\linewidth}}}]{% 
	  \tt {\tiny [GM$|$GM]}  
	 no 
	  } 
	   } 
	   } 
	 & & \\ 
 

    \theutterance \stepcounter{utterance}  

    & & & \multicolumn{2}{p{0.3\linewidth}}{\cellcolor[rgb]{0.95,0.95,0.95}{%
	\makecell[{{p{\linewidth}}}]{% 
	  \tt {\tiny [GM$|$GM]}  
	 game_result = WIN 
	  } 
	   } 
	   } 
	 & & \\ 
 

\end{supertabular}
}

\end{document}
