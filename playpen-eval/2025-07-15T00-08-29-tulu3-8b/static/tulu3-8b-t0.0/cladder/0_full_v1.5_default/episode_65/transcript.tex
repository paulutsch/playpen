
\documentclass{article}
\usepackage{colortbl}
\usepackage{makecell}
\usepackage{multirow}
\usepackage{supertabular}

\begin{document}

\newcounter{utterance}

\twocolumn

{ \footnotesize  \setcounter{utterance}{1}
\setlength{\tabcolsep}{0pt}
\begin{supertabular}{c@{$\;$}|p{.15\linewidth}@{}p{.15\linewidth}p{.15\linewidth}p{.15\linewidth}p{.15\linewidth}p{.15\linewidth}}

    \# & $\;$A & \multicolumn{4}{c}{Game Master} & $\;\:$B\\
    \hline 

    \theutterance \stepcounter{utterance}  

    & & \multicolumn{4}{p{0.6\linewidth}}{\cellcolor[rgb]{0.9,0.9,0.9}{%
	\makecell[{{p{\linewidth}}}]{% 
	  \tt {\tiny [A$\langle$GM]}  
	 You are an expert in causal inference. The following question is not a typical commonsense query, but rather a meticulously designed question created by a professor specializing in causal inference, intended to assess the students' mastery of the course content.\\ \tt Imagine a self-contained, hypothetical world with only the following conditions, and without any unmentioned factors or causal relationships: Swoy has a direct effect on rixq and xevu. Rixq has a direct effect on zuph. Zuph has a direct effect on xevu. Swoy is unobserved. For those who are not rixq and are not zuph, the probability of xevu is 46%. For those who are not rixq and are zuph, the probability of xevu is 41%. For those who are rixq and are not zuph, the probability of xevu is 66%. For those who are rixq and are zuph, the probability of xevu is 60%. For those who are not rixq, the probability of zuph is 61%. For those who are rixq, the probability of zuph is 34%. For those who are rixq, would it be more likely to see xevu if the individual was not rixq?\\ \tt Start your answer with 'Yes' or 'No', followed by additional reasoning or evidence to support your explanation. 
	  } 
	   } 
	   } 
	 & \\ 
 

    \theutterance \stepcounter{utterance}  

    & \multicolumn{4}{p{0.6\linewidth}}{\cellcolor[rgb]{0.8,1,0.9}{%
	\makecell[{{p{\linewidth}}}]{% 
	  \tt {\tiny [A$\rangle$GM]}  
	 No, it would not be more likely to see xevu if the individual was not rixq.\\ \tt \\ \tt To analyze this, we need to consider the conditional probabilities provided:\\ \tt \\ \tt - Probability of xevu given not rixq and not zuph: 46%\\ \tt - Probability of xevu given not rixq and zuph: 41%\\ \tt - Probability of xevu given rixq and not zuph: 66%\\ \tt - Probability of xevu given rixq and zuph: 60%\\ \tt \\ \tt Additionally, we have the probabilities of zuph given rixq:\\ \tt \\ \tt - Probability of zuph given rixq: 34%\\ \tt \\ \tt We are asked whether, for those who are rixq, it would be more likely to see xevu compared to those who are not rixq. To answer this, we need to consider the overall probabilities of xevu for rixq individuals compared to non-rixq individuals.\\ \tt \\ \tt First, calculate the overall probability of xevu for rixq individuals:\\ \tt \\ \tt \[ P(Xevu | Rixq) = (P(Xevu | Rixq, Zuph) \cdot P(Zuph | Rixq)) + (P(Xevu | Rixq, \neg Zuph) \cdot P(\neg Zuph | Rixq)) \]\\ \tt \\ \tt Substitute the given probabilities:\\ \tt \\ \tt \[ P(Xevu | Rixq 
	  } 
	   } 
	   } 
	 & & \\ 
 

    \theutterance \stepcounter{utterance}  

    & & & \multicolumn{2}{p{0.3\linewidth}}{\cellcolor[rgb]{0.95,0.95,0.95}{%
	\makecell[{{p{\linewidth}}}]{% 
	  \tt {\tiny [GM$|$GM]}  
	 No 
	  } 
	   } 
	   } 
	 & & \\ 
 

    \theutterance \stepcounter{utterance}  

    & & & \multicolumn{2}{p{0.3\linewidth}}{\cellcolor[rgb]{0.95,0.95,0.95}{%
	\makecell[{{p{\linewidth}}}]{% 
	  \tt {\tiny [GM$|$GM]}  
	 no 
	  } 
	   } 
	   } 
	 & & \\ 
 

    \theutterance \stepcounter{utterance}  

    & & & \multicolumn{2}{p{0.3\linewidth}}{\cellcolor[rgb]{0.95,0.95,0.95}{%
	\makecell[{{p{\linewidth}}}]{% 
	  \tt {\tiny [GM$|$GM]}  
	 game_result = WIN 
	  } 
	   } 
	   } 
	 & & \\ 
 

\end{supertabular}
}

\end{document}
