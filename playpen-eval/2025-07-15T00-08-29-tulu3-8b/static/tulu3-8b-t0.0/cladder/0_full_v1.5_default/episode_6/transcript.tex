
\documentclass{article}
\usepackage{colortbl}
\usepackage{makecell}
\usepackage{multirow}
\usepackage{supertabular}

\begin{document}

\newcounter{utterance}

\twocolumn

{ \footnotesize  \setcounter{utterance}{1}
\setlength{\tabcolsep}{0pt}
\begin{supertabular}{c@{$\;$}|p{.15\linewidth}@{}p{.15\linewidth}p{.15\linewidth}p{.15\linewidth}p{.15\linewidth}p{.15\linewidth}}

    \# & $\;$A & \multicolumn{4}{c}{Game Master} & $\;\:$B\\
    \hline 

    \theutterance \stepcounter{utterance}  

    & & \multicolumn{4}{p{0.6\linewidth}}{\cellcolor[rgb]{0.9,0.9,0.9}{%
	\makecell[{{p{\linewidth}}}]{% 
	  \tt {\tiny [A$\langle$GM]}  
	 You are an expert in causal inference. The following question is not a typical commonsense query, but rather a meticulously designed question created by a professor specializing in causal inference, intended to assess the students' mastery of the course content.\\ \tt Imagine a self-contained, hypothetical world with only the following conditions, and without any unmentioned factors or causal relationships: Appearance has a direct effect on fame. Talent has a direct effect on fame. The overall probability of attractive appearance is 56%. For people considered unattractive, the probability of talent is 9%. For people considered attractive, the probability of talent is 9%. Is talent more likely than lack of talent overall?\\ \tt Start your answer with 'Yes' or 'No', followed by additional reasoning or evidence to support your explanation. 
	  } 
	   } 
	   } 
	 & \\ 
 

    \theutterance \stepcounter{utterance}  

    & \multicolumn{4}{p{0.6\linewidth}}{\cellcolor[rgb]{0.8,1,0.9}{%
	\makecell[{{p{\linewidth}}}]{% 
	  \tt {\tiny [A$\rangle$GM]}  
	 No, talent is not more likely than lack of talent overall. \\ \tt \\ \tt To understand this, we need to consider the joint probabilities and the law of total probability. Let's define:\\ \tt \\ \tt - P(A) = Probability of being attractive = 56% = 0.56\\ \tt - P(¬A) = Probability of being unattractive = 44% = 0.44\\ \tt - P(T|A) = Probability of having talent given attractiveness = 9% = 0.09\\ \tt - P(T|¬A) = Probability of having talent given unattractiveness = 9% = 0.09\\ \tt \\ \tt We want to determine the overall probability of having talent (P(T)), which can be calculated using the law of total probability:\\ \tt \\ \tt P(T) = P(T|A) * P(A) + P(T|¬A) * P(¬A)\\ \tt \\ \tt Substituting the given values:\\ \tt \\ \tt P(T) = (0.09 * 0.56) + (0.09 * 0.44)\\ \tt \\ \tt P(T) = 0.0504 + 0.0396\\ \tt \\ \tt P(T) = 0.0900\\ \tt \\ \tt So, the overall probability of having talent is 9%. \\ \tt \\ \tt Now, let's calculate the probability of lacking talent (P(¬T)):\\ \tt \\ \tt P(¬T) = 1 - P(T) = 1 - 0.09 = 0.91\\ \tt \\ \tt Therefore, talent is not more likely 
	  } 
	   } 
	   } 
	 & & \\ 
 

    \theutterance \stepcounter{utterance}  

    & & & \multicolumn{2}{p{0.3\linewidth}}{\cellcolor[rgb]{0.95,0.95,0.95}{%
	\makecell[{{p{\linewidth}}}]{% 
	  \tt {\tiny [GM$|$GM]}  
	 No 
	  } 
	   } 
	   } 
	 & & \\ 
 

    \theutterance \stepcounter{utterance}  

    & & & \multicolumn{2}{p{0.3\linewidth}}{\cellcolor[rgb]{0.95,0.95,0.95}{%
	\makecell[{{p{\linewidth}}}]{% 
	  \tt {\tiny [GM$|$GM]}  
	 no 
	  } 
	   } 
	   } 
	 & & \\ 
 

    \theutterance \stepcounter{utterance}  

    & & & \multicolumn{2}{p{0.3\linewidth}}{\cellcolor[rgb]{0.95,0.95,0.95}{%
	\makecell[{{p{\linewidth}}}]{% 
	  \tt {\tiny [GM$|$GM]}  
	 game_result = WIN 
	  } 
	   } 
	   } 
	 & & \\ 
 

\end{supertabular}
}

\end{document}
