
\documentclass{article}
\usepackage{colortbl}
\usepackage{makecell}
\usepackage{multirow}
\usepackage{supertabular}

\begin{document}

\newcounter{utterance}

\twocolumn

{ \footnotesize  \setcounter{utterance}{1}
\setlength{\tabcolsep}{0pt}
\begin{supertabular}{c@{$\;$}|p{.15\linewidth}@{}p{.15\linewidth}p{.15\linewidth}p{.15\linewidth}p{.15\linewidth}p{.15\linewidth}}

    \# & $\;$A & \multicolumn{4}{c}{Game Master} & $\;\:$B\\
    \hline 

    \theutterance \stepcounter{utterance}  

    & & \multicolumn{4}{p{0.6\linewidth}}{\cellcolor[rgb]{0.9,0.9,0.9}{%
	\makecell[{{p{\linewidth}}}]{% 
	  \tt {\tiny [A$\langle$GM]}  
	 You are an expert in causal inference. The following question is not a typical commonsense query, but rather a meticulously designed question created by a professor specializing in causal inference, intended to assess the students' mastery of the course content.\\ \tt Imagine a self-contained, hypothetical world with only the following conditions, and without any unmentioned factors or causal relationships: Personality has a direct effect on relationship status. Appearance has a direct effect on relationship status. The overall probability of kindness is 85%. For people with mean personalities and single people, the probability of attractive appearance is 38%. For people with mean personalities and in a relationship, the probability of attractive appearance is 71%. For people with kind personalities and single people, the probability of attractive appearance is 32%. For people with kind personalities and in a relationship, the probability of attractive appearance is 57%. If we look at people in a relationship, does the chance of attractive appearance decrease when kindness?\\ \tt Start your answer with 'Yes' or 'No', followed by additional reasoning or evidence to support your explanation. 
	  } 
	   } 
	   } 
	 & \\ 
 

    \theutterance \stepcounter{utterance}  

    & \multicolumn{4}{p{0.6\linewidth}}{\cellcolor[rgb]{0.8,1,0.9}{%
	\makecell[{{p{\linewidth}}}]{% 
	  \tt {\tiny [A$\rangle$GM]}  
	 No, the chance of attractive appearance does not necessarily decrease when kindness is considered for people in a relationship.\\ \tt \\ \tt To analyze this, we need to compare the probabilities of attractive appearance for people with kind personalities who are in a relationship versus those with mean personalities who are in a relationship.\\ \tt \\ \tt For people with kind personalities and in a relationship, the probability of attractive appearance is 57%.\\ \tt \\ \tt For people with mean personalities and in a relationship, the probability of attractive appearance is 71%.\\ \tt \\ \tt The probability of attractive appearance is higher for those with mean personalities compared to those with kind personalities when both are in a relationship. This suggests that, under the given conditions, having a mean personality might be associated with a higher likelihood of being perceived as attractive when in a relationship, compared to having a kind personality.\\ \tt \\ \tt However, it's important to note that these probabilities are not absolute measures of attractiveness but rather conditional probabilities given the specified personality and relationship status. The overall probability of kindness being 85% does not directly influence the conditional probabilities provided for attractiveness.\\ \tt \\ \tt The key takeaway here is that the relationship between kindness, personality, and attractiveness is complex and context-dependent. In this hypothetical scenario, the direct effect of personality on attractiveness seems to overshadow the effect of kindness, at least when considering the provided probabilities.\\ \tt \\ \tt Therefore, based on the given data, the chance of attractive appearance does not decrease when kindness is considered for people in a relationship. Instead, the data suggests that the personality trait (mean vs. kind) might have a more significant impact on 
	  } 
	   } 
	   } 
	 & & \\ 
 

    \theutterance \stepcounter{utterance}  

    & & & \multicolumn{2}{p{0.3\linewidth}}{\cellcolor[rgb]{0.95,0.95,0.95}{%
	\makecell[{{p{\linewidth}}}]{% 
	  \tt {\tiny [GM$|$GM]}  
	 No 
	  } 
	   } 
	   } 
	 & & \\ 
 

    \theutterance \stepcounter{utterance}  

    & & & \multicolumn{2}{p{0.3\linewidth}}{\cellcolor[rgb]{0.95,0.95,0.95}{%
	\makecell[{{p{\linewidth}}}]{% 
	  \tt {\tiny [GM$|$GM]}  
	 yes 
	  } 
	   } 
	   } 
	 & & \\ 
 

    \theutterance \stepcounter{utterance}  

    & & & \multicolumn{2}{p{0.3\linewidth}}{\cellcolor[rgb]{0.95,0.95,0.95}{%
	\makecell[{{p{\linewidth}}}]{% 
	  \tt {\tiny [GM$|$GM]}  
	 game_result = LOSE 
	  } 
	   } 
	   } 
	 & & \\ 
 

\end{supertabular}
}

\end{document}
