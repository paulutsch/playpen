
\documentclass{article}
\usepackage{colortbl}
\usepackage{makecell}
\usepackage{multirow}
\usepackage{supertabular}

\begin{document}

\newcounter{utterance}

\twocolumn

{ \footnotesize  \setcounter{utterance}{1}
\setlength{\tabcolsep}{0pt}
\begin{supertabular}{c@{$\;$}|p{.15\linewidth}@{}p{.15\linewidth}p{.15\linewidth}p{.15\linewidth}p{.15\linewidth}p{.15\linewidth}}

    \# & $\;$A & \multicolumn{4}{c}{Game Master} & $\;\:$B\\
    \hline 

    \theutterance \stepcounter{utterance}  

    & & \multicolumn{4}{p{0.6\linewidth}}{\cellcolor[rgb]{0.9,0.9,0.9}{%
	\makecell[{{p{\linewidth}}}]{% 
	  \tt {\tiny [A$\langle$GM]}  
	 You are an expert in causal inference. The following question is not a typical commonsense query, but rather a meticulously designed question created by a professor specializing in causal inference, intended to assess the students' mastery of the course content.\\ \tt Imagine a self-contained, hypothetical world with only the following conditions, and without any unmentioned factors or causal relationships: CEO has a direct effect on director and manager. Manager has a direct effect on foot size. Director has a direct effect on foot size. For CEOs who fire employees and managers who don't sign termination letters, the probability of large feet is 62%. For CEOs who fire employees and managers who sign termination letters, the probability of large feet is 15%. For CEOs who fire employees and managers who don't sign termination letters, the probability of large feet is 83%. For CEOs who fire employees and managers who sign termination letters, the probability of large feet is 35%. The overall probability of CEO's decision to fire the employee is 94%. Will manager signing the termination letter increase the chance of large feet?\\ \tt Start your answer with 'Yes' or 'No', followed by additional reasoning or evidence to support your explanation. 
	  } 
	   } 
	   } 
	 & \\ 
 

    \theutterance \stepcounter{utterance}  

    & \multicolumn{4}{p{0.6\linewidth}}{\cellcolor[rgb]{0.8,1,0.9}{%
	\makecell[{{p{\linewidth}}}]{% 
	  \tt {\tiny [A$\rangle$GM]}  
	 No, the manager signing the termination letter does not increase the chance of large feet. \\ \tt \\ \tt To analyze this, we need to consider the potential confounding variables and the direct effects as described in the problem. We have the following data points:\\ \tt \\ \tt 1. For CEOs who fire employees and managers who don't sign termination letters, the probability of large feet is 83%.\\ \tt 2. For CEOs who fire employees and managers who sign termination letters, the probability of large feet is 35%.\\ \tt \\ \tt The key here is to understand the direct effects and the potential confounding variables:\\ \tt \\ \tt - **Direct Effect of CEO's Decision to Fire:** This decision has a significant impact on the probability of large feet, as seen by the different probabilities when the manager signs or does not sign the termination letter.\\ \tt - **Direct Effect of Manager's Action (Signing vs. Not Signing):** The problem states that the manager's action (signing or not signing the termination letter) does not directly affect foot size; instead, it interacts with the CEO's decision to fire.\\ \tt \\ \tt To determine the effect of the manager signing the termination letter on the probability of large feet, we should consider the following:\\ \tt \\ \tt - The probability of large feet is higher (83%) when both the CEO fires and the manager does not sign the termination letter.\\ \tt - The probability of large feet is lower (35%) when both the CEO fires and the manager signs the termination letter.\\ \tt \\ \tt The difference in probabilities (83% vs. 35%) suggests that the CEO's decision to 
	  } 
	   } 
	   } 
	 & & \\ 
 

    \theutterance \stepcounter{utterance}  

    & & & \multicolumn{2}{p{0.3\linewidth}}{\cellcolor[rgb]{0.95,0.95,0.95}{%
	\makecell[{{p{\linewidth}}}]{% 
	  \tt {\tiny [GM$|$GM]}  
	 No 
	  } 
	   } 
	   } 
	 & & \\ 
 

    \theutterance \stepcounter{utterance}  

    & & & \multicolumn{2}{p{0.3\linewidth}}{\cellcolor[rgb]{0.95,0.95,0.95}{%
	\makecell[{{p{\linewidth}}}]{% 
	  \tt {\tiny [GM$|$GM]}  
	 no 
	  } 
	   } 
	   } 
	 & & \\ 
 

    \theutterance \stepcounter{utterance}  

    & & & \multicolumn{2}{p{0.3\linewidth}}{\cellcolor[rgb]{0.95,0.95,0.95}{%
	\makecell[{{p{\linewidth}}}]{% 
	  \tt {\tiny [GM$|$GM]}  
	 game_result = WIN 
	  } 
	   } 
	   } 
	 & & \\ 
 

\end{supertabular}
}

\end{document}
