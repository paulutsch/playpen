
\documentclass{article}
\usepackage{colortbl}
\usepackage{makecell}
\usepackage{multirow}
\usepackage{supertabular}

\begin{document}

\newcounter{utterance}

\twocolumn

{ \footnotesize  \setcounter{utterance}{1}
\setlength{\tabcolsep}{0pt}
\begin{supertabular}{c@{$\;$}|p{.15\linewidth}@{}p{.15\linewidth}p{.15\linewidth}p{.15\linewidth}p{.15\linewidth}p{.15\linewidth}}

    \# & $\;$A & \multicolumn{4}{c}{Game Master} & $\;\:$B\\
    \hline 

    \theutterance \stepcounter{utterance}  

    & & \multicolumn{4}{p{0.6\linewidth}}{\cellcolor[rgb]{0.9,0.9,0.9}{%
	\makecell[{{p{\linewidth}}}]{% 
	  \tt {\tiny [A$\langle$GM]}  
	 You are an expert in causal inference. The following question is not a typical commonsense query, but rather a meticulously designed question created by a professor specializing in causal inference, intended to assess the students' mastery of the course content.\\ \tt Imagine a self-contained, hypothetical world with only the following conditions, and without any unmentioned factors or causal relationships: Smoking has a direct effect on effort and college admission. Effort has a direct effect on college admission. For nonsmokers, the probability of college admission is 67%. For smokers, the probability of college admission is 55%. For smokers, would it be less likely to see college admission if the student had been a nonsmoker?\\ \tt Start your answer with 'Yes' or 'No', followed by additional reasoning or evidence to support your explanation. 
	  } 
	   } 
	   } 
	 & \\ 
 

    \theutterance \stepcounter{utterance}  

    & \multicolumn{4}{p{0.6\linewidth}}{\cellcolor[rgb]{0.8,1,0.9}{%
	\makecell[{{p{\linewidth}}}]{% 
	  \tt {\tiny [A$\rangle$GM]}  
	 Yes, it would be less likely to see college admission for smokers compared to nonsmokers, given the provided conditions. \\ \tt \\ \tt The problem states that smoking has a direct effect on effort and college admission, and effort has a direct effect on college admission. We are also given the probabilities: 67% for nonsmokers and 55% for smokers. \\ \tt \\ \tt This setup suggests a causal relationship where smoking negatively impacts both effort and the likelihood of college admission. The lower probability of admission for smokers (55%) compared to nonsmokers (67%) indicates that smoking is associated with a lower likelihood of college admission, even after accounting for the effect of effort.\\ \tt \\ \tt To further clarify, let's consider the potential causal pathway:\\ \tt \\ \tt 1. **Smoking affects effort**: Smoking can impair cognitive functions, reduce motivation, and generally decrease the amount of effort a student might put into their studies.\\ \tt 2. **Effort affects college admission**: Higher effort typically correlates with better academic performance, which is a significant factor in college admission decisions.\\ \tt \\ \tt Given these direct effects, even if we assume that smokers and nonsmokers have the same average level of effort (which is unlikely given the first point), the direct negative effect of smoking on college admission probability remains. This is because smoking itself, independent of effort, lowers the probability of college admission to 55%.\\ \tt \\ \tt Therefore, under the given conditions, smokers would indeed be less likely to see college admission compared to nonsmokers, due to the direct negative effect of smoking on college admission 
	  } 
	   } 
	   } 
	 & & \\ 
 

    \theutterance \stepcounter{utterance}  

    & & & \multicolumn{2}{p{0.3\linewidth}}{\cellcolor[rgb]{0.95,0.95,0.95}{%
	\makecell[{{p{\linewidth}}}]{% 
	  \tt {\tiny [GM$|$GM]}  
	 Yes 
	  } 
	   } 
	   } 
	 & & \\ 
 

    \theutterance \stepcounter{utterance}  

    & & & \multicolumn{2}{p{0.3\linewidth}}{\cellcolor[rgb]{0.95,0.95,0.95}{%
	\makecell[{{p{\linewidth}}}]{% 
	  \tt {\tiny [GM$|$GM]}  
	 no 
	  } 
	   } 
	   } 
	 & & \\ 
 

    \theutterance \stepcounter{utterance}  

    & & & \multicolumn{2}{p{0.3\linewidth}}{\cellcolor[rgb]{0.95,0.95,0.95}{%
	\makecell[{{p{\linewidth}}}]{% 
	  \tt {\tiny [GM$|$GM]}  
	 game_result = LOSE 
	  } 
	   } 
	   } 
	 & & \\ 
 

\end{supertabular}
}

\end{document}
