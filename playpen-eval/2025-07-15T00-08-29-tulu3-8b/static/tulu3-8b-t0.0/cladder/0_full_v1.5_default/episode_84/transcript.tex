
\documentclass{article}
\usepackage{colortbl}
\usepackage{makecell}
\usepackage{multirow}
\usepackage{supertabular}

\begin{document}

\newcounter{utterance}

\twocolumn

{ \footnotesize  \setcounter{utterance}{1}
\setlength{\tabcolsep}{0pt}
\begin{supertabular}{c@{$\;$}|p{.15\linewidth}@{}p{.15\linewidth}p{.15\linewidth}p{.15\linewidth}p{.15\linewidth}p{.15\linewidth}}

    \# & $\;$A & \multicolumn{4}{c}{Game Master} & $\;\:$B\\
    \hline 

    \theutterance \stepcounter{utterance}  

    & & \multicolumn{4}{p{0.6\linewidth}}{\cellcolor[rgb]{0.9,0.9,0.9}{%
	\makecell[{{p{\linewidth}}}]{% 
	  \tt {\tiny [A$\langle$GM]}  
	 You are an expert in causal inference. The following question is not a typical commonsense query, but rather a meticulously designed question created by a professor specializing in causal inference, intended to assess the students' mastery of the course content.\\ \tt Imagine a self-contained, hypothetical world with only the following conditions, and without any unmentioned factors or causal relationships: Yomx has a direct effect on gwet. Gwet has a direct effect on xevu. For those who are not yomx, the probability of xevu is 68%. For those who are yomx, the probability of xevu is 40%. For those who are yomx, would it be less likely to see xevu if the individual was not yomx?\\ \tt Start your answer with 'Yes' or 'No', followed by additional reasoning or evidence to support your explanation. 
	  } 
	   } 
	   } 
	 & \\ 
 

    \theutterance \stepcounter{utterance}  

    & \multicolumn{4}{p{0.6\linewidth}}{\cellcolor[rgb]{0.8,1,0.9}{%
	\makecell[{{p{\linewidth}}}]{% 
	  \tt {\tiny [A$\rangle$GM]}  
	 No, it would not be less likely to see xevu if an individual who is yomx was not yomx. The question seems to imply a misunderstanding of the causal relationships described.\\ \tt \\ \tt Firstly, let's clarify the given information:\\ \tt \\ \tt - Yomx has a direct effect on gwet.\\ \tt - Gwet has a direct effect on xevu.\\ \tt - For those who are yomx, the probability of xevu is 40%.\\ \tt - For those who are not yomx, the probability of xevu is 68%.\\ \tt \\ \tt The key here is to understand the direction of the causal relationships. Yomx affects gwet, and gwet affects xevu. The probability of xevu given yomx (40%) is lower than the probability of xevu given not-yomx (68%). This suggests that being yomx has a causal effect that reduces the likelihood of xevu.\\ \tt \\ \tt Now, the question asks if it would be less likely to see xevu if an individual who is yomx was not yomx. The answer is 'No' because:\\ \tt \\ \tt - The presence of yomx directly affects gwet, which in turn affects xevu.\\ \tt - The probability of xevu is already lower for those who are yomx (40%) compared to those who are not yomx (68%).\\ \tt - Changing the status of yomx (from being yomx to not being 
	  } 
	   } 
	   } 
	 & & \\ 
 

    \theutterance \stepcounter{utterance}  

    & & & \multicolumn{2}{p{0.3\linewidth}}{\cellcolor[rgb]{0.95,0.95,0.95}{%
	\makecell[{{p{\linewidth}}}]{% 
	  \tt {\tiny [GM$|$GM]}  
	 No 
	  } 
	   } 
	   } 
	 & & \\ 
 

    \theutterance \stepcounter{utterance}  

    & & & \multicolumn{2}{p{0.3\linewidth}}{\cellcolor[rgb]{0.95,0.95,0.95}{%
	\makecell[{{p{\linewidth}}}]{% 
	  \tt {\tiny [GM$|$GM]}  
	 no 
	  } 
	   } 
	   } 
	 & & \\ 
 

    \theutterance \stepcounter{utterance}  

    & & & \multicolumn{2}{p{0.3\linewidth}}{\cellcolor[rgb]{0.95,0.95,0.95}{%
	\makecell[{{p{\linewidth}}}]{% 
	  \tt {\tiny [GM$|$GM]}  
	 game_result = WIN 
	  } 
	   } 
	   } 
	 & & \\ 
 

\end{supertabular}
}

\end{document}
