
\documentclass{article}
\usepackage{colortbl}
\usepackage{makecell}
\usepackage{multirow}
\usepackage{supertabular}

\begin{document}

\newcounter{utterance}

\twocolumn

{ \footnotesize  \setcounter{utterance}{1}
\setlength{\tabcolsep}{0pt}
\begin{supertabular}{c@{$\;$}|p{.15\linewidth}@{}p{.15\linewidth}p{.15\linewidth}p{.15\linewidth}p{.15\linewidth}p{.15\linewidth}}

    \# & $\;$A & \multicolumn{4}{c}{Game Master} & $\;\:$B\\
    \hline 

    \theutterance \stepcounter{utterance}  

    & & \multicolumn{4}{p{0.6\linewidth}}{\cellcolor[rgb]{0.9,0.9,0.9}{%
	\makecell[{{p{\linewidth}}}]{% 
	  \tt {\tiny [A$\langle$GM]}  
	 You are an expert in causal inference. The following question is not a typical commonsense query, but rather a meticulously designed question created by a professor specializing in causal inference, intended to assess the students' mastery of the course content.\\ \tt Imagine a self-contained, hypothetical world with only the following conditions, and without any unmentioned factors or causal relationships: Rixq has a direct effect on xevu. Zuph has a direct effect on xevu. The overall probability of rixq is 40%. The probability of not rixq and xevu is 9%. The probability of rixq and xevu is 22%. Is the chance of xevu smaller when observing rixq?\\ \tt Start your answer with 'Yes' or 'No', followed by additional reasoning or evidence to support your explanation. 
	  } 
	   } 
	   } 
	 & \\ 
 

    \theutterance \stepcounter{utterance}  

    & \multicolumn{4}{p{0.6\linewidth}}{\cellcolor[rgb]{0.8,1,0.9}{%
	\makecell[{{p{\linewidth}}}]{% 
	  \tt {\tiny [A$\rangle$GM]}  
	 No, the chance of xevu is not smaller when observing rixq. \\ \tt \\ \tt To determine this, we can use the law of total probability. We are given the following probabilities:\\ \tt \\ \tt - P(Rixq) = 0.40\\ \tt - P(Not Rixq) = 0.60 (since P(Rixq) + P(Not Rixq) = 1)\\ \tt - P(Xevu | Not Rixq) = 0.09\\ \tt - P(Rixq and Xevu) = 0.22\\ \tt \\ \tt First, calculate the probability of Xevu occurring given Rixq (P(Xevu | Rixq)) using the definition of conditional probability:\\ \tt \\ \tt P(Xevu | Rixq) = P(Rixq and Xevu) / P(Rixq)\\ \tt \\ \tt Substitute the given values:\\ \tt \\ \tt P(Xevu | Rixq) = 0.22 / 0.40 = 0.55\\ \tt \\ \tt Now compare P(Xevu | Rixq) with P(Xevu | Not Rixq):\\ \tt \\ \tt - P(Xevu | Rixq) = 0.55\\ \tt - P(Xevu | Not Rixq) = 0.09\\ \tt \\ \tt Since 0.55 > 0.09, the chance of xevu is actually higher when observing rixq, not lower.\\ \tt \\ \tt Therefore, the answer is No, the chance of xevu is 
	  } 
	   } 
	   } 
	 & & \\ 
 

    \theutterance \stepcounter{utterance}  

    & & & \multicolumn{2}{p{0.3\linewidth}}{\cellcolor[rgb]{0.95,0.95,0.95}{%
	\makecell[{{p{\linewidth}}}]{% 
	  \tt {\tiny [GM$|$GM]}  
	 No 
	  } 
	   } 
	   } 
	 & & \\ 
 

    \theutterance \stepcounter{utterance}  

    & & & \multicolumn{2}{p{0.3\linewidth}}{\cellcolor[rgb]{0.95,0.95,0.95}{%
	\makecell[{{p{\linewidth}}}]{% 
	  \tt {\tiny [GM$|$GM]}  
	 no 
	  } 
	   } 
	   } 
	 & & \\ 
 

    \theutterance \stepcounter{utterance}  

    & & & \multicolumn{2}{p{0.3\linewidth}}{\cellcolor[rgb]{0.95,0.95,0.95}{%
	\makecell[{{p{\linewidth}}}]{% 
	  \tt {\tiny [GM$|$GM]}  
	 game_result = WIN 
	  } 
	   } 
	   } 
	 & & \\ 
 

\end{supertabular}
}

\end{document}
