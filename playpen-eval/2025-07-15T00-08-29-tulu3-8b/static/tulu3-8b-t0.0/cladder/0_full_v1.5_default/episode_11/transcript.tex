
\documentclass{article}
\usepackage{colortbl}
\usepackage{makecell}
\usepackage{multirow}
\usepackage{supertabular}

\begin{document}

\newcounter{utterance}

\twocolumn

{ \footnotesize  \setcounter{utterance}{1}
\setlength{\tabcolsep}{0pt}
\begin{supertabular}{c@{$\;$}|p{.15\linewidth}@{}p{.15\linewidth}p{.15\linewidth}p{.15\linewidth}p{.15\linewidth}p{.15\linewidth}}

    \# & $\;$A & \multicolumn{4}{c}{Game Master} & $\;\:$B\\
    \hline 

    \theutterance \stepcounter{utterance}  

    & & \multicolumn{4}{p{0.6\linewidth}}{\cellcolor[rgb]{0.9,0.9,0.9}{%
	\makecell[{{p{\linewidth}}}]{% 
	  \tt {\tiny [A$\langle$GM]}  
	 You are an expert in causal inference. The following question is not a typical commonsense query, but rather a meticulously designed question created by a professor specializing in causal inference, intended to assess the students' mastery of the course content.\\ \tt Imagine a self-contained, hypothetical world with only the following conditions, and without any unmentioned factors or causal relationships: Obesity has a direct effect on diabetes and lifespan. Smoking has a direct effect on diabetes and lifespan. Diabetes has a direct effect on lifespan. Smoking is unobserved. For normal weight people and without diabetes, the probability of long lifespan is 91%. For normal weight people and with diabetes, the probability of long lifespan is 71%. For obese people and without diabetes, the probability of long lifespan is 69%. For obese people and with diabetes, the probability of long lifespan is 43%. For normal weight people and nonsmokers, the probability of having diabetes is 28%. For normal weight people and smokers, the probability of having diabetes is 20%. For obese people and nonsmokers, the probability of having diabetes is 57%. For obese people and smokers, the probability of having diabetes is 70%. The overall probability of smoker is 8%. If we disregard the mediation effect through diabetes, would obesity negatively affect lifespan?\\ \tt Start your answer with 'Yes' or 'No', followed by additional reasoning or evidence to support your explanation. 
	  } 
	   } 
	   } 
	 & \\ 
 

    \theutterance \stepcounter{utterance}  

    & \multicolumn{4}{p{0.6\linewidth}}{\cellcolor[rgb]{0.8,1,0.9}{%
	\makecell[{{p{\linewidth}}}]{% 
	  \tt {\tiny [A$\rangle$GM]}  
	 Yes, obesity negatively affects lifespan when disregarding the mediation effect through diabetes. \\ \tt \\ \tt To understand this, we need to compare the probabilities of long lifespan between normal weight and obese individuals, while considering only the direct effects and ignoring the indirect effect through diabetes.\\ \tt \\ \tt For normal weight individuals, the probability of a long lifespan is 91% when they do not have diabetes. For obese individuals, this probability drops to 69% when they do not have diabetes. This direct comparison already suggests that obesity, on its own, negatively affects the probability of a long lifespan.\\ \tt \\ \tt The data provided does not include a direct comparison for smokers versus nonsmokers among obese individuals, but we can infer the general effect of obesity on lifespan without considering diabetes:\\ \tt \\ \tt - Normal weight individuals have a higher probability of a long lifespan (91%) compared to obese individuals (69%), even when diabetes is not present.\\ \tt \\ \tt This difference in probabilities indicates that obesity, independent of its effect on diabetes, has a negative impact on lifespan. The direct effect of obesity on lifespan is clearly observed in the provided data, as obese individuals have a lower probability of a long lifespan compared to their normal weight counterparts, even when they do not have diabetes.\\ \tt \\ \tt Therefore, based on the given data and disregarding the mediation effect through diabetes, obesity negatively affects lifespan.<|end_of_text|> 
	  } 
	   } 
	   } 
	 & & \\ 
 

    \theutterance \stepcounter{utterance}  

    & & & \multicolumn{2}{p{0.3\linewidth}}{\cellcolor[rgb]{0.95,0.95,0.95}{%
	\makecell[{{p{\linewidth}}}]{% 
	  \tt {\tiny [GM$|$GM]}  
	 Yes 
	  } 
	   } 
	   } 
	 & & \\ 
 

    \theutterance \stepcounter{utterance}  

    & & & \multicolumn{2}{p{0.3\linewidth}}{\cellcolor[rgb]{0.95,0.95,0.95}{%
	\makecell[{{p{\linewidth}}}]{% 
	  \tt {\tiny [GM$|$GM]}  
	 yes 
	  } 
	   } 
	   } 
	 & & \\ 
 

    \theutterance \stepcounter{utterance}  

    & & & \multicolumn{2}{p{0.3\linewidth}}{\cellcolor[rgb]{0.95,0.95,0.95}{%
	\makecell[{{p{\linewidth}}}]{% 
	  \tt {\tiny [GM$|$GM]}  
	 game_result = WIN 
	  } 
	   } 
	   } 
	 & & \\ 
 

\end{supertabular}
}

\end{document}
