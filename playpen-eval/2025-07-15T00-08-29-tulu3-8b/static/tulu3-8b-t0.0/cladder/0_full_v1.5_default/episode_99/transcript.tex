
\documentclass{article}
\usepackage{colortbl}
\usepackage{makecell}
\usepackage{multirow}
\usepackage{supertabular}

\begin{document}

\newcounter{utterance}

\twocolumn

{ \footnotesize  \setcounter{utterance}{1}
\setlength{\tabcolsep}{0pt}
\begin{supertabular}{c@{$\;$}|p{.15\linewidth}@{}p{.15\linewidth}p{.15\linewidth}p{.15\linewidth}p{.15\linewidth}p{.15\linewidth}}

    \# & $\;$A & \multicolumn{4}{c}{Game Master} & $\;\:$B\\
    \hline 

    \theutterance \stepcounter{utterance}  

    & & \multicolumn{4}{p{0.6\linewidth}}{\cellcolor[rgb]{0.9,0.9,0.9}{%
	\makecell[{{p{\linewidth}}}]{% 
	  \tt {\tiny [A$\langle$GM]}  
	 You are an expert in causal inference. The following question is not a typical commonsense query, but rather a meticulously designed question created by a professor specializing in causal inference, intended to assess the students' mastery of the course content.\\ \tt Imagine a self-contained, hypothetical world with only the following conditions, and without any unmentioned factors or causal relationships: Zory has a direct effect on zuph and glimx. Zuph has a direct effect on jyka. Jyka has a direct effect on glimx. Zory is unobserved. We know that zory causes zuph. zuph causes not jyka. zory or jyka causes glimx. We observed an individual is zory. Would an individual is not glimx if not zuph instead of zuph?\\ \tt Start your answer with 'Yes' or 'No', followed by additional reasoning or evidence to support your explanation. 
	  } 
	   } 
	   } 
	 & \\ 
 

    \theutterance \stepcounter{utterance}  

    & \multicolumn{4}{p{0.6\linewidth}}{\cellcolor[rgb]{0.8,1,0.9}{%
	\makecell[{{p{\linewidth}}}]{% 
	  \tt {\tiny [A$\rangle$GM]}  
	 No, an individual would not necessarily be not glimx if not zuph instead of zuph. \\ \tt \\ \tt The reasoning behind this is based on the given causal relationships and the observed condition:\\ \tt \\ \tt 1. **Zory causes Zuph**: This establishes a direct causal link from Zory to Zuph. Since we observed an individual is Zory, it implies that Zuph is also present because Zory's effect on Zuph is direct and unidirectional.\\ \tt \\ \tt 2. **Zuph causes not Jyka**: This indicates that Zuph has a direct effect on Jyka, but it is such that Zuph prevents Jyka from occurring. However, this does not directly affect the relationship between Zory and Glimx.\\ \tt \\ \tt 3. **Zory or Jyka causes Glimx**: This condition introduces a crucial ambiguity. It suggests that Glimx can be caused either by Zory or by Jyka. Since we know Zory causes Zuph, and Zuph prevents Jyka, the presence of Zory indirectly influences Glimx through Zuph's effect on Jyka.\\ \tt \\ \tt Given these relationships, if Zuph were not present (which would only happen if Zory were not present, according to the direct causal link), then Jyka would be present (since Zuph prevents Jyka). However, the crucial point is that Zory still causes Glimx directly, 
	  } 
	   } 
	   } 
	 & & \\ 
 

    \theutterance \stepcounter{utterance}  

    & & & \multicolumn{2}{p{0.3\linewidth}}{\cellcolor[rgb]{0.95,0.95,0.95}{%
	\makecell[{{p{\linewidth}}}]{% 
	  \tt {\tiny [GM$|$GM]}  
	 No 
	  } 
	   } 
	   } 
	 & & \\ 
 

    \theutterance \stepcounter{utterance}  

    & & & \multicolumn{2}{p{0.3\linewidth}}{\cellcolor[rgb]{0.95,0.95,0.95}{%
	\makecell[{{p{\linewidth}}}]{% 
	  \tt {\tiny [GM$|$GM]}  
	 no 
	  } 
	   } 
	   } 
	 & & \\ 
 

    \theutterance \stepcounter{utterance}  

    & & & \multicolumn{2}{p{0.3\linewidth}}{\cellcolor[rgb]{0.95,0.95,0.95}{%
	\makecell[{{p{\linewidth}}}]{% 
	  \tt {\tiny [GM$|$GM]}  
	 game_result = WIN 
	  } 
	   } 
	   } 
	 & & \\ 
 

\end{supertabular}
}

\end{document}
