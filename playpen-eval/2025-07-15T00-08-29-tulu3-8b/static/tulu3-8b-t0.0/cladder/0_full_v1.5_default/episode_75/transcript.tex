
\documentclass{article}
\usepackage{colortbl}
\usepackage{makecell}
\usepackage{multirow}
\usepackage{supertabular}

\begin{document}

\newcounter{utterance}

\twocolumn

{ \footnotesize  \setcounter{utterance}{1}
\setlength{\tabcolsep}{0pt}
\begin{supertabular}{c@{$\;$}|p{.15\linewidth}@{}p{.15\linewidth}p{.15\linewidth}p{.15\linewidth}p{.15\linewidth}p{.15\linewidth}}

    \# & $\;$A & \multicolumn{4}{c}{Game Master} & $\;\:$B\\
    \hline 

    \theutterance \stepcounter{utterance}  

    & & \multicolumn{4}{p{0.6\linewidth}}{\cellcolor[rgb]{0.9,0.9,0.9}{%
	\makecell[{{p{\linewidth}}}]{% 
	  \tt {\tiny [A$\langle$GM]}  
	 You are an expert in causal inference. The following question is not a typical commonsense query, but rather a meticulously designed question created by a professor specializing in causal inference, intended to assess the students' mastery of the course content.\\ \tt Imagine a self-contained, hypothetical world with only the following conditions, and without any unmentioned factors or causal relationships: Jyka has a direct effect on yupt. Yupt has a direct effect on kwox. The overall probability of jyka is 25%. The probability of not jyka and kwox is 71%. The probability of jyka and kwox is 24%. Is the chance of kwox smaller when observing jyka?\\ \tt Start your answer with 'Yes' or 'No', followed by additional reasoning or evidence to support your explanation. 
	  } 
	   } 
	   } 
	 & \\ 
 

    \theutterance \stepcounter{utterance}  

    & \multicolumn{4}{p{0.6\linewidth}}{\cellcolor[rgb]{0.8,1,0.9}{%
	\makecell[{{p{\linewidth}}}]{% 
	  \tt {\tiny [A$\rangle$GM]}  
	 No, the chance of kwox is not smaller when observing jyka. \\ \tt \\ \tt To understand this, we need to use the principles of conditional probability and the given probabilities. We are given the following:\\ \tt \\ \tt - P(Jyka) = 0.25\\ \tt - P(Not Jyka and Kwox) = 0.71\\ \tt - P(Jyka and Kwox) = 0.24\\ \tt \\ \tt First, let's calculate the probability of Kwox (P(Kwox)):\\ \tt \\ \tt P(Kwox) = P(Kwox and Jyka) + P(Kwox and Not Jyka)\\ \tt \\ \tt From the given data, P(Jyka and Kwox) = 0.24, which is the probability of both Jyka and Kwox occurring together.\\ \tt \\ \tt To find P(Kwox and Not Jyka), we use the fact that the probability of Not Jyka and Kwox is 0.71. This can be broken down as:\\ \tt \\ \tt P(Not Jyka and Kwox) = P(Kwox and Not Jyka)\\ \tt \\ \tt So, P(Kwox and Not Jyka) = 0.71.\\ \tt \\ \tt Now, we can calculate P(Kwox):\\ \tt \\ \tt P(Kwox) = P(Jyka and Kwox) + P(Kwox and Not Jyka)\\ \tt P(Kwox) = 0.24 + 0.71 
	  } 
	   } 
	   } 
	 & & \\ 
 

    \theutterance \stepcounter{utterance}  

    & & & \multicolumn{2}{p{0.3\linewidth}}{\cellcolor[rgb]{0.95,0.95,0.95}{%
	\makecell[{{p{\linewidth}}}]{% 
	  \tt {\tiny [GM$|$GM]}  
	 No 
	  } 
	   } 
	   } 
	 & & \\ 
 

    \theutterance \stepcounter{utterance}  

    & & & \multicolumn{2}{p{0.3\linewidth}}{\cellcolor[rgb]{0.95,0.95,0.95}{%
	\makecell[{{p{\linewidth}}}]{% 
	  \tt {\tiny [GM$|$GM]}  
	 no 
	  } 
	   } 
	   } 
	 & & \\ 
 

    \theutterance \stepcounter{utterance}  

    & & & \multicolumn{2}{p{0.3\linewidth}}{\cellcolor[rgb]{0.95,0.95,0.95}{%
	\makecell[{{p{\linewidth}}}]{% 
	  \tt {\tiny [GM$|$GM]}  
	 game_result = WIN 
	  } 
	   } 
	   } 
	 & & \\ 
 

\end{supertabular}
}

\end{document}
