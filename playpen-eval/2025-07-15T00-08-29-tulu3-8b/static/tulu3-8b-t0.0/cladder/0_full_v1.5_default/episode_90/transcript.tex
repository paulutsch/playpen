
\documentclass{article}
\usepackage{colortbl}
\usepackage{makecell}
\usepackage{multirow}
\usepackage{supertabular}

\begin{document}

\newcounter{utterance}

\twocolumn

{ \footnotesize  \setcounter{utterance}{1}
\setlength{\tabcolsep}{0pt}
\begin{supertabular}{c@{$\;$}|p{.15\linewidth}@{}p{.15\linewidth}p{.15\linewidth}p{.15\linewidth}p{.15\linewidth}p{.15\linewidth}}

    \# & $\;$A & \multicolumn{4}{c}{Game Master} & $\;\:$B\\
    \hline 

    \theutterance \stepcounter{utterance}  

    & & \multicolumn{4}{p{0.6\linewidth}}{\cellcolor[rgb]{0.9,0.9,0.9}{%
	\makecell[{{p{\linewidth}}}]{% 
	  \tt {\tiny [A$\langle$GM]}  
	 You are an expert in causal inference. The following question is not a typical commonsense query, but rather a meticulously designed question created by a professor specializing in causal inference, intended to assess the students' mastery of the course content.\\ \tt Imagine a self-contained, hypothetical world with only the following conditions, and without any unmentioned factors or causal relationships: Pexu has a direct effect on rukz. Hwax has a direct effect on rukz. For those who are not pexu, the probability of rukz is 9%. For those who are pexu, the probability of rukz is 18%. For those who are pexu, would it be less likely to see rukz if the individual was not pexu?\\ \tt Start your answer with 'Yes' or 'No', followed by additional reasoning or evidence to support your explanation. 
	  } 
	   } 
	   } 
	 & \\ 
 

    \theutterance \stepcounter{utterance}  

    & \multicolumn{4}{p{0.6\linewidth}}{\cellcolor[rgb]{0.8,1,0.9}{%
	\makecell[{{p{\linewidth}}}]{% 
	  \tt {\tiny [A$\rangle$GM]}  
	 No, it would not be less likely to see rukz if an individual who is pexu were not pexu. \\ \tt \\ \tt The question is asking whether the probability of rukz would decrease if we were to condition on an individual not being pexu, given that they are currently pexu. However, the provided information does not support this claim.\\ \tt \\ \tt The given data states:\\ \tt - Probability of rukz for non-pexu individuals (not pexu) is 9%.\\ \tt - Probability of rukz for pexu individuals (pexu) is 18%.\\ \tt \\ \tt When we condition on an individual being pexu, we are already considering the group for which rukz is more likely (18% chance compared to 9% for non-pexu). The presence of the pexu status already accounts for a higher likelihood of rukz. \\ \tt \\ \tt To clarify, the question seems to be asking whether the probability of rukz would drop from 18% to something less than 9% if we were to condition on the individual not being pexu, given that they are currently pexu. However, this is not a valid comparison based on the provided data. \\ \tt \\ \tt The correct interpretation of the data is that pexu status increases the likelihood of rukz from 9% to 18%. Therefore, if we were to condition on an individual not being pexu (i.e., considering only those who are not pexu), the probability of 
	  } 
	   } 
	   } 
	 & & \\ 
 

    \theutterance \stepcounter{utterance}  

    & & & \multicolumn{2}{p{0.3\linewidth}}{\cellcolor[rgb]{0.95,0.95,0.95}{%
	\makecell[{{p{\linewidth}}}]{% 
	  \tt {\tiny [GM$|$GM]}  
	 No 
	  } 
	   } 
	   } 
	 & & \\ 
 

    \theutterance \stepcounter{utterance}  

    & & & \multicolumn{2}{p{0.3\linewidth}}{\cellcolor[rgb]{0.95,0.95,0.95}{%
	\makecell[{{p{\linewidth}}}]{% 
	  \tt {\tiny [GM$|$GM]}  
	 yes 
	  } 
	   } 
	   } 
	 & & \\ 
 

    \theutterance \stepcounter{utterance}  

    & & & \multicolumn{2}{p{0.3\linewidth}}{\cellcolor[rgb]{0.95,0.95,0.95}{%
	\makecell[{{p{\linewidth}}}]{% 
	  \tt {\tiny [GM$|$GM]}  
	 game_result = LOSE 
	  } 
	   } 
	   } 
	 & & \\ 
 

\end{supertabular}
}

\end{document}
