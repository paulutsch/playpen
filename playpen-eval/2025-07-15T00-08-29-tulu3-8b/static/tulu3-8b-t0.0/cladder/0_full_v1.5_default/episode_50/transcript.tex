
\documentclass{article}
\usepackage{colortbl}
\usepackage{makecell}
\usepackage{multirow}
\usepackage{supertabular}

\begin{document}

\newcounter{utterance}

\twocolumn

{ \footnotesize  \setcounter{utterance}{1}
\setlength{\tabcolsep}{0pt}
\begin{supertabular}{c@{$\;$}|p{.15\linewidth}@{}p{.15\linewidth}p{.15\linewidth}p{.15\linewidth}p{.15\linewidth}p{.15\linewidth}}

    \# & $\;$A & \multicolumn{4}{c}{Game Master} & $\;\:$B\\
    \hline 

    \theutterance \stepcounter{utterance}  

    & & \multicolumn{4}{p{0.6\linewidth}}{\cellcolor[rgb]{0.9,0.9,0.9}{%
	\makecell[{{p{\linewidth}}}]{% 
	  \tt {\tiny [A$\langle$GM]}  
	 You are an expert in causal inference. The following question is not a typical commonsense query, but rather a meticulously designed question created by a professor specializing in causal inference, intended to assess the students' mastery of the course content.\\ \tt Imagine a self-contained, hypothetical world with only the following conditions, and without any unmentioned factors or causal relationships: Having a brother has a direct effect on wife and alarm clock. Wife has a direct effect on alarm clock. The overall probability of having a brother is 1%. The probability of not having a brother and ringing alarm is 64%. The probability of having a brother and ringing alarm is 0%. Is the chance of ringing alarm larger when observing having a brother?\\ \tt Start your answer with 'Yes' or 'No', followed by additional reasoning or evidence to support your explanation. 
	  } 
	   } 
	   } 
	 & \\ 
 

    \theutterance \stepcounter{utterance}  

    & \multicolumn{4}{p{0.6\linewidth}}{\cellcolor[rgb]{0.8,1,0.9}{%
	\makecell[{{p{\linewidth}}}]{% 
	  \tt {\tiny [A$\rangle$GM]}  
	 No, the chance of ringing an alarm is not larger when observing having a brother. \\ \tt \\ \tt To analyze this, we start by defining the probabilities based on the given conditions:\\ \tt \\ \tt - Probability of having a brother (B) = 1% = 0.01\\ \tt - Probability of not having a brother (¬B) = 99% = 0.99\\ \tt - Probability of ringing an alarm given not having a brother (P(Alarm | ¬B)) = 64% = 0.64\\ \tt \\ \tt Given that the probability of having a brother and ringing an alarm (P(Alarm & B)) is 0%, we can infer that:\\ \tt \\ \tt - P(Alarm | B) = 0% = 0\\ \tt \\ \tt This means that the probability of ringing an alarm given that someone has a brother is 0%. In contrast, the probability of ringing an alarm given that someone does not have a brother is 64%.\\ \tt \\ \tt Therefore, the chance of ringing an alarm is significantly larger when observing that someone does not have a brother (0.64) compared to when observing that someone has a brother (0).\\ \tt \\ \tt The key point here is that the direct effect of having a brother on the alarm clock is null (P(Alarm | B) = 0), whereas the direct effect of not having a brother and ringing an alarm is substantial (P(Alarm | ¬B) = 0.64). Thus, observing having a brother does not increase the likelihood of the alarm ringing, and in 
	  } 
	   } 
	   } 
	 & & \\ 
 

    \theutterance \stepcounter{utterance}  

    & & & \multicolumn{2}{p{0.3\linewidth}}{\cellcolor[rgb]{0.95,0.95,0.95}{%
	\makecell[{{p{\linewidth}}}]{% 
	  \tt {\tiny [GM$|$GM]}  
	 No 
	  } 
	   } 
	   } 
	 & & \\ 
 

    \theutterance \stepcounter{utterance}  

    & & & \multicolumn{2}{p{0.3\linewidth}}{\cellcolor[rgb]{0.95,0.95,0.95}{%
	\makecell[{{p{\linewidth}}}]{% 
	  \tt {\tiny [GM$|$GM]}  
	 no 
	  } 
	   } 
	   } 
	 & & \\ 
 

    \theutterance \stepcounter{utterance}  

    & & & \multicolumn{2}{p{0.3\linewidth}}{\cellcolor[rgb]{0.95,0.95,0.95}{%
	\makecell[{{p{\linewidth}}}]{% 
	  \tt {\tiny [GM$|$GM]}  
	 game_result = WIN 
	  } 
	   } 
	   } 
	 & & \\ 
 

\end{supertabular}
}

\end{document}
