
\documentclass{article}
\usepackage{colortbl}
\usepackage{makecell}
\usepackage{multirow}
\usepackage{supertabular}

\begin{document}

\newcounter{utterance}

\twocolumn

{ \footnotesize  \setcounter{utterance}{1}
\setlength{\tabcolsep}{0pt}
\begin{supertabular}{c@{$\;$}|p{.15\linewidth}@{}p{.15\linewidth}p{.15\linewidth}p{.15\linewidth}p{.15\linewidth}p{.15\linewidth}}

    \# & $\;$A & \multicolumn{4}{c}{Game Master} & $\;\:$B\\
    \hline 

    \theutterance \stepcounter{utterance}  

    & & \multicolumn{4}{p{0.6\linewidth}}{\cellcolor[rgb]{0.9,0.9,0.9}{%
	\makecell[{{p{\linewidth}}}]{% 
	  \tt {\tiny [A$\langle$GM]}  
	 You are an expert in causal inference. The following question is not a typical commonsense query, but rather a meticulously designed question created by a professor specializing in causal inference, intended to assess the students' mastery of the course content.\\ \tt Imagine a self-contained, hypothetical world with only the following conditions, and without any unmentioned factors or causal relationships: Gender has a direct effect on department competitiveness and admission status. Residency status has a direct effect on department competitiveness and admission status. Department competitiveness has a direct effect on admission status. Residency status is unobserved. For individuals who are not male, the probability of admission acceptance is 40%. For individuals who are male, the probability of admission acceptance is 32%. Will male gender increase the chance of admission acceptance?\\ \tt Start your answer with 'Yes' or 'No', followed by additional reasoning or evidence to support your explanation. 
	  } 
	   } 
	   } 
	 & \\ 
 

    \theutterance \stepcounter{utterance}  

    & \multicolumn{4}{p{0.6\linewidth}}{\cellcolor[rgb]{0.8,1,0.9}{%
	\makecell[{{p{\linewidth}}}]{% 
	  \tt {\tiny [A$\rangle$GM]}  
	 No, male gender does not increase the chance of admission acceptance based solely on the provided information. \\ \tt \\ \tt The problem states that for individuals who are not male, the probability of admission acceptance is 40%, and for males, it is 32%. At first glance, it might seem that being male decreases the chance of admission acceptance. However, we must consider the direct effects and the causal relationships between the variables as described:\\ \tt \\ \tt 1. **Gender has a direct effect on department competitiveness and admission status.**\\ \tt 2. **Residency status has a direct effect on department competitiveness and admission status.**\\ \tt 3. **Department competitiveness has a direct effect on admission status.**\\ \tt 4. **Residency status is unobserved.**\\ \tt \\ \tt Given these conditions, we know:\\ \tt \\ \tt - Gender directly affects admission status, with males having a lower probability of acceptance (32%) compared to non-males (40%).\\ \tt - Department competitiveness, which is directly affected by both gender and residency status (though residency status is unobserved), in turn affects admission status.\\ \tt \\ \tt The key point here is that the direct effect of gender on admission status is negative (males have a lower acceptance rate). However, the problem does not provide information on how department competitiveness mediates this relationship. \\ \tt \\ \tt To determine whether male gender increases, decreases, or has no effect on the chance of admission acceptance, we would need to know:\\ \tt \\ \tt - How department competitiveness is influenced by gender and residency status.\\ \tt - The extent to which department competitiveness affects the admission decision, 
	  } 
	   } 
	   } 
	 & & \\ 
 

    \theutterance \stepcounter{utterance}  

    & & & \multicolumn{2}{p{0.3\linewidth}}{\cellcolor[rgb]{0.95,0.95,0.95}{%
	\makecell[{{p{\linewidth}}}]{% 
	  \tt {\tiny [GM$|$GM]}  
	 No 
	  } 
	   } 
	   } 
	 & & \\ 
 

    \theutterance \stepcounter{utterance}  

    & & & \multicolumn{2}{p{0.3\linewidth}}{\cellcolor[rgb]{0.95,0.95,0.95}{%
	\makecell[{{p{\linewidth}}}]{% 
	  \tt {\tiny [GM$|$GM]}  
	 no 
	  } 
	   } 
	   } 
	 & & \\ 
 

    \theutterance \stepcounter{utterance}  

    & & & \multicolumn{2}{p{0.3\linewidth}}{\cellcolor[rgb]{0.95,0.95,0.95}{%
	\makecell[{{p{\linewidth}}}]{% 
	  \tt {\tiny [GM$|$GM]}  
	 game_result = WIN 
	  } 
	   } 
	   } 
	 & & \\ 
 

\end{supertabular}
}

\end{document}
