
\documentclass{article}
\usepackage{colortbl}
\usepackage{makecell}
\usepackage{multirow}
\usepackage{supertabular}

\begin{document}

\newcounter{utterance}

\twocolumn

{ \footnotesize  \setcounter{utterance}{1}
\setlength{\tabcolsep}{0pt}
\begin{supertabular}{c@{$\;$}|p{.15\linewidth}@{}p{.15\linewidth}p{.15\linewidth}p{.15\linewidth}p{.15\linewidth}p{.15\linewidth}}

    \# & $\;$A & \multicolumn{4}{c}{Game Master} & $\;\:$B\\
    \hline 

    \theutterance \stepcounter{utterance}  

    & & \multicolumn{4}{p{0.6\linewidth}}{\cellcolor[rgb]{0.9,0.9,0.9}{%
	\makecell[{{p{\linewidth}}}]{% 
	  \tt {\tiny [A$\langle$GM]}  
	 You are an expert in causal inference. The following question is not a typical commonsense query, but rather a meticulously designed question created by a professor specializing in causal inference, intended to assess the students' mastery of the course content.\\ \tt Imagine a self-contained, hypothetical world with only the following conditions, and without any unmentioned factors or causal relationships: Husband has a direct effect on wife and alarm clock. Wife has a direct effect on alarm clock. The overall probability of alarm set by husband is 8%. For husbands that don't set the alarm, the probability of ringing alarm is 75%. For husbands that set the alarm, the probability of ringing alarm is 16%. Is ringing alarm more likely than silent alarm overall?\\ \tt Start your answer with 'Yes' or 'No', followed by additional reasoning or evidence to support your explanation. 
	  } 
	   } 
	   } 
	 & \\ 
 

    \theutterance \stepcounter{utterance}  

    & \multicolumn{4}{p{0.6\linewidth}}{\cellcolor[rgb]{0.8,1,0.9}{%
	\makecell[{{p{\linewidth}}}]{% 
	  \tt {\tiny [A$\rangle$GM]}  
	 No, ringing alarm is not more likely than a silent alarm overall. \\ \tt \\ \tt To determine this, we need to calculate the overall probability of a ringing alarm versus a silent alarm. We'll do this by considering the two groups: husbands who set the alarm and those who do not.\\ \tt \\ \tt **Step 1: Calculate the probability of a ringing alarm for husbands who set the alarm.**\\ \tt \\ \tt Given:\\ \tt - Probability that a husband sets the alarm = 8% = 0.08\\ \tt - Probability of ringing alarm given that the husband sets it = 16% = 0.16\\ \tt \\ \tt The joint probability of a ringing alarm given that a husband sets the alarm is:\\ \tt \[ P(\text{Ringing} \mid \text{Set by Husband}) = 0.08 \times 0.16 = 0.0128 \]\\ \tt \\ \tt **Step 2: Calculate the probability of a ringing alarm for husbands who do not set the alarm.**\\ \tt \\ \tt Given:\\ \tt - Probability that a husband does not set the alarm = 1 - 0.08 = 0.92\\ \tt - Probability of ringing alarm given that the husband does not set it = 75% = 0.75\\ \tt \\ \tt The joint probability of a ringing alarm given that a husband does not set the alarm is:\\ \tt \[ P(\text{Ringing} \mid \text{Not Set by Husband}) = 0.92 \times 0.75 = 0.69 \]\\ \tt \\ \tt **Step 3: Calculate 
	  } 
	   } 
	   } 
	 & & \\ 
 

    \theutterance \stepcounter{utterance}  

    & & & \multicolumn{2}{p{0.3\linewidth}}{\cellcolor[rgb]{0.95,0.95,0.95}{%
	\makecell[{{p{\linewidth}}}]{% 
	  \tt {\tiny [GM$|$GM]}  
	 No 
	  } 
	   } 
	   } 
	 & & \\ 
 

    \theutterance \stepcounter{utterance}  

    & & & \multicolumn{2}{p{0.3\linewidth}}{\cellcolor[rgb]{0.95,0.95,0.95}{%
	\makecell[{{p{\linewidth}}}]{% 
	  \tt {\tiny [GM$|$GM]}  
	 yes 
	  } 
	   } 
	   } 
	 & & \\ 
 

    \theutterance \stepcounter{utterance}  

    & & & \multicolumn{2}{p{0.3\linewidth}}{\cellcolor[rgb]{0.95,0.95,0.95}{%
	\makecell[{{p{\linewidth}}}]{% 
	  \tt {\tiny [GM$|$GM]}  
	 game_result = LOSE 
	  } 
	   } 
	   } 
	 & & \\ 
 

\end{supertabular}
}

\end{document}
