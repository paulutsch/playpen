
\documentclass{article}
\usepackage{colortbl}
\usepackage{makecell}
\usepackage{multirow}
\usepackage{supertabular}

\begin{document}

\newcounter{utterance}

\twocolumn

{ \footnotesize  \setcounter{utterance}{1}
\setlength{\tabcolsep}{0pt}
\begin{supertabular}{c@{$\;$}|p{.15\linewidth}@{}p{.15\linewidth}p{.15\linewidth}p{.15\linewidth}p{.15\linewidth}p{.15\linewidth}}

    \# & $\;$A & \multicolumn{4}{c}{Game Master} & $\;\:$B\\
    \hline 

    \theutterance \stepcounter{utterance}  

    & & \multicolumn{4}{p{0.6\linewidth}}{\cellcolor[rgb]{0.9,0.9,0.9}{%
	\makecell[{{p{\linewidth}}}]{% 
	  \tt {\tiny [A$\langle$GM]}  
	 You are an expert in causal inference. The following question is not a typical commonsense query, but rather a meticulously designed question created by a professor specializing in causal inference, intended to assess the students' mastery of the course content.\\ \tt Imagine a self-contained, hypothetical world with only the following conditions, and without any unmentioned factors or causal relationships: Alice waking up has a direct effect on Alice arriving to school. Traffic has a direct effect on Alice arriving to school. For days when Alice wakes up on time, the probability of arriving to school on time is 46%. For days when Alice wakes up late, the probability of arriving to school on time is 78%. Will waking up late increase the chance of arriving to school on time?\\ \tt Start your answer with 'Yes' or 'No', followed by additional reasoning or evidence to support your explanation. 
	  } 
	   } 
	   } 
	 & \\ 
 

    \theutterance \stepcounter{utterance}  

    & \multicolumn{4}{p{0.6\linewidth}}{\cellcolor[rgb]{0.8,1,0.9}{%
	\makecell[{{p{\linewidth}}}]{% 
	  \tt {\tiny [A$\rangle$GM]}  
	 No, waking up late does not necessarily increase the chance of arriving to school on time. While it is true that the probability of arriving on time is higher (78%) when Alice wakes up late compared to when she wakes up on time (46%), this comparison alone does not establish a causal relationship.\\ \tt \\ \tt In causal inference, we look for evidence that the change in the "treatment" (waking up late) is the direct cause of the change in the "outcome" (arriving on time). Here, we have only observed associations between these two variables, not a controlled experiment where we randomly assign waking up times and measure the effect on school arrival times.\\ \tt \\ \tt There could be confounding variables at play that we haven't accounted for in this hypothetical scenario. For instance, Alice might have different behaviors or strategies when she wakes up late that could contribute to her higher probability of arriving on time. These could include:\\ \tt \\ \tt 1. **Prior Planning**: Alice might plan her schedule better when she knows she has less time, leading to more efficient use of her time.\\ \tt 2. **Motivation**: Being late might serve as a stronger motivator for Alice to ensure she arrives on time.\\ \tt 3. **External Factors**: On days when Alice wakes up late, there might be fewer traffic delays or other unforeseen circumstances that make it easier for her to arrive on time.\\ \tt \\ \tt Without controlling for these potential confounders, we cannot definitively say that waking up late causes Alice to arrive on time. The observed association 
	  } 
	   } 
	   } 
	 & & \\ 
 

    \theutterance \stepcounter{utterance}  

    & & & \multicolumn{2}{p{0.3\linewidth}}{\cellcolor[rgb]{0.95,0.95,0.95}{%
	\makecell[{{p{\linewidth}}}]{% 
	  \tt {\tiny [GM$|$GM]}  
	 No 
	  } 
	   } 
	   } 
	 & & \\ 
 

    \theutterance \stepcounter{utterance}  

    & & & \multicolumn{2}{p{0.3\linewidth}}{\cellcolor[rgb]{0.95,0.95,0.95}{%
	\makecell[{{p{\linewidth}}}]{% 
	  \tt {\tiny [GM$|$GM]}  
	 yes 
	  } 
	   } 
	   } 
	 & & \\ 
 

    \theutterance \stepcounter{utterance}  

    & & & \multicolumn{2}{p{0.3\linewidth}}{\cellcolor[rgb]{0.95,0.95,0.95}{%
	\makecell[{{p{\linewidth}}}]{% 
	  \tt {\tiny [GM$|$GM]}  
	 game_result = LOSE 
	  } 
	   } 
	   } 
	 & & \\ 
 

\end{supertabular}
}

\end{document}
