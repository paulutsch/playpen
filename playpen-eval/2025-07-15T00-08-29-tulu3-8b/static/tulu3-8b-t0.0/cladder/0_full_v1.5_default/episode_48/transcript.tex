
\documentclass{article}
\usepackage{colortbl}
\usepackage{makecell}
\usepackage{multirow}
\usepackage{supertabular}

\begin{document}

\newcounter{utterance}

\twocolumn

{ \footnotesize  \setcounter{utterance}{1}
\setlength{\tabcolsep}{0pt}
\begin{supertabular}{c@{$\;$}|p{.15\linewidth}@{}p{.15\linewidth}p{.15\linewidth}p{.15\linewidth}p{.15\linewidth}p{.15\linewidth}}

    \# & $\;$A & \multicolumn{4}{c}{Game Master} & $\;\:$B\\
    \hline 

    \theutterance \stepcounter{utterance}  

    & & \multicolumn{4}{p{0.6\linewidth}}{\cellcolor[rgb]{0.9,0.9,0.9}{%
	\makecell[{{p{\linewidth}}}]{% 
	  \tt {\tiny [A$\langle$GM]}  
	 You are an expert in causal inference. The following question is not a typical commonsense query, but rather a meticulously designed question created by a professor specializing in causal inference, intended to assess the students' mastery of the course content.\\ \tt Imagine a self-contained, hypothetical world with only the following conditions, and without any unmentioned factors or causal relationships: Listening to jazz has a direct effect on tar deposit. Tar deposit has a direct effect on lung cancer. For people who do not listen to jazz, the probability of lung cancer is 61%. For people who listen to jazz, the probability of lung cancer is 79%. For people who listen to jazz, would it be less likely to see lung cancer if the person did not listen to jazz?\\ \tt Start your answer with 'Yes' or 'No', followed by additional reasoning or evidence to support your explanation. 
	  } 
	   } 
	   } 
	 & \\ 
 

    \theutterance \stepcounter{utterance}  

    & \multicolumn{4}{p{0.6\linewidth}}{\cellcolor[rgb]{0.8,1,0.9}{%
	\makecell[{{p{\linewidth}}}]{% 
	  \tt {\tiny [A$\rangle$GM]}  
	 No, it would not be less likely to see lung cancer in people who listen to jazz if they did not listen to jazz. \\ \tt \\ \tt The given information suggests a causal relationship where listening to jazz increases the probability of lung cancer. Specifically, the probability of lung cancer is 61% for those who do not listen to jazz and 79% for those who do listen to jazz. This indicates that listening to jazz is associated with a higher risk of lung cancer.\\ \tt \\ \tt To answer the question directly: No, if a person who listens to jazz stopped listening to jazz, their risk of lung cancer would not decrease to the baseline level of 61% observed in non-jazz listeners. The reason is that the observed difference in probabilities (79% vs. 61%) suggests that the effect of listening to jazz on lung cancer risk is not fully reversible. \\ \tt \\ \tt This conclusion is based on the assumption that the observed association is causal, which is a common assumption in causal inference but should be treated with caution. In real-world scenarios, unmeasured confounders or other complex causal relationships could exist, potentially affecting this interpretation. However, based solely on the provided information, the direct effect of stopping jazz listening would not be expected to reduce the risk of lung cancer to the level of non-jazz listeners.\\ \tt \\ \tt In causal inference, establishing a causal relationship requires more than just association; it often involves experimental evidence (like randomized controlled trials) or sophisticated statistical methods (like instrumental variables, difference-in-differences, or causal 
	  } 
	   } 
	   } 
	 & & \\ 
 

    \theutterance \stepcounter{utterance}  

    & & & \multicolumn{2}{p{0.3\linewidth}}{\cellcolor[rgb]{0.95,0.95,0.95}{%
	\makecell[{{p{\linewidth}}}]{% 
	  \tt {\tiny [GM$|$GM]}  
	 No 
	  } 
	   } 
	   } 
	 & & \\ 
 

    \theutterance \stepcounter{utterance}  

    & & & \multicolumn{2}{p{0.3\linewidth}}{\cellcolor[rgb]{0.95,0.95,0.95}{%
	\makecell[{{p{\linewidth}}}]{% 
	  \tt {\tiny [GM$|$GM]}  
	 yes 
	  } 
	   } 
	   } 
	 & & \\ 
 

    \theutterance \stepcounter{utterance}  

    & & & \multicolumn{2}{p{0.3\linewidth}}{\cellcolor[rgb]{0.95,0.95,0.95}{%
	\makecell[{{p{\linewidth}}}]{% 
	  \tt {\tiny [GM$|$GM]}  
	 game_result = LOSE 
	  } 
	   } 
	   } 
	 & & \\ 
 

\end{supertabular}
}

\end{document}
