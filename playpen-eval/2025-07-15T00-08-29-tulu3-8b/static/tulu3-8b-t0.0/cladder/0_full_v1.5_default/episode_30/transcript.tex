
\documentclass{article}
\usepackage{colortbl}
\usepackage{makecell}
\usepackage{multirow}
\usepackage{supertabular}

\begin{document}

\newcounter{utterance}

\twocolumn

{ \footnotesize  \setcounter{utterance}{1}
\setlength{\tabcolsep}{0pt}
\begin{supertabular}{c@{$\;$}|p{.15\linewidth}@{}p{.15\linewidth}p{.15\linewidth}p{.15\linewidth}p{.15\linewidth}p{.15\linewidth}}

    \# & $\;$A & \multicolumn{4}{c}{Game Master} & $\;\:$B\\
    \hline 

    \theutterance \stepcounter{utterance}  

    & & \multicolumn{4}{p{0.6\linewidth}}{\cellcolor[rgb]{0.9,0.9,0.9}{%
	\makecell[{{p{\linewidth}}}]{% 
	  \tt {\tiny [A$\langle$GM]}  
	 You are an expert in causal inference. The following question is not a typical commonsense query, but rather a meticulously designed question created by a professor specializing in causal inference, intended to assess the students' mastery of the course content.\\ \tt Imagine a self-contained, hypothetical world with only the following conditions, and without any unmentioned factors or causal relationships: Demand has a direct effect on supply and price. Yield per acre has a direct effect on supply. Supply has a direct effect on price. Demand is unobserved. The overall probability of increased supply is 78%. For situations with reduced supply, the probability of increased price is 44%. For situations with increased supply, the probability of increased price is 81%. Is increased price less likely than reduced price overall?\\ \tt Start your answer with 'Yes' or 'No', followed by additional reasoning or evidence to support your explanation. 
	  } 
	   } 
	   } 
	 & \\ 
 

    \theutterance \stepcounter{utterance}  

    & \multicolumn{4}{p{0.6\linewidth}}{\cellcolor[rgb]{0.8,1,0.9}{%
	\makecell[{{p{\linewidth}}}]{% 
	  \tt {\tiny [A$\rangle$GM]}  
	 No, increased price is not less likely than reduced price overall. \\ \tt \\ \tt To understand this, we need to consider the probabilities given in the problem:\\ \tt \\ \tt 1. The overall probability of increased supply is 78%.\\ \tt 2. For situations with reduced supply, the probability of increased price is 44%.\\ \tt 3. For situations with increased supply, the probability of increased price is 81%.\\ \tt \\ \tt To find the overall probability of an increased price, we need to consider both scenarios: increased supply and reduced supply.\\ \tt \\ \tt **Step 1: Calculate the probability of increased price given increased supply.**\\ \tt \\ \tt The probability of increased price given increased supply is 81%.\\ \tt \\ \tt **Step 2: Calculate the probability of reduced supply.**\\ \tt \\ \tt The probability of reduced supply is 1 (certainty) minus the probability of increased supply, which is 78%, or 22%.\\ \tt \\ \tt **Step 3: Calculate the probability of increased price given reduced supply.**\\ \tt \\ \tt The probability of increased price given reduced supply is 44%.\\ \tt \\ \tt **Step 4: Calculate the overall probability of increased price.**\\ \tt \\ \tt The overall probability of increased price (P(increased price)) can be calculated using the law of total probability:\\ \tt \\ \tt P(increased price) = P(increased price | increased supply) * P(increased supply) + P(increased price | reduced supply) * P(reduced supply)\\ \tt \\ \tt Substituting the given probabilities:\\ \tt \\ \tt P(increased price) = (81% * 78%) + (44% * 22%) 
	  } 
	   } 
	   } 
	 & & \\ 
 

    \theutterance \stepcounter{utterance}  

    & & & \multicolumn{2}{p{0.3\linewidth}}{\cellcolor[rgb]{0.95,0.95,0.95}{%
	\makecell[{{p{\linewidth}}}]{% 
	  \tt {\tiny [GM$|$GM]}  
	 No 
	  } 
	   } 
	   } 
	 & & \\ 
 

    \theutterance \stepcounter{utterance}  

    & & & \multicolumn{2}{p{0.3\linewidth}}{\cellcolor[rgb]{0.95,0.95,0.95}{%
	\makecell[{{p{\linewidth}}}]{% 
	  \tt {\tiny [GM$|$GM]}  
	 no 
	  } 
	   } 
	   } 
	 & & \\ 
 

    \theutterance \stepcounter{utterance}  

    & & & \multicolumn{2}{p{0.3\linewidth}}{\cellcolor[rgb]{0.95,0.95,0.95}{%
	\makecell[{{p{\linewidth}}}]{% 
	  \tt {\tiny [GM$|$GM]}  
	 game_result = WIN 
	  } 
	   } 
	   } 
	 & & \\ 
 

\end{supertabular}
}

\end{document}
