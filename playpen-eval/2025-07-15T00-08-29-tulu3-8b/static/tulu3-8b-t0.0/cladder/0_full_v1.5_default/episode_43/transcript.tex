
\documentclass{article}
\usepackage{colortbl}
\usepackage{makecell}
\usepackage{multirow}
\usepackage{supertabular}

\begin{document}

\newcounter{utterance}

\twocolumn

{ \footnotesize  \setcounter{utterance}{1}
\setlength{\tabcolsep}{0pt}
\begin{supertabular}{c@{$\;$}|p{.15\linewidth}@{}p{.15\linewidth}p{.15\linewidth}p{.15\linewidth}p{.15\linewidth}p{.15\linewidth}}

    \# & $\;$A & \multicolumn{4}{c}{Game Master} & $\;\:$B\\
    \hline 

    \theutterance \stepcounter{utterance}  

    & & \multicolumn{4}{p{0.6\linewidth}}{\cellcolor[rgb]{0.9,0.9,0.9}{%
	\makecell[{{p{\linewidth}}}]{% 
	  \tt {\tiny [A$\langle$GM]}  
	 You are an expert in causal inference. The following question is not a typical commonsense query, but rather a meticulously designed question created by a professor specializing in causal inference, intended to assess the students' mastery of the course content.\\ \tt Imagine a self-contained, hypothetical world with only the following conditions, and without any unmentioned factors or causal relationships: Ability to speak english has a direct effect on diabetes and lifespan. Smoking has a direct effect on diabetes and lifespan. Diabetes has a direct effect on lifespan. Smoking is unobserved. For people who do not speak english, the probability of long lifespan is 77%. For people who speak english, the probability of long lifespan is 45%. Will speaking english decrease the chance of long lifespan?\\ \tt Start your answer with 'Yes' or 'No', followed by additional reasoning or evidence to support your explanation. 
	  } 
	   } 
	   } 
	 & \\ 
 

    \theutterance \stepcounter{utterance}  

    & \multicolumn{4}{p{0.6\linewidth}}{\cellcolor[rgb]{0.8,1,0.9}{%
	\makecell[{{p{\linewidth}}}]{% 
	  \tt {\tiny [A$\rangle$GM]}  
	 No, speaking English does not decrease the chance of a long lifespan based on the given information alone. \\ \tt \\ \tt The problem states that speaking English has a direct effect on diabetes and lifespan, but it does not specify the nature of this effect. It could be that speaking English is associated with certain behaviors or lifestyles that indirectly affect lifespan, such as access to healthcare, diet, or socioeconomic status, which are not explicitly mentioned.\\ \tt \\ \tt Here's the breakdown of the causal relationships:\\ \tt \\ \tt 1. Smoking has a direct effect on diabetes and lifespan. Smoking is unobserved, meaning we do not have direct information about smoking habits in this hypothetical world.\\ \tt \\ \tt 2. Diabetes has a direct effect on lifespan. This is a well-established fact; diabetes can lead to various complications that shorten lifespan.\\ \tt \\ \tt 3. For people who do not speak English, the probability of a long lifespan is 77%. \\ \tt 4. For people who speak English, the probability of a long lifespan is 45%.\\ \tt \\ \tt Given these points, we cannot conclude that speaking English directly decreases the chance of a long lifespan. Here are a few reasons:\\ \tt \\ \tt - The lower probability of a long lifespan among English speakers could be due to confounding factors. For instance, if English speakers are more likely to smoke (an unobserved variable), this could explain the lower probability of a long lifespan, rather than speaking English itself.\\ \tt \\ \tt - The effect of speaking English might be mediated by other variables. For example, English speakers might have less access to health information or support, which could 
	  } 
	   } 
	   } 
	 & & \\ 
 

    \theutterance \stepcounter{utterance}  

    & & & \multicolumn{2}{p{0.3\linewidth}}{\cellcolor[rgb]{0.95,0.95,0.95}{%
	\makecell[{{p{\linewidth}}}]{% 
	  \tt {\tiny [GM$|$GM]}  
	 No 
	  } 
	   } 
	   } 
	 & & \\ 
 

    \theutterance \stepcounter{utterance}  

    & & & \multicolumn{2}{p{0.3\linewidth}}{\cellcolor[rgb]{0.95,0.95,0.95}{%
	\makecell[{{p{\linewidth}}}]{% 
	  \tt {\tiny [GM$|$GM]}  
	 yes 
	  } 
	   } 
	   } 
	 & & \\ 
 

    \theutterance \stepcounter{utterance}  

    & & & \multicolumn{2}{p{0.3\linewidth}}{\cellcolor[rgb]{0.95,0.95,0.95}{%
	\makecell[{{p{\linewidth}}}]{% 
	  \tt {\tiny [GM$|$GM]}  
	 game_result = LOSE 
	  } 
	   } 
	   } 
	 & & \\ 
 

\end{supertabular}
}

\end{document}
