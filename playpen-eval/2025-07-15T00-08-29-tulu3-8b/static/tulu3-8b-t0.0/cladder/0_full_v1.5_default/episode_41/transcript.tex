
\documentclass{article}
\usepackage{colortbl}
\usepackage{makecell}
\usepackage{multirow}
\usepackage{supertabular}

\begin{document}

\newcounter{utterance}

\twocolumn

{ \footnotesize  \setcounter{utterance}{1}
\setlength{\tabcolsep}{0pt}
\begin{supertabular}{c@{$\;$}|p{.15\linewidth}@{}p{.15\linewidth}p{.15\linewidth}p{.15\linewidth}p{.15\linewidth}p{.15\linewidth}}

    \# & $\;$A & \multicolumn{4}{c}{Game Master} & $\;\:$B\\
    \hline 

    \theutterance \stepcounter{utterance}  

    & & \multicolumn{4}{p{0.6\linewidth}}{\cellcolor[rgb]{0.9,0.9,0.9}{%
	\makecell[{{p{\linewidth}}}]{% 
	  \tt {\tiny [A$\langle$GM]}  
	 You are an expert in causal inference. The following question is not a typical commonsense query, but rather a meticulously designed question created by a professor specializing in causal inference, intended to assess the students' mastery of the course content.\\ \tt Imagine a self-contained, hypothetical world with only the following conditions, and without any unmentioned factors or causal relationships: Education level has a direct effect on skill. Skill has a direct effect on black hair. The overall probability of college degree or higher is 6%. The probability of high school degree or lower and black hair is 25%. The probability of college degree or higher and black hair is 3%. Is the chance of black hair smaller when observing college degree or higher?\\ \tt Start your answer with 'Yes' or 'No', followed by additional reasoning or evidence to support your explanation. 
	  } 
	   } 
	   } 
	 & \\ 
 

    \theutterance \stepcounter{utterance}  

    & \multicolumn{4}{p{0.6\linewidth}}{\cellcolor[rgb]{0.8,1,0.9}{%
	\makecell[{{p{\linewidth}}}]{% 
	  \tt {\tiny [A$\rangle$GM]}  
	 No, the chance of black hair is not smaller when observing a college degree or higher. \\ \tt \\ \tt To understand this, we need to use the principle of conditional probability. We are given the following probabilities:\\ \tt \\ \tt - P(College) = 0.06 (Probability of having a college degree or higher)\\ \tt - P(Black Hair | High School or Lower) = 0.25 (Probability of having black hair given the person has a high school degree or lower)\\ \tt - P(Black Hair | College) = 0.03 (Probability of having black hair given the person has a college degree or higher)\\ \tt \\ \tt We want to determine if P(Black Hair | College) < P(Black Hair | High School or Lower).\\ \tt \\ \tt Using Bayes' theorem, we can express these conditional probabilities as:\\ \tt \\ \tt P(Black Hair | College) = P(Black Hair ∩ College) / P(College)\\ \tt \\ \tt We know P(Black Hair ∩ College) = 0.03 (from the given data) and P(College) = 0.06.\\ \tt \\ \tt So, P(Black Hair | College) = 0.03 / 0.06 = 0.5.\\ \tt \\ \tt Now, let's consider P(Black Hair | High School or Lower):\\ \tt \\ \tt P(Black Hair | High School or Lower) = P(Black Hair ∩ High School or Lower) / P(High School or Lower)\\ \tt \\ \tt We know P(Black Hair ∩ High School or Lower) = 
	  } 
	   } 
	   } 
	 & & \\ 
 

    \theutterance \stepcounter{utterance}  

    & & & \multicolumn{2}{p{0.3\linewidth}}{\cellcolor[rgb]{0.95,0.95,0.95}{%
	\makecell[{{p{\linewidth}}}]{% 
	  \tt {\tiny [GM$|$GM]}  
	 No 
	  } 
	   } 
	   } 
	 & & \\ 
 

    \theutterance \stepcounter{utterance}  

    & & & \multicolumn{2}{p{0.3\linewidth}}{\cellcolor[rgb]{0.95,0.95,0.95}{%
	\makecell[{{p{\linewidth}}}]{% 
	  \tt {\tiny [GM$|$GM]}  
	 no 
	  } 
	   } 
	   } 
	 & & \\ 
 

    \theutterance \stepcounter{utterance}  

    & & & \multicolumn{2}{p{0.3\linewidth}}{\cellcolor[rgb]{0.95,0.95,0.95}{%
	\makecell[{{p{\linewidth}}}]{% 
	  \tt {\tiny [GM$|$GM]}  
	 game_result = WIN 
	  } 
	   } 
	   } 
	 & & \\ 
 

\end{supertabular}
}

\end{document}
