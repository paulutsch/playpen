
\documentclass{article}
\usepackage{colortbl}
\usepackage{makecell}
\usepackage{multirow}
\usepackage{supertabular}

\begin{document}

\newcounter{utterance}

\twocolumn

{ \footnotesize  \setcounter{utterance}{1}
\setlength{\tabcolsep}{0pt}
\begin{supertabular}{c@{$\;$}|p{.15\linewidth}@{}p{.15\linewidth}p{.15\linewidth}p{.15\linewidth}p{.15\linewidth}p{.15\linewidth}}

    \# & $\;$A & \multicolumn{4}{c}{Game Master} & $\;\:$B\\
    \hline 

    \theutterance \stepcounter{utterance}  

    & & \multicolumn{4}{p{0.6\linewidth}}{\cellcolor[rgb]{0.9,0.9,0.9}{%
	\makecell[{{p{\linewidth}}}]{% 
	  \tt {\tiny [A$\langle$GM]}  
	 You are an expert in causal inference. The following question is not a typical commonsense query, but rather a meticulously designed question created by a professor specializing in causal inference, intended to assess the students' mastery of the course content.\\ \tt Imagine a self-contained, hypothetical world with only the following conditions, and without any unmentioned factors or causal relationships: Jyka has a direct effect on zory and zuph. Zuph has a direct effect on glimx. Zory has a direct effect on glimx. The overall probability of zuph is 34%. The probability of not zuph and glimx is 64%. The probability of zuph and glimx is 28%. Is the chance of glimx smaller when observing zuph?\\ \tt Start your answer with 'Yes' or 'No', followed by additional reasoning or evidence to support your explanation. 
	  } 
	   } 
	   } 
	 & \\ 
 

    \theutterance \stepcounter{utterance}  

    & \multicolumn{4}{p{0.6\linewidth}}{\cellcolor[rgb]{0.8,1,0.9}{%
	\makecell[{{p{\linewidth}}}]{% 
	  \tt {\tiny [A$\rangle$GM]}  
	 No, the chance of glimx is not smaller when observing zuph. \\ \tt \\ \tt To understand this, we need to use the principle of conditional probability. We are given the following information:\\ \tt \\ \tt - P(Zuph) = 34% = 0.34\\ \tt - P(Not Zuph and Glimx) = 64% = 0.64\\ \tt - P(Zuph and Glimx) = 28% = 0.28\\ \tt \\ \tt We want to determine if P(Glimx | Zuph) (the probability of Glimx given Zuph) is smaller than P(Glimx).\\ \tt \\ \tt First, calculate P(Glimx) using the law of total probability:\\ \tt \\ \tt P(Glimx) = P(Glimx and Zuph) + P(Glimx and Not Zuph)\\ \tt          = P(Zuph and Glimx) + P(Not Zuph and Glimx)\\ \tt          = 0.28 + 0.64\\ \tt          = 0.92\\ \tt \\ \tt Next, calculate P(Glimx | Zuph) using the definition of conditional probability:\\ \tt \\ \tt P(Glimx | Zuph) = P(Glimx and Zuph) / P(Zuph)\\ \tt                 = 0.28 / 0.34\\ \tt                 ≈ 0.8235\\ \tt \\ \tt Now, compare P(Glimx) and P(Glimx | Zuph):\\ \tt \\ \tt - P(Glim 
	  } 
	   } 
	   } 
	 & & \\ 
 

    \theutterance \stepcounter{utterance}  

    & & & \multicolumn{2}{p{0.3\linewidth}}{\cellcolor[rgb]{0.95,0.95,0.95}{%
	\makecell[{{p{\linewidth}}}]{% 
	  \tt {\tiny [GM$|$GM]}  
	 No 
	  } 
	   } 
	   } 
	 & & \\ 
 

    \theutterance \stepcounter{utterance}  

    & & & \multicolumn{2}{p{0.3\linewidth}}{\cellcolor[rgb]{0.95,0.95,0.95}{%
	\makecell[{{p{\linewidth}}}]{% 
	  \tt {\tiny [GM$|$GM]}  
	 yes 
	  } 
	   } 
	   } 
	 & & \\ 
 

    \theutterance \stepcounter{utterance}  

    & & & \multicolumn{2}{p{0.3\linewidth}}{\cellcolor[rgb]{0.95,0.95,0.95}{%
	\makecell[{{p{\linewidth}}}]{% 
	  \tt {\tiny [GM$|$GM]}  
	 game_result = LOSE 
	  } 
	   } 
	   } 
	 & & \\ 
 

\end{supertabular}
}

\end{document}
