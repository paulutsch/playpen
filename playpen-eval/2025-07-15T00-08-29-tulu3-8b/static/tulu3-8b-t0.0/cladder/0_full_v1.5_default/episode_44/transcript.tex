
\documentclass{article}
\usepackage{colortbl}
\usepackage{makecell}
\usepackage{multirow}
\usepackage{supertabular}

\begin{document}

\newcounter{utterance}

\twocolumn

{ \footnotesize  \setcounter{utterance}{1}
\setlength{\tabcolsep}{0pt}
\begin{supertabular}{c@{$\;$}|p{.15\linewidth}@{}p{.15\linewidth}p{.15\linewidth}p{.15\linewidth}p{.15\linewidth}p{.15\linewidth}}

    \# & $\;$A & \multicolumn{4}{c}{Game Master} & $\;\:$B\\
    \hline 

    \theutterance \stepcounter{utterance}  

    & & \multicolumn{4}{p{0.6\linewidth}}{\cellcolor[rgb]{0.9,0.9,0.9}{%
	\makecell[{{p{\linewidth}}}]{% 
	  \tt {\tiny [A$\langle$GM]}  
	 You are an expert in causal inference. The following question is not a typical commonsense query, but rather a meticulously designed question created by a professor specializing in causal inference, intended to assess the students' mastery of the course content.\\ \tt Imagine a self-contained, hypothetical world with only the following conditions, and without any unmentioned factors or causal relationships: Ability to speak english has a direct effect on diabetes and lifespan. Smoking has a direct effect on diabetes and lifespan. Diabetes has a direct effect on lifespan. Smoking is unobserved. For people who do not speak english, the probability of long lifespan is 70%. For people who speak english, the probability of long lifespan is 38%. For people who speak english, would it be less likely to see long lifespan if the person did not speak english?\\ \tt Start your answer with 'Yes' or 'No', followed by additional reasoning or evidence to support your explanation. 
	  } 
	   } 
	   } 
	 & \\ 
 

    \theutterance \stepcounter{utterance}  

    & \multicolumn{4}{p{0.6\linewidth}}{\cellcolor[rgb]{0.8,1,0.9}{%
	\makecell[{{p{\linewidth}}}]{% 
	  \tt {\tiny [A$\rangle$GM]}  
	 No, it would not be less likely to see a long lifespan for people who speak English if they did not speak English. \\ \tt \\ \tt The given information suggests a complex interplay between speaking English, smoking, diabetes, and lifespan. However, we need to carefully consider the direct and indirect effects at play:\\ \tt \\ \tt 1. **Direct Effects**: \\ \tt    - Speaking English has a direct effect on diabetes and lifespan.\\ \tt    - Smoking has a direct effect on diabetes and lifespan.\\ \tt    - Diabetes has a direct effect on lifespan.\\ \tt \\ \tt 2. **Observed Probabilities**:\\ \tt    - Probability of long lifespan for those who do not speak English: 70%.\\ \tt    - Probability of long lifespan for those who speak English: 38%.\\ \tt \\ \tt 3. **Unobserved Variable**: \\ \tt    - Smoking is unobserved, which means we do not have direct information about smoking habits for different language groups.\\ \tt \\ \tt Given these conditions, we cannot directly infer that not speaking English would lead to a higher probability of a long lifespan for those who already speak English. Here's why:\\ \tt \\ \tt - **Confounding by Smoking**: The unobserved variable (smoking) could be a significant confounder. People who speak English might have different smoking habits compared to those who do not speak English. If English speakers are more likely to smoke (a possibility given the direct effect of smoking on diabetes and lifespan), then the lower probability of long lifespan among English speakers could be due to higher smoking rates rather than the act of speaking English itself.\\ \tt \\ \tt - ** 
	  } 
	   } 
	   } 
	 & & \\ 
 

    \theutterance \stepcounter{utterance}  

    & & & \multicolumn{2}{p{0.3\linewidth}}{\cellcolor[rgb]{0.95,0.95,0.95}{%
	\makecell[{{p{\linewidth}}}]{% 
	  \tt {\tiny [GM$|$GM]}  
	 No 
	  } 
	   } 
	   } 
	 & & \\ 
 

    \theutterance \stepcounter{utterance}  

    & & & \multicolumn{2}{p{0.3\linewidth}}{\cellcolor[rgb]{0.95,0.95,0.95}{%
	\makecell[{{p{\linewidth}}}]{% 
	  \tt {\tiny [GM$|$GM]}  
	 no 
	  } 
	   } 
	   } 
	 & & \\ 
 

    \theutterance \stepcounter{utterance}  

    & & & \multicolumn{2}{p{0.3\linewidth}}{\cellcolor[rgb]{0.95,0.95,0.95}{%
	\makecell[{{p{\linewidth}}}]{% 
	  \tt {\tiny [GM$|$GM]}  
	 game_result = WIN 
	  } 
	   } 
	   } 
	 & & \\ 
 

\end{supertabular}
}

\end{document}
