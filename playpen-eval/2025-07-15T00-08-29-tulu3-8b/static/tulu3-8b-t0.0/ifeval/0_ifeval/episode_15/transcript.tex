
\documentclass{article}
\usepackage{colortbl}
\usepackage{makecell}
\usepackage{multirow}
\usepackage{supertabular}

\begin{document}

\newcounter{utterance}

\twocolumn

{ \footnotesize  \setcounter{utterance}{1}
\setlength{\tabcolsep}{0pt}
\begin{supertabular}{c@{$\;$}|p{.15\linewidth}@{}p{.15\linewidth}p{.15\linewidth}p{.15\linewidth}p{.15\linewidth}p{.15\linewidth}}

    \# & $\;$A & \multicolumn{4}{c}{Game Master} & $\;\:$B\\
    \hline 

    \theutterance \stepcounter{utterance}  

    & & \multicolumn{4}{p{0.6\linewidth}}{\cellcolor[rgb]{0.9,0.9,0.9}{%
	\makecell[{{p{\linewidth}}}]{% 
	  \tt {\tiny [A$\langle$GM]}  
	 Write an interesting and funny article about the biology of a banana peel. In your response, the word disappointed should appear at least 2 times, and at least six section should be highlighted with markdown,  i.e *banana peel*. 
	  } 
	   } 
	   } 
	 & \\ 
 

    \theutterance \stepcounter{utterance}  

    & \multicolumn{4}{p{0.6\linewidth}}{\cellcolor[rgb]{0.8,1,0.9}{%
	\makecell[{{p{\linewidth}}}]{% 
	  \tt {\tiny [A$\rangle$GM]}  
	 **The Unbelievable World of the *Banana Peel***\\ \tt \\ \tt Banana peels are not just the neglected leftovers of our fruity indulgence; they are the unsung heroes of the botanical world. Let's embark on a whimsical journey to explore the *banana peel*, a surprisingly fascinating subject that might just leave you *disappointed* in your previous disregard for it.\\ \tt \\ \tt **The Origin Story of the *Banana Peel***  \\ \tt Every banana peel has a humble beginning, growing as a protective sheath around the banana itself. Imagine the banana as a VIP, the peel as its bodyguard, ensuring a safe and snug journey from tree to table. However, many people *disappointed* by its tough texture and unappetizing appearance often overlook this natural armor.\\ \tt \\ \tt **The *Banana Peel*'s Nutritional Secrets**  \\ \tt Despite its rough exterior, the *banana peel* harbors a treasure trove of nutrients. Rich in dietary fiber, potassium, and various antioxidants, it's a shame that so many of us *disappointed* by its appearance miss out on these health benefits. Next time you peel a banana, consider saving that peel for a nutritious snack or infusion.\\ \tt \\ \tt **The *Banana Peel* in Popular Culture**  \\ \tt Banana peels have made quite the splash in pop culture, often used in slapstick comedy for their slip-and-slide potential. Who hasn't seen a character *disappointed* by a sudden slip on a banana peel? This cultural icon highlights the peel's practical, if not comedic, utility.\\ \tt \\ \tt **The *Banana Peel* as a Sustainable Resource**  \\ \tt Eco-conscious consumers are increasingly recognizing the *banana peel* as a sustainable resource. It can be composted, used as a natural pesticide, or even turned into biofuel. It's a pity that so many people *disappointed* by its initial use might be missing out on these eco-friendly opportunities.\\ \tt \\ \tt **The *Banana Peel* in the Animal Kingdom**  \\ \tt Banana peels aren't just for humans; they're a favorite snack for many animals. Monkeys, birds, and even some insects find the *banana peel* to be a delightful treat. It's a reminder that even the most *disappointed* human waste products can be a vital part of the ecosystem.\\ \tt \\ \tt **The *Banana Peel* as a DIY Project**  \\ \tt Creative minds have found numerous ways to reuse *banana peels*. From homemade facial scrubs to natural dyes for fabric, the possibilities are endless. It's a shame that many people *disappointed* by the peel's initial presentation might be missing out on these fun and eco-friendly projects.\\ \tt \\ \tt In conclusion, the *banana peel* is more than just a discarded piece of fruit; it's a versatile, nutritious, and even entertaining part of our world. Next time you peel a banana, remember that you might be *disappointed* to throw away a potential ally in health, sustainability, and fun. So, the next time you see a *banana peel*, think twice before discarding it—there's more to this humble sheath than meets the eye!<|end_of_text|> 
	  } 
	   } 
	   } 
	 & & \\ 
 

    \theutterance \stepcounter{utterance}  

    & & & \multicolumn{2}{p{0.3\linewidth}}{\cellcolor[rgb]{0.95,0.95,0.95}{%
	\makecell[{{p{\linewidth}}}]{% 
	  \tt {\tiny [GM$|$GM]}  
	 {'keywords:frequency': {'relation': 'at least', 'keyword': 'disappointed', 'frequency': 2}, 'detectable_format:number_highlighted_sections': {'num_highlights': 6}} 
	  } 
	   } 
	   } 
	 & & \\ 
 

    \theutterance \stepcounter{utterance}  

    & & & \multicolumn{2}{p{0.3\linewidth}}{\cellcolor[rgb]{0.95,0.95,0.95}{%
	\makecell[{{p{\linewidth}}}]{% 
	  \tt {\tiny [GM$|$GM]}  
	 game_result = WIN 
	  } 
	   } 
	   } 
	 & & \\ 
 

\end{supertabular}
}

\end{document}
