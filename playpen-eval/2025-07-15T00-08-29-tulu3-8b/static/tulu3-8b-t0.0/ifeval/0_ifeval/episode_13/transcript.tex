
\documentclass{article}
\usepackage{colortbl}
\usepackage{makecell}
\usepackage{multirow}
\usepackage{supertabular}

\begin{document}

\newcounter{utterance}

\twocolumn

{ \footnotesize  \setcounter{utterance}{1}
\setlength{\tabcolsep}{0pt}
\begin{supertabular}{c@{$\;$}|p{.15\linewidth}@{}p{.15\linewidth}p{.15\linewidth}p{.15\linewidth}p{.15\linewidth}p{.15\linewidth}}

    \# & $\;$A & \multicolumn{4}{c}{Game Master} & $\;\:$B\\
    \hline 

    \theutterance \stepcounter{utterance}  

    & & \multicolumn{4}{p{0.6\linewidth}}{\cellcolor[rgb]{0.9,0.9,0.9}{%
	\makecell[{{p{\linewidth}}}]{% 
	  \tt {\tiny [A$\langle$GM]}  
	 What's a good way to ask Sonia out? Please reply with exactly 4 paragraphs and separate each paragraph with two new lines. Put double quotation marks around your entire response. The very first paragraph must start with the word "weekend". 
	  } 
	   } 
	   } 
	 & \\ 
 

    \theutterance \stepcounter{utterance}  

    & \multicolumn{4}{p{0.6\linewidth}}{\cellcolor[rgb]{0.8,1,0.9}{%
	\makecell[{{p{\linewidth}}}]{% 
	  \tt {\tiny [A$\rangle$GM]}  
	 "Weekend plans are always a great opportunity to connect with someone you're interested in, and Sonia might just appreciate a spontaneous invitation. Consider proposing a casual outing that allows for natural conversation and a relaxed atmosphere. Whether it's a picnic in the park, a walk by the river, or even a movie at a cozy café, the key is to choose an activity that both of you might enjoy. This way, you create a setting where the focus is on the experience rather than the pressure of a formal date.\\ \tt \\ \tt Two new lines\\ \tt \\ \tt When asking Sonia out, it's important to be genuine and straightforward. You could say something like, 'Hey Sonia, I was thinking this weekend might be a great time for a little adventure. How about we go for a picnic at the park? It would be nice to catch up and enjoy the sunshine together.' This approach is direct yet friendly, showing that you value her company and are open to a casual and enjoyable time. Being clear about your intentions helps set the right tone and reduces any potential misunderstandings.\\ \tt \\ \tt Two new lines\\ \tt \\ \tt Preparation is key to making a good impression, so consider a few thoughtful details that could enhance the experience. Perhaps bring a favorite snack or a book you think she might like, showing that you've put some thought into the idea. During the outing, focus on engaging in meaningful conversation, listening actively, and showing genuine interest in what she has to say. This not only makes her feel valued but also helps build a connection. Remember, the goal is to enjoy each other's company and see if there's a spark that could lead to something more.\\ \tt \\ \tt Two new lines\\ \tt \\ \tt Finally, be prepared for any outcome with grace and positivity. Whether she accepts or declines, respond with kindness and appreciation for her time. If she says yes, enjoy the moment and savor the experience, being present and attentive. If she says no, thank her for considering it and assure her that you hope to hang out again in the future. Either way, approach the situation with confidence and a positive attitude, as this will leave a lasting impression. In the end, the most important thing is to be true to yourself and enjoy the process of getting to know someone new."<|end_of_text|> 
	  } 
	   } 
	   } 
	 & & \\ 
 

    \theutterance \stepcounter{utterance}  

    & & & \multicolumn{2}{p{0.3\linewidth}}{\cellcolor[rgb]{0.95,0.95,0.95}{%
	\makecell[{{p{\linewidth}}}]{% 
	  \tt {\tiny [GM$|$GM]}  
	 {'length_constraints:nth_paragraph_first_word': {'num_paragraphs': 4, 'first_word': 'weekend', 'nth_paragraph': 1}, 'startend:quotation': {}} 
	  } 
	   } 
	   } 
	 & & \\ 
 

    \theutterance \stepcounter{utterance}  

    & & & \multicolumn{2}{p{0.3\linewidth}}{\cellcolor[rgb]{0.95,0.95,0.95}{%
	\makecell[{{p{\linewidth}}}]{% 
	  \tt {\tiny [GM$|$GM]}  
	 game_result = LOSE 
	  } 
	   } 
	   } 
	 & & \\ 
 

\end{supertabular}
}

\end{document}
