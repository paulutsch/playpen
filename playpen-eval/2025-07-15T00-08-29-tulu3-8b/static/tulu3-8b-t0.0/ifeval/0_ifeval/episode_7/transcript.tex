
\documentclass{article}
\usepackage{colortbl}
\usepackage{makecell}
\usepackage{multirow}
\usepackage{supertabular}

\begin{document}

\newcounter{utterance}

\twocolumn

{ \footnotesize  \setcounter{utterance}{1}
\setlength{\tabcolsep}{0pt}
\begin{supertabular}{c@{$\;$}|p{.15\linewidth}@{}p{.15\linewidth}p{.15\linewidth}p{.15\linewidth}p{.15\linewidth}p{.15\linewidth}}

    \# & $\;$A & \multicolumn{4}{c}{Game Master} & $\;\:$B\\
    \hline 

    \theutterance \stepcounter{utterance}  

    & & \multicolumn{4}{p{0.6\linewidth}}{\cellcolor[rgb]{0.9,0.9,0.9}{%
	\makecell[{{p{\linewidth}}}]{% 
	  \tt {\tiny [A$\langle$GM]}  
	 Take the text below as a starting point, and make it a complete article: "You may have to meet with a helper to work out a parenting plan. The first would be to estimate how many time you have everyday for parenting, and is that enough...."\\ \tt Avoid using the following keywords: sleep, cook, feed\\ \tt Mention the keyword 'schedule' for more than 5 times. 
	  } 
	   } 
	   } 
	 & \\ 
 

    \theutterance \stepcounter{utterance}  

    & \multicolumn{4}{p{0.6\linewidth}}{\cellcolor[rgb]{0.8,1,0.9}{%
	\makecell[{{p{\linewidth}}}]{% 
	  \tt {\tiny [A$\rangle$GM]}  
	 Title: Crafting a Comprehensive Parenting Schedule\\ \tt \\ \tt When embarking on the journey of parenthood, one of the most crucial steps is establishing a solid parenting schedule. This plan is the backbone of a harmonious family life, ensuring that both parents and children know what to expect and when. You may have to meet with a helper to work out a parenting schedule that suits your family's unique needs.\\ \tt \\ \tt The first step in creating a parenting schedule is to estimate how much time you have every day for parenting responsibilities. It's essential to assess whether the available time is sufficient to meet the needs of your children. A well-structured schedule can help maximize the quality time you spend with your children, while also allowing for personal time and responsibilities.\\ \tt \\ \tt Once you have a rough idea of your daily time allocation, the next step is to break down the day into manageable chunks. A typical parenting schedule might include waking up, morning routines, educational activities, playtime, meals, and relaxation periods. Each of these segments should be allocated a specific time slot to ensure a smooth flow throughout the day.\\ \tt \\ \tt It's important to involve both parents in the schedule creation process. This collaborative approach helps in establishing a schedule that is fair, flexible, and mutually agreeable. Both parents should have input on the daily schedule, as well as any adjustments that may need to be made due to work commitments, appointments, or other obligations.\\ \tt \\ \tt A successful parenting schedule should also account for flexibility. Life is unpredictable, and there will be days when the plan needs to be adjusted. It's crucial to have a backup plan or a way to adapt the schedule on the fly. This might mean having a list of alternative activities or being open to spontaneous playtime.\\ \tt \\ \tt In addition to daily scheduling, it's also beneficial to plan for the week ahead. A weekly schedule can help with grocery shopping, planning for extracurricular activities, and ensuring that there is a balance between structured activities and free play. This long-term planning can prevent last-minute scrambles and help maintain a sense of order in the household.\\ \tt \\ \tt When creating a schedule, it's important to consider the individual needs and preferences of each child. Some children thrive with a rigid schedule, while others may benefit from more flexibility. Listening to your children and being open to their input can make the schedule more palatable and help them feel respected and understood.\\ \tt \\ \tt Furthermore, the schedule should also include time for family bonding. Whether it's a shared meal, a family game night, or a weekend outing, these moments are crucial for strengthening family ties. By prioritizing these times in the schedule, you're investing in the emotional well-being of your family.\\ \tt \\ \tt Reviewing and adjusting the parenting schedule regularly is also key. As children grow and their needs change, so too should the schedule. Quarterly reviews can help ensure that the schedule remains relevant and effective. This adaptability is crucial for maintaining a positive and nurturing environment for your children.\\ \tt \\ \tt In conclusion, a well-thought-out parenting schedule is an invaluable tool for any family. By dedicating time to create, review, and adjust the schedule, parents can provide a stable and loving environment for their children. With a clear schedule in place, families can navigate the daily demands of parenthood with confidence, ensuring that every moment counts and every child feels valued and cared for. Remember, a schedule is not a rigid set of rules, but rather a flexible guide that can be adapted to meet the ever-changing needs of your family.<|end_of_text|> 
	  } 
	   } 
	   } 
	 & & \\ 
 

    \theutterance \stepcounter{utterance}  

    & & & \multicolumn{2}{p{0.3\linewidth}}{\cellcolor[rgb]{0.95,0.95,0.95}{%
	\makecell[{{p{\linewidth}}}]{% 
	  \tt {\tiny [GM$|$GM]}  
	 {'keywords:forbidden_words': {'forbidden_words': ['sleep', 'cook', 'feed']}, 'keywords:frequency': {'relation': 'at least', 'keyword': 'schedule', 'frequency': 6}} 
	  } 
	   } 
	   } 
	 & & \\ 
 

    \theutterance \stepcounter{utterance}  

    & & & \multicolumn{2}{p{0.3\linewidth}}{\cellcolor[rgb]{0.95,0.95,0.95}{%
	\makecell[{{p{\linewidth}}}]{% 
	  \tt {\tiny [GM$|$GM]}  
	 game_result = WIN 
	  } 
	   } 
	   } 
	 & & \\ 
 

\end{supertabular}
}

\end{document}
