
\documentclass{article}
\usepackage{colortbl}
\usepackage{makecell}
\usepackage{multirow}
\usepackage{supertabular}

\begin{document}

\newcounter{utterance}

\twocolumn

{ \footnotesize  \setcounter{utterance}{1}
\setlength{\tabcolsep}{0pt}
\begin{supertabular}{c@{$\;$}|p{.15\linewidth}@{}p{.15\linewidth}p{.15\linewidth}p{.15\linewidth}p{.15\linewidth}p{.15\linewidth}}

    \# & $\;$A & \multicolumn{4}{c}{Game Master} & $\;\:$B\\
    \hline 

    \theutterance \stepcounter{utterance}  

    & & \multicolumn{4}{p{0.6\linewidth}}{\cellcolor[rgb]{0.9,0.9,0.9}{%
	\makecell[{{p{\linewidth}}}]{% 
	  \tt {\tiny [A$\langle$GM]}  
	 Explain what happens when you sniff a flower to 3rd grade students. Please answer in Finnish, no other language is allowed throughout your answer. Make sure your response contains a title wrapped in double angular brackets, i.e. <<title>>. 
	  } 
	   } 
	   } 
	 & \\ 
 

    \theutterance \stepcounter{utterance}  

    & \multicolumn{4}{p{0.6\linewidth}}{\cellcolor[rgb]{0.8,1,0.9}{%
	\makecell[{{p{\linewidth}}}]{% 
	  \tt {\tiny [A$\rangle$GM]}  
	 <<Kuinka hajua tuoksuvasta kukasta>>\\ \tt \\ \tt Kun sinä hajua tuoksuvasta kukasta, sinun aistisi mahtavan kauniin ja voimakkaan tuoksun. Kukka päästää ilmaan erilaisia aromeja, jotka ovat luonnostaan hajua. Näitä aromeja kutsutaan yleisnimellä "hajusteet". Kun sinä tuoksee kukkaa, sinun hengityksesi kulkee läpi nämä hajusteet, ja ne tarttuvat sinun hengitysteitseisiin. Sinun aivot tunnistavat nämä hajut ja lähettävät sinulle tietynlaisen tuntemuksen, joka tunnetaan nimellä "tuoksu". Tuoksut voivat muistuttaa sinua esimerkiksi lehtiä, puuta tai hedelmää. Kukat ovat luonnollinen ja ihania keinoja tutustua eri maistojen maailmaan.<|end_of_text|> 
	  } 
	   } 
	   } 
	 & & \\ 
 

    \theutterance \stepcounter{utterance}  

    & & & \multicolumn{2}{p{0.3\linewidth}}{\cellcolor[rgb]{0.95,0.95,0.95}{%
	\makecell[{{p{\linewidth}}}]{% 
	  \tt {\tiny [GM$|$GM]}  
	 {'language:response_language': {'language': 'fi'}, 'detectable_format:title': {}} 
	  } 
	   } 
	   } 
	 & & \\ 
 

    \theutterance \stepcounter{utterance}  

    & & & \multicolumn{2}{p{0.3\linewidth}}{\cellcolor[rgb]{0.95,0.95,0.95}{%
	\makecell[{{p{\linewidth}}}]{% 
	  \tt {\tiny [GM$|$GM]}  
	 game_result = WIN 
	  } 
	   } 
	   } 
	 & & \\ 
 

\end{supertabular}
}

\end{document}
