
\documentclass{article}
\usepackage{colortbl}
\usepackage{makecell}
\usepackage{multirow}
\usepackage{supertabular}

\begin{document}

\newcounter{utterance}

\twocolumn

{ \footnotesize  \setcounter{utterance}{1}
\setlength{\tabcolsep}{0pt}
\begin{supertabular}{c@{$\;$}|p{.15\linewidth}@{}p{.15\linewidth}p{.15\linewidth}p{.15\linewidth}p{.15\linewidth}p{.15\linewidth}}

    \# & $\;$A & \multicolumn{4}{c}{Game Master} & $\;\:$B\\
    \hline 

    \theutterance \stepcounter{utterance}  

    & & \multicolumn{4}{p{0.6\linewidth}}{\cellcolor[rgb]{0.9,0.9,0.9}{%
	\makecell[{{p{\linewidth}}}]{% 
	  \tt {\tiny [A$\langle$GM]}  
	 Write a casual blog post about how the outer solar system is different from the inner solar system, and what that means for the possibility of life. Wrap your entire response with double quotation marks. Your response should contain 17 or more sentences. 
	  } 
	   } 
	   } 
	 & \\ 
 

    \theutterance \stepcounter{utterance}  

    & \multicolumn{4}{p{0.6\linewidth}}{\cellcolor[rgb]{0.8,1,0.9}{%
	\makecell[{{p{\linewidth}}}]{% 
	  \tt {\tiny [A$\rangle$GM]}  
	 "Exploring the vast expanse of our solar system reveals striking differences between its inner and outer regions, each with implications for the potential harboring of life. The inner solar system, comprising the orbits of Mercury, Venus, Earth, and Mars, is characterized by its proximity to the Sun. This closeness means these planets experience higher surface temperatures and stronger gravitational forces. The inner system's rocky composition is a result of its formation process, where material closer to the Sun coalesced into denser, more solid bodies. Earth, in particular, has a diverse range of environments, from lush forests to icy poles, supporting a rich tapestry of life thanks to its moderate temperatures and liquid water.\\ \tt \\ \tt In contrast, the outer solar system, home to Jupiter, Saturn, Uranus, and Neptune, is a realm of gas giants and ice giants, far removed from the Sun's warming rays. Here, temperatures plummet, and the pressure and density of the atmospheres are significantly higher. These conditions make direct life as we know it less likely, as liquid water, a crucial ingredient for life, is scarce. Instead, the outer system is a frozen expanse, with moons like Europa and Enceladus capturing our attention. These moons harbor subsurface oceans, hidden beneath thick crusts of ice, offering potential habitats for life. The cold temperatures and pressure could preserve organic molecules, creating environments where life might thrive in unconventional ways.\\ \tt \\ \tt The differences in composition and temperature between the inner and outer solar system have profound implications for the possibility of life. In the inner system, the presence of liquid water and moderate temperatures have been key factors in the development of complex life. However, the outer system's extreme conditions suggest that any life forms there might be fundamentally different from those on Earth. The presence of organic materials and subsurface oceans on moons like Europa and Enceladus hints at the potential for life, albeit in forms we have yet to discover. Scientists are eager to explore these distant worlds, using probes and telescopes to gather more data.\\ \tt \\ \tt The study of these distant regions not only expands our understanding of planetary science but also challenges our notions of life itself. It prompts us to consider the diverse conditions under which life might exist. While the inner solar system offers a familiar template for life, the outer system presents a frontier of unknown possibilities. The search for life beyond Earth is not just about finding duplicates of ourselves but understanding the myriad ways life can emerge and evolve. As we continue to explore, we may uncover new principles that redefine our understanding of life's origins and potential.\\ \tt \\ \tt In summary, the inner and outer solar system offer distinct environments, each with its own potential for hosting life. While the inner system provides a familiar and hospitable setting for life as we know it, the outer system presents a cold, yet intriguing, possibility for life in hidden oceans. The quest to understand these differences is not only a scientific endeavor but also a philosophical one, inviting us to ponder the nature of life and its potential ubiquity in the universe. As we peer deeper into the outer solar system, we are reminded of the vastness and mystery of the cosmos, urging us to keep exploring and questioning."<|end_of_text|> 
	  } 
	   } 
	   } 
	 & & \\ 
 

    \theutterance \stepcounter{utterance}  

    & & & \multicolumn{2}{p{0.3\linewidth}}{\cellcolor[rgb]{0.95,0.95,0.95}{%
	\makecell[{{p{\linewidth}}}]{% 
	  \tt {\tiny [GM$|$GM]}  
	 {'startend:quotation': {}, 'length_constraints:number_sentences': {'relation': 'at least', 'num_sentences': 17}} 
	  } 
	   } 
	   } 
	 & & \\ 
 

    \theutterance \stepcounter{utterance}  

    & & & \multicolumn{2}{p{0.3\linewidth}}{\cellcolor[rgb]{0.95,0.95,0.95}{%
	\makecell[{{p{\linewidth}}}]{% 
	  \tt {\tiny [GM$|$GM]}  
	 game_result = LOSE 
	  } 
	   } 
	   } 
	 & & \\ 
 

\end{supertabular}
}

\end{document}
