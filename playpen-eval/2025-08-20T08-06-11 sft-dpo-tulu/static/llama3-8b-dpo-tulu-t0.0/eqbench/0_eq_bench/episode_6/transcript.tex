
\documentclass{article}
\usepackage{colortbl}
\usepackage{makecell}
\usepackage{multirow}
\usepackage{supertabular}

\begin{document}

\newcounter{utterance}

\twocolumn

{ \footnotesize  \setcounter{utterance}{1}
\setlength{\tabcolsep}{0pt}
\begin{supertabular}{c@{$\;$}|p{.15\linewidth}@{}p{.15\linewidth}p{.15\linewidth}p{.15\linewidth}p{.15\linewidth}p{.15\linewidth}}

    \# & $\;$A & \multicolumn{4}{c}{Game Master} & $\;\:$B\\
    \hline 

    \theutterance \stepcounter{utterance}  

    & & \multicolumn{4}{p{0.6\linewidth}}{\cellcolor[rgb]{0.9,0.9,0.9}{%
	\makecell[{{p{\linewidth}}}]{% 
	  \tt {\tiny [A$\langle$GM]}  
	 \\ \tt Your task is to predict the likely emotional responses of a character in this dialogue:\\ \tt \\ \tt James: I can't believe we're doing this again, Rachel. \\ \tt Rachel: What, the play? Or you mean us, playing lovers?\\ \tt James: Both, I guess. \\ \tt Rachel: Well, it's not like we have a choice. We were cast for these roles. \\ \tt James: Yeah, but it's... complicated.\\ \tt Rachel: Complicated? Are you still stuck on that scene from three years ago?\\ \tt James: I wouldn't say stuck. More like... haunted.\\ \tt Rachel: Haunted? That's a bit dramatic, even for you, James.\\ \tt James: Maybe, but it's the truth.\\ \tt Rachel: You know, for someone who loves romance novels, you're terrible at expressing your feelings.\\ \tt James: And you're too good at it. It's intimidating.\\ \tt Rachel: Intimidating? Me? You're the one who's always retreating into your shell.\\ \tt James: It's not like I do it on purpose, Rachel. It's just... hard.\\ \tt [End dialogue]\\ \tt \\ \tt At the end of this dialogue, James would feel...\\ \tt Inadequate\\ \tt Angry\\ \tt Confident\\ \tt Relieved\\ \tt \\ \tt Give each of these possible emotions a score from 0-10 for the relative intensity that they are likely to be feeling each.\\ \tt \\ \tt You must output in the following format, including headings (of course, you should give your own scores), with no additional commentary:\\ \tt \\ \tt Inadequate: <score>\\ \tt Angry: <score>\\ \tt Confident: <score>\\ \tt Relieved: <score>\\ \tt \\ \tt \\ \tt [End of answer]\\ \tt \\ \tt Remember: zero is a valid score, meaning they are likely not feeling that emotion. You must score at least one emotion > 0.\\ \tt \\ \tt Your answer:\\ \tt  
	  } 
	   } 
	   } 
	 & \\ 
 

    \theutterance \stepcounter{utterance}  

    & \multicolumn{4}{p{0.6\linewidth}}{\cellcolor[rgb]{0.8,1,0.9}{%
	\makecell[{{p{\linewidth}}}]{% 
	  \tt {\tiny [A$\rangle$GM]}  
	 Inadequate: 8\\ \tt Angry: 2\\ \tt Confident: 0\\ \tt Relieved: 0 
	  } 
	   } 
	   } 
	 & & \\ 
 

    \theutterance \stepcounter{utterance}  

    & & & \multicolumn{2}{p{0.3\linewidth}}{\cellcolor[rgb]{0.95,0.95,0.95}{%
	\makecell[{{p{\linewidth}}}]{% 
	  \tt {\tiny [GM$|$GM]}  
	 Inadequate: 8\\ \tt Angry: 2\\ \tt Confident: 0\\ \tt Relieved: 0 
	  } 
	   } 
	   } 
	 & & \\ 
 

    \theutterance \stepcounter{utterance}  

    & & & \multicolumn{2}{p{0.3\linewidth}}{\cellcolor[rgb]{0.95,0.95,0.95}{%
	\makecell[{{p{\linewidth}}}]{% 
	  \tt {\tiny [GM$|$GM]}  
	 Inadequate: 8\\ \tt Angry: 0\\ \tt Confident: 0\\ \tt Relieved: 0 
	  } 
	   } 
	   } 
	 & & \\ 
 

    \theutterance \stepcounter{utterance}  

    & & & \multicolumn{2}{p{0.3\linewidth}}{\cellcolor[rgb]{0.95,0.95,0.95}{%
	\makecell[{{p{\linewidth}}}]{% 
	  \tt {\tiny [GM$|$GM]}  
	 game_result = LOSE 
	  } 
	   } 
	   } 
	 & & \\ 
 

\end{supertabular}
}

\end{document}
