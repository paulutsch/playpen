
\documentclass{article}
\usepackage{colortbl}
\usepackage{makecell}
\usepackage{multirow}
\usepackage{supertabular}

\begin{document}

\newcounter{utterance}

\twocolumn

{ \footnotesize  \setcounter{utterance}{1}
\setlength{\tabcolsep}{0pt}
\begin{supertabular}{c@{$\;$}|p{.15\linewidth}@{}p{.15\linewidth}p{.15\linewidth}p{.15\linewidth}p{.15\linewidth}p{.15\linewidth}}

    \# & $\;$A & \multicolumn{4}{c}{Game Master} & $\;\:$B\\
    \hline 

    \theutterance \stepcounter{utterance}  

    & & \multicolumn{4}{p{0.6\linewidth}}{\cellcolor[rgb]{0.9,0.9,0.9}{%
	\makecell[{{p{\linewidth}}}]{% 
	  \tt {\tiny [A$\langle$GM]}  
	 You are an expert in causal inference. The following question is not a typical commonsense query, but rather a meticulously designed question created by a professor specializing in causal inference, intended to assess the students' mastery of the course content.\\ \tt Imagine a self-contained, hypothetical world with only the following conditions, and without any unmentioned factors or causal relationships: Poverty has a direct effect on water quality and cholera. Water company has a direct effect on water quality. Water quality has a direct effect on cholera. Poverty is unobserved. We know that high poverty or global water company causes clean water. high poverty or clean water causes cholera contraction. We observed the person is served by a global water company and the region has low poverty. Would the person gets cholera if polluted water instead of clean water?\\ \tt Start your answer with 'Yes' or 'No', followed by additional reasoning or evidence to support your explanation. 
	  } 
	   } 
	   } 
	 & \\ 
 

    \theutterance \stepcounter{utterance}  

    & \multicolumn{4}{p{0.6\linewidth}}{\cellcolor[rgb]{0.8,1,0.9}{%
	\makecell[{{p{\linewidth}}}]{% 
	  \tt {\tiny [A$\rangle$GM]}  
	 No\\ \tt \\ \tt To answer this question, we need to apply the principles of causal inference, specifically the concept of counterfactuals and the backdoor criterion. \\ \tt \\ \tt Given the conditions:\\ \tt \\ \tt - Poverty has a direct effect on water quality and cholera.\\ \tt - Water company has a direct effect on water quality.\\ \tt - Water quality has a direct effect on cholera.\\ \tt - Poverty is unobserved.\\ \tt - High poverty or global water company causes clean water.\\ \tt - High poverty or clean water causes cholera contraction.\\ \tt \\ \tt We are given that the person is served by a global water company and the region has low poverty. \\ \tt \\ \tt To determine the effect of polluted water on the person, we need to consider the counterfactual: what would happen if the person was exposed to polluted water instead of clean water, given the observed conditions.\\ \tt \\ \tt Since the person is served by a global water company, we know that this is a sufficient condition for clean water. Therefore, the person is already receiving clean water, and the global water company is a confounder that blocks the direct effect of poverty on water quality.\\ \tt \\ \tt The backdoor criterion states that a path from the cause to the effect is blocked if there is a confounder that affects both the cause and the effect. In this case, the global water company is a confounder that blocks the direct effect of poverty on water quality.\\ \tt \\ \tt Given that the region has low poverty, we can infer that the person would still receive clean water even if the global water company were not present, because low 
	  } 
	   } 
	   } 
	 & & \\ 
 

    \theutterance \stepcounter{utterance}  

    & & & \multicolumn{2}{p{0.3\linewidth}}{\cellcolor[rgb]{0.95,0.95,0.95}{%
	\makecell[{{p{\linewidth}}}]{% 
	  \tt {\tiny [GM$|$GM]}  
	 No 
	  } 
	   } 
	   } 
	 & & \\ 
 

    \theutterance \stepcounter{utterance}  

    & & & \multicolumn{2}{p{0.3\linewidth}}{\cellcolor[rgb]{0.95,0.95,0.95}{%
	\makecell[{{p{\linewidth}}}]{% 
	  \tt {\tiny [GM$|$GM]}  
	 no 
	  } 
	   } 
	   } 
	 & & \\ 
 

    \theutterance \stepcounter{utterance}  

    & & & \multicolumn{2}{p{0.3\linewidth}}{\cellcolor[rgb]{0.95,0.95,0.95}{%
	\makecell[{{p{\linewidth}}}]{% 
	  \tt {\tiny [GM$|$GM]}  
	 game_result = WIN 
	  } 
	   } 
	   } 
	 & & \\ 
 

\end{supertabular}
}

\end{document}
