
\documentclass{article}
\usepackage{colortbl}
\usepackage{makecell}
\usepackage{multirow}
\usepackage{supertabular}

\begin{document}

\newcounter{utterance}

\twocolumn

{ \footnotesize  \setcounter{utterance}{1}
\setlength{\tabcolsep}{0pt}
\begin{supertabular}{c@{$\;$}|p{.15\linewidth}@{}p{.15\linewidth}p{.15\linewidth}p{.15\linewidth}p{.15\linewidth}p{.15\linewidth}}

    \# & $\;$A & \multicolumn{4}{c}{Game Master} & $\;\:$B\\
    \hline 

    \theutterance \stepcounter{utterance}  

    & & \multicolumn{4}{p{0.6\linewidth}}{\cellcolor[rgb]{0.9,0.9,0.9}{%
	\makecell[{{p{\linewidth}}}]{% 
	  \tt {\tiny [A$\langle$GM]}  
	 You are an expert in causal inference. The following question is not a typical commonsense query, but rather a meticulously designed question created by a professor specializing in causal inference, intended to assess the students' mastery of the course content.\\ \tt Imagine a self-contained, hypothetical world with only the following conditions, and without any unmentioned factors or causal relationships: Alice waking up has a direct effect on Alice arriving to school. Traffic has a direct effect on Alice arriving to school. The overall probability of waking up late is 5%. For days when Alice wakes up on time, the probability of arriving to school on time is 52%. For days when Alice wakes up late, the probability of arriving to school on time is 87%. Is arriving to school on time more likely than arriving to school late overall?\\ \tt Start your answer with 'Yes' or 'No', followed by additional reasoning or evidence to support your explanation. 
	  } 
	   } 
	   } 
	 & \\ 
 

    \theutterance \stepcounter{utterance}  

    & \multicolumn{4}{p{0.6\linewidth}}{\cellcolor[rgb]{0.8,1,0.9}{%
	\makecell[{{p{\linewidth}}}]{% 
	  \tt {\tiny [A$\rangle$GM]}  
	 Yes, arriving to school on time is more likely than arriving to school late overall. \\ \tt \\ \tt To answer this question, we need to calculate the overall probability of arriving to school on time and the overall probability of arriving to school late. \\ \tt \\ \tt Let's denote the probability of waking up on time as P(Wake On Time) = 0.95 (since the probability of waking up late is 5%) and the probability of arriving to school on time given waking up on time as P(Arrive On Time | Wake On Time) = 0.52. \\ \tt \\ \tt We also know the probability of arriving to school on time given waking up late as P(Arrive On Time | Wake Late) = 0.87.\\ \tt \\ \tt Using the law of total probability, we can calculate the overall probability of arriving to school on time as follows:\\ \tt \\ \tt P(Arrive On Time) = P(Wake On Time) * P(Arrive On Time | Wake On Time) + P(Wake Late) * P(Arrive On Time | Wake Late)\\ \tt = 0.95 * 0.52 + 0.05 * 0.87\\ \tt = 0.494 + 0.0435\\ \tt = 0.5375\\ \tt \\ \tt Similarly, we can calculate the overall probability of arriving to school late as follows:\\ \tt \\ \tt P(Arrive Late) = 1 - P(Arrive On Time)\\ \tt = 1 - 0.5375\\ \tt = 0.4625 
	  } 
	   } 
	   } 
	 & & \\ 
 

    \theutterance \stepcounter{utterance}  

    & & & \multicolumn{2}{p{0.3\linewidth}}{\cellcolor[rgb]{0.95,0.95,0.95}{%
	\makecell[{{p{\linewidth}}}]{% 
	  \tt {\tiny [GM$|$GM]}  
	 Yes 
	  } 
	   } 
	   } 
	 & & \\ 
 

    \theutterance \stepcounter{utterance}  

    & & & \multicolumn{2}{p{0.3\linewidth}}{\cellcolor[rgb]{0.95,0.95,0.95}{%
	\makecell[{{p{\linewidth}}}]{% 
	  \tt {\tiny [GM$|$GM]}  
	 yes 
	  } 
	   } 
	   } 
	 & & \\ 
 

    \theutterance \stepcounter{utterance}  

    & & & \multicolumn{2}{p{0.3\linewidth}}{\cellcolor[rgb]{0.95,0.95,0.95}{%
	\makecell[{{p{\linewidth}}}]{% 
	  \tt {\tiny [GM$|$GM]}  
	 game_result = WIN 
	  } 
	   } 
	   } 
	 & & \\ 
 

\end{supertabular}
}

\end{document}
