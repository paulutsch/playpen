
\documentclass{article}
\usepackage{colortbl}
\usepackage{makecell}
\usepackage{multirow}
\usepackage{supertabular}

\begin{document}

\newcounter{utterance}

\twocolumn

{ \footnotesize  \setcounter{utterance}{1}
\setlength{\tabcolsep}{0pt}
\begin{supertabular}{c@{$\;$}|p{.15\linewidth}@{}p{.15\linewidth}p{.15\linewidth}p{.15\linewidth}p{.15\linewidth}p{.15\linewidth}}

    \# & $\;$A & \multicolumn{4}{c}{Game Master} & $\;\:$B\\
    \hline 

    \theutterance \stepcounter{utterance}  

    & & \multicolumn{4}{p{0.6\linewidth}}{\cellcolor[rgb]{0.9,0.9,0.9}{%
	\makecell[{{p{\linewidth}}}]{% 
	  \tt {\tiny [A$\langle$GM]}  
	 You are an expert in causal inference. The following question is not a typical commonsense query, but rather a meticulously designed question created by a professor specializing in causal inference, intended to assess the students' mastery of the course content.\\ \tt Imagine a self-contained, hypothetical world with only the following conditions, and without any unmentioned factors or causal relationships: Husband has a direct effect on wife and alarm clock. Wife has a direct effect on alarm clock. The overall probability of alarm set by husband is 8%. For husbands that don't set the alarm, the probability of ringing alarm is 75%. For husbands that set the alarm, the probability of ringing alarm is 16%. Is ringing alarm more likely than silent alarm overall?\\ \tt Start your answer with 'Yes' or 'No', followed by additional reasoning or evidence to support your explanation. 
	  } 
	   } 
	   } 
	 & \\ 
 

    \theutterance \stepcounter{utterance}  

    & \multicolumn{4}{p{0.6\linewidth}}{\cellcolor[rgb]{0.8,1,0.9}{%
	\makecell[{{p{\linewidth}}}]{% 
	  \tt {\tiny [A$\rangle$GM]}  
	 No\\ \tt \\ \tt To answer this question, we need to calculate the overall probability of a ringing alarm and compare it to the probability of a silent alarm. \\ \tt \\ \tt Let's denote the event of a husband setting the alarm as S, and the event of the alarm ringing as R. We are given the following probabilities:\\ \tt \\ \tt - P(R|S) = 16% (probability of alarm ringing given that the husband sets the alarm)\\ \tt - P(R|~S) = 75% (probability of alarm ringing given that the husband does not set the alarm)\\ \tt - P(S) = 8% (probability of the husband setting the alarm)\\ \tt \\ \tt We can calculate the overall probability of the alarm ringing (P(R)) using the law of total probability:\\ \tt \\ \tt P(R) = P(R|S) * P(S) + P(R|~S) * P(~S)\\ \tt \\ \tt First, we need to find P(~S), which is the probability of the husband not setting the alarm. Since P(S) = 8%, we have:\\ \tt \\ \tt P(~S) = 1 - P(S) = 1 - 0.08 = 0.92\\ \tt \\ \tt Now, we can plug in the values:\\ \tt \\ \tt P(R) = 0.16 * 0.08 + 0.75 * 0.92\\ \tt = 0.0128 + 0.69\\ \tt = 0.7028\\ \tt \\ \tt The overall probability of the alarm ringing is approximately 70.28%.\\ \tt \\ \tt To find the 
	  } 
	   } 
	   } 
	 & & \\ 
 

    \theutterance \stepcounter{utterance}  

    & & & \multicolumn{2}{p{0.3\linewidth}}{\cellcolor[rgb]{0.95,0.95,0.95}{%
	\makecell[{{p{\linewidth}}}]{% 
	  \tt {\tiny [GM$|$GM]}  
	 No 
	  } 
	   } 
	   } 
	 & & \\ 
 

    \theutterance \stepcounter{utterance}  

    & & & \multicolumn{2}{p{0.3\linewidth}}{\cellcolor[rgb]{0.95,0.95,0.95}{%
	\makecell[{{p{\linewidth}}}]{% 
	  \tt {\tiny [GM$|$GM]}  
	 yes 
	  } 
	   } 
	   } 
	 & & \\ 
 

    \theutterance \stepcounter{utterance}  

    & & & \multicolumn{2}{p{0.3\linewidth}}{\cellcolor[rgb]{0.95,0.95,0.95}{%
	\makecell[{{p{\linewidth}}}]{% 
	  \tt {\tiny [GM$|$GM]}  
	 game_result = LOSE 
	  } 
	   } 
	   } 
	 & & \\ 
 

\end{supertabular}
}

\end{document}
