
\documentclass{article}
\usepackage{colortbl}
\usepackage{makecell}
\usepackage{multirow}
\usepackage{supertabular}

\begin{document}

\newcounter{utterance}

\twocolumn

{ \footnotesize  \setcounter{utterance}{1}
\setlength{\tabcolsep}{0pt}
\begin{supertabular}{c@{$\;$}|p{.15\linewidth}@{}p{.15\linewidth}p{.15\linewidth}p{.15\linewidth}p{.15\linewidth}p{.15\linewidth}}

    \# & $\;$A & \multicolumn{4}{c}{Game Master} & $\;\:$B\\
    \hline 

    \theutterance \stepcounter{utterance}  

    & & \multicolumn{4}{p{0.6\linewidth}}{\cellcolor[rgb]{0.9,0.9,0.9}{%
	\makecell[{{p{\linewidth}}}]{% 
	  \tt {\tiny [A$\langle$GM]}  
	 You are an expert in causal inference. The following question is not a typical commonsense query, but rather a meticulously designed question created by a professor specializing in causal inference, intended to assess the students' mastery of the course content.\\ \tt Imagine a self-contained, hypothetical world with only the following conditions, and without any unmentioned factors or causal relationships: Unobserved confounders has a direct effect on education level and salary. Proximity to a college has a direct effect on education level. Education level has a direct effect on salary. Unobserved confounders is unobserved. For people living far from a college, the probability of high salary is 39%. For people living close to a college, the probability of high salary is 49%. For people living far from a college, the probability of college degree or higher is 57%. For people living close to a college, the probability of college degree or higher is 30%. Will college degree or higher increase the chance of high salary?\\ \tt Start your answer with 'Yes' or 'No', followed by additional reasoning or evidence to support your explanation. 
	  } 
	   } 
	   } 
	 & \\ 
 

    \theutterance \stepcounter{utterance}  

    & \multicolumn{4}{p{0.6\linewidth}}{\cellcolor[rgb]{0.8,1,0.9}{%
	\makecell[{{p{\linewidth}}}]{% 
	  \tt {\tiny [A$\rangle$GM]}  
	 No\\ \tt \\ \tt To answer this question, we need to use causal inference techniques, specifically to address the issue of unobserved confounders. Let's denote the unobserved confounders as U, education level as E, proximity to a college as P, and salary as S.\\ \tt \\ \tt Given the conditions:\\ \tt \\ \tt 1. U has a direct effect on E and S.\\ \tt 2. P has a direct effect on E.\\ \tt 3. E has a direct effect on S.\\ \tt 4. U is unobserved.\\ \tt \\ \tt We are given the following probabilities:\\ \tt \\ \tt - P(S=high | P=far) = 0.39\\ \tt - P(S=high | P=close) = 0.49\\ \tt - P(E=college | P=far) = 0.57\\ \tt - P(E=college | P=close) = 0.30\\ \tt \\ \tt We want to determine if E (college degree or higher) increases the chance of S (high salary). To do this, we need to use the potential outcomes framework and the concept of causal effects.\\ \tt \\ \tt Let's define the potential outcomes:\\ \tt \\ \tt - S(E=college, P=far) = the salary if someone has a college degree and lives far from a college\\ \tt - S(E=college, P=close) = the salary if someone has a college degree and lives close to a college\\ \tt - S(E=non-college, P=far) = the salary if someone does not have a college degree and lives far from a 
	  } 
	   } 
	   } 
	 & & \\ 
 

    \theutterance \stepcounter{utterance}  

    & & & \multicolumn{2}{p{0.3\linewidth}}{\cellcolor[rgb]{0.95,0.95,0.95}{%
	\makecell[{{p{\linewidth}}}]{% 
	  \tt {\tiny [GM$|$GM]}  
	 No 
	  } 
	   } 
	   } 
	 & & \\ 
 

    \theutterance \stepcounter{utterance}  

    & & & \multicolumn{2}{p{0.3\linewidth}}{\cellcolor[rgb]{0.95,0.95,0.95}{%
	\makecell[{{p{\linewidth}}}]{% 
	  \tt {\tiny [GM$|$GM]}  
	 no 
	  } 
	   } 
	   } 
	 & & \\ 
 

    \theutterance \stepcounter{utterance}  

    & & & \multicolumn{2}{p{0.3\linewidth}}{\cellcolor[rgb]{0.95,0.95,0.95}{%
	\makecell[{{p{\linewidth}}}]{% 
	  \tt {\tiny [GM$|$GM]}  
	 game_result = WIN 
	  } 
	   } 
	   } 
	 & & \\ 
 

\end{supertabular}
}

\end{document}
