
\documentclass{article}
\usepackage{colortbl}
\usepackage{makecell}
\usepackage{multirow}
\usepackage{supertabular}

\begin{document}

\newcounter{utterance}

\twocolumn

{ \footnotesize  \setcounter{utterance}{1}
\setlength{\tabcolsep}{0pt}
\begin{supertabular}{c@{$\;$}|p{.15\linewidth}@{}p{.15\linewidth}p{.15\linewidth}p{.15\linewidth}p{.15\linewidth}p{.15\linewidth}}

    \# & $\;$A & \multicolumn{4}{c}{Game Master} & $\;\:$B\\
    \hline 

    \theutterance \stepcounter{utterance}  

    & & \multicolumn{4}{p{0.6\linewidth}}{\cellcolor[rgb]{0.9,0.9,0.9}{%
	\makecell[{{p{\linewidth}}}]{% 
	  \tt {\tiny [A$\langle$GM]}  
	 You are an expert in causal inference. The following question is not a typical commonsense query, but rather a meticulously designed question created by a professor specializing in causal inference, intended to assess the students' mastery of the course content.\\ \tt Imagine a self-contained, hypothetical world with only the following conditions, and without any unmentioned factors or causal relationships: Listening to jazz has a direct effect on tar deposit. Tar deposit has a direct effect on lung cancer. For people who do not listen to jazz, the probability of lung cancer is 61%. For people who listen to jazz, the probability of lung cancer is 79%. For people who listen to jazz, would it be less likely to see lung cancer if the person did not listen to jazz?\\ \tt Start your answer with 'Yes' or 'No', followed by additional reasoning or evidence to support your explanation. 
	  } 
	   } 
	   } 
	 & \\ 
 

    \theutterance \stepcounter{utterance}  

    & \multicolumn{4}{p{0.6\linewidth}}{\cellcolor[rgb]{0.8,1,0.9}{%
	\makecell[{{p{\linewidth}}}]{% 
	  \tt {\tiny [A$\rangle$GM]}  
	 No\\ \tt \\ \tt To answer this question, we need to apply the concept of counterfactuals and potential outcomes. We are comparing the probability of lung cancer for a person who listens to jazz (P(Lung Cancer | Jazz)) to the probability of lung cancer for the same person if they did not listen to jazz (P(Lung Cancer | no Jazz)).\\ \tt \\ \tt Given that the probability of lung cancer for people who do not listen to jazz is 61%, we want to know whether it would be less likely to see lung cancer if the person did not listen to jazz, given that they do listen to jazz.\\ \tt \\ \tt The key insight here is that the probability of lung cancer for people who listen to jazz (79%) is higher than for those who do not listen to jazz (61%). However, this does not directly inform us about the counterfactual probability of lung cancer for a person who listens to jazz if they did not listen to jazz.\\ \tt \\ \tt To answer the question, we need to consider the causal relationship between listening to jazz and tar deposit, and then between tar deposit and lung cancer. The direct effect of listening to jazz on tar deposit is not specified, but we can infer that the presence of tar deposit is a mediator in the relationship between listening to jazz and lung cancer.\\ \tt \\ \tt If we assume that the effect of listening to jazz on tar deposit is positive (i.e., listening to jazz increases tar deposit), and the effect of tar deposit on lung cancer is also positive (i.e., higher tar deposit 
	  } 
	   } 
	   } 
	 & & \\ 
 

    \theutterance \stepcounter{utterance}  

    & & & \multicolumn{2}{p{0.3\linewidth}}{\cellcolor[rgb]{0.95,0.95,0.95}{%
	\makecell[{{p{\linewidth}}}]{% 
	  \tt {\tiny [GM$|$GM]}  
	 No 
	  } 
	   } 
	   } 
	 & & \\ 
 

    \theutterance \stepcounter{utterance}  

    & & & \multicolumn{2}{p{0.3\linewidth}}{\cellcolor[rgb]{0.95,0.95,0.95}{%
	\makecell[{{p{\linewidth}}}]{% 
	  \tt {\tiny [GM$|$GM]}  
	 yes 
	  } 
	   } 
	   } 
	 & & \\ 
 

    \theutterance \stepcounter{utterance}  

    & & & \multicolumn{2}{p{0.3\linewidth}}{\cellcolor[rgb]{0.95,0.95,0.95}{%
	\makecell[{{p{\linewidth}}}]{% 
	  \tt {\tiny [GM$|$GM]}  
	 game_result = LOSE 
	  } 
	   } 
	   } 
	 & & \\ 
 

\end{supertabular}
}

\end{document}
