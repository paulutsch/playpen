
\documentclass{article}
\usepackage{colortbl}
\usepackage{makecell}
\usepackage{multirow}
\usepackage{supertabular}

\begin{document}

\newcounter{utterance}

\twocolumn

{ \footnotesize  \setcounter{utterance}{1}
\setlength{\tabcolsep}{0pt}
\begin{supertabular}{c@{$\;$}|p{.15\linewidth}@{}p{.15\linewidth}p{.15\linewidth}p{.15\linewidth}p{.15\linewidth}p{.15\linewidth}}

    \# & $\;$A & \multicolumn{4}{c}{Game Master} & $\;\:$B\\
    \hline 

    \theutterance \stepcounter{utterance}  

    & & \multicolumn{4}{p{0.6\linewidth}}{\cellcolor[rgb]{0.9,0.9,0.9}{%
	\makecell[{{p{\linewidth}}}]{% 
	  \tt {\tiny [A$\langle$GM]}  
	 You are an expert in causal inference. The following question is not a typical commonsense query, but rather a meticulously designed question created by a professor specializing in causal inference, intended to assess the students' mastery of the course content.\\ \tt Imagine a self-contained, hypothetical world with only the following conditions, and without any unmentioned factors or causal relationships: Solar eclipse has a direct effect on Alice arriving to school. Traffic has a direct effect on Alice arriving to school. Method 1: We look at how solar eclipse correlates with Alice arriving to school case by case according to traffic. Method 2: We look directly at how solar eclipse correlates with Alice arriving to school in general. To understand how solar eclipse affects Alice arriving to school, is it more correct to use the Method 1 than Method 2?\\ \tt Start your answer with 'Yes' or 'No', followed by additional reasoning or evidence to support your explanation. 
	  } 
	   } 
	   } 
	 & \\ 
 

    \theutterance \stepcounter{utterance}  

    & \multicolumn{4}{p{0.6\linewidth}}{\cellcolor[rgb]{0.8,1,0.9}{%
	\makecell[{{p{\linewidth}}}]{% 
	  \tt {\tiny [A$\rangle$GM]}  
	 No, it is not more correct to use Method 1 than Method 2. \\ \tt \\ \tt In this hypothetical world, we have a direct effect of solar eclipse on Alice arriving to school and a direct effect of traffic on Alice arriving to school. This situation describes a classic case of confounding variable, where traffic is a confounder that affects both the exposure (solar eclipse) and the outcome (Alice arriving to school). \\ \tt \\ \tt Method 2, which looks directly at the correlation between solar eclipse and Alice arriving to school in general, is more likely to yield a biased estimate of the causal effect of solar eclipse on Alice arriving to school. This is because it does not account for the confounding effect of traffic, which can lead to an association between solar eclipse and Alice arriving to school that is not due to the causal effect of solar eclipse.\\ \tt \\ \tt Method 1, on the other hand, attempts to control for the confounding effect of traffic by examining the correlation between solar eclipse and Alice arriving to school case by case according to traffic. This approach is more likely to provide a more accurate estimate of the causal effect of solar eclipse on Alice arriving to school, as it takes into account the varying levels of traffic and its potential impact on Alice's arrival time.\\ \tt \\ \tt However, it's worth noting that even Method 1 may not fully eliminate the bias due to confounding, as it still relies on the assumption that traffic is the only confounder and that the relationship between solar eclipse and Alice arriving to school is linear 
	  } 
	   } 
	   } 
	 & & \\ 
 

    \theutterance \stepcounter{utterance}  

    & & & \multicolumn{2}{p{0.3\linewidth}}{\cellcolor[rgb]{0.95,0.95,0.95}{%
	\makecell[{{p{\linewidth}}}]{% 
	  \tt {\tiny [GM$|$GM]}  
	 No 
	  } 
	   } 
	   } 
	 & & \\ 
 

    \theutterance \stepcounter{utterance}  

    & & & \multicolumn{2}{p{0.3\linewidth}}{\cellcolor[rgb]{0.95,0.95,0.95}{%
	\makecell[{{p{\linewidth}}}]{% 
	  \tt {\tiny [GM$|$GM]}  
	 no 
	  } 
	   } 
	   } 
	 & & \\ 
 

    \theutterance \stepcounter{utterance}  

    & & & \multicolumn{2}{p{0.3\linewidth}}{\cellcolor[rgb]{0.95,0.95,0.95}{%
	\makecell[{{p{\linewidth}}}]{% 
	  \tt {\tiny [GM$|$GM]}  
	 game_result = WIN 
	  } 
	   } 
	   } 
	 & & \\ 
 

\end{supertabular}
}

\end{document}
