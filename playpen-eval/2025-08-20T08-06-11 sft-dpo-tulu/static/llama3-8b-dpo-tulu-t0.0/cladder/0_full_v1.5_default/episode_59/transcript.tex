
\documentclass{article}
\usepackage{colortbl}
\usepackage{makecell}
\usepackage{multirow}
\usepackage{supertabular}

\begin{document}

\newcounter{utterance}

\twocolumn

{ \footnotesize  \setcounter{utterance}{1}
\setlength{\tabcolsep}{0pt}
\begin{supertabular}{c@{$\;$}|p{.15\linewidth}@{}p{.15\linewidth}p{.15\linewidth}p{.15\linewidth}p{.15\linewidth}p{.15\linewidth}}

    \# & $\;$A & \multicolumn{4}{c}{Game Master} & $\;\:$B\\
    \hline 

    \theutterance \stepcounter{utterance}  

    & & \multicolumn{4}{p{0.6\linewidth}}{\cellcolor[rgb]{0.9,0.9,0.9}{%
	\makecell[{{p{\linewidth}}}]{% 
	  \tt {\tiny [A$\langle$GM]}  
	 You are an expert in causal inference. The following question is not a typical commonsense query, but rather a meticulously designed question created by a professor specializing in causal inference, intended to assess the students' mastery of the course content.\\ \tt Imagine a self-contained, hypothetical world with only the following conditions, and without any unmentioned factors or causal relationships: Unobserved confounders has a direct effect on having a sister and cholesterol level. Treatment assignment has a direct effect on having a sister. Having a sister has a direct effect on cholesterol level. Unobserved confounders is unobserved. The overall probability of having a sister is 20%. The probability of not having a sister and low cholesterol is 65%. The probability of having a sister and low cholesterol is 14%. Is the chance of low cholesterol larger when observing having a sister?\\ \tt Start your answer with 'Yes' or 'No', followed by additional reasoning or evidence to support your explanation. 
	  } 
	   } 
	   } 
	 & \\ 
 

    \theutterance \stepcounter{utterance}  

    & \multicolumn{4}{p{0.6\linewidth}}{\cellcolor[rgb]{0.8,1,0.9}{%
	\makecell[{{p{\linewidth}}}]{% 
	  \tt {\tiny [A$\rangle$GM]}  
	 No\\ \tt \\ \tt To answer this question, we need to use the concept of causal inference, specifically the backdoor criterion or the frontdoor adjustment. Given the conditions:\\ \tt \\ \tt - Unobserved confounders (U) has a direct effect on having a sister (S) and cholesterol level (C).\\ \tt - Treatment assignment (T) has a direct effect on having a sister (S).\\ \tt - Having a sister (S) has a direct effect on cholesterol level (C).\\ \tt - Unobserved confounders (U) is unobserved.\\ \tt \\ \tt We are asked to compare the probability of low cholesterol (C=0) when observing having a sister (S=1) versus not observing having a sister (S=0).\\ \tt \\ \tt First, let's define the probability distributions:\\ \tt \\ \tt - P(S=1) = 0.2 (probability of having a sister)\\ \tt - P(S=0) = 0.8 (probability of not having a sister)\\ \tt - P(C=0, S=0) = 0.65 (probability of low cholesterol and not having a sister)\\ \tt - P(C=0, S=1) = 0.14 (probability of low cholesterol and having a sister)\\ \tt \\ \tt We can use the law of total probability to calculate the probability of low cholesterol (C=0) when not observing having a sister (S=0):\\ \tt \\ \tt P(C=0) = P(C=0, S=0) + P(C=0, S=1) * P(S 
	  } 
	   } 
	   } 
	 & & \\ 
 

    \theutterance \stepcounter{utterance}  

    & & & \multicolumn{2}{p{0.3\linewidth}}{\cellcolor[rgb]{0.95,0.95,0.95}{%
	\makecell[{{p{\linewidth}}}]{% 
	  \tt {\tiny [GM$|$GM]}  
	 No 
	  } 
	   } 
	   } 
	 & & \\ 
 

    \theutterance \stepcounter{utterance}  

    & & & \multicolumn{2}{p{0.3\linewidth}}{\cellcolor[rgb]{0.95,0.95,0.95}{%
	\makecell[{{p{\linewidth}}}]{% 
	  \tt {\tiny [GM$|$GM]}  
	 no 
	  } 
	   } 
	   } 
	 & & \\ 
 

    \theutterance \stepcounter{utterance}  

    & & & \multicolumn{2}{p{0.3\linewidth}}{\cellcolor[rgb]{0.95,0.95,0.95}{%
	\makecell[{{p{\linewidth}}}]{% 
	  \tt {\tiny [GM$|$GM]}  
	 game_result = WIN 
	  } 
	   } 
	   } 
	 & & \\ 
 

\end{supertabular}
}

\end{document}
