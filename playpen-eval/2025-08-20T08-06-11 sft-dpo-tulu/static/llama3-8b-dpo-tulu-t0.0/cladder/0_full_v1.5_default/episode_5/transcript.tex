
\documentclass{article}
\usepackage{colortbl}
\usepackage{makecell}
\usepackage{multirow}
\usepackage{supertabular}

\begin{document}

\newcounter{utterance}

\twocolumn

{ \footnotesize  \setcounter{utterance}{1}
\setlength{\tabcolsep}{0pt}
\begin{supertabular}{c@{$\;$}|p{.15\linewidth}@{}p{.15\linewidth}p{.15\linewidth}p{.15\linewidth}p{.15\linewidth}p{.15\linewidth}}

    \# & $\;$A & \multicolumn{4}{c}{Game Master} & $\;\:$B\\
    \hline 

    \theutterance \stepcounter{utterance}  

    & & \multicolumn{4}{p{0.6\linewidth}}{\cellcolor[rgb]{0.9,0.9,0.9}{%
	\makecell[{{p{\linewidth}}}]{% 
	  \tt {\tiny [A$\langle$GM]}  
	 You are an expert in causal inference. The following question is not a typical commonsense query, but rather a meticulously designed question created by a professor specializing in causal inference, intended to assess the students' mastery of the course content.\\ \tt Imagine a self-contained, hypothetical world with only the following conditions, and without any unmentioned factors or causal relationships: Vaccination status has a direct effect on vaccination reaction and getting smallpox. Getting smallpox has a direct effect on smallpox survival. Vaccination reaction has a direct effect on smallpox survival. Method 1: We look directly at how vaccination status correlates with smallpox survival in general. Method 2: We look at this correlation case by case according to vaccination reaction. To understand how vaccination status affects smallpox survival, is it more correct to use the Method 1 than Method 2?\\ \tt Start your answer with 'Yes' or 'No', followed by additional reasoning or evidence to support your explanation. 
	  } 
	   } 
	   } 
	 & \\ 
 

    \theutterance \stepcounter{utterance}  

    & \multicolumn{4}{p{0.6\linewidth}}{\cellcolor[rgb]{0.8,1,0.9}{%
	\makecell[{{p{\linewidth}}}]{% 
	  \tt {\tiny [A$\rangle$GM]}  
	 No, it is not more correct to use Method 1 than Method 2. \\ \tt \\ \tt In this hypothetical world, we have a complex causal structure involving vaccination status, vaccination reaction, getting smallpox, and smallpox survival. The presence of a common effect (getting smallpox) and a common cause (vaccination status) between vaccination reaction and smallpox survival creates a problem of confounding. \\ \tt \\ \tt Method 1, which looks directly at the correlation between vaccination status and smallpox survival in general, will be biased due to the confounding effect of getting smallpox. Individuals who are vaccinated are less likely to get smallpox, and those who get smallpox are more likely to have a poor smallpox survival outcome. This confounding variable will lead to an underestimation of the effect of vaccination status on smallpox survival.\\ \tt \\ \tt Method 2, which looks at the correlation case by case according to vaccination reaction, is a form of stratification or matching that can help to control for the confounding effect of getting smallpox. By conditioning on vaccination reaction, we can isolate the effect of vaccination status on smallpox survival within each stratum, thereby reducing the bias caused by the common effect.\\ \tt \\ \tt In this scenario, Method 2 is a more correct approach because it allows us to identify the causal effect of vaccination status on smallpox survival while controlling for the confounding variable. 
	  } 
	   } 
	   } 
	 & & \\ 
 

    \theutterance \stepcounter{utterance}  

    & & & \multicolumn{2}{p{0.3\linewidth}}{\cellcolor[rgb]{0.95,0.95,0.95}{%
	\makecell[{{p{\linewidth}}}]{% 
	  \tt {\tiny [GM$|$GM]}  
	 No 
	  } 
	   } 
	   } 
	 & & \\ 
 

    \theutterance \stepcounter{utterance}  

    & & & \multicolumn{2}{p{0.3\linewidth}}{\cellcolor[rgb]{0.95,0.95,0.95}{%
	\makecell[{{p{\linewidth}}}]{% 
	  \tt {\tiny [GM$|$GM]}  
	 yes 
	  } 
	   } 
	   } 
	 & & \\ 
 

    \theutterance \stepcounter{utterance}  

    & & & \multicolumn{2}{p{0.3\linewidth}}{\cellcolor[rgb]{0.95,0.95,0.95}{%
	\makecell[{{p{\linewidth}}}]{% 
	  \tt {\tiny [GM$|$GM]}  
	 game_result = LOSE 
	  } 
	   } 
	   } 
	 & & \\ 
 

\end{supertabular}
}

\end{document}
