
\documentclass{article}
\usepackage{colortbl}
\usepackage{makecell}
\usepackage{multirow}
\usepackage{supertabular}

\begin{document}

\newcounter{utterance}

\twocolumn

{ \footnotesize  \setcounter{utterance}{1}
\setlength{\tabcolsep}{0pt}
\begin{supertabular}{c@{$\;$}|p{.15\linewidth}@{}p{.15\linewidth}p{.15\linewidth}p{.15\linewidth}p{.15\linewidth}p{.15\linewidth}}

    \# & $\;$A & \multicolumn{4}{c}{Game Master} & $\;\:$B\\
    \hline 

    \theutterance \stepcounter{utterance}  

    & & \multicolumn{4}{p{0.6\linewidth}}{\cellcolor[rgb]{0.9,0.9,0.9}{%
	\makecell[{{p{\linewidth}}}]{% 
	  \tt {\tiny [A$\langle$GM]}  
	 You are an expert in causal inference. The following question is not a typical commonsense query, but rather a meticulously designed question created by a professor specializing in causal inference, intended to assess the students' mastery of the course content.\\ \tt Imagine a self-contained, hypothetical world with only the following conditions, and without any unmentioned factors or causal relationships: Maternal smoking status has a direct effect on infant's birth weight and freckles. Health condition has a direct effect on infant's birth weight and freckles. Infant's birth weight has a direct effect on freckles. Health condition is unobserved. Method 1: We look directly at how maternal smoking status correlates with freckles in general. Method 2: We look at this correlation case by case according to infant's birth weight. To understand how maternal smoking status affects freckles, is it more correct to use the Method 1 than Method 2?\\ \tt Start your answer with 'Yes' or 'No', followed by additional reasoning or evidence to support your explanation. 
	  } 
	   } 
	   } 
	 & \\ 
 

    \theutterance \stepcounter{utterance}  

    & \multicolumn{4}{p{0.6\linewidth}}{\cellcolor[rgb]{0.8,1,0.9}{%
	\makecell[{{p{\linewidth}}}]{% 
	  \tt {\tiny [A$\rangle$GM]}  
	 No, it is not more correct to use Method 1 than Method 2. \\ \tt \\ \tt In this hypothetical world, we have a complex causal structure involving direct effects between maternal smoking status, infant's birth weight, and freckles. The presence of unobserved health condition further complicates the analysis. \\ \tt \\ \tt Method 1, which looks directly at the correlation between maternal smoking status and freckles in general, is subject to confounding bias due to the unobserved health condition. Since health condition affects both maternal smoking status and freckles, and is not accounted for in Method 1, the observed correlation between maternal smoking status and freckles will be biased.\\ \tt \\ \tt Method 2, which looks at the correlation case by case according to infant's birth weight, is a form of stratification or matching that can help to control for the confounding effect of birth weight. Since birth weight is a direct effect of both maternal smoking status and health condition, and also affects freckles, stratifying by birth weight can help to isolate the effect of maternal smoking status on freckles, while still accounting for the effect of health condition.\\ \tt \\ \tt In this scenario, Method 2 is a more correct approach because it allows for the control of a common effect (birth weight) that is influenced by both the treatment (maternal smoking status) and the confounder (health condition). This can help to provide a more accurate estimate of the causal effect of maternal smoking status on freckles. 
	  } 
	   } 
	   } 
	 & & \\ 
 

    \theutterance \stepcounter{utterance}  

    & & & \multicolumn{2}{p{0.3\linewidth}}{\cellcolor[rgb]{0.95,0.95,0.95}{%
	\makecell[{{p{\linewidth}}}]{% 
	  \tt {\tiny [GM$|$GM]}  
	 No 
	  } 
	   } 
	   } 
	 & & \\ 
 

    \theutterance \stepcounter{utterance}  

    & & & \multicolumn{2}{p{0.3\linewidth}}{\cellcolor[rgb]{0.95,0.95,0.95}{%
	\makecell[{{p{\linewidth}}}]{% 
	  \tt {\tiny [GM$|$GM]}  
	 yes 
	  } 
	   } 
	   } 
	 & & \\ 
 

    \theutterance \stepcounter{utterance}  

    & & & \multicolumn{2}{p{0.3\linewidth}}{\cellcolor[rgb]{0.95,0.95,0.95}{%
	\makecell[{{p{\linewidth}}}]{% 
	  \tt {\tiny [GM$|$GM]}  
	 game_result = LOSE 
	  } 
	   } 
	   } 
	 & & \\ 
 

\end{supertabular}
}

\end{document}
