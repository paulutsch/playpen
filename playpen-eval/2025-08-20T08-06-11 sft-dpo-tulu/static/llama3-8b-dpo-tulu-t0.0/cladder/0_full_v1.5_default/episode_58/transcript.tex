
\documentclass{article}
\usepackage{colortbl}
\usepackage{makecell}
\usepackage{multirow}
\usepackage{supertabular}

\begin{document}

\newcounter{utterance}

\twocolumn

{ \footnotesize  \setcounter{utterance}{1}
\setlength{\tabcolsep}{0pt}
\begin{supertabular}{c@{$\;$}|p{.15\linewidth}@{}p{.15\linewidth}p{.15\linewidth}p{.15\linewidth}p{.15\linewidth}p{.15\linewidth}}

    \# & $\;$A & \multicolumn{4}{c}{Game Master} & $\;\:$B\\
    \hline 

    \theutterance \stepcounter{utterance}  

    & & \multicolumn{4}{p{0.6\linewidth}}{\cellcolor[rgb]{0.9,0.9,0.9}{%
	\makecell[{{p{\linewidth}}}]{% 
	  \tt {\tiny [A$\langle$GM]}  
	 You are an expert in causal inference. The following question is not a typical commonsense query, but rather a meticulously designed question created by a professor specializing in causal inference, intended to assess the students' mastery of the course content.\\ \tt Imagine a self-contained, hypothetical world with only the following conditions, and without any unmentioned factors or causal relationships: Vaccination status has a direct effect on vaccination reaction and getting smallpox. Getting smallpox has a direct effect on black hair. Vaccination reaction has a direct effect on black hair. For unvaccinated individuals, the probability of black hair is 27%. For vaccinated individuals, the probability of black hair is 37%. Will vaccination decrease the chance of black hair?\\ \tt Start your answer with 'Yes' or 'No', followed by additional reasoning or evidence to support your explanation. 
	  } 
	   } 
	   } 
	 & \\ 
 

    \theutterance \stepcounter{utterance}  

    & \multicolumn{4}{p{0.6\linewidth}}{\cellcolor[rgb]{0.8,1,0.9}{%
	\makecell[{{p{\linewidth}}}]{% 
	  \tt {\tiny [A$\rangle$GM]}  
	 No\\ \tt \\ \tt To answer this question, we need to carefully consider the causal relationships and their effects. We have the following direct effects:\\ \tt \\ \tt 1. Vaccination status affects vaccination reaction.\\ \tt 2. Vaccination status affects getting smallpox.\\ \tt 3. Vaccination reaction affects black hair.\\ \tt 4. Getting smallpox affects black hair.\\ \tt \\ \tt We are given the following probabilities:\\ \tt \\ \tt - For unvaccinated individuals, the probability of black hair is 27%.\\ \tt - For vaccinated individuals, the probability of black hair is 37%.\\ \tt \\ \tt To determine whether vaccination decreases the chance of black hair, we need to consider the total effect of vaccination on black hair, which involves both the direct effect of vaccination on black hair and the indirect effects through getting smallpox and vaccination reaction.\\ \tt \\ \tt The direct effect of vaccination on black hair is not explicitly stated, but we can infer it from the given probabilities. The probability of black hair for vaccinated individuals (37%) is higher than for unvaccinated individuals (27%). This suggests that vaccination has a positive direct effect on black hair.\\ \tt \\ \tt However, the indirect effects of vaccination on black hair through getting smallpox and vaccination reaction are also important. Getting smallpox has a direct effect on black hair, and vaccination status affects getting smallpox. Since vaccination reduces the chance of getting smallpox, this might seem to suggest that vaccination decreases the chance of black hair. However, we also need to consider the effect of vaccination reaction on black hair.\\ \tt \\ \tt The key insight here is that 
	  } 
	   } 
	   } 
	 & & \\ 
 

    \theutterance \stepcounter{utterance}  

    & & & \multicolumn{2}{p{0.3\linewidth}}{\cellcolor[rgb]{0.95,0.95,0.95}{%
	\makecell[{{p{\linewidth}}}]{% 
	  \tt {\tiny [GM$|$GM]}  
	 No 
	  } 
	   } 
	   } 
	 & & \\ 
 

    \theutterance \stepcounter{utterance}  

    & & & \multicolumn{2}{p{0.3\linewidth}}{\cellcolor[rgb]{0.95,0.95,0.95}{%
	\makecell[{{p{\linewidth}}}]{% 
	  \tt {\tiny [GM$|$GM]}  
	 no 
	  } 
	   } 
	   } 
	 & & \\ 
 

    \theutterance \stepcounter{utterance}  

    & & & \multicolumn{2}{p{0.3\linewidth}}{\cellcolor[rgb]{0.95,0.95,0.95}{%
	\makecell[{{p{\linewidth}}}]{% 
	  \tt {\tiny [GM$|$GM]}  
	 game_result = WIN 
	  } 
	   } 
	   } 
	 & & \\ 
 

\end{supertabular}
}

\end{document}
