
\documentclass{article}
\usepackage{colortbl}
\usepackage{makecell}
\usepackage{multirow}
\usepackage{supertabular}

\begin{document}

\newcounter{utterance}

\twocolumn

{ \footnotesize  \setcounter{utterance}{1}
\setlength{\tabcolsep}{0pt}
\begin{supertabular}{c@{$\;$}|p{.15\linewidth}@{}p{.15\linewidth}p{.15\linewidth}p{.15\linewidth}p{.15\linewidth}p{.15\linewidth}}

    \# & $\;$A & \multicolumn{4}{c}{Game Master} & $\;\:$B\\
    \hline 

    \theutterance \stepcounter{utterance}  

    & & \multicolumn{4}{p{0.6\linewidth}}{\cellcolor[rgb]{0.9,0.9,0.9}{%
	\makecell[{{p{\linewidth}}}]{% 
	  \tt {\tiny [A$\langle$GM]}  
	 You are an expert in causal inference. The following question is not a typical commonsense query, but rather a meticulously designed question created by a professor specializing in causal inference, intended to assess the students' mastery of the course content.\\ \tt Imagine a self-contained, hypothetical world with only the following conditions, and without any unmentioned factors or causal relationships: Kidney stone size has a direct effect on treatment and lip thickness. Treatment has a direct effect on lip thickness. For patients who have small kidney stones and not receiving treatment, the probability of thick lips is 6%. For patients who have small kidney stones and receiving treatment, the probability of thick lips is 38%. For patients who have large kidney stones and not receiving treatment, the probability of thick lips is 63%. For patients who have large kidney stones and receiving treatment, the probability of thick lips is 95%. The overall probability of large kidney stone is 50%. For patients receiving treatment, would it be more likely to see thick lips if the patient had received no treatment?\\ \tt Start your answer with 'Yes' or 'No', followed by additional reasoning or evidence to support your explanation. 
	  } 
	   } 
	   } 
	 & \\ 
 

    \theutterance \stepcounter{utterance}  

    & \multicolumn{4}{p{0.6\linewidth}}{\cellcolor[rgb]{0.8,1,0.9}{%
	\makecell[{{p{\linewidth}}}]{% 
	  \tt {\tiny [A$\rangle$GM]}  
	 No\\ \tt \\ \tt To answer this question, we need to apply the principles of causal inference, specifically the concept of counterfactuals. We are asked to compare the probability of thick lips among patients who received treatment with the probability of thick lips among patients who would have had no treatment, had they not received treatment.\\ \tt \\ \tt Let's denote the treatment as T, the kidney stone size as S (small or large), and the lip thickness as L (thick or not thick). We are given the following probabilities:\\ \tt \\ \tt - P(L|S = small, T = no) = 0.06\\ \tt - P(L|S = small, T = yes) = 0.38\\ \tt - P(L|S = large, T = no) = 0.63\\ \tt - P(L|S = large, T = yes) = 0.95\\ \tt - P(S = large) = 0.5\\ \tt \\ \tt We want to compare P(L|T = yes) with P(L|T = no), where T = no is the counterfactual scenario of not receiving treatment.\\ \tt \\ \tt To do this, we need to calculate the probability of thick lips among patients who received treatment, P(L|T = yes). We can use the law of total probability:\\ \tt \\ \tt P(L|T = yes) = P(S = small|T = yes) \* P(L|S = small, T = yes) + P(S = large|T = yes) \* P(L|S 
	  } 
	   } 
	   } 
	 & & \\ 
 

    \theutterance \stepcounter{utterance}  

    & & & \multicolumn{2}{p{0.3\linewidth}}{\cellcolor[rgb]{0.95,0.95,0.95}{%
	\makecell[{{p{\linewidth}}}]{% 
	  \tt {\tiny [GM$|$GM]}  
	 No 
	  } 
	   } 
	   } 
	 & & \\ 
 

    \theutterance \stepcounter{utterance}  

    & & & \multicolumn{2}{p{0.3\linewidth}}{\cellcolor[rgb]{0.95,0.95,0.95}{%
	\makecell[{{p{\linewidth}}}]{% 
	  \tt {\tiny [GM$|$GM]}  
	 no 
	  } 
	   } 
	   } 
	 & & \\ 
 

    \theutterance \stepcounter{utterance}  

    & & & \multicolumn{2}{p{0.3\linewidth}}{\cellcolor[rgb]{0.95,0.95,0.95}{%
	\makecell[{{p{\linewidth}}}]{% 
	  \tt {\tiny [GM$|$GM]}  
	 game_result = WIN 
	  } 
	   } 
	   } 
	 & & \\ 
 

\end{supertabular}
}

\end{document}
