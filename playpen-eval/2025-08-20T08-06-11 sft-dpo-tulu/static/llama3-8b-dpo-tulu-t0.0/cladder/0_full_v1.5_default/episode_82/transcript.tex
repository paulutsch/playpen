
\documentclass{article}
\usepackage{colortbl}
\usepackage{makecell}
\usepackage{multirow}
\usepackage{supertabular}

\begin{document}

\newcounter{utterance}

\twocolumn

{ \footnotesize  \setcounter{utterance}{1}
\setlength{\tabcolsep}{0pt}
\begin{supertabular}{c@{$\;$}|p{.15\linewidth}@{}p{.15\linewidth}p{.15\linewidth}p{.15\linewidth}p{.15\linewidth}p{.15\linewidth}}

    \# & $\;$A & \multicolumn{4}{c}{Game Master} & $\;\:$B\\
    \hline 

    \theutterance \stepcounter{utterance}  

    & & \multicolumn{4}{p{0.6\linewidth}}{\cellcolor[rgb]{0.9,0.9,0.9}{%
	\makecell[{{p{\linewidth}}}]{% 
	  \tt {\tiny [A$\langle$GM]}  
	 You are an expert in causal inference. The following question is not a typical commonsense query, but rather a meticulously designed question created by a professor specializing in causal inference, intended to assess the students' mastery of the course content.\\ \tt Imagine a self-contained, hypothetical world with only the following conditions, and without any unmentioned factors or causal relationships: Kwox has a direct effect on kwoz. Swoq has a direct effect on kwoz. The overall probability of kwox is 1%. For those who are not kwox, the probability of kwoz is 47%. For those who are kwox, the probability of kwoz is 92%. Is kwoz less likely than not kwoz overall?\\ \tt Start your answer with 'Yes' or 'No', followed by additional reasoning or evidence to support your explanation. 
	  } 
	   } 
	   } 
	 & \\ 
 

    \theutterance \stepcounter{utterance}  

    & \multicolumn{4}{p{0.6\linewidth}}{\cellcolor[rgb]{0.8,1,0.9}{%
	\makecell[{{p{\linewidth}}}]{% 
	  \tt {\tiny [A$\rangle$GM]}  
	 No\\ \tt \\ \tt To determine whether kwoz is less likely than not kwoz overall, we need to calculate the overall probability of kwoz. We can do this by considering two groups: those who are kwox and those who are not kwox.\\ \tt \\ \tt - The probability of kwox is 1%.\\ \tt - For those who are not kwox, the probability of kwoz is 47%.\\ \tt - For those who are kwox, the probability of kwoz is 92%.\\ \tt \\ \tt We can calculate the overall probability of kwoz by considering the probability of kwox and the conditional probability of kwoz given kwox, as well as the probability of not kwox and the conditional probability of kwoz given not kwox. This is known as the law of total probability.\\ \tt \\ \tt Let P(kwoz) be the overall probability of kwoz, P(kwox) be the probability of kwox, P(kwoz|kwox) be the probability of kwoz given kwox, P(kwoz|not kwox) be the probability of kwoz given not kwox, and P(not kwox) be the probability of not kwox.\\ \tt \\ \tt We know that P(kwox) = 0.01, P(kwoz|kwox) = 0.92, and P(kwoz|not kwox) = 0 
	  } 
	   } 
	   } 
	 & & \\ 
 

    \theutterance \stepcounter{utterance}  

    & & & \multicolumn{2}{p{0.3\linewidth}}{\cellcolor[rgb]{0.95,0.95,0.95}{%
	\makecell[{{p{\linewidth}}}]{% 
	  \tt {\tiny [GM$|$GM]}  
	 No 
	  } 
	   } 
	   } 
	 & & \\ 
 

    \theutterance \stepcounter{utterance}  

    & & & \multicolumn{2}{p{0.3\linewidth}}{\cellcolor[rgb]{0.95,0.95,0.95}{%
	\makecell[{{p{\linewidth}}}]{% 
	  \tt {\tiny [GM$|$GM]}  
	 yes 
	  } 
	   } 
	   } 
	 & & \\ 
 

    \theutterance \stepcounter{utterance}  

    & & & \multicolumn{2}{p{0.3\linewidth}}{\cellcolor[rgb]{0.95,0.95,0.95}{%
	\makecell[{{p{\linewidth}}}]{% 
	  \tt {\tiny [GM$|$GM]}  
	 game_result = LOSE 
	  } 
	   } 
	   } 
	 & & \\ 
 

\end{supertabular}
}

\end{document}
