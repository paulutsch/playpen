
\documentclass{article}
\usepackage{colortbl}
\usepackage{makecell}
\usepackage{multirow}
\usepackage{supertabular}

\begin{document}

\newcounter{utterance}

\twocolumn

{ \footnotesize  \setcounter{utterance}{1}
\setlength{\tabcolsep}{0pt}
\begin{supertabular}{c@{$\;$}|p{.15\linewidth}@{}p{.15\linewidth}p{.15\linewidth}p{.15\linewidth}p{.15\linewidth}p{.15\linewidth}}

    \# & $\;$A & \multicolumn{4}{c}{Game Master} & $\;\:$B\\
    \hline 

    \theutterance \stepcounter{utterance}  

    & & \multicolumn{4}{p{0.6\linewidth}}{\cellcolor[rgb]{0.9,0.9,0.9}{%
	\makecell[{{p{\linewidth}}}]{% 
	  \tt {\tiny [A$\langle$GM]}  
	 You are an expert in causal inference. The following question is not a typical commonsense query, but rather a meticulously designed question created by a professor specializing in causal inference, intended to assess the students' mastery of the course content.\\ \tt Imagine a self-contained, hypothetical world with only the following conditions, and without any unmentioned factors or causal relationships: Xevo has a direct effect on tijv and gyzp. Tijw has a direct effect on tijv and gyzp. Tijv has a direct effect on gyzp. Tijw is unobserved. We know that xevo or tijw causes tijv. xevo or tijw or tijv causes gyzp. We observed an individual is tijw. Would an individual is not gyzp if not xevo instead of xevo?\\ \tt Start your answer with 'Yes' or 'No', followed by additional reasoning or evidence to support your explanation. 
	  } 
	   } 
	   } 
	 & \\ 
 

    \theutterance \stepcounter{utterance}  

    & \multicolumn{4}{p{0.6\linewidth}}{\cellcolor[rgb]{0.8,1,0.9}{%
	\makecell[{{p{\linewidth}}}]{% 
	  \tt {\tiny [A$\rangle$GM]}  
	 No\\ \tt \\ \tt To answer this question, we need to apply the principles of causal inference, specifically the concept of causal graphs and the backdoor criterion. \\ \tt \\ \tt First, let's draw the causal graph based on the given conditions:\\ \tt \\ \tt - Xevo -> Tijv\\ \tt - Xevo -> Gyzp\\ \tt - Tijw -> Tijv\\ \tt - Tijw -> Gyzp\\ \tt - Tijv -> Gyzp\\ \tt \\ \tt Since Tijw is unobserved, we need to use the backdoor criterion to identify the causal effect of Xevo on Gyzp. The backdoor criterion states that a path from Xevo to Gyzp is blocked if we condition on all the variables that are on the path from Xevo to Gyzp, except for the variable that is the target of the causal effect (Gyzp).\\ \tt \\ \tt In this case, the path from Xevo to Gyzp is Xevo -> Tijv -> Gyzp. To block this path, we need to condition on Tijv.\\ \tt \\ \tt However, we are given that Xevo or Tijw causes Tijv. This means that Tijv is a common effect of both Xevo and Tijw. Therefore, conditioning on Tijv will not block the path from Xevo to Gyzp, because Tijv is not a collider (a variable that is the target of two causal effects).\\ \tt \\ \tt Now, let's 
	  } 
	   } 
	   } 
	 & & \\ 
 

    \theutterance \stepcounter{utterance}  

    & & & \multicolumn{2}{p{0.3\linewidth}}{\cellcolor[rgb]{0.95,0.95,0.95}{%
	\makecell[{{p{\linewidth}}}]{% 
	  \tt {\tiny [GM$|$GM]}  
	 No 
	  } 
	   } 
	   } 
	 & & \\ 
 

    \theutterance \stepcounter{utterance}  

    & & & \multicolumn{2}{p{0.3\linewidth}}{\cellcolor[rgb]{0.95,0.95,0.95}{%
	\makecell[{{p{\linewidth}}}]{% 
	  \tt {\tiny [GM$|$GM]}  
	 no 
	  } 
	   } 
	   } 
	 & & \\ 
 

    \theutterance \stepcounter{utterance}  

    & & & \multicolumn{2}{p{0.3\linewidth}}{\cellcolor[rgb]{0.95,0.95,0.95}{%
	\makecell[{{p{\linewidth}}}]{% 
	  \tt {\tiny [GM$|$GM]}  
	 game_result = WIN 
	  } 
	   } 
	   } 
	 & & \\ 
 

\end{supertabular}
}

\end{document}
