
\documentclass{article}
\usepackage{colortbl}
\usepackage{makecell}
\usepackage{multirow}
\usepackage{supertabular}

\begin{document}

\newcounter{utterance}

\twocolumn

{ \footnotesize  \setcounter{utterance}{1}
\setlength{\tabcolsep}{0pt}
\begin{supertabular}{c@{$\;$}|p{.15\linewidth}@{}p{.15\linewidth}p{.15\linewidth}p{.15\linewidth}p{.15\linewidth}p{.15\linewidth}}

    \# & $\;$A & \multicolumn{4}{c}{Game Master} & $\;\:$B\\
    \hline 

    \theutterance \stepcounter{utterance}  

    & & \multicolumn{4}{p{0.6\linewidth}}{\cellcolor[rgb]{0.9,0.9,0.9}{%
	\makecell[{{p{\linewidth}}}]{% 
	  \tt {\tiny [A$\langle$GM]}  
	 You are an expert in causal inference. The following question is not a typical commonsense query, but rather a meticulously designed question created by a professor specializing in causal inference, intended to assess the students' mastery of the course content.\\ \tt Imagine a self-contained, hypothetical world with only the following conditions, and without any unmentioned factors or causal relationships: Kidney stone size has a direct effect on treatment and recovery. Treatment has a direct effect on recovery. Method 1: We look directly at how treatment correlates with recovery in general. Method 2: We look at this correlation case by case according to kidney stone size. To understand how treatment affects recovery, is it more correct to use the Method 1 than Method 2?\\ \tt Start your answer with 'Yes' or 'No', followed by additional reasoning or evidence to support your explanation. 
	  } 
	   } 
	   } 
	 & \\ 
 

    \theutterance \stepcounter{utterance}  

    & \multicolumn{4}{p{0.6\linewidth}}{\cellcolor[rgb]{0.8,1,0.9}{%
	\makecell[{{p{\linewidth}}}]{% 
	  \tt {\tiny [A$\rangle$GM]}  
	 No, it is not more correct to use Method 1 than Method 2. \\ \tt \\ \tt This question is a classic example of the problem of confounding variables in causal inference. In this hypothetical world, kidney stone size is a confounding variable that affects both treatment and recovery. When we look directly at the correlation between treatment and recovery in general (Method 1), we are not accounting for the effect of kidney stone size on both variables. This can lead to biased estimates of the treatment effect, as the relationship between treatment and recovery may be influenced by the size of the kidney stone.\\ \tt \\ \tt On the other hand, when we look at the correlation between treatment and recovery case by case according to kidney stone size (Method 2), we are controlling for the confounding effect of kidney stone size. This is an example of stratification or matching, which is a common technique used in causal inference to account for confounding variables. By stratifying the data by kidney stone size, we can estimate the treatment effect within each stratum, which can provide a more accurate estimate of the causal relationship between treatment and recovery.\\ \tt \\ \tt In this case, Method 2 is more correct because it accounts for the confounding effect of kidney stone size, which is a key factor that affects both treatment and recovery. This is an example of the principle of "like treats like" in causal inference, which suggests that we should compare similar units (in this case, units with similar kidney stone sizes) to estimate the treatment effect. 
	  } 
	   } 
	   } 
	 & & \\ 
 

    \theutterance \stepcounter{utterance}  

    & & & \multicolumn{2}{p{0.3\linewidth}}{\cellcolor[rgb]{0.95,0.95,0.95}{%
	\makecell[{{p{\linewidth}}}]{% 
	  \tt {\tiny [GM$|$GM]}  
	 No 
	  } 
	   } 
	   } 
	 & & \\ 
 

    \theutterance \stepcounter{utterance}  

    & & & \multicolumn{2}{p{0.3\linewidth}}{\cellcolor[rgb]{0.95,0.95,0.95}{%
	\makecell[{{p{\linewidth}}}]{% 
	  \tt {\tiny [GM$|$GM]}  
	 no 
	  } 
	   } 
	   } 
	 & & \\ 
 

    \theutterance \stepcounter{utterance}  

    & & & \multicolumn{2}{p{0.3\linewidth}}{\cellcolor[rgb]{0.95,0.95,0.95}{%
	\makecell[{{p{\linewidth}}}]{% 
	  \tt {\tiny [GM$|$GM]}  
	 game_result = WIN 
	  } 
	   } 
	   } 
	 & & \\ 
 

\end{supertabular}
}

\end{document}
