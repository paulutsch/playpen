
\documentclass{article}
\usepackage{colortbl}
\usepackage{makecell}
\usepackage{multirow}
\usepackage{supertabular}

\begin{document}

\newcounter{utterance}

\twocolumn

{ \footnotesize  \setcounter{utterance}{1}
\setlength{\tabcolsep}{0pt}
\begin{supertabular}{c@{$\;$}|p{.15\linewidth}@{}p{.15\linewidth}p{.15\linewidth}p{.15\linewidth}p{.15\linewidth}p{.15\linewidth}}

    \# & $\;$A & \multicolumn{4}{c}{Game Master} & $\;\:$B\\
    \hline 

    \theutterance \stepcounter{utterance}  

    & & \multicolumn{4}{p{0.6\linewidth}}{\cellcolor[rgb]{0.9,0.9,0.9}{%
	\makecell[{{p{\linewidth}}}]{% 
	  \tt {\tiny [A$\langle$GM]}  
	 You are an expert in causal inference. The following question is not a typical commonsense query, but rather a meticulously designed question created by a professor specializing in causal inference, intended to assess the students' mastery of the course content.\\ \tt Imagine a self-contained, hypothetical world with only the following conditions, and without any unmentioned factors or causal relationships: Obesity has a direct effect on diabetes and lifespan. Smoking has a direct effect on diabetes and lifespan. Diabetes has a direct effect on lifespan. Smoking is unobserved. For normal weight people and without diabetes, the probability of long lifespan is 91%. For normal weight people and with diabetes, the probability of long lifespan is 71%. For obese people and without diabetes, the probability of long lifespan is 69%. For obese people and with diabetes, the probability of long lifespan is 43%. For normal weight people and nonsmokers, the probability of having diabetes is 28%. For normal weight people and smokers, the probability of having diabetes is 20%. For obese people and nonsmokers, the probability of having diabetes is 57%. For obese people and smokers, the probability of having diabetes is 70%. The overall probability of smoker is 8%. If we disregard the mediation effect through diabetes, would obesity negatively affect lifespan?\\ \tt Start your answer with 'Yes' or 'No', followed by additional reasoning or evidence to support your explanation. 
	  } 
	   } 
	   } 
	 & \\ 
 

    \theutterance \stepcounter{utterance}  

    & \multicolumn{4}{p{0.6\linewidth}}{\cellcolor[rgb]{0.8,1,0.9}{%
	\makecell[{{p{\linewidth}}}]{% 
	  \tt {\tiny [A$\rangle$GM]}  
	 No, obesity would not negatively affect lifespan when disregarding the mediation effect through diabetes.\\ \tt \\ \tt To address this question, we need to compare the probability of long lifespan between normal weight and obese individuals, while controlling for the effect of smoking and diabetes. \\ \tt \\ \tt First, we need to estimate the probability of long lifespan for normal weight and obese individuals, assuming they are nonsmokers and do not have diabetes. We can use the given probabilities to calculate these values:\\ \tt \\ \tt - For normal weight people and without diabetes, the probability of long lifespan is 91%.\\ \tt - For obese people and without diabetes, the probability of long lifespan is 69%.\\ \tt \\ \tt However, we need to account for the effect of smoking on diabetes. We know that the probability of having diabetes for normal weight people and smokers is 20%, and for obese people and smokers is 70%. We also know that the overall probability of being a smoker is 8%. \\ \tt \\ \tt Using the law of total probability, we can estimate the probability of long lifespan for normal weight and obese individuals, assuming they are nonsmokers and do not have diabetes. Let's denote the probability of being a nonsmoker as 1 - 0.08 = 0.92.\\ \tt \\ \tt - The probability of long lifespan for normal weight people and without diabetes is 0.91.\\ \tt - The probability of long lifespan for normal weight people and without diabetes, given that they are smokers, is 0.91 * (1 - 0.20) = 0.728 
	  } 
	   } 
	   } 
	 & & \\ 
 

    \theutterance \stepcounter{utterance}  

    & & & \multicolumn{2}{p{0.3\linewidth}}{\cellcolor[rgb]{0.95,0.95,0.95}{%
	\makecell[{{p{\linewidth}}}]{% 
	  \tt {\tiny [GM$|$GM]}  
	 No 
	  } 
	   } 
	   } 
	 & & \\ 
 

    \theutterance \stepcounter{utterance}  

    & & & \multicolumn{2}{p{0.3\linewidth}}{\cellcolor[rgb]{0.95,0.95,0.95}{%
	\makecell[{{p{\linewidth}}}]{% 
	  \tt {\tiny [GM$|$GM]}  
	 yes 
	  } 
	   } 
	   } 
	 & & \\ 
 

    \theutterance \stepcounter{utterance}  

    & & & \multicolumn{2}{p{0.3\linewidth}}{\cellcolor[rgb]{0.95,0.95,0.95}{%
	\makecell[{{p{\linewidth}}}]{% 
	  \tt {\tiny [GM$|$GM]}  
	 game_result = LOSE 
	  } 
	   } 
	   } 
	 & & \\ 
 

\end{supertabular}
}

\end{document}
