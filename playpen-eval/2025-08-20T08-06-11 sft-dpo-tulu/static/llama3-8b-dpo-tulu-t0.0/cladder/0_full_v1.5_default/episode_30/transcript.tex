
\documentclass{article}
\usepackage{colortbl}
\usepackage{makecell}
\usepackage{multirow}
\usepackage{supertabular}

\begin{document}

\newcounter{utterance}

\twocolumn

{ \footnotesize  \setcounter{utterance}{1}
\setlength{\tabcolsep}{0pt}
\begin{supertabular}{c@{$\;$}|p{.15\linewidth}@{}p{.15\linewidth}p{.15\linewidth}p{.15\linewidth}p{.15\linewidth}p{.15\linewidth}}

    \# & $\;$A & \multicolumn{4}{c}{Game Master} & $\;\:$B\\
    \hline 

    \theutterance \stepcounter{utterance}  

    & & \multicolumn{4}{p{0.6\linewidth}}{\cellcolor[rgb]{0.9,0.9,0.9}{%
	\makecell[{{p{\linewidth}}}]{% 
	  \tt {\tiny [A$\langle$GM]}  
	 You are an expert in causal inference. The following question is not a typical commonsense query, but rather a meticulously designed question created by a professor specializing in causal inference, intended to assess the students' mastery of the course content.\\ \tt Imagine a self-contained, hypothetical world with only the following conditions, and without any unmentioned factors or causal relationships: Demand has a direct effect on supply and price. Yield per acre has a direct effect on supply. Supply has a direct effect on price. Demand is unobserved. The overall probability of increased supply is 78%. For situations with reduced supply, the probability of increased price is 44%. For situations with increased supply, the probability of increased price is 81%. Is increased price less likely than reduced price overall?\\ \tt Start your answer with 'Yes' or 'No', followed by additional reasoning or evidence to support your explanation. 
	  } 
	   } 
	   } 
	 & \\ 
 

    \theutterance \stepcounter{utterance}  

    & \multicolumn{4}{p{0.6\linewidth}}{\cellcolor[rgb]{0.8,1,0.9}{%
	\makecell[{{p{\linewidth}}}]{% 
	  \tt {\tiny [A$\rangle$GM]}  
	 No, increased price is not less likely than reduced price overall. \\ \tt \\ \tt To answer this question, we need to apply the principles of causal inference, specifically the concept of counterfactuals and potential outcomes. Let's break down the given information:\\ \tt \\ \tt 1. Demand has a direct effect on supply and price.\\ \tt 2. Yield per acre has a direct effect on supply.\\ \tt 3. Supply has a direct effect on price.\\ \tt 4. Demand is unobserved.\\ \tt 5. The overall probability of increased supply is 78%.\\ \tt 6. For situations with reduced supply, the probability of increased price is 44%.\\ \tt 7. For situations with increased supply, the probability of increased price is 81%.\\ \tt \\ \tt We are interested in comparing the likelihood of increased price versus reduced price. To do this, we need to consider the potential outcomes of supply changes. Let's define two counterfactuals:\\ \tt \\ \tt - Counterfactual 1 (CF1): Reduced supply\\ \tt - Counterfactual 2 (CF2): Increased supply\\ \tt \\ \tt From the given information, we know that:\\ \tt \\ \tt - The probability of CF1 is 1 - 0.78 = 0.22 (since the overall probability of increased supply is 78%).\\ \tt - For CF1, the probability of increased price is 0.44.\\ \tt - The probability of CF2 is 0.78.\\ \tt - For CF2, the probability of increased price is 0.81.\\ \tt \\ \tt Now, let's consider the potential outcomes of price changes. We are 
	  } 
	   } 
	   } 
	 & & \\ 
 

    \theutterance \stepcounter{utterance}  

    & & & \multicolumn{2}{p{0.3\linewidth}}{\cellcolor[rgb]{0.95,0.95,0.95}{%
	\makecell[{{p{\linewidth}}}]{% 
	  \tt {\tiny [GM$|$GM]}  
	 No 
	  } 
	   } 
	   } 
	 & & \\ 
 

    \theutterance \stepcounter{utterance}  

    & & & \multicolumn{2}{p{0.3\linewidth}}{\cellcolor[rgb]{0.95,0.95,0.95}{%
	\makecell[{{p{\linewidth}}}]{% 
	  \tt {\tiny [GM$|$GM]}  
	 no 
	  } 
	   } 
	   } 
	 & & \\ 
 

    \theutterance \stepcounter{utterance}  

    & & & \multicolumn{2}{p{0.3\linewidth}}{\cellcolor[rgb]{0.95,0.95,0.95}{%
	\makecell[{{p{\linewidth}}}]{% 
	  \tt {\tiny [GM$|$GM]}  
	 game_result = WIN 
	  } 
	   } 
	   } 
	 & & \\ 
 

\end{supertabular}
}

\end{document}
