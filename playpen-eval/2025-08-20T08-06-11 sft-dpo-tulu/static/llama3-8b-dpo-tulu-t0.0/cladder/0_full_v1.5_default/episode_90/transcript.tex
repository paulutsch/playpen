
\documentclass{article}
\usepackage{colortbl}
\usepackage{makecell}
\usepackage{multirow}
\usepackage{supertabular}

\begin{document}

\newcounter{utterance}

\twocolumn

{ \footnotesize  \setcounter{utterance}{1}
\setlength{\tabcolsep}{0pt}
\begin{supertabular}{c@{$\;$}|p{.15\linewidth}@{}p{.15\linewidth}p{.15\linewidth}p{.15\linewidth}p{.15\linewidth}p{.15\linewidth}}

    \# & $\;$A & \multicolumn{4}{c}{Game Master} & $\;\:$B\\
    \hline 

    \theutterance \stepcounter{utterance}  

    & & \multicolumn{4}{p{0.6\linewidth}}{\cellcolor[rgb]{0.9,0.9,0.9}{%
	\makecell[{{p{\linewidth}}}]{% 
	  \tt {\tiny [A$\langle$GM]}  
	 You are an expert in causal inference. The following question is not a typical commonsense query, but rather a meticulously designed question created by a professor specializing in causal inference, intended to assess the students' mastery of the course content.\\ \tt Imagine a self-contained, hypothetical world with only the following conditions, and without any unmentioned factors or causal relationships: Pexu has a direct effect on rukz. Hwax has a direct effect on rukz. For those who are not pexu, the probability of rukz is 9%. For those who are pexu, the probability of rukz is 18%. For those who are pexu, would it be less likely to see rukz if the individual was not pexu?\\ \tt Start your answer with 'Yes' or 'No', followed by additional reasoning or evidence to support your explanation. 
	  } 
	   } 
	   } 
	 & \\ 
 

    \theutterance \stepcounter{utterance}  

    & \multicolumn{4}{p{0.6\linewidth}}{\cellcolor[rgb]{0.8,1,0.9}{%
	\makecell[{{p{\linewidth}}}]{% 
	  \tt {\tiny [A$\rangle$GM]}  
	 No, it would not be less likely to see rukz if the individual was not pexu for those who are pexu.\\ \tt \\ \tt To understand this, let's break down the given information:\\ \tt \\ \tt - The probability of rukz for those who are not pexu is 9%.\\ \tt - The probability of rukz for those who are pexu is 18%.\\ \tt \\ \tt The question asks whether it would be less likely to see rukz if the individual was not pexu, given that they are pexu. This is a counterfactual question, which is a fundamental concept in causal inference. \\ \tt \\ \tt To answer this, we need to consider the potential outcome if the individual was not pexu, but we know they are pexu. The given information does not provide a direct comparison of the probability of rukz between being pexu and not being pexu for an individual who is pexu. However, we can infer the answer by considering the effect of pexu on rukz.\\ \tt \\ \tt The presence of pexu increases the probability of rukz from 9% to 18%. This indicates that pexu has a positive effect on rukz. If we were to remove the effect of pexu (i.e., consider the individual as not pexu), the probability of rukz would decrease from 18% to 9%. Therefore, it would be less likely to see rukz if the individual was not pexu, but 
	  } 
	   } 
	   } 
	 & & \\ 
 

    \theutterance \stepcounter{utterance}  

    & & & \multicolumn{2}{p{0.3\linewidth}}{\cellcolor[rgb]{0.95,0.95,0.95}{%
	\makecell[{{p{\linewidth}}}]{% 
	  \tt {\tiny [GM$|$GM]}  
	 No 
	  } 
	   } 
	   } 
	 & & \\ 
 

    \theutterance \stepcounter{utterance}  

    & & & \multicolumn{2}{p{0.3\linewidth}}{\cellcolor[rgb]{0.95,0.95,0.95}{%
	\makecell[{{p{\linewidth}}}]{% 
	  \tt {\tiny [GM$|$GM]}  
	 yes 
	  } 
	   } 
	   } 
	 & & \\ 
 

    \theutterance \stepcounter{utterance}  

    & & & \multicolumn{2}{p{0.3\linewidth}}{\cellcolor[rgb]{0.95,0.95,0.95}{%
	\makecell[{{p{\linewidth}}}]{% 
	  \tt {\tiny [GM$|$GM]}  
	 game_result = LOSE 
	  } 
	   } 
	   } 
	 & & \\ 
 

\end{supertabular}
}

\end{document}
