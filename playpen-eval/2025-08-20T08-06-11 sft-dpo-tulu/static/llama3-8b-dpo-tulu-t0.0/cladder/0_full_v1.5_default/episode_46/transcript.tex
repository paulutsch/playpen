
\documentclass{article}
\usepackage{colortbl}
\usepackage{makecell}
\usepackage{multirow}
\usepackage{supertabular}

\begin{document}

\newcounter{utterance}

\twocolumn

{ \footnotesize  \setcounter{utterance}{1}
\setlength{\tabcolsep}{0pt}
\begin{supertabular}{c@{$\;$}|p{.15\linewidth}@{}p{.15\linewidth}p{.15\linewidth}p{.15\linewidth}p{.15\linewidth}p{.15\linewidth}}

    \# & $\;$A & \multicolumn{4}{c}{Game Master} & $\;\:$B\\
    \hline 

    \theutterance \stepcounter{utterance}  

    & & \multicolumn{4}{p{0.6\linewidth}}{\cellcolor[rgb]{0.9,0.9,0.9}{%
	\makecell[{{p{\linewidth}}}]{% 
	  \tt {\tiny [A$\langle$GM]}  
	 You are an expert in causal inference. The following question is not a typical commonsense query, but rather a meticulously designed question created by a professor specializing in causal inference, intended to assess the students' mastery of the course content.\\ \tt Imagine a self-contained, hypothetical world with only the following conditions, and without any unmentioned factors or causal relationships: Kidney stone size has a direct effect on treatment and lip thickness. Treatment has a direct effect on lip thickness. The overall probability of receives treatment is 48%. For patients not receiving treatment, the probability of thick lips is 64%. For patients receiving treatment, the probability of thick lips is 29%. Is thick lips more likely than thin lips overall?\\ \tt Start your answer with 'Yes' or 'No', followed by additional reasoning or evidence to support your explanation. 
	  } 
	   } 
	   } 
	 & \\ 
 

    \theutterance \stepcounter{utterance}  

    & \multicolumn{4}{p{0.6\linewidth}}{\cellcolor[rgb]{0.8,1,0.9}{%
	\makecell[{{p{\linewidth}}}]{% 
	  \tt {\tiny [A$\rangle$GM]}  
	 No, thick lips are not more likely than thin lips overall.\\ \tt \\ \tt To answer this question, we need to apply the principles of causal inference, specifically the concept of counterfactuals and potential outcomes. Let's break down the given information:\\ \tt \\ \tt 1. Kidney stone size affects treatment and lip thickness.\\ \tt 2. Treatment affects lip thickness.\\ \tt 3. The probability of receiving treatment is 48%.\\ \tt 4. For patients not receiving treatment, the probability of thick lips is 64%.\\ \tt 5. For patients receiving treatment, the probability of thick lips is 29%.\\ \tt \\ \tt We are asked to determine whether thick lips are more likely than thin lips overall. To do this, we need to consider the potential outcomes for both treated and untreated individuals.\\ \tt \\ \tt Let's define two potential outcomes for lip thickness: T (thick lips) and N (thin lips). We can use the given probabilities to estimate the potential outcomes for treated and untreated individuals:\\ \tt \\ \tt - For untreated individuals (1 - 0.48 = 0.52): P(T|not treated) = 0.64, P(N|not treated) = 0.36\\ \tt - For treated individuals (0.48): P(T|treated) = 0.29, P(N|treated) = 0.71\\ \tt \\ \tt Now, we need to consider the counterfactual: what would happen if all individuals were untreated? In this hypothetical scenario, the probability of thick lips would be 0.64 for all individuals. This is 
	  } 
	   } 
	   } 
	 & & \\ 
 

    \theutterance \stepcounter{utterance}  

    & & & \multicolumn{2}{p{0.3\linewidth}}{\cellcolor[rgb]{0.95,0.95,0.95}{%
	\makecell[{{p{\linewidth}}}]{% 
	  \tt {\tiny [GM$|$GM]}  
	 No 
	  } 
	   } 
	   } 
	 & & \\ 
 

    \theutterance \stepcounter{utterance}  

    & & & \multicolumn{2}{p{0.3\linewidth}}{\cellcolor[rgb]{0.95,0.95,0.95}{%
	\makecell[{{p{\linewidth}}}]{% 
	  \tt {\tiny [GM$|$GM]}  
	 no 
	  } 
	   } 
	   } 
	 & & \\ 
 

    \theutterance \stepcounter{utterance}  

    & & & \multicolumn{2}{p{0.3\linewidth}}{\cellcolor[rgb]{0.95,0.95,0.95}{%
	\makecell[{{p{\linewidth}}}]{% 
	  \tt {\tiny [GM$|$GM]}  
	 game_result = WIN 
	  } 
	   } 
	   } 
	 & & \\ 
 

\end{supertabular}
}

\end{document}
