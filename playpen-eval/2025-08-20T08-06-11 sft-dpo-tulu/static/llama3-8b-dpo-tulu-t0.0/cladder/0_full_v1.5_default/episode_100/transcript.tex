
\documentclass{article}
\usepackage{colortbl}
\usepackage{makecell}
\usepackage{multirow}
\usepackage{supertabular}

\begin{document}

\newcounter{utterance}

\twocolumn

{ \footnotesize  \setcounter{utterance}{1}
\setlength{\tabcolsep}{0pt}
\begin{supertabular}{c@{$\;$}|p{.15\linewidth}@{}p{.15\linewidth}p{.15\linewidth}p{.15\linewidth}p{.15\linewidth}p{.15\linewidth}}

    \# & $\;$A & \multicolumn{4}{c}{Game Master} & $\;\:$B\\
    \hline 

    \theutterance \stepcounter{utterance}  

    & & \multicolumn{4}{p{0.6\linewidth}}{\cellcolor[rgb]{0.9,0.9,0.9}{%
	\makecell[{{p{\linewidth}}}]{% 
	  \tt {\tiny [A$\langle$GM]}  
	 You are an expert in causal inference. The following question is not a typical commonsense query, but rather a meticulously designed question created by a professor specializing in causal inference, intended to assess the students' mastery of the course content.\\ \tt Imagine a self-contained, hypothetical world with only the following conditions, and without any unmentioned factors or causal relationships: Jyka has a direct effect on zory and zuph. Zuph has a direct effect on glimx. Zory has a direct effect on glimx. We know that jyka causes not zuph and zory. zuph and zory causes glimx. We observed an individual is jyka. Would an individual is glimx if zuph instead of not zuph?\\ \tt Start your answer with 'Yes' or 'No', followed by additional reasoning or evidence to support your explanation. 
	  } 
	   } 
	   } 
	 & \\ 
 

    \theutterance \stepcounter{utterance}  

    & \multicolumn{4}{p{0.6\linewidth}}{\cellcolor[rgb]{0.8,1,0.9}{%
	\makecell[{{p{\linewidth}}}]{% 
	  \tt {\tiny [A$\rangle$GM]}  
	 No\\ \tt \\ \tt The given conditions imply a causal graph with the following structure:\\ \tt \\ \tt - Jyka -> Zory\\ \tt - Jyka -> Zuph\\ \tt - Zuph -> Glimx\\ \tt - Zory -> Glimx\\ \tt \\ \tt We are given that Jyka causes both Zuph and Zory, and both Zuph and Zory cause Glimx. We are asked to consider a counterfactual scenario where Zuph is instead not Zuph (i.e., Zuph is 0).\\ \tt \\ \tt To determine the effect of this change on Glimx, we need to consider the causal paths from Jyka to Glimx. There are two paths: Jyka -> Zory -> Glimx and Jyka -> Zuph -> Glimx.\\ \tt \\ \tt If Zuph is 0, the first path (Jyka -> Zory -> Glimx) remains unaffected, as Zory's value is determined by Jyka. However, the second path (Jyka -> Zuph -> Glimx) is blocked, as Zuph is 0.\\ \tt \\ \tt Since the second path is blocked, the effect of Jyka on Glimx is mediated solely through the first path (Jyka -> Zory -> Glimx). Therefore, the value of Glimx is determined solely by the value of Zory, which is determined by Jyka.\\ \tt \\ \tt In other words, the causal effect 
	  } 
	   } 
	   } 
	 & & \\ 
 

    \theutterance \stepcounter{utterance}  

    & & & \multicolumn{2}{p{0.3\linewidth}}{\cellcolor[rgb]{0.95,0.95,0.95}{%
	\makecell[{{p{\linewidth}}}]{% 
	  \tt {\tiny [GM$|$GM]}  
	 No 
	  } 
	   } 
	   } 
	 & & \\ 
 

    \theutterance \stepcounter{utterance}  

    & & & \multicolumn{2}{p{0.3\linewidth}}{\cellcolor[rgb]{0.95,0.95,0.95}{%
	\makecell[{{p{\linewidth}}}]{% 
	  \tt {\tiny [GM$|$GM]}  
	 no 
	  } 
	   } 
	   } 
	 & & \\ 
 

    \theutterance \stepcounter{utterance}  

    & & & \multicolumn{2}{p{0.3\linewidth}}{\cellcolor[rgb]{0.95,0.95,0.95}{%
	\makecell[{{p{\linewidth}}}]{% 
	  \tt {\tiny [GM$|$GM]}  
	 game_result = WIN 
	  } 
	   } 
	   } 
	 & & \\ 
 

\end{supertabular}
}

\end{document}
