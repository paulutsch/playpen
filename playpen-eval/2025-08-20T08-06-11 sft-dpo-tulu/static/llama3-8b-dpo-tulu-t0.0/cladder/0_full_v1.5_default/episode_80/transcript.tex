
\documentclass{article}
\usepackage{colortbl}
\usepackage{makecell}
\usepackage{multirow}
\usepackage{supertabular}

\begin{document}

\newcounter{utterance}

\twocolumn

{ \footnotesize  \setcounter{utterance}{1}
\setlength{\tabcolsep}{0pt}
\begin{supertabular}{c@{$\;$}|p{.15\linewidth}@{}p{.15\linewidth}p{.15\linewidth}p{.15\linewidth}p{.15\linewidth}p{.15\linewidth}}

    \# & $\;$A & \multicolumn{4}{c}{Game Master} & $\;\:$B\\
    \hline 

    \theutterance \stepcounter{utterance}  

    & & \multicolumn{4}{p{0.6\linewidth}}{\cellcolor[rgb]{0.9,0.9,0.9}{%
	\makecell[{{p{\linewidth}}}]{% 
	  \tt {\tiny [A$\langle$GM]}  
	 You are an expert in causal inference. The following question is not a typical commonsense query, but rather a meticulously designed question created by a professor specializing in causal inference, intended to assess the students' mastery of the course content.\\ \tt Imagine a self-contained, hypothetical world with only the following conditions, and without any unmentioned factors or causal relationships: Zory has a direct effect on zuph and glimx. Zuph has a direct effect on jyka. Jyka has a direct effect on glimx. Zory is unobserved. The overall probability of zuph is 8%. The probability of not zuph and glimx is 58%. The probability of zuph and glimx is 1%. Is the chance of glimx smaller when observing zuph?\\ \tt Start your answer with 'Yes' or 'No', followed by additional reasoning or evidence to support your explanation. 
	  } 
	   } 
	   } 
	 & \\ 
 

    \theutterance \stepcounter{utterance}  

    & \multicolumn{4}{p{0.6\linewidth}}{\cellcolor[rgb]{0.8,1,0.9}{%
	\makecell[{{p{\linewidth}}}]{% 
	  \tt {\tiny [A$\rangle$GM]}  
	 No\\ \tt \\ \tt To answer this question, we need to apply the principles of causal inference, specifically the concept of confounding variables and the backdoor criterion. \\ \tt \\ \tt Given the conditions:\\ \tt \\ \tt 1. Zory has a direct effect on zuph and glimx.\\ \tt 2. Zuph has a direct effect on jyka.\\ \tt 3. Jyka has a direct effect on glimx.\\ \tt 4. Zory is unobserved.\\ \tt \\ \tt We can represent these relationships using a directed acyclic graph (DAG):\\ \tt \\ \tt Zory -> Zuph\\ \tt Zory -> Glimx\\ \tt Zuph -> Jyka\\ \tt Jyka -> Glimx\\ \tt \\ \tt Since Zory is unobserved, it is a confounding variable. The presence of Zory affects both Zuph and Glimx, which in turn affects Jyka and Glimx. This creates a confounding effect where the relationship between Zuph and Glimx is not causal, but rather due to the common cause Zory.\\ \tt \\ \tt To determine if the chance of Glimx is smaller when observing Zuph, we need to consider the backdoor criterion. The backdoor path from Zuph to Glimx is:\\ \tt \\ \tt Zuph -> Jyka -> Glimx\\ \tt \\ \tt This path is blocked by the observation of Zuph, as we are conditioning on Zuph. However, there is another path from Zuph to Glimx that is not blocked by the 
	  } 
	   } 
	   } 
	 & & \\ 
 

    \theutterance \stepcounter{utterance}  

    & & & \multicolumn{2}{p{0.3\linewidth}}{\cellcolor[rgb]{0.95,0.95,0.95}{%
	\makecell[{{p{\linewidth}}}]{% 
	  \tt {\tiny [GM$|$GM]}  
	 No 
	  } 
	   } 
	   } 
	 & & \\ 
 

    \theutterance \stepcounter{utterance}  

    & & & \multicolumn{2}{p{0.3\linewidth}}{\cellcolor[rgb]{0.95,0.95,0.95}{%
	\makecell[{{p{\linewidth}}}]{% 
	  \tt {\tiny [GM$|$GM]}  
	 yes 
	  } 
	   } 
	   } 
	 & & \\ 
 

    \theutterance \stepcounter{utterance}  

    & & & \multicolumn{2}{p{0.3\linewidth}}{\cellcolor[rgb]{0.95,0.95,0.95}{%
	\makecell[{{p{\linewidth}}}]{% 
	  \tt {\tiny [GM$|$GM]}  
	 game_result = LOSE 
	  } 
	   } 
	   } 
	 & & \\ 
 

\end{supertabular}
}

\end{document}
