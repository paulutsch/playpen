
\documentclass{article}
\usepackage{colortbl}
\usepackage{makecell}
\usepackage{multirow}
\usepackage{supertabular}

\begin{document}

\newcounter{utterance}

\twocolumn

{ \footnotesize  \setcounter{utterance}{1}
\setlength{\tabcolsep}{0pt}
\begin{supertabular}{c@{$\;$}|p{.15\linewidth}@{}p{.15\linewidth}p{.15\linewidth}p{.15\linewidth}p{.15\linewidth}p{.15\linewidth}}

    \# & $\;$A & \multicolumn{4}{c}{Game Master} & $\;\:$B\\
    \hline 

    \theutterance \stepcounter{utterance}  

    & & \multicolumn{4}{p{0.6\linewidth}}{\cellcolor[rgb]{0.9,0.9,0.9}{%
	\makecell[{{p{\linewidth}}}]{% 
	  \tt {\tiny [A$\langle$GM]}  
	 You are an expert in causal inference. The following question is not a typical commonsense query, but rather a meticulously designed question created by a professor specializing in causal inference, intended to assess the students' mastery of the course content.\\ \tt Imagine a self-contained, hypothetical world with only the following conditions, and without any unmentioned factors or causal relationships: Eating citrus has a direct effect on vitmain C. Vitmain C has a direct effect on scurvy. For patients not consuming citrus, the probability of scurvy is 73%. For patients consuming citrus, the probability of scurvy is 54%. Does eating citrus positively affect scurvy through vitmain C?\\ \tt Start your answer with 'Yes' or 'No', followed by additional reasoning or evidence to support your explanation. 
	  } 
	   } 
	   } 
	 & \\ 
 

    \theutterance \stepcounter{utterance}  

    & \multicolumn{4}{p{0.6\linewidth}}{\cellcolor[rgb]{0.8,1,0.9}{%
	\makecell[{{p{\linewidth}}}]{% 
	  \tt {\tiny [A$\rangle$GM]}  
	 No\\ \tt \\ \tt To determine whether eating citrus positively affects scurvy through vitamin C, we need to apply the concept of mediation analysis. A positive effect of the mediator (vitamin C) on the outcome (scurvy) means that the mediator is a causal pathway through which the exposure (eating citrus) affects the outcome.\\ \tt \\ \tt Given the conditions, we know that eating citrus has a direct effect on vitamin C, and vitamin C has a direct effect on scurvy. However, to establish a positive effect of vitamin C on scurvy, we need to examine the relationship between vitamin C and scurvy.\\ \tt \\ \tt The provided information does not directly state the probability of scurvy for patients with varying levels of vitamin C. Instead, it provides the probabilities of scurvy for patients consuming or not consuming citrus. This suggests that the relationship between vitamin C and scurvy is not a direct one, but rather a mediated effect through the consumption of citrus.\\ \tt \\ \tt To assess the effect of vitamin C on scurvy, we would need to know the probability of scurvy for patients with varying levels of vitamin C, regardless of their citrus consumption. Without this information, we cannot conclude that vitamin C has a positive effect on scurvy.\\ \tt \\ \tt However, we can still examine the effect of eating citrus on scurvy through vitamin C. The probability of scurvy for patients consuming citrus (54%) is lower than for those not consuming citrus (73%). This suggests that eating citrus has a positive 
	  } 
	   } 
	   } 
	 & & \\ 
 

    \theutterance \stepcounter{utterance}  

    & & & \multicolumn{2}{p{0.3\linewidth}}{\cellcolor[rgb]{0.95,0.95,0.95}{%
	\makecell[{{p{\linewidth}}}]{% 
	  \tt {\tiny [GM$|$GM]}  
	 No 
	  } 
	   } 
	   } 
	 & & \\ 
 

    \theutterance \stepcounter{utterance}  

    & & & \multicolumn{2}{p{0.3\linewidth}}{\cellcolor[rgb]{0.95,0.95,0.95}{%
	\makecell[{{p{\linewidth}}}]{% 
	  \tt {\tiny [GM$|$GM]}  
	 no 
	  } 
	   } 
	   } 
	 & & \\ 
 

    \theutterance \stepcounter{utterance}  

    & & & \multicolumn{2}{p{0.3\linewidth}}{\cellcolor[rgb]{0.95,0.95,0.95}{%
	\makecell[{{p{\linewidth}}}]{% 
	  \tt {\tiny [GM$|$GM]}  
	 game_result = WIN 
	  } 
	   } 
	   } 
	 & & \\ 
 

\end{supertabular}
}

\end{document}
