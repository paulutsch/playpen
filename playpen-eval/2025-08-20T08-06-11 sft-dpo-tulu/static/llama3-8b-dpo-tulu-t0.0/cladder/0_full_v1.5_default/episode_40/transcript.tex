
\documentclass{article}
\usepackage{colortbl}
\usepackage{makecell}
\usepackage{multirow}
\usepackage{supertabular}

\begin{document}

\newcounter{utterance}

\twocolumn

{ \footnotesize  \setcounter{utterance}{1}
\setlength{\tabcolsep}{0pt}
\begin{supertabular}{c@{$\;$}|p{.15\linewidth}@{}p{.15\linewidth}p{.15\linewidth}p{.15\linewidth}p{.15\linewidth}p{.15\linewidth}}

    \# & $\;$A & \multicolumn{4}{c}{Game Master} & $\;\:$B\\
    \hline 

    \theutterance \stepcounter{utterance}  

    & & \multicolumn{4}{p{0.6\linewidth}}{\cellcolor[rgb]{0.9,0.9,0.9}{%
	\makecell[{{p{\linewidth}}}]{% 
	  \tt {\tiny [A$\langle$GM]}  
	 You are an expert in causal inference. The following question is not a typical commonsense query, but rather a meticulously designed question created by a professor specializing in causal inference, intended to assess the students' mastery of the course content.\\ \tt Imagine a self-contained, hypothetical world with only the following conditions, and without any unmentioned factors or causal relationships: Appearance has a direct effect on fame. Lip thickness has a direct effect on fame. Method 1: We look at how appearance correlates with lip thickness case by case according to fame. Method 2: We look directly at how appearance correlates with lip thickness in general. To understand how appearance affects lip thickness, is it more correct to use the Method 1 than Method 2?\\ \tt Start your answer with 'Yes' or 'No', followed by additional reasoning or evidence to support your explanation. 
	  } 
	   } 
	   } 
	 & \\ 
 

    \theutterance \stepcounter{utterance}  

    & \multicolumn{4}{p{0.6\linewidth}}{\cellcolor[rgb]{0.8,1,0.9}{%
	\makecell[{{p{\linewidth}}}]{% 
	  \tt {\tiny [A$\rangle$GM]}  
	 No, it is not more correct to use Method 1 than Method 2 to understand how appearance affects lip thickness.\\ \tt \\ \tt In this hypothetical world, we are trying to estimate the causal effect of appearance on lip thickness. However, the problem statement also mentions that lip thickness has a direct effect on fame, and appearance has a direct effect on fame. This creates a problem of confounding by the common effect, fame.\\ \tt \\ \tt Method 1, which examines the correlation between appearance and lip thickness case by case according to fame, is essentially a form of regression adjustment for the confounder fame. However, this method is still susceptible to issues such as selection bias and collider bias, especially if the cases are not randomly selected or if there are other unobserved variables that affect the relationship between appearance, lip thickness, and fame.\\ \tt \\ \tt Method 2, which directly examines the correlation between appearance and lip thickness in general, is a more straightforward approach that avoids the confounding by fame. This method is more likely to provide an unbiased estimate of the causal effect of appearance on lip thickness, as it does not rely on the potentially problematic case-by-case analysis.\\ \tt \\ \tt In general, when estimating causal effects, it is often better to use methods that directly address the confounding variables, such as instrumental variables, regression adjustment, or structural equation modeling, rather than relying on indirect methods that may be susceptible to bias. In this case, Method 2 is a more direct and unbiased approach to estimating the causal effect of appearance on lip thickness 
	  } 
	   } 
	   } 
	 & & \\ 
 

    \theutterance \stepcounter{utterance}  

    & & & \multicolumn{2}{p{0.3\linewidth}}{\cellcolor[rgb]{0.95,0.95,0.95}{%
	\makecell[{{p{\linewidth}}}]{% 
	  \tt {\tiny [GM$|$GM]}  
	 No 
	  } 
	   } 
	   } 
	 & & \\ 
 

    \theutterance \stepcounter{utterance}  

    & & & \multicolumn{2}{p{0.3\linewidth}}{\cellcolor[rgb]{0.95,0.95,0.95}{%
	\makecell[{{p{\linewidth}}}]{% 
	  \tt {\tiny [GM$|$GM]}  
	 no 
	  } 
	   } 
	   } 
	 & & \\ 
 

    \theutterance \stepcounter{utterance}  

    & & & \multicolumn{2}{p{0.3\linewidth}}{\cellcolor[rgb]{0.95,0.95,0.95}{%
	\makecell[{{p{\linewidth}}}]{% 
	  \tt {\tiny [GM$|$GM]}  
	 game_result = WIN 
	  } 
	   } 
	   } 
	 & & \\ 
 

\end{supertabular}
}

\end{document}
