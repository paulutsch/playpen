
\documentclass{article}
\usepackage{colortbl}
\usepackage{makecell}
\usepackage{multirow}
\usepackage{supertabular}

\begin{document}

\newcounter{utterance}

\twocolumn

{ \footnotesize  \setcounter{utterance}{1}
\setlength{\tabcolsep}{0pt}
\begin{supertabular}{c@{$\;$}|p{.15\linewidth}@{}p{.15\linewidth}p{.15\linewidth}p{.15\linewidth}p{.15\linewidth}p{.15\linewidth}}

    \# & $\;$A & \multicolumn{4}{c}{Game Master} & $\;\:$B\\
    \hline 

    \theutterance \stepcounter{utterance}  

    & & \multicolumn{4}{p{0.6\linewidth}}{\cellcolor[rgb]{0.9,0.9,0.9}{%
	\makecell[{{p{\linewidth}}}]{% 
	  \tt {\tiny [A$\langle$GM]}  
	 You are an expert in causal inference. The following question is not a typical commonsense query, but rather a meticulously designed question created by a professor specializing in causal inference, intended to assess the students' mastery of the course content.\\ \tt Imagine a self-contained, hypothetical world with only the following conditions, and without any unmentioned factors or causal relationships: Husband has a direct effect on wife and alarm clock. Wife has a direct effect on alarm clock. For husbands that don't set the alarm and wives that don't set the alarm, the probability of ringing alarm is 11%. For husbands that don't set the alarm and wives that set the alarm, the probability of ringing alarm is 60%. For husbands that set the alarm and wives that don't set the alarm, the probability of ringing alarm is 46%. For husbands that set the alarm and wives that set the alarm, the probability of ringing alarm is 92%. For husbands that don't set the alarm, the probability of alarm set by wife is 61%. For husbands that set the alarm, the probability of alarm set by wife is 1%. Does husband positively affect alarm clock through wife?\\ \tt Start your answer with 'Yes' or 'No', followed by additional reasoning or evidence to support your explanation. 
	  } 
	   } 
	   } 
	 & \\ 
 

    \theutterance \stepcounter{utterance}  

    & \multicolumn{4}{p{0.6\linewidth}}{\cellcolor[rgb]{0.8,1,0.9}{%
	\makecell[{{p{\linewidth}}}]{% 
	  \tt {\tiny [A$\rangle$GM]}  
	 No, husband does not positively affect alarm clock through wife.\\ \tt \\ \tt To assess this, we can use the backdoor criterion, which is a method for identifying causal relationships in the presence of confounding variables. In this case, the wife is the potential confounding variable.\\ \tt \\ \tt We can start by identifying the causal graph:\\ \tt \\ \tt - Husband (H) -> Wife (W)\\ \tt - Wife (W) -> Alarm Clock (AC)\\ \tt - Husband (H) -> Alarm Clock (AC) (direct effect)\\ \tt \\ \tt We are interested in the effect of H on AC through W. To apply the backdoor criterion, we need to check the following conditions:\\ \tt \\ \tt 1. No unmeasured confounders: We assume that there are no unmentioned factors or causal relationships in this hypothetical world.\\ \tt 2. No colliders on the path: There are no variables that have a direct effect on both H and W, or W and AC.\\ \tt 3. No descendants of the intermediate variable (W) on the path: W is a direct parent of AC, so this condition is satisfied.\\ \tt 4. The backdoor path is blocked: We need to check if the backdoor path from H to AC through W is blocked by the instrumental variables or the confounding variables.\\ \tt \\ \tt To check the backdoor criterion, we can use the formula:\\ \tt \\ \tt P(AC|do(H)) = P(AC|H, W)\\ \tt \\ \tt We can estimate this using the given probabilities:\\ \tt \\ \tt - P(AC|H=0, W=0) = 
	  } 
	   } 
	   } 
	 & & \\ 
 

    \theutterance \stepcounter{utterance}  

    & & & \multicolumn{2}{p{0.3\linewidth}}{\cellcolor[rgb]{0.95,0.95,0.95}{%
	\makecell[{{p{\linewidth}}}]{% 
	  \tt {\tiny [GM$|$GM]}  
	 No 
	  } 
	   } 
	   } 
	 & & \\ 
 

    \theutterance \stepcounter{utterance}  

    & & & \multicolumn{2}{p{0.3\linewidth}}{\cellcolor[rgb]{0.95,0.95,0.95}{%
	\makecell[{{p{\linewidth}}}]{% 
	  \tt {\tiny [GM$|$GM]}  
	 no 
	  } 
	   } 
	   } 
	 & & \\ 
 

    \theutterance \stepcounter{utterance}  

    & & & \multicolumn{2}{p{0.3\linewidth}}{\cellcolor[rgb]{0.95,0.95,0.95}{%
	\makecell[{{p{\linewidth}}}]{% 
	  \tt {\tiny [GM$|$GM]}  
	 game_result = WIN 
	  } 
	   } 
	   } 
	 & & \\ 
 

\end{supertabular}
}

\end{document}
