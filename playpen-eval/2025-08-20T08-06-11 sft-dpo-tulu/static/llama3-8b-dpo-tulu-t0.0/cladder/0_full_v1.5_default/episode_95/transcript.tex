
\documentclass{article}
\usepackage{colortbl}
\usepackage{makecell}
\usepackage{multirow}
\usepackage{supertabular}

\begin{document}

\newcounter{utterance}

\twocolumn

{ \footnotesize  \setcounter{utterance}{1}
\setlength{\tabcolsep}{0pt}
\begin{supertabular}{c@{$\;$}|p{.15\linewidth}@{}p{.15\linewidth}p{.15\linewidth}p{.15\linewidth}p{.15\linewidth}p{.15\linewidth}}

    \# & $\;$A & \multicolumn{4}{c}{Game Master} & $\;\:$B\\
    \hline 

    \theutterance \stepcounter{utterance}  

    & & \multicolumn{4}{p{0.6\linewidth}}{\cellcolor[rgb]{0.9,0.9,0.9}{%
	\makecell[{{p{\linewidth}}}]{% 
	  \tt {\tiny [A$\langle$GM]}  
	 You are an expert in causal inference. The following question is not a typical commonsense query, but rather a meticulously designed question created by a professor specializing in causal inference, intended to assess the students' mastery of the course content.\\ \tt Imagine a self-contained, hypothetical world with only the following conditions, and without any unmentioned factors or causal relationships: Qwiu has a direct effect on jyka and kwox. Jyka has a direct effect on yupt. Yupt has a direct effect on kwox. Qwiu is unobserved. We know that qwiu causes jyka. jyka causes not yupt. qwiu or yupt causes kwox. We observed an individual is qwiu. Would an individual is kwox if jyka instead of not jyka?\\ \tt Start your answer with 'Yes' or 'No', followed by additional reasoning or evidence to support your explanation. 
	  } 
	   } 
	   } 
	 & \\ 
 

    \theutterance \stepcounter{utterance}  

    & \multicolumn{4}{p{0.6\linewidth}}{\cellcolor[rgb]{0.8,1,0.9}{%
	\makecell[{{p{\linewidth}}}]{% 
	  \tt {\tiny [A$\rangle$GM]}  
	 No\\ \tt \\ \tt To answer this question, we need to carefully analyze the given causal relationships and the conditions provided. We know that:\\ \tt \\ \tt 1. Qwiu causes Jyka.\\ \tt 2. Jyka causes not Yupt (i.e., Yupt is the opposite of what Jyka causes).\\ \tt 3. Qwiu or Yupt causes Kwox (this is a disjunctive relationship, meaning either Qwiu or Yupt can cause Kwox).\\ \tt 4. We observed an individual is Qwiu.\\ \tt \\ \tt Given that we observed Qwiu, we can infer that Jyka is caused by Qwiu (from condition 1). Now, let's consider the two scenarios:\\ \tt \\ \tt Scenario 1: Jyka is present.\\ \tt Scenario 2: Jyka is not present (i.e., Jyka is the opposite of what it would be if Qwiu were the cause).\\ \tt \\ \tt In both scenarios, the presence of Qwiu is the same. The only difference is the presence or absence of Jyka. Since Jyka causes not Yupt, if Jyka is present, then Yupt is not present (Scenario 1). If Jyka is not present, then Yupt is present (Scenario 2).\\ \tt \\ \tt Now, considering the disjunctive relationship between Qwiu and Yupt causing Kwox, we need to determine which of these two causes Kwox in each scenario. In Scenario 1, 
	  } 
	   } 
	   } 
	 & & \\ 
 

    \theutterance \stepcounter{utterance}  

    & & & \multicolumn{2}{p{0.3\linewidth}}{\cellcolor[rgb]{0.95,0.95,0.95}{%
	\makecell[{{p{\linewidth}}}]{% 
	  \tt {\tiny [GM$|$GM]}  
	 No 
	  } 
	   } 
	   } 
	 & & \\ 
 

    \theutterance \stepcounter{utterance}  

    & & & \multicolumn{2}{p{0.3\linewidth}}{\cellcolor[rgb]{0.95,0.95,0.95}{%
	\makecell[{{p{\linewidth}}}]{% 
	  \tt {\tiny [GM$|$GM]}  
	 yes 
	  } 
	   } 
	   } 
	 & & \\ 
 

    \theutterance \stepcounter{utterance}  

    & & & \multicolumn{2}{p{0.3\linewidth}}{\cellcolor[rgb]{0.95,0.95,0.95}{%
	\makecell[{{p{\linewidth}}}]{% 
	  \tt {\tiny [GM$|$GM]}  
	 game_result = LOSE 
	  } 
	   } 
	   } 
	 & & \\ 
 

\end{supertabular}
}

\end{document}
