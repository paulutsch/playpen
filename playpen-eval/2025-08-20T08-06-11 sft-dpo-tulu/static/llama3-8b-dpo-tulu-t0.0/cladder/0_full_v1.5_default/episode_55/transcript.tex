
\documentclass{article}
\usepackage{colortbl}
\usepackage{makecell}
\usepackage{multirow}
\usepackage{supertabular}

\begin{document}

\newcounter{utterance}

\twocolumn

{ \footnotesize  \setcounter{utterance}{1}
\setlength{\tabcolsep}{0pt}
\begin{supertabular}{c@{$\;$}|p{.15\linewidth}@{}p{.15\linewidth}p{.15\linewidth}p{.15\linewidth}p{.15\linewidth}p{.15\linewidth}}

    \# & $\;$A & \multicolumn{4}{c}{Game Master} & $\;\:$B\\
    \hline 

    \theutterance \stepcounter{utterance}  

    & & \multicolumn{4}{p{0.6\linewidth}}{\cellcolor[rgb]{0.9,0.9,0.9}{%
	\makecell[{{p{\linewidth}}}]{% 
	  \tt {\tiny [A$\langle$GM]}  
	 You are an expert in causal inference. The following question is not a typical commonsense query, but rather a meticulously designed question created by a professor specializing in causal inference, intended to assess the students' mastery of the course content.\\ \tt Imagine a self-contained, hypothetical world with only the following conditions, and without any unmentioned factors or causal relationships: Alice waking up has a direct effect on rainfall. Traffic has a direct effect on rainfall. The overall probability of waking up late is 9%. The probability of waking up on time and high rainfall is 53%. The probability of waking up late and high rainfall is 8%. Is the chance of high rainfall smaller when observing waking up late?\\ \tt Start your answer with 'Yes' or 'No', followed by additional reasoning or evidence to support your explanation. 
	  } 
	   } 
	   } 
	 & \\ 
 

    \theutterance \stepcounter{utterance}  

    & \multicolumn{4}{p{0.6\linewidth}}{\cellcolor[rgb]{0.8,1,0.9}{%
	\makecell[{{p{\linewidth}}}]{% 
	  \tt {\tiny [A$\rangle$GM]}  
	 Yes, the chance of high rainfall is smaller when observing waking up late.\\ \tt \\ \tt To answer this question, we can use the concept of conditional probability and the backdoor criterion for causal inference. Let's define the variables:\\ \tt \\ \tt - W: Waking up on time (1) or late (0)\\ \tt - R: High rainfall (1) or not (0)\\ \tt \\ \tt We are given the following probabilities:\\ \tt \\ \tt - P(W = 1) = 0.09 (probability of waking up late)\\ \tt - P(R|W = 1) = 0.53 (probability of high rainfall given waking up on time)\\ \tt - P(R|W = 0) = 0.08 (probability of high rainfall given waking up late)\\ \tt \\ \tt We want to find P(R|W = 0) / P(R), which represents the ratio of the probability of high rainfall given waking up late to the overall probability of high rainfall.\\ \tt \\ \tt Using the law of total probability, we can express P(R) as:\\ \tt \\ \tt P(R) = P(R|W = 1) \* P(W = 1) + P(R|W = 0) \* P(W = 0)\\ \tt = 0.53 \* 0.09 + 0.08 \* 0.91\\ \tt = 0.0477 + 0.0728\\ \tt = 0.1205\\ \tt \\ \tt Now, we can calculate the ratio:\\ \tt \\ \tt P(R|W = 0) / P(R) 
	  } 
	   } 
	   } 
	 & & \\ 
 

    \theutterance \stepcounter{utterance}  

    & & & \multicolumn{2}{p{0.3\linewidth}}{\cellcolor[rgb]{0.95,0.95,0.95}{%
	\makecell[{{p{\linewidth}}}]{% 
	  \tt {\tiny [GM$|$GM]}  
	 Yes 
	  } 
	   } 
	   } 
	 & & \\ 
 

    \theutterance \stepcounter{utterance}  

    & & & \multicolumn{2}{p{0.3\linewidth}}{\cellcolor[rgb]{0.95,0.95,0.95}{%
	\makecell[{{p{\linewidth}}}]{% 
	  \tt {\tiny [GM$|$GM]}  
	 no 
	  } 
	   } 
	   } 
	 & & \\ 
 

    \theutterance \stepcounter{utterance}  

    & & & \multicolumn{2}{p{0.3\linewidth}}{\cellcolor[rgb]{0.95,0.95,0.95}{%
	\makecell[{{p{\linewidth}}}]{% 
	  \tt {\tiny [GM$|$GM]}  
	 game_result = LOSE 
	  } 
	   } 
	   } 
	 & & \\ 
 

\end{supertabular}
}

\end{document}
