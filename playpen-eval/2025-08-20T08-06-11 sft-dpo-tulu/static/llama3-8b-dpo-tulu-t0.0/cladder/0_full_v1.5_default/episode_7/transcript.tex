
\documentclass{article}
\usepackage{colortbl}
\usepackage{makecell}
\usepackage{multirow}
\usepackage{supertabular}

\begin{document}

\newcounter{utterance}

\twocolumn

{ \footnotesize  \setcounter{utterance}{1}
\setlength{\tabcolsep}{0pt}
\begin{supertabular}{c@{$\;$}|p{.15\linewidth}@{}p{.15\linewidth}p{.15\linewidth}p{.15\linewidth}p{.15\linewidth}p{.15\linewidth}}

    \# & $\;$A & \multicolumn{4}{c}{Game Master} & $\;\:$B\\
    \hline 

    \theutterance \stepcounter{utterance}  

    & & \multicolumn{4}{p{0.6\linewidth}}{\cellcolor[rgb]{0.9,0.9,0.9}{%
	\makecell[{{p{\linewidth}}}]{% 
	  \tt {\tiny [A$\langle$GM]}  
	 You are an expert in causal inference. The following question is not a typical commonsense query, but rather a meticulously designed question created by a professor specializing in causal inference, intended to assess the students' mastery of the course content.\\ \tt Imagine a self-contained, hypothetical world with only the following conditions, and without any unmentioned factors or causal relationships: Gender has a direct effect on department competitiveness and admission status. Department competitiveness has a direct effect on admission status. For individuals who are not male and applicants to a non-competitive department, the probability of admission acceptance is 58%. For individuals who are not male and applicants to a competitive department, the probability of admission acceptance is 25%. For individuals who are male and applicants to a non-competitive department, the probability of admission acceptance is 59%. For individuals who are male and applicants to a competitive department, the probability of admission acceptance is 28%. For individuals who are not male, the probability of competitive department is 29%. For individuals who are male, the probability of competitive department is 77%. Does gender positively affect admission status through department competitiveness?\\ \tt Start your answer with 'Yes' or 'No', followed by additional reasoning or evidence to support your explanation. 
	  } 
	   } 
	   } 
	 & \\ 
 

    \theutterance \stepcounter{utterance}  

    & \multicolumn{4}{p{0.6\linewidth}}{\cellcolor[rgb]{0.8,1,0.9}{%
	\makecell[{{p{\linewidth}}}]{% 
	  \tt {\tiny [A$\rangle$GM]}  
	 No\\ \tt \\ \tt To determine whether gender positively affects admission status through department competitiveness, we need to assess the causal relationship between gender and admission status, mediated by department competitiveness. \\ \tt \\ \tt First, we need to establish the causal relationship between gender and department competitiveness. The given data shows that the probability of being in a competitive department is 77% for males and 29% for non-males. This indicates a direct effect of gender on department competitiveness.\\ \tt \\ \tt Next, we need to establish the causal relationship between department competitiveness and admission status. The given data shows that the probability of admission acceptance is 25% for non-males in a competitive department and 28% for males in a competitive department. However, the probability of admission acceptance is 58% for non-males in a non-competitive department and 59% for males in a non-competitive department. This indicates that department competitiveness has a negative effect on admission status.\\ \tt \\ \tt Now, to assess the mediated effect of gender on admission status through department competitiveness, we need to compare the effect of gender on admission status when controlling for department competitiveness. We can do this by calculating the probability of admission acceptance for each gender group, given that they are in a competitive department.\\ \tt \\ \tt For non-males, the probability of admission acceptance in a competitive department is 25%. For males, the probability of admission acceptance in a competitive department is 28%. \\ \tt \\ \tt However, to accurately assess the mediated effect, we need to calculate the probability of admission acceptance for each gender group, given 
	  } 
	   } 
	   } 
	 & & \\ 
 

    \theutterance \stepcounter{utterance}  

    & & & \multicolumn{2}{p{0.3\linewidth}}{\cellcolor[rgb]{0.95,0.95,0.95}{%
	\makecell[{{p{\linewidth}}}]{% 
	  \tt {\tiny [GM$|$GM]}  
	 No 
	  } 
	   } 
	   } 
	 & & \\ 
 

    \theutterance \stepcounter{utterance}  

    & & & \multicolumn{2}{p{0.3\linewidth}}{\cellcolor[rgb]{0.95,0.95,0.95}{%
	\makecell[{{p{\linewidth}}}]{% 
	  \tt {\tiny [GM$|$GM]}  
	 no 
	  } 
	   } 
	   } 
	 & & \\ 
 

    \theutterance \stepcounter{utterance}  

    & & & \multicolumn{2}{p{0.3\linewidth}}{\cellcolor[rgb]{0.95,0.95,0.95}{%
	\makecell[{{p{\linewidth}}}]{% 
	  \tt {\tiny [GM$|$GM]}  
	 game_result = WIN 
	  } 
	   } 
	   } 
	 & & \\ 
 

\end{supertabular}
}

\end{document}
