
\documentclass{article}
\usepackage{colortbl}
\usepackage{makecell}
\usepackage{multirow}
\usepackage{supertabular}

\begin{document}

\newcounter{utterance}

\twocolumn

{ \footnotesize  \setcounter{utterance}{1}
\setlength{\tabcolsep}{0pt}
\begin{supertabular}{c@{$\;$}|p{.15\linewidth}@{}p{.15\linewidth}p{.15\linewidth}p{.15\linewidth}p{.15\linewidth}p{.15\linewidth}}

    \# & $\;$A & \multicolumn{4}{c}{Game Master} & $\;\:$B\\
    \hline 

    \theutterance \stepcounter{utterance}  

    & & \multicolumn{4}{p{0.6\linewidth}}{\cellcolor[rgb]{0.9,0.9,0.9}{%
	\makecell[{{p{\linewidth}}}]{% 
	  \tt {\tiny [A$\langle$GM]}  
	 You are an expert in causal inference. The following question is not a typical commonsense query, but rather a meticulously designed question created by a professor specializing in causal inference, intended to assess the students' mastery of the course content.\\ \tt Imagine a self-contained, hypothetical world with only the following conditions, and without any unmentioned factors or causal relationships: Unobserved confounders has a direct effect on education level and foot size. Proximity to a college has a direct effect on education level. Education level has a direct effect on foot size. Unobserved confounders is unobserved. The overall probability of college degree or higher is 47%. The probability of high school degree or lower and large feet is 32%. The probability of college degree or higher and large feet is 10%. Is the chance of large feet smaller when observing college degree or higher?\\ \tt Start your answer with 'Yes' or 'No', followed by additional reasoning or evidence to support your explanation. 
	  } 
	   } 
	   } 
	 & \\ 
 

    \theutterance \stepcounter{utterance}  

    & \multicolumn{4}{p{0.6\linewidth}}{\cellcolor[rgb]{0.8,1,0.9}{%
	\makecell[{{p{\linewidth}}}]{% 
	  \tt {\tiny [A$\rangle$GM]}  
	 Yes. \\ \tt \\ \tt To solve this problem, we can use the backdoor adjustment formula from Pearl's do-calculus, which states that the causal effect of a treatment (in this case, college degree or higher) on an outcome (in this case, large feet) can be estimated by adjusting for the confounders (in this case, unobserved confounders and education level). \\ \tt \\ \tt Let's denote the treatment as T (college degree or higher), the outcome as Y (large feet), and the confounders as U (unobserved confounders) and E (education level). We can use the following formula to estimate the causal effect:\\ \tt \\ \tt P(Y|do(T=1)) = P(Y|T=1, U, E) = P(Y|T=1, E) = P(Y|E)\\ \tt \\ \tt Since we are interested in the probability of large feet given college degree or higher, we need to calculate P(Y|T=1). \\ \tt \\ \tt We can use the law of total probability to calculate P(Y|T=1):\\ \tt \\ \tt P(Y|T=1) = P(Y|T=1, E=1) * P(T=1|E=1) + P(Y|T=1, E=0) * P(T=1|E=0)\\ \tt \\ \tt We know that P(T=1) = 0.47, P(T=1|E=1) = 1 (since everyone with a college degree or higher 
	  } 
	   } 
	   } 
	 & & \\ 
 

    \theutterance \stepcounter{utterance}  

    & & & \multicolumn{2}{p{0.3\linewidth}}{\cellcolor[rgb]{0.95,0.95,0.95}{%
	\makecell[{{p{\linewidth}}}]{% 
	  \tt {\tiny [GM$|$GM]}  
	 Yes 
	  } 
	   } 
	   } 
	 & & \\ 
 

    \theutterance \stepcounter{utterance}  

    & & & \multicolumn{2}{p{0.3\linewidth}}{\cellcolor[rgb]{0.95,0.95,0.95}{%
	\makecell[{{p{\linewidth}}}]{% 
	  \tt {\tiny [GM$|$GM]}  
	 yes 
	  } 
	   } 
	   } 
	 & & \\ 
 

    \theutterance \stepcounter{utterance}  

    & & & \multicolumn{2}{p{0.3\linewidth}}{\cellcolor[rgb]{0.95,0.95,0.95}{%
	\makecell[{{p{\linewidth}}}]{% 
	  \tt {\tiny [GM$|$GM]}  
	 game_result = WIN 
	  } 
	   } 
	   } 
	 & & \\ 
 

\end{supertabular}
}

\end{document}
