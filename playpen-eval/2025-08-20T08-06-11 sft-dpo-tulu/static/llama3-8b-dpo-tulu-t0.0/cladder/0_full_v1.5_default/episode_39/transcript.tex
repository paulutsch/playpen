
\documentclass{article}
\usepackage{colortbl}
\usepackage{makecell}
\usepackage{multirow}
\usepackage{supertabular}

\begin{document}

\newcounter{utterance}

\twocolumn

{ \footnotesize  \setcounter{utterance}{1}
\setlength{\tabcolsep}{0pt}
\begin{supertabular}{c@{$\;$}|p{.15\linewidth}@{}p{.15\linewidth}p{.15\linewidth}p{.15\linewidth}p{.15\linewidth}p{.15\linewidth}}

    \# & $\;$A & \multicolumn{4}{c}{Game Master} & $\;\:$B\\
    \hline 

    \theutterance \stepcounter{utterance}  

    & & \multicolumn{4}{p{0.6\linewidth}}{\cellcolor[rgb]{0.9,0.9,0.9}{%
	\makecell[{{p{\linewidth}}}]{% 
	  \tt {\tiny [A$\langle$GM]}  
	 You are an expert in causal inference. The following question is not a typical commonsense query, but rather a meticulously designed question created by a professor specializing in causal inference, intended to assess the students' mastery of the course content.\\ \tt Imagine a self-contained, hypothetical world with only the following conditions, and without any unmentioned factors or causal relationships: Tanning salon treatment has a direct effect on foot size. Going to the beach has a direct effect on foot size. The overall probability of tanning salon treatment is 10%. For people not using tanning salon treatments, the probability of large feet is 13%. For people who went to tanning salons, the probability of large feet is 70%. Is large feet more likely than small feet overall?\\ \tt Start your answer with 'Yes' or 'No', followed by additional reasoning or evidence to support your explanation. 
	  } 
	   } 
	   } 
	 & \\ 
 

    \theutterance \stepcounter{utterance}  

    & \multicolumn{4}{p{0.6\linewidth}}{\cellcolor[rgb]{0.8,1,0.9}{%
	\makecell[{{p{\linewidth}}}]{% 
	  \tt {\tiny [A$\rangle$GM]}  
	 No, large feet are not more likely than small feet overall.\\ \tt \\ \tt To answer this question, we need to apply the principles of causal inference, specifically the concept of counterfactuals and potential outcomes. Let's break down the information given:\\ \tt \\ \tt 1. Tanning salon treatment has a direct effect on foot size.\\ \tt 2. Going to the beach has a direct effect on foot size.\\ \tt 3. The overall probability of tanning salon treatment is 10%.\\ \tt 4. For people not using tanning salon treatments, the probability of large feet is 13%.\\ \tt 5. For people who went to tanning salons, the probability of large feet is 70%.\\ \tt \\ \tt We are asked to determine whether large feet are more likely than small feet overall. To do this, we need to consider the potential outcomes for individuals who did and did not receive tanning salon treatment.\\ \tt \\ \tt Let's define two potential outcomes for foot size: large (L) and small (S). We can use the notation P(L|T) to denote the probability of large feet given tanning salon treatment (T), and P(L|¬T) to denote the probability of large feet given no tanning salon treatment (¬T).\\ \tt \\ \tt From the given information, we know that P(T) = 0.1 (10% probability of tanning salon treatment) and P(L|¬T) = 0.13 (13% probability of large feet given no tanning salon treatment). We also know that P(L|T) 
	  } 
	   } 
	   } 
	 & & \\ 
 

    \theutterance \stepcounter{utterance}  

    & & & \multicolumn{2}{p{0.3\linewidth}}{\cellcolor[rgb]{0.95,0.95,0.95}{%
	\makecell[{{p{\linewidth}}}]{% 
	  \tt {\tiny [GM$|$GM]}  
	 No 
	  } 
	   } 
	   } 
	 & & \\ 
 

    \theutterance \stepcounter{utterance}  

    & & & \multicolumn{2}{p{0.3\linewidth}}{\cellcolor[rgb]{0.95,0.95,0.95}{%
	\makecell[{{p{\linewidth}}}]{% 
	  \tt {\tiny [GM$|$GM]}  
	 no 
	  } 
	   } 
	   } 
	 & & \\ 
 

    \theutterance \stepcounter{utterance}  

    & & & \multicolumn{2}{p{0.3\linewidth}}{\cellcolor[rgb]{0.95,0.95,0.95}{%
	\makecell[{{p{\linewidth}}}]{% 
	  \tt {\tiny [GM$|$GM]}  
	 game_result = WIN 
	  } 
	   } 
	   } 
	 & & \\ 
 

\end{supertabular}
}

\end{document}
