
\documentclass{article}
\usepackage{colortbl}
\usepackage{makecell}
\usepackage{multirow}
\usepackage{supertabular}

\begin{document}

\newcounter{utterance}

\twocolumn

{ \footnotesize  \setcounter{utterance}{1}
\setlength{\tabcolsep}{0pt}
\begin{supertabular}{c@{$\;$}|p{.15\linewidth}@{}p{.15\linewidth}p{.15\linewidth}p{.15\linewidth}p{.15\linewidth}p{.15\linewidth}}

    \# & $\;$A & \multicolumn{4}{c}{Game Master} & $\;\:$B\\
    \hline 

    \theutterance \stepcounter{utterance}  

    & & \multicolumn{4}{p{0.6\linewidth}}{\cellcolor[rgb]{0.9,0.9,0.9}{%
	\makecell[{{p{\linewidth}}}]{% 
	  \tt {\tiny [A$\langle$GM]}  
	 You are an expert in causal inference. The following question is not a typical commonsense query, but rather a meticulously designed question created by a professor specializing in causal inference, intended to assess the students' mastery of the course content.\\ \tt Imagine a self-contained, hypothetical world with only the following conditions, and without any unmentioned factors or causal relationships: CEO has a direct effect on director and having visited England. Having visited England has a direct effect on employee. Director has a direct effect on employee. For people who have not visited England and directors who don't sign termination letters, the probability of employee being fired is 9%. For people who have not visited England and directors who sign termination letters, the probability of employee being fired is 54%. For people who have visited England and directors who don't sign termination letters, the probability of employee being fired is 46%. For people who have visited England and directors who sign termination letters, the probability of employee being fired is 90%. For people who have not visited England, the probability of director signing the termination letter is 17%. For people who have visited England, the probability of director signing the termination letter is 27%. For people who have visited England, would it be less likely to see employee being fired if the person had not visited England?\\ \tt Start your answer with 'Yes' or 'No', followed by additional reasoning or evidence to support your explanation. 
	  } 
	   } 
	   } 
	 & \\ 
 

    \theutterance \stepcounter{utterance}  

    & \multicolumn{4}{p{0.6\linewidth}}{\cellcolor[rgb]{0.8,1,0.9}{%
	\makecell[{{p{\linewidth}}}]{% 
	  \tt {\tiny [A$\rangle$GM]}  
	 No, it would not be less likely to see an employee being fired if the person had not visited England.\\ \tt \\ \tt To answer this question, we need to apply the principles of causal inference, specifically the concept of counterfactuals. We are asked to compare the probability of an employee being fired in the actual world (where the employee has visited England) with the probability of an employee being fired in a counterfactual world (where the employee had not visited England).\\ \tt \\ \tt Let's denote the following:\\ \tt \\ \tt - A: "The employee has visited England"\\ \tt - B: "The employee is fired"\\ \tt - C: "The director signs the termination letter"\\ \tt \\ \tt We are given the following probabilities:\\ \tt \\ \tt - P(B|A, C) = 90% (employee is fired if they have visited England and the director signs the termination letter)\\ \tt - P(B|A, ¬C) = 46% (employee is fired if they have visited England and the director does not sign the termination letter)\\ \tt - P(B|¬A, C) = 54% (employee is fired if they have not visited England and the director signs the termination letter)\\ \tt - P(B|¬A, ¬C) = 9% (employee is fired if they have not visited England and the director does not sign the termination letter)\\ \tt - P(C|¬A) = 17% (probability of the director signing the termination letter if the employee has not visited England)\\ \tt - P(C|A) = 27% 
	  } 
	   } 
	   } 
	 & & \\ 
 

    \theutterance \stepcounter{utterance}  

    & & & \multicolumn{2}{p{0.3\linewidth}}{\cellcolor[rgb]{0.95,0.95,0.95}{%
	\makecell[{{p{\linewidth}}}]{% 
	  \tt {\tiny [GM$|$GM]}  
	 No 
	  } 
	   } 
	   } 
	 & & \\ 
 

    \theutterance \stepcounter{utterance}  

    & & & \multicolumn{2}{p{0.3\linewidth}}{\cellcolor[rgb]{0.95,0.95,0.95}{%
	\makecell[{{p{\linewidth}}}]{% 
	  \tt {\tiny [GM$|$GM]}  
	 yes 
	  } 
	   } 
	   } 
	 & & \\ 
 

    \theutterance \stepcounter{utterance}  

    & & & \multicolumn{2}{p{0.3\linewidth}}{\cellcolor[rgb]{0.95,0.95,0.95}{%
	\makecell[{{p{\linewidth}}}]{% 
	  \tt {\tiny [GM$|$GM]}  
	 game_result = LOSE 
	  } 
	   } 
	   } 
	 & & \\ 
 

\end{supertabular}
}

\end{document}
