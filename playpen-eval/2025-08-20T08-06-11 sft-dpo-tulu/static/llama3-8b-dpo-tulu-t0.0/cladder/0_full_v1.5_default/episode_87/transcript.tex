
\documentclass{article}
\usepackage{colortbl}
\usepackage{makecell}
\usepackage{multirow}
\usepackage{supertabular}

\begin{document}

\newcounter{utterance}

\twocolumn

{ \footnotesize  \setcounter{utterance}{1}
\setlength{\tabcolsep}{0pt}
\begin{supertabular}{c@{$\;$}|p{.15\linewidth}@{}p{.15\linewidth}p{.15\linewidth}p{.15\linewidth}p{.15\linewidth}p{.15\linewidth}}

    \# & $\;$A & \multicolumn{4}{c}{Game Master} & $\;\:$B\\
    \hline 

    \theutterance \stepcounter{utterance}  

    & & \multicolumn{4}{p{0.6\linewidth}}{\cellcolor[rgb]{0.9,0.9,0.9}{%
	\makecell[{{p{\linewidth}}}]{% 
	  \tt {\tiny [A$\langle$GM]}  
	 You are an expert in causal inference. The following question is not a typical commonsense query, but rather a meticulously designed question created by a professor specializing in causal inference, intended to assess the students' mastery of the course content.\\ \tt Imagine a self-contained, hypothetical world with only the following conditions, and without any unmentioned factors or causal relationships: Yupt has a direct effect on muvq. Zupj has a direct effect on muvq. Method 1: We look directly at how yupt correlates with muvq in general. Method 2: We look at this correlation case by case according to zupj. To understand how yupt affects muvq, is it more correct to use the Method 1 than Method 2?\\ \tt Start your answer with 'Yes' or 'No', followed by additional reasoning or evidence to support your explanation. 
	  } 
	   } 
	   } 
	 & \\ 
 

    \theutterance \stepcounter{utterance}  

    & \multicolumn{4}{p{0.6\linewidth}}{\cellcolor[rgb]{0.8,1,0.9}{%
	\makecell[{{p{\linewidth}}}]{% 
	  \tt {\tiny [A$\rangle$GM]}  
	 No, it is not more correct to use Method 1 than Method 2. \\ \tt \\ \tt In this hypothetical world, both Yupt and Zupj have direct effects on Muvq. This means that the effect of Yupt on Muvq is not constant and may vary depending on the level of Zupj. When we look at the correlation between Yupt and Muvq in general (Method 1), we are essentially averaging out the varying effects of Yupt on Muvq across different levels of Zupj. This can lead to biased estimates of the effect of Yupt on Muvq, as the effect may be different at different levels of Zupj.\\ \tt \\ \tt On the other hand, Method 2, which looks at the correlation case by case according to Zupj, allows us to estimate the effect of Yupt on Muvq at different levels of Zupj. This is an example of effect modification, where the effect of Yupt on Muvq is modified by the level of Zupj. By accounting for the varying effects of Yupt on Muvq across different levels of Zupj, Method 2 provides a more accurate and nuanced understanding of the relationship between Yupt and Muvq.\\ \tt \\ \tt In particular, this scenario illustrates the importance of considering interaction effects in causal inference. The interaction between Yupt and Zupj can lead to a phenomenon known as "confounding by interaction," where the effect of Yupt on Muv 
	  } 
	   } 
	   } 
	 & & \\ 
 

    \theutterance \stepcounter{utterance}  

    & & & \multicolumn{2}{p{0.3\linewidth}}{\cellcolor[rgb]{0.95,0.95,0.95}{%
	\makecell[{{p{\linewidth}}}]{% 
	  \tt {\tiny [GM$|$GM]}  
	 No 
	  } 
	   } 
	   } 
	 & & \\ 
 

    \theutterance \stepcounter{utterance}  

    & & & \multicolumn{2}{p{0.3\linewidth}}{\cellcolor[rgb]{0.95,0.95,0.95}{%
	\makecell[{{p{\linewidth}}}]{% 
	  \tt {\tiny [GM$|$GM]}  
	 yes 
	  } 
	   } 
	   } 
	 & & \\ 
 

    \theutterance \stepcounter{utterance}  

    & & & \multicolumn{2}{p{0.3\linewidth}}{\cellcolor[rgb]{0.95,0.95,0.95}{%
	\makecell[{{p{\linewidth}}}]{% 
	  \tt {\tiny [GM$|$GM]}  
	 game_result = LOSE 
	  } 
	   } 
	   } 
	 & & \\ 
 

\end{supertabular}
}

\end{document}
