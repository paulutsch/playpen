
\documentclass{article}
\usepackage{colortbl}
\usepackage{makecell}
\usepackage{multirow}
\usepackage{supertabular}

\begin{document}

\newcounter{utterance}

\twocolumn

{ \footnotesize  \setcounter{utterance}{1}
\setlength{\tabcolsep}{0pt}
\begin{supertabular}{c@{$\;$}|p{.15\linewidth}@{}p{.15\linewidth}p{.15\linewidth}p{.15\linewidth}p{.15\linewidth}p{.15\linewidth}}

    \# & $\;$A & \multicolumn{4}{c}{Game Master} & $\;\:$B\\
    \hline 

    \theutterance \stepcounter{utterance}  

    & & \multicolumn{4}{p{0.6\linewidth}}{\cellcolor[rgb]{0.9,0.9,0.9}{%
	\makecell[{{p{\linewidth}}}]{% 
	  \tt {\tiny [A$\langle$GM]}  
	 You are an expert in causal inference. The following question is not a typical commonsense query, but rather a meticulously designed question created by a professor specializing in causal inference, intended to assess the students' mastery of the course content.\\ \tt Imagine a self-contained, hypothetical world with only the following conditions, and without any unmentioned factors or causal relationships: My decision has a direct effect on penguin mood and curly hair. Penguin mood has a direct effect on curly hair. For those who choose to take the stairs, the probability of curly hair is 70%. For those who choose to take the elevator, the probability of curly hair is 17%. For those who choose to take the elevator, would it be less likely to see curly hair if I had taken the stairs?\\ \tt Start your answer with 'Yes' or 'No', followed by additional reasoning or evidence to support your explanation. 
	  } 
	   } 
	   } 
	 & \\ 
 

    \theutterance \stepcounter{utterance}  

    & \multicolumn{4}{p{0.6\linewidth}}{\cellcolor[rgb]{0.8,1,0.9}{%
	\makecell[{{p{\linewidth}}}]{% 
	  \tt {\tiny [A$\rangle$GM]}  
	 No, it would not be less likely to see curly hair if I had taken the stairs. \\ \tt \\ \tt To understand this, we need to apply the concept of counterfactuals and potential outcomes. The question is asking us to compare the probability of curly hair among those who take the elevator, under two different scenarios: one where I take the stairs and another where I take the elevator. \\ \tt \\ \tt Let's denote the event of taking the stairs as S, the event of taking the elevator as E, and the event of having curly hair as C. We are given the following probabilities:\\ \tt \\ \tt - P(C|S) = 0.7 (probability of curly hair given that I take the stairs)\\ \tt - P(C|E) = 0.17 (probability of curly hair given that I take the elevator)\\ \tt \\ \tt We want to compare P(C|E, S) (the probability of curly hair given that I take the elevator and you take the stairs) with P(C|E) (the probability of curly hair given that I take the elevator). \\ \tt \\ \tt To do this, we need to consider the causal relationships between the variables. The decision to take the stairs or the elevator affects the probability of curly hair, and the probability of curly hair also affects the probability of curly hair (this is a trivial relationship, but it's part of the problem statement). However, the key relationship here is that the decision to take the stairs or the elevator affects the probability of curly hair.\\ \tt \\ \tt The problem statement does not 
	  } 
	   } 
	   } 
	 & & \\ 
 

    \theutterance \stepcounter{utterance}  

    & & & \multicolumn{2}{p{0.3\linewidth}}{\cellcolor[rgb]{0.95,0.95,0.95}{%
	\makecell[{{p{\linewidth}}}]{% 
	  \tt {\tiny [GM$|$GM]}  
	 No 
	  } 
	   } 
	   } 
	 & & \\ 
 

    \theutterance \stepcounter{utterance}  

    & & & \multicolumn{2}{p{0.3\linewidth}}{\cellcolor[rgb]{0.95,0.95,0.95}{%
	\makecell[{{p{\linewidth}}}]{% 
	  \tt {\tiny [GM$|$GM]}  
	 no 
	  } 
	   } 
	   } 
	 & & \\ 
 

    \theutterance \stepcounter{utterance}  

    & & & \multicolumn{2}{p{0.3\linewidth}}{\cellcolor[rgb]{0.95,0.95,0.95}{%
	\makecell[{{p{\linewidth}}}]{% 
	  \tt {\tiny [GM$|$GM]}  
	 game_result = WIN 
	  } 
	   } 
	   } 
	 & & \\ 
 

\end{supertabular}
}

\end{document}
