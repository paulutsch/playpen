
\documentclass{article}
\usepackage{colortbl}
\usepackage{makecell}
\usepackage{multirow}
\usepackage{supertabular}

\begin{document}

\newcounter{utterance}

\twocolumn

{ \footnotesize  \setcounter{utterance}{1}
\setlength{\tabcolsep}{0pt}
\begin{supertabular}{c@{$\;$}|p{.15\linewidth}@{}p{.15\linewidth}p{.15\linewidth}p{.15\linewidth}p{.15\linewidth}p{.15\linewidth}}

    \# & $\;$A & \multicolumn{4}{c}{Game Master} & $\;\:$B\\
    \hline 

    \theutterance \stepcounter{utterance}  

    & & \multicolumn{4}{p{0.6\linewidth}}{\cellcolor[rgb]{0.9,0.9,0.9}{%
	\makecell[{{p{\linewidth}}}]{% 
	  \tt {\tiny [A$\langle$GM]}  
	 You are an expert in causal inference. The following question is not a typical commonsense query, but rather a meticulously designed question created by a professor specializing in causal inference, intended to assess the students' mastery of the course content.\\ \tt Imagine a self-contained, hypothetical world with only the following conditions, and without any unmentioned factors or causal relationships: Jyka has a direct effect on zory and zuph. Zuph has a direct effect on glimx. Zory has a direct effect on glimx. The overall probability of zuph is 34%. The probability of not zuph and glimx is 64%. The probability of zuph and glimx is 28%. Is the chance of glimx smaller when observing zuph?\\ \tt Start your answer with 'Yes' or 'No', followed by additional reasoning or evidence to support your explanation. 
	  } 
	   } 
	   } 
	 & \\ 
 

    \theutterance \stepcounter{utterance}  

    & \multicolumn{4}{p{0.6\linewidth}}{\cellcolor[rgb]{0.8,1,0.9}{%
	\makecell[{{p{\linewidth}}}]{% 
	  \tt {\tiny [A$\rangle$GM]}  
	 No, the chance of glimx is not smaller when observing zuph.\\ \tt \\ \tt To analyze this, we can use the concept of conditional probability and the backdoor criterion from causal inference. Let's denote the variables as follows: Jyka = J, zory = Z, zuph = U, and glimx = G.\\ \tt \\ \tt We are given the following information:\\ \tt \\ \tt 1. P(U) = 0.34\\ \tt 2. P(¬U ∧ G) = 0.64\\ \tt 3. P(U ∧ G) = 0.28\\ \tt \\ \tt We want to determine if P(G|U) < P(G). To do this, we can use the law of total probability to calculate P(G|U) and compare it to P(G).\\ \tt \\ \tt First, we can calculate P(G) using the law of total probability:\\ \tt \\ \tt P(G) = P(¬U ∧ G) + P(U ∧ G)\\ \tt = 0.64 + 0.28\\ \tt = 0.92\\ \tt \\ \tt Next, we can calculate P(G|U) using the definition of conditional probability:\\ \tt \\ \tt P(G|U) = P(U ∧ G) / P(U)\\ \tt = 0.28 / 0.34\\ \tt = 0.8235\\ \tt \\ \tt Since P(G|U) > P(G), the chance of glimx is not smaller when observing zuph.\\ \tt \\ \tt However, we need to consider the direct effects between the variables. We are given that zuph 
	  } 
	   } 
	   } 
	 & & \\ 
 

    \theutterance \stepcounter{utterance}  

    & & & \multicolumn{2}{p{0.3\linewidth}}{\cellcolor[rgb]{0.95,0.95,0.95}{%
	\makecell[{{p{\linewidth}}}]{% 
	  \tt {\tiny [GM$|$GM]}  
	 No 
	  } 
	   } 
	   } 
	 & & \\ 
 

    \theutterance \stepcounter{utterance}  

    & & & \multicolumn{2}{p{0.3\linewidth}}{\cellcolor[rgb]{0.95,0.95,0.95}{%
	\makecell[{{p{\linewidth}}}]{% 
	  \tt {\tiny [GM$|$GM]}  
	 yes 
	  } 
	   } 
	   } 
	 & & \\ 
 

    \theutterance \stepcounter{utterance}  

    & & & \multicolumn{2}{p{0.3\linewidth}}{\cellcolor[rgb]{0.95,0.95,0.95}{%
	\makecell[{{p{\linewidth}}}]{% 
	  \tt {\tiny [GM$|$GM]}  
	 game_result = LOSE 
	  } 
	   } 
	   } 
	 & & \\ 
 

\end{supertabular}
}

\end{document}
