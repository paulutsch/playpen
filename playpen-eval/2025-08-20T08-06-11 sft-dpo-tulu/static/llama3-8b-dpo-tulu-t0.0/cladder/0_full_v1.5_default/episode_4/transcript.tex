
\documentclass{article}
\usepackage{colortbl}
\usepackage{makecell}
\usepackage{multirow}
\usepackage{supertabular}

\begin{document}

\newcounter{utterance}

\twocolumn

{ \footnotesize  \setcounter{utterance}{1}
\setlength{\tabcolsep}{0pt}
\begin{supertabular}{c@{$\;$}|p{.15\linewidth}@{}p{.15\linewidth}p{.15\linewidth}p{.15\linewidth}p{.15\linewidth}p{.15\linewidth}}

    \# & $\;$A & \multicolumn{4}{c}{Game Master} & $\;\:$B\\
    \hline 

    \theutterance \stepcounter{utterance}  

    & & \multicolumn{4}{p{0.6\linewidth}}{\cellcolor[rgb]{0.9,0.9,0.9}{%
	\makecell[{{p{\linewidth}}}]{% 
	  \tt {\tiny [A$\langle$GM]}  
	 You are an expert in causal inference. The following question is not a typical commonsense query, but rather a meticulously designed question created by a professor specializing in causal inference, intended to assess the students' mastery of the course content.\\ \tt Imagine a self-contained, hypothetical world with only the following conditions, and without any unmentioned factors or causal relationships: Vaccination status has a direct effect on vaccination reaction and getting smallpox. Getting smallpox has a direct effect on smallpox survival. Vaccination reaction has a direct effect on smallpox survival. For unvaccinated individuals, the probability of smallpox survival is 39%. For vaccinated individuals, the probability of smallpox survival is 46%. For vaccinated individuals, would it be more likely to see smallpox survival if the person had been vaccinated?\\ \tt Start your answer with 'Yes' or 'No', followed by additional reasoning or evidence to support your explanation. 
	  } 
	   } 
	   } 
	 & \\ 
 

    \theutterance \stepcounter{utterance}  

    & \multicolumn{4}{p{0.6\linewidth}}{\cellcolor[rgb]{0.8,1,0.9}{%
	\makecell[{{p{\linewidth}}}]{% 
	  \tt {\tiny [A$\rangle$GM]}  
	 No, it would not be more likely to see smallpox survival if the person had been vaccinated, given the information that the probability of smallpox survival is 46% for vaccinated individuals, which is lower than the 39% probability for unvaccinated individuals.\\ \tt \\ \tt This result may seem counterintuitive at first, but it is a consequence of the complex relationships between vaccination status, vaccination reaction, and smallpox survival. The key insight here is that the effect of vaccination status on smallpox survival is mediated by the vaccination reaction. \\ \tt \\ \tt To understand this, we need to consider the following:\\ \tt \\ \tt 1. Vaccination status has a direct effect on vaccination reaction.\\ \tt 2. Vaccination reaction has a direct effect on smallpox survival.\\ \tt 3. Getting smallpox has a direct effect on smallpox survival.\\ \tt \\ \tt The question asks whether it is more likely to see smallpox survival if the person had been vaccinated. However, the provided information does not directly answer this question. Instead, it provides the probabilities of smallpox survival for vaccinated and unvaccinated individuals.\\ \tt \\ \tt To answer the question, we need to consider the potential pathways through which vaccination status affects smallpox survival. There are two possible pathways:\\ \tt \\ \tt 1. Vaccination status → Vaccination reaction → Smallpox survival\\ \tt 2. Vaccination status → Getting smallpox → Smallpox survival\\ \tt \\ \tt The first pathway suggests that vaccination status affects smallpox survival through the vaccination reaction. However, the information 
	  } 
	   } 
	   } 
	 & & \\ 
 

    \theutterance \stepcounter{utterance}  

    & & & \multicolumn{2}{p{0.3\linewidth}}{\cellcolor[rgb]{0.95,0.95,0.95}{%
	\makecell[{{p{\linewidth}}}]{% 
	  \tt {\tiny [GM$|$GM]}  
	 No 
	  } 
	   } 
	   } 
	 & & \\ 
 

    \theutterance \stepcounter{utterance}  

    & & & \multicolumn{2}{p{0.3\linewidth}}{\cellcolor[rgb]{0.95,0.95,0.95}{%
	\makecell[{{p{\linewidth}}}]{% 
	  \tt {\tiny [GM$|$GM]}  
	 no 
	  } 
	   } 
	   } 
	 & & \\ 
 

    \theutterance \stepcounter{utterance}  

    & & & \multicolumn{2}{p{0.3\linewidth}}{\cellcolor[rgb]{0.95,0.95,0.95}{%
	\makecell[{{p{\linewidth}}}]{% 
	  \tt {\tiny [GM$|$GM]}  
	 game_result = WIN 
	  } 
	   } 
	   } 
	 & & \\ 
 

\end{supertabular}
}

\end{document}
