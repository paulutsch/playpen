
\documentclass{article}
\usepackage{colortbl}
\usepackage{makecell}
\usepackage{multirow}
\usepackage{supertabular}

\begin{document}

\newcounter{utterance}

\twocolumn

{ \footnotesize  \setcounter{utterance}{1}
\setlength{\tabcolsep}{0pt}
\begin{supertabular}{c@{$\;$}|p{.15\linewidth}@{}p{.15\linewidth}p{.15\linewidth}p{.15\linewidth}p{.15\linewidth}p{.15\linewidth}}

    \# & $\;$A & \multicolumn{4}{c}{Game Master} & $\;\:$B\\
    \hline 

    \theutterance \stepcounter{utterance}  

    & & \multicolumn{4}{p{0.6\linewidth}}{\cellcolor[rgb]{0.9,0.9,0.9}{%
	\makecell[{{p{\linewidth}}}]{% 
	  \tt {\tiny [A$\langle$GM]}  
	 You are an expert in causal inference. The following question is not a typical commonsense query, but rather a meticulously designed question created by a professor specializing in causal inference, intended to assess the students' mastery of the course content.\\ \tt Imagine a self-contained, hypothetical world with only the following conditions, and without any unmentioned factors or causal relationships: Parents' intelligence has a direct effect on parents' social status and child's intelligence. Other unobserved factors has a direct effect on parents' social status and child's intelligence. Parents' social status has a direct effect on child's intelligence. Other unobserved factors is unobserved. We know that intelligent parents and confounder active causes high parental social status. intelligent parents and confounder active and high parental social status causes intelligent child. We observed confounder active. Would the child is intelligent if intelligent parents instead of unintelligent parents?\\ \tt Start your answer with 'Yes' or 'No', followed by additional reasoning or evidence to support your explanation. 
	  } 
	   } 
	   } 
	 & \\ 
 

    \theutterance \stepcounter{utterance}  

    & \multicolumn{4}{p{0.6\linewidth}}{\cellcolor[rgb]{0.8,1,0.9}{%
	\makecell[{{p{\linewidth}}}]{% 
	  \tt {\tiny [A$\rangle$GM]}  
	 Yes. \\ \tt \\ \tt To answer this question, we need to apply the principles of causal inference, specifically the concept of confounding and the adjustment for confounding variables. \\ \tt \\ \tt Given the conditions:\\ \tt \\ \tt 1. Parents' intelligence has a direct effect on parents' social status and child's intelligence.\\ \tt 2. Other unobserved factors have a direct effect on parents' social status and child's intelligence.\\ \tt 3. Parents' social status has a direct effect on child's intelligence.\\ \tt 4. Other unobserved factors is unobserved.\\ \tt 5. Intelligent parents and confounder active causes high parental social status.\\ \tt 6. Intelligent parents and confounder active and high parental social status causes intelligent child.\\ \tt 7. We observed confounder active.\\ \tt \\ \tt We are asked to compare the outcome (child's intelligence) under two scenarios: (a) intelligent parents, and (b) unintelligent parents.\\ \tt \\ \tt First, we need to identify the confounding variable, which is the "confounder active" mentioned in conditions 5 and 6. The confounder active is a common cause of both parents' intelligence and parents' social status, and also a common cause of parents' social status and child's intelligence.\\ \tt \\ \tt To adjust for the confounding variable, we can use the concept of counterfactuals. We want to know what would happen to the child's intelligence if intelligent parents instead of unintelligent parents, while holding the confounder active constant.\\ \tt \\ \tt Since the confounder active is observed (condition 7), we can use it 
	  } 
	   } 
	   } 
	 & & \\ 
 

    \theutterance \stepcounter{utterance}  

    & & & \multicolumn{2}{p{0.3\linewidth}}{\cellcolor[rgb]{0.95,0.95,0.95}{%
	\makecell[{{p{\linewidth}}}]{% 
	  \tt {\tiny [GM$|$GM]}  
	 Yes 
	  } 
	   } 
	   } 
	 & & \\ 
 

    \theutterance \stepcounter{utterance}  

    & & & \multicolumn{2}{p{0.3\linewidth}}{\cellcolor[rgb]{0.95,0.95,0.95}{%
	\makecell[{{p{\linewidth}}}]{% 
	  \tt {\tiny [GM$|$GM]}  
	 yes 
	  } 
	   } 
	   } 
	 & & \\ 
 

    \theutterance \stepcounter{utterance}  

    & & & \multicolumn{2}{p{0.3\linewidth}}{\cellcolor[rgb]{0.95,0.95,0.95}{%
	\makecell[{{p{\linewidth}}}]{% 
	  \tt {\tiny [GM$|$GM]}  
	 game_result = WIN 
	  } 
	   } 
	   } 
	 & & \\ 
 

\end{supertabular}
}

\end{document}
