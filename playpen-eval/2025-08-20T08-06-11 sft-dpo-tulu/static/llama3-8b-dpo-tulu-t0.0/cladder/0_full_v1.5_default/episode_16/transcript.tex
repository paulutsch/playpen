
\documentclass{article}
\usepackage{colortbl}
\usepackage{makecell}
\usepackage{multirow}
\usepackage{supertabular}

\begin{document}

\newcounter{utterance}

\twocolumn

{ \footnotesize  \setcounter{utterance}{1}
\setlength{\tabcolsep}{0pt}
\begin{supertabular}{c@{$\;$}|p{.15\linewidth}@{}p{.15\linewidth}p{.15\linewidth}p{.15\linewidth}p{.15\linewidth}p{.15\linewidth}}

    \# & $\;$A & \multicolumn{4}{c}{Game Master} & $\;\:$B\\
    \hline 

    \theutterance \stepcounter{utterance}  

    & & \multicolumn{4}{p{0.6\linewidth}}{\cellcolor[rgb]{0.9,0.9,0.9}{%
	\makecell[{{p{\linewidth}}}]{% 
	  \tt {\tiny [A$\langle$GM]}  
	 You are an expert in causal inference. The following question is not a typical commonsense query, but rather a meticulously designed question created by a professor specializing in causal inference, intended to assess the students' mastery of the course content.\\ \tt Imagine a self-contained, hypothetical world with only the following conditions, and without any unmentioned factors or causal relationships: Husband has a direct effect on wife and alarm clock. Wife has a direct effect on alarm clock. The overall probability of alarm set by husband is 17%. The probability of alarm not set by husband and ringing alarm is 69%. The probability of alarm set by husband and ringing alarm is 3%. Is the chance of ringing alarm smaller when observing alarm set by husband?\\ \tt Start your answer with 'Yes' or 'No', followed by additional reasoning or evidence to support your explanation. 
	  } 
	   } 
	   } 
	 & \\ 
 

    \theutterance \stepcounter{utterance}  

    & \multicolumn{4}{p{0.6\linewidth}}{\cellcolor[rgb]{0.8,1,0.9}{%
	\makecell[{{p{\linewidth}}}]{% 
	  \tt {\tiny [A$\rangle$GM]}  
	 Yes, the chance of a ringing alarm is smaller when observing an alarm set by husband.\\ \tt \\ \tt To answer this question, we need to apply the concept of conditional probability and the causal relationships given in the problem. Let's denote the event of an alarm being set by the husband as H, the event of an alarm being set by the wife as W, and the event of an alarm ringing as R.\\ \tt \\ \tt We are given the following probabilities:\\ \tt \\ \tt - P(H) = 17% (probability of alarm set by husband)\\ \tt - P(R|H̄) = 69% (probability of alarm ringing given that it was not set by husband)\\ \tt - P(H and R) = 3% (probability of alarm set by husband and ringing)\\ \tt \\ \tt We can calculate the probability of an alarm ringing given that it was set by the husband (P(R|H)) using the law of total probability:\\ \tt \\ \tt P(R|H) = P(R and H) / P(H)\\ \tt = 3% / 17%\\ \tt = 0.1765 (approximately 17.65%)\\ \tt \\ \tt Now, we can compare this to the probability of an alarm ringing given that it was not set by the husband (P(R|H̄)), which is 69%. Since 17.65% is less than 69%, the chance of a ringing alarm is indeed smaller when observing an alarm set by the husband.\\ \tt \\ \tt This result is consistent with the causal relationships given in the problem. The wife's effect on the alarm 
	  } 
	   } 
	   } 
	 & & \\ 
 

    \theutterance \stepcounter{utterance}  

    & & & \multicolumn{2}{p{0.3\linewidth}}{\cellcolor[rgb]{0.95,0.95,0.95}{%
	\makecell[{{p{\linewidth}}}]{% 
	  \tt {\tiny [GM$|$GM]}  
	 Yes 
	  } 
	   } 
	   } 
	 & & \\ 
 

    \theutterance \stepcounter{utterance}  

    & & & \multicolumn{2}{p{0.3\linewidth}}{\cellcolor[rgb]{0.95,0.95,0.95}{%
	\makecell[{{p{\linewidth}}}]{% 
	  \tt {\tiny [GM$|$GM]}  
	 yes 
	  } 
	   } 
	   } 
	 & & \\ 
 

    \theutterance \stepcounter{utterance}  

    & & & \multicolumn{2}{p{0.3\linewidth}}{\cellcolor[rgb]{0.95,0.95,0.95}{%
	\makecell[{{p{\linewidth}}}]{% 
	  \tt {\tiny [GM$|$GM]}  
	 game_result = WIN 
	  } 
	   } 
	   } 
	 & & \\ 
 

\end{supertabular}
}

\end{document}
