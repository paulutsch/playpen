
\documentclass{article}
\usepackage{colortbl}
\usepackage{makecell}
\usepackage{multirow}
\usepackage{supertabular}

\begin{document}

\newcounter{utterance}

\twocolumn

{ \footnotesize  \setcounter{utterance}{1}
\setlength{\tabcolsep}{0pt}
\begin{supertabular}{c@{$\;$}|p{.15\linewidth}@{}p{.15\linewidth}p{.15\linewidth}p{.15\linewidth}p{.15\linewidth}p{.15\linewidth}}

    \# & $\;$A & \multicolumn{4}{c}{Game Master} & $\;\:$B\\
    \hline 

    \theutterance \stepcounter{utterance}  

    & & \multicolumn{4}{p{0.6\linewidth}}{\cellcolor[rgb]{0.9,0.9,0.9}{%
	\makecell[{{p{\linewidth}}}]{% 
	  \tt {\tiny [A$\langle$GM]}  
	 You are an expert in causal inference. The following question is not a typical commonsense query, but rather a meticulously designed question created by a professor specializing in causal inference, intended to assess the students' mastery of the course content.\\ \tt Imagine a self-contained, hypothetical world with only the following conditions, and without any unmentioned factors or causal relationships: Gender has a direct effect on department competitiveness and peanut allergy. Department competitiveness has a direct effect on peanut allergy. For individuals who are not male, the probability of being allergic to peanuts is 48%. For individuals who are male, the probability of being allergic to peanuts is 36%. For individuals who are male, would it be more likely to see being allergic to peanuts if the individual had not been male?\\ \tt Start your answer with 'Yes' or 'No', followed by additional reasoning or evidence to support your explanation. 
	  } 
	   } 
	   } 
	 & \\ 
 

    \theutterance \stepcounter{utterance}  

    & \multicolumn{4}{p{0.6\linewidth}}{\cellcolor[rgb]{0.8,1,0.9}{%
	\makecell[{{p{\linewidth}}}]{% 
	  \tt {\tiny [A$\rangle$GM]}  
	 No, it would not be more likely to see being allergic to peanuts if the individual had not been male.\\ \tt \\ \tt To understand this, let's break down the given information:\\ \tt \\ \tt 1. Gender has a direct effect on peanut allergy.\\ \tt 2. Department competitiveness has a direct effect on peanut allergy.\\ \tt 3. For individuals who are male, the probability of being allergic to peanuts is 36%.\\ \tt 4. For individuals who are not male (i.e., female), the probability of being allergic to peanuts is 48%.\\ \tt \\ \tt The question asks whether it would be more likely to see being allergic to peanuts if the individual had not been male. This is essentially asking whether the probability of being allergic to peanuts is higher for females than for males, given that the individual is not male.\\ \tt \\ \tt To answer this, we need to consider the causal relationships between the variables. The given information implies that there is a direct effect of gender on peanut allergy, and a direct effect of department competitiveness on peanut allergy. However, we are not given any information about the relationship between gender and department competitiveness.\\ \tt \\ \tt Since we are not given any information about the relationship between gender and department competitiveness, we cannot use the fact that department competitiveness has a direct effect on peanut allergy to infer anything about the relationship between gender and peanut allergy.\\ \tt \\ \tt However, we can use the given probabilities to compare the likelihood of being allergic to peanuts for males and females. The probability of being allergic to peanuts is 36% for males and 48% for females. This suggests that, 
	  } 
	   } 
	   } 
	 & & \\ 
 

    \theutterance \stepcounter{utterance}  

    & & & \multicolumn{2}{p{0.3\linewidth}}{\cellcolor[rgb]{0.95,0.95,0.95}{%
	\makecell[{{p{\linewidth}}}]{% 
	  \tt {\tiny [GM$|$GM]}  
	 No 
	  } 
	   } 
	   } 
	 & & \\ 
 

    \theutterance \stepcounter{utterance}  

    & & & \multicolumn{2}{p{0.3\linewidth}}{\cellcolor[rgb]{0.95,0.95,0.95}{%
	\makecell[{{p{\linewidth}}}]{% 
	  \tt {\tiny [GM$|$GM]}  
	 yes 
	  } 
	   } 
	   } 
	 & & \\ 
 

    \theutterance \stepcounter{utterance}  

    & & & \multicolumn{2}{p{0.3\linewidth}}{\cellcolor[rgb]{0.95,0.95,0.95}{%
	\makecell[{{p{\linewidth}}}]{% 
	  \tt {\tiny [GM$|$GM]}  
	 game_result = LOSE 
	  } 
	   } 
	   } 
	 & & \\ 
 

\end{supertabular}
}

\end{document}
