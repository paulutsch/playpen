
\documentclass{article}
\usepackage{colortbl}
\usepackage{makecell}
\usepackage{multirow}
\usepackage{supertabular}

\begin{document}

\newcounter{utterance}

\twocolumn

{ \footnotesize  \setcounter{utterance}{1}
\setlength{\tabcolsep}{0pt}
\begin{supertabular}{c@{$\;$}|p{.15\linewidth}@{}p{.15\linewidth}p{.15\linewidth}p{.15\linewidth}p{.15\linewidth}p{.15\linewidth}}

    \# & $\;$A & \multicolumn{4}{c}{Game Master} & $\;\:$B\\
    \hline 

    \theutterance \stepcounter{utterance}  

    & & \multicolumn{4}{p{0.6\linewidth}}{\cellcolor[rgb]{0.9,0.9,0.9}{%
	\makecell[{{p{\linewidth}}}]{% 
	  \tt {\tiny [A$\langle$GM]}  
	 You are an expert in causal inference. The following question is not a typical commonsense query, but rather a meticulously designed question created by a professor specializing in causal inference, intended to assess the students' mastery of the course content.\\ \tt Imagine a self-contained, hypothetical world with only the following conditions, and without any unmentioned factors or causal relationships: Gwet has a direct effect on yomx and xevu. Yomx has a direct effect on xevu. For those who are not gwet and are not yomx, the probability of xevu is 82%. For those who are not gwet and are yomx, the probability of xevu is 63%. For those who are gwet and are not yomx, the probability of xevu is 36%. For those who are gwet and are yomx, the probability of xevu is 18%. The overall probability of gwet is 41%. Will yomx increase the chance of xevu?\\ \tt Start your answer with 'Yes' or 'No', followed by additional reasoning or evidence to support your explanation. 
	  } 
	   } 
	   } 
	 & \\ 
 

    \theutterance \stepcounter{utterance}  

    & \multicolumn{4}{p{0.6\linewidth}}{\cellcolor[rgb]{0.8,1,0.9}{%
	\makecell[{{p{\linewidth}}}]{% 
	  \tt {\tiny [A$\rangle$GM]}  
	 Yes, yomx will increase the chance of xevu. \\ \tt \\ \tt To determine the causal effect of yomx on xevu, we can use the backdoor adjustment formula, which is based on the do-calculus. However, given the information provided, we can also use the g-formula to estimate the causal effect.\\ \tt \\ \tt Let's denote the probability of xevu as P(xevu), the probability of yomx as P(yomx), and the probability of gwet as P(gwet). We are given the following probabilities:\\ \tt \\ \tt - P(xevu | not gwet, not yomx) = 0.82\\ \tt - P(xevu | not gwet, yomx) = 0.63\\ \tt - P(xevu | gwet, not yomx) = 0.36\\ \tt - P(xevu | gwet, yomx) = 0.18\\ \tt - P(gwet) = 0.41\\ \tt \\ \tt We can estimate the causal effect of yomx on xevu by comparing the probability of xevu among those who are yomx to the probability of xevu among those who are not yomx, while adjusting for the effect of gwet.\\ \tt \\ \tt First, we need to calculate the probability of xevu among those who are not yomx and not gwet, and among those who are yomx and not gwet. We can 
	  } 
	   } 
	   } 
	 & & \\ 
 

    \theutterance \stepcounter{utterance}  

    & & & \multicolumn{2}{p{0.3\linewidth}}{\cellcolor[rgb]{0.95,0.95,0.95}{%
	\makecell[{{p{\linewidth}}}]{% 
	  \tt {\tiny [GM$|$GM]}  
	 Yes 
	  } 
	   } 
	   } 
	 & & \\ 
 

    \theutterance \stepcounter{utterance}  

    & & & \multicolumn{2}{p{0.3\linewidth}}{\cellcolor[rgb]{0.95,0.95,0.95}{%
	\makecell[{{p{\linewidth}}}]{% 
	  \tt {\tiny [GM$|$GM]}  
	 no 
	  } 
	   } 
	   } 
	 & & \\ 
 

    \theutterance \stepcounter{utterance}  

    & & & \multicolumn{2}{p{0.3\linewidth}}{\cellcolor[rgb]{0.95,0.95,0.95}{%
	\makecell[{{p{\linewidth}}}]{% 
	  \tt {\tiny [GM$|$GM]}  
	 game_result = LOSE 
	  } 
	   } 
	   } 
	 & & \\ 
 

\end{supertabular}
}

\end{document}
