
\documentclass{article}
\usepackage{colortbl}
\usepackage{makecell}
\usepackage{multirow}
\usepackage{supertabular}

\begin{document}

\newcounter{utterance}

\twocolumn

{ \footnotesize  \setcounter{utterance}{1}
\setlength{\tabcolsep}{0pt}
\begin{supertabular}{c@{$\;$}|p{.15\linewidth}@{}p{.15\linewidth}p{.15\linewidth}p{.15\linewidth}p{.15\linewidth}p{.15\linewidth}}

    \# & $\;$A & \multicolumn{4}{c}{Game Master} & $\;\:$B\\
    \hline 

    \theutterance \stepcounter{utterance}  

    & & \multicolumn{4}{p{0.6\linewidth}}{\cellcolor[rgb]{0.9,0.9,0.9}{%
	\makecell[{{p{\linewidth}}}]{% 
	  \tt {\tiny [A$\langle$GM]}  
	 You are an expert in causal inference. The following question is not a typical commonsense query, but rather a meticulously designed question created by a professor specializing in causal inference, intended to assess the students' mastery of the course content.\\ \tt Imagine a self-contained, hypothetical world with only the following conditions, and without any unmentioned factors or causal relationships: Swoy has a direct effect on rixq and xevu. Rixq has a direct effect on zuph. Zuph has a direct effect on xevu. Swoy is unobserved. For those who are not rixq and are not zuph, the probability of xevu is 46%. For those who are not rixq and are zuph, the probability of xevu is 41%. For those who are rixq and are not zuph, the probability of xevu is 66%. For those who are rixq and are zuph, the probability of xevu is 60%. For those who are not rixq, the probability of zuph is 61%. For those who are rixq, the probability of zuph is 34%. For those who are rixq, would it be more likely to see xevu if the individual was not rixq?\\ \tt Start your answer with 'Yes' or 'No', followed by additional reasoning or evidence to support your explanation. 
	  } 
	   } 
	   } 
	 & \\ 
 

    \theutterance \stepcounter{utterance}  

    & \multicolumn{4}{p{0.6\linewidth}}{\cellcolor[rgb]{0.8,1,0.9}{%
	\makecell[{{p{\linewidth}}}]{% 
	  \tt {\tiny [A$\rangle$GM]}  
	 No, it would not be more likely to see xevu if the individual was not rixq. \\ \tt \\ \tt To address this question, we need to apply the principles of causal inference, specifically the concept of confounding variables and the potential outcomes framework. \\ \tt \\ \tt Let's denote the potential outcomes of xevu as Y(1) and Y(0), where Y(1) is the outcome when the individual is rixq and Y(0) is the outcome when the individual is not rixq. We are interested in the average treatment effect (ATE) of being rixq on the probability of xevu.\\ \tt \\ \tt We can use the potential outcomes framework to estimate the ATE. The ATE is defined as E[Y(1) - Y(0)], where E denotes the expectation. \\ \tt \\ \tt To estimate the ATE, we need to estimate E[Y(1)] and E[Y(0)]. We can use the given probabilities to estimate these expectations.\\ \tt \\ \tt For those who are rixq, the probability of xevu is 66% when zuph is 0 and 60% when zuph is 1. We can use these probabilities to estimate E[Y(1)]:\\ \tt \\ \tt E[Y(1)] = P(zuph=0) * Y(1|rixq, zuph=0) + P(zuph=1) * Y(1|rixq, zuph=1)\\ \tt = 0.66 
	  } 
	   } 
	   } 
	 & & \\ 
 

    \theutterance \stepcounter{utterance}  

    & & & \multicolumn{2}{p{0.3\linewidth}}{\cellcolor[rgb]{0.95,0.95,0.95}{%
	\makecell[{{p{\linewidth}}}]{% 
	  \tt {\tiny [GM$|$GM]}  
	 No 
	  } 
	   } 
	   } 
	 & & \\ 
 

    \theutterance \stepcounter{utterance}  

    & & & \multicolumn{2}{p{0.3\linewidth}}{\cellcolor[rgb]{0.95,0.95,0.95}{%
	\makecell[{{p{\linewidth}}}]{% 
	  \tt {\tiny [GM$|$GM]}  
	 no 
	  } 
	   } 
	   } 
	 & & \\ 
 

    \theutterance \stepcounter{utterance}  

    & & & \multicolumn{2}{p{0.3\linewidth}}{\cellcolor[rgb]{0.95,0.95,0.95}{%
	\makecell[{{p{\linewidth}}}]{% 
	  \tt {\tiny [GM$|$GM]}  
	 game_result = WIN 
	  } 
	   } 
	   } 
	 & & \\ 
 

\end{supertabular}
}

\end{document}
