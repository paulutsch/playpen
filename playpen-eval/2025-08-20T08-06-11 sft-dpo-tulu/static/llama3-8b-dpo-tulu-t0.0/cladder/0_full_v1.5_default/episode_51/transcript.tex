
\documentclass{article}
\usepackage{colortbl}
\usepackage{makecell}
\usepackage{multirow}
\usepackage{supertabular}

\begin{document}

\newcounter{utterance}

\twocolumn

{ \footnotesize  \setcounter{utterance}{1}
\setlength{\tabcolsep}{0pt}
\begin{supertabular}{c@{$\;$}|p{.15\linewidth}@{}p{.15\linewidth}p{.15\linewidth}p{.15\linewidth}p{.15\linewidth}p{.15\linewidth}}

    \# & $\;$A & \multicolumn{4}{c}{Game Master} & $\;\:$B\\
    \hline 

    \theutterance \stepcounter{utterance}  

    & & \multicolumn{4}{p{0.6\linewidth}}{\cellcolor[rgb]{0.9,0.9,0.9}{%
	\makecell[{{p{\linewidth}}}]{% 
	  \tt {\tiny [A$\langle$GM]}  
	 You are an expert in causal inference. The following question is not a typical commonsense query, but rather a meticulously designed question created by a professor specializing in causal inference, intended to assess the students' mastery of the course content.\\ \tt Imagine a self-contained, hypothetical world with only the following conditions, and without any unmentioned factors or causal relationships: Having a brother has a direct effect on wife and alarm clock. Wife has a direct effect on alarm clock. For people who do not have a brother and wives that don't set the alarm, the probability of ringing alarm is 88%. For people who do not have a brother and wives that set the alarm, the probability of ringing alarm is 45%. For people who have a brother and wives that don't set the alarm, the probability of ringing alarm is 58%. For people who have a brother and wives that set the alarm, the probability of ringing alarm is 13%. For people who do not have a brother, the probability of alarm set by wife is 9%. For people who have a brother, the probability of alarm set by wife is 64%. Does having a brother negatively affect alarm clock through wife?\\ \tt Start your answer with 'Yes' or 'No', followed by additional reasoning or evidence to support your explanation. 
	  } 
	   } 
	   } 
	 & \\ 
 

    \theutterance \stepcounter{utterance}  

    & \multicolumn{4}{p{0.6\linewidth}}{\cellcolor[rgb]{0.8,1,0.9}{%
	\makecell[{{p{\linewidth}}}]{% 
	  \tt {\tiny [A$\rangle$GM]}  
	 No, having a brother does not negatively affect the alarm clock through the wife. \\ \tt \\ \tt To assess this, we need to consider the causal relationships and the conditional probabilities provided. The presence of a brother affects the probability of the wife setting the alarm (64% vs 9%), but the effect of having a brother on the alarm clock is not straightforward. \\ \tt \\ \tt We can use the backdoor criterion to assess the causal effect. The backdoor criterion states that if we can block all backdoor paths from the cause (having a brother) to the effect (alarm clock), we can estimate the causal effect. \\ \tt \\ \tt In this case, the backdoor path is through the wife's behavior (setting the alarm). We need to check if the presence of a brother affects the wife's behavior in a way that would lead to a negative effect on the alarm clock.\\ \tt \\ \tt To do this, we can compare the probability of the alarm ringing when the wife sets the alarm (45% and 13%) to the probability of the alarm ringing when the wife doesn't set the alarm (88% and 58%). \\ \tt \\ \tt The presence of a brother reduces the probability of the alarm ringing when the wife sets the alarm (from 45% to 13%), but it also increases the probability of the alarm ringing when the wife doesn't set the alarm (from 88% to 58%). \\ \tt \\ \tt This suggests that the presence of a brother has a complex effect on the alarm clock, but it does not negatively affect it through the 
	  } 
	   } 
	   } 
	 & & \\ 
 

    \theutterance \stepcounter{utterance}  

    & & & \multicolumn{2}{p{0.3\linewidth}}{\cellcolor[rgb]{0.95,0.95,0.95}{%
	\makecell[{{p{\linewidth}}}]{% 
	  \tt {\tiny [GM$|$GM]}  
	 No 
	  } 
	   } 
	   } 
	 & & \\ 
 

    \theutterance \stepcounter{utterance}  

    & & & \multicolumn{2}{p{0.3\linewidth}}{\cellcolor[rgb]{0.95,0.95,0.95}{%
	\makecell[{{p{\linewidth}}}]{% 
	  \tt {\tiny [GM$|$GM]}  
	 yes 
	  } 
	   } 
	   } 
	 & & \\ 
 

    \theutterance \stepcounter{utterance}  

    & & & \multicolumn{2}{p{0.3\linewidth}}{\cellcolor[rgb]{0.95,0.95,0.95}{%
	\makecell[{{p{\linewidth}}}]{% 
	  \tt {\tiny [GM$|$GM]}  
	 game_result = LOSE 
	  } 
	   } 
	   } 
	 & & \\ 
 

\end{supertabular}
}

\end{document}
