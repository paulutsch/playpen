
\documentclass{article}
\usepackage{colortbl}
\usepackage{makecell}
\usepackage{multirow}
\usepackage{supertabular}

\begin{document}

\newcounter{utterance}

\twocolumn

{ \footnotesize  \setcounter{utterance}{1}
\setlength{\tabcolsep}{0pt}
\begin{supertabular}{c@{$\;$}|p{.15\linewidth}@{}p{.15\linewidth}p{.15\linewidth}p{.15\linewidth}p{.15\linewidth}p{.15\linewidth}}

    \# & $\;$A & \multicolumn{4}{c}{Game Master} & $\;\:$B\\
    \hline 

    \theutterance \stepcounter{utterance}  

    & & \multicolumn{4}{p{0.6\linewidth}}{\cellcolor[rgb]{0.9,0.9,0.9}{%
	\makecell[{{p{\linewidth}}}]{% 
	  \tt {\tiny [A$\langle$GM]}  
	 You are an expert in causal inference. The following question is not a typical commonsense query, but rather a meticulously designed question created by a professor specializing in causal inference, intended to assess the students' mastery of the course content.\\ \tt Imagine a self-contained, hypothetical world with only the following conditions, and without any unmentioned factors or causal relationships: Drinking coffee has a direct effect on hospitalization status. Broken bones has a direct effect on hospitalization status. Method 1: We look directly at how drinking coffee correlates with broken bones in general. Method 2: We look at this correlation case by case according to hospitalization status. To understand how drinking coffee affects broken bones, is it more correct to use the Method 1 than Method 2?\\ \tt Start your answer with 'Yes' or 'No', followed by additional reasoning or evidence to support your explanation. 
	  } 
	   } 
	   } 
	 & \\ 
 

    \theutterance \stepcounter{utterance}  

    & \multicolumn{4}{p{0.6\linewidth}}{\cellcolor[rgb]{0.8,1,0.9}{%
	\makecell[{{p{\linewidth}}}]{% 
	  \tt {\tiny [A$\rangle$GM]}  
	 No, it is not more correct to use Method 1 than Method 2 to understand how drinking coffee affects broken bones.\\ \tt \\ \tt Method 1, which looks directly at the correlation between drinking coffee and broken bones in general, is subject to confounding bias. In this hypothetical world, drinking coffee has a direct effect on hospitalization status, and broken bones also have a direct effect on hospitalization status. This means that the correlation between drinking coffee and broken bones is likely to be influenced by their shared effect on hospitalization status. For example, people who drink more coffee may be more likely to be hospitalized due to broken bones, not because coffee causes broken bones, but because both coffee consumption and broken bones are associated with increased hospitalization. This confounding bias can lead to an incorrect conclusion about the causal relationship between coffee consumption and broken bones.\\ \tt \\ \tt Method 2, which looks at the correlation case by case according to hospitalization status, is a more appropriate approach to understanding the causal relationship between coffee consumption and broken bones. By conditioning on hospitalization status, we can control for the confounding effect of hospitalization status on the relationship between coffee consumption and broken bones. This approach is known as "conditional association" or "conditional correlation," and it is a key concept in causal inference. By using Method 2, we can more accurately estimate the causal effect of coffee consumption on broken bones, while controlling for the shared effect of hospitalization status. 
	  } 
	   } 
	   } 
	 & & \\ 
 

    \theutterance \stepcounter{utterance}  

    & & & \multicolumn{2}{p{0.3\linewidth}}{\cellcolor[rgb]{0.95,0.95,0.95}{%
	\makecell[{{p{\linewidth}}}]{% 
	  \tt {\tiny [GM$|$GM]}  
	 No 
	  } 
	   } 
	   } 
	 & & \\ 
 

    \theutterance \stepcounter{utterance}  

    & & & \multicolumn{2}{p{0.3\linewidth}}{\cellcolor[rgb]{0.95,0.95,0.95}{%
	\makecell[{{p{\linewidth}}}]{% 
	  \tt {\tiny [GM$|$GM]}  
	 yes 
	  } 
	   } 
	   } 
	 & & \\ 
 

    \theutterance \stepcounter{utterance}  

    & & & \multicolumn{2}{p{0.3\linewidth}}{\cellcolor[rgb]{0.95,0.95,0.95}{%
	\makecell[{{p{\linewidth}}}]{% 
	  \tt {\tiny [GM$|$GM]}  
	 game_result = LOSE 
	  } 
	   } 
	   } 
	 & & \\ 
 

\end{supertabular}
}

\end{document}
