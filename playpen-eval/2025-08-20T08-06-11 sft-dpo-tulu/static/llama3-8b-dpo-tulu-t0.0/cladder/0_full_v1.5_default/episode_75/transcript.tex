
\documentclass{article}
\usepackage{colortbl}
\usepackage{makecell}
\usepackage{multirow}
\usepackage{supertabular}

\begin{document}

\newcounter{utterance}

\twocolumn

{ \footnotesize  \setcounter{utterance}{1}
\setlength{\tabcolsep}{0pt}
\begin{supertabular}{c@{$\;$}|p{.15\linewidth}@{}p{.15\linewidth}p{.15\linewidth}p{.15\linewidth}p{.15\linewidth}p{.15\linewidth}}

    \# & $\;$A & \multicolumn{4}{c}{Game Master} & $\;\:$B\\
    \hline 

    \theutterance \stepcounter{utterance}  

    & & \multicolumn{4}{p{0.6\linewidth}}{\cellcolor[rgb]{0.9,0.9,0.9}{%
	\makecell[{{p{\linewidth}}}]{% 
	  \tt {\tiny [A$\langle$GM]}  
	 You are an expert in causal inference. The following question is not a typical commonsense query, but rather a meticulously designed question created by a professor specializing in causal inference, intended to assess the students' mastery of the course content.\\ \tt Imagine a self-contained, hypothetical world with only the following conditions, and without any unmentioned factors or causal relationships: Jyka has a direct effect on yupt. Yupt has a direct effect on kwox. The overall probability of jyka is 25%. The probability of not jyka and kwox is 71%. The probability of jyka and kwox is 24%. Is the chance of kwox smaller when observing jyka?\\ \tt Start your answer with 'Yes' or 'No', followed by additional reasoning or evidence to support your explanation. 
	  } 
	   } 
	   } 
	 & \\ 
 

    \theutterance \stepcounter{utterance}  

    & \multicolumn{4}{p{0.6\linewidth}}{\cellcolor[rgb]{0.8,1,0.9}{%
	\makecell[{{p{\linewidth}}}]{% 
	  \tt {\tiny [A$\rangle$GM]}  
	 No\\ \tt \\ \tt To determine if the chance of kwox is smaller when observing jyka, we need to use the concept of conditional probability and the backdoor criterion from causal inference. \\ \tt \\ \tt First, let's establish the causal relationships and their corresponding probabilities:\\ \tt \\ \tt - P(Jyka) = 0.25\\ \tt - P(Not Jyka and Kwox) = 0.71\\ \tt - P(Jyka and Kwox) = 0.24\\ \tt \\ \tt We can calculate the probability of Kwox as follows:\\ \tt \\ \tt P(Kwox) = P(Not Jyka and Kwox) + P(Jyka and Kwox)\\ \tt = 0.71 + 0.24\\ \tt = 0.95\\ \tt \\ \tt Now, let's calculate the probability of Kwox given Jyka (P(Kwox|Jyka)):\\ \tt \\ \tt P(Kwox|Jyka) = P(Jyka and Kwox) / P(Jyka)\\ \tt = 0.24 / 0.25\\ \tt = 0.96\\ \tt \\ \tt Comparing P(Kwox) and P(Kwox|Jyka), we can see that P(Kwox|Jyka) is actually greater than P(Kwox). This indicates that observing Jyka does not make the chance of Kwox smaller; instead, it makes the chance of Kwox larger. \\ \tt \\ \tt This result can be explained by the direct effect 
	  } 
	   } 
	   } 
	 & & \\ 
 

    \theutterance \stepcounter{utterance}  

    & & & \multicolumn{2}{p{0.3\linewidth}}{\cellcolor[rgb]{0.95,0.95,0.95}{%
	\makecell[{{p{\linewidth}}}]{% 
	  \tt {\tiny [GM$|$GM]}  
	 No 
	  } 
	   } 
	   } 
	 & & \\ 
 

    \theutterance \stepcounter{utterance}  

    & & & \multicolumn{2}{p{0.3\linewidth}}{\cellcolor[rgb]{0.95,0.95,0.95}{%
	\makecell[{{p{\linewidth}}}]{% 
	  \tt {\tiny [GM$|$GM]}  
	 no 
	  } 
	   } 
	   } 
	 & & \\ 
 

    \theutterance \stepcounter{utterance}  

    & & & \multicolumn{2}{p{0.3\linewidth}}{\cellcolor[rgb]{0.95,0.95,0.95}{%
	\makecell[{{p{\linewidth}}}]{% 
	  \tt {\tiny [GM$|$GM]}  
	 game_result = WIN 
	  } 
	   } 
	   } 
	 & & \\ 
 

\end{supertabular}
}

\end{document}
