
\documentclass{article}
\usepackage{colortbl}
\usepackage{makecell}
\usepackage{multirow}
\usepackage{supertabular}

\begin{document}

\newcounter{utterance}

\twocolumn

{ \footnotesize  \setcounter{utterance}{1}
\setlength{\tabcolsep}{0pt}
\begin{supertabular}{c@{$\;$}|p{.15\linewidth}@{}p{.15\linewidth}p{.15\linewidth}p{.15\linewidth}p{.15\linewidth}p{.15\linewidth}}

    \# & $\;$A & \multicolumn{4}{c}{Game Master} & $\;\:$B\\
    \hline 

    \theutterance \stepcounter{utterance}  

    & & \multicolumn{4}{p{0.6\linewidth}}{\cellcolor[rgb]{0.9,0.9,0.9}{%
	\makecell[{{p{\linewidth}}}]{% 
	  \tt {\tiny [A$\langle$GM]}  
	 You are an expert in causal inference. The following question is not a typical commonsense query, but rather a meticulously designed question created by a professor specializing in causal inference, intended to assess the students' mastery of the course content.\\ \tt Imagine a self-contained, hypothetical world with only the following conditions, and without any unmentioned factors or causal relationships: Having a brother has a direct effect on wife and alarm clock. Wife has a direct effect on alarm clock. The overall probability of having a brother is 1%. The probability of not having a brother and ringing alarm is 64%. The probability of having a brother and ringing alarm is 0%. Is the chance of ringing alarm larger when observing having a brother?\\ \tt Start your answer with 'Yes' or 'No', followed by additional reasoning or evidence to support your explanation. 
	  } 
	   } 
	   } 
	 & \\ 
 

    \theutterance \stepcounter{utterance}  

    & \multicolumn{4}{p{0.6\linewidth}}{\cellcolor[rgb]{0.8,1,0.9}{%
	\makecell[{{p{\linewidth}}}]{% 
	  \tt {\tiny [A$\rangle$GM]}  
	 No\\ \tt \\ \tt To answer this question, we need to apply the principles of causal inference, specifically the concept of causal effects and the potential outcomes framework. \\ \tt \\ \tt Let's denote the presence of a brother as B, the presence of a ringing alarm as A, and the presence of a wife as W. We are given the following information:\\ \tt \\ \tt 1. P(B) = 0.01 (probability of having a brother)\\ \tt 2. P(¬B, A) = 0.64 (probability of not having a brother and ringing alarm)\\ \tt 3. P(B, A) = 0 (probability of having a brother and ringing alarm)\\ \tt \\ \tt We want to determine if the chance of ringing alarm (A) is larger when observing having a brother (B). To do this, we need to compare the probability of ringing alarm in the presence of a brother (P(A|B)) to the probability of ringing alarm in the absence of a brother (P(A|¬B)).\\ \tt \\ \tt Using the law of total probability, we can express P(A|B) as:\\ \tt \\ \tt P(A|B) = P(A, B) / P(B) = 0 / 0.01 = 0\\ \tt \\ \tt Similarly, we can express P(A|¬B) as:\\ \tt \\ \tt P(A|¬B) = P(A, ¬B) / P(¬B) = P(¬B, A) / (1 - P(B)) = 0.64 / (1 - 0.01 
	  } 
	   } 
	   } 
	 & & \\ 
 

    \theutterance \stepcounter{utterance}  

    & & & \multicolumn{2}{p{0.3\linewidth}}{\cellcolor[rgb]{0.95,0.95,0.95}{%
	\makecell[{{p{\linewidth}}}]{% 
	  \tt {\tiny [GM$|$GM]}  
	 No 
	  } 
	   } 
	   } 
	 & & \\ 
 

    \theutterance \stepcounter{utterance}  

    & & & \multicolumn{2}{p{0.3\linewidth}}{\cellcolor[rgb]{0.95,0.95,0.95}{%
	\makecell[{{p{\linewidth}}}]{% 
	  \tt {\tiny [GM$|$GM]}  
	 no 
	  } 
	   } 
	   } 
	 & & \\ 
 

    \theutterance \stepcounter{utterance}  

    & & & \multicolumn{2}{p{0.3\linewidth}}{\cellcolor[rgb]{0.95,0.95,0.95}{%
	\makecell[{{p{\linewidth}}}]{% 
	  \tt {\tiny [GM$|$GM]}  
	 game_result = WIN 
	  } 
	   } 
	   } 
	 & & \\ 
 

\end{supertabular}
}

\end{document}
