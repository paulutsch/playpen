
\documentclass{article}
\usepackage{colortbl}
\usepackage{makecell}
\usepackage{multirow}
\usepackage{supertabular}

\begin{document}

\newcounter{utterance}

\twocolumn

{ \footnotesize  \setcounter{utterance}{1}
\setlength{\tabcolsep}{0pt}
\begin{supertabular}{c@{$\;$}|p{.15\linewidth}@{}p{.15\linewidth}p{.15\linewidth}p{.15\linewidth}p{.15\linewidth}p{.15\linewidth}}

    \# & $\;$A & \multicolumn{4}{c}{Game Master} & $\;\:$B\\
    \hline 

    \theutterance \stepcounter{utterance}  

    & & \multicolumn{4}{p{0.6\linewidth}}{\cellcolor[rgb]{0.9,0.9,0.9}{%
	\makecell[{{p{\linewidth}}}]{% 
	  \tt {\tiny [A$\langle$GM]}  
	 Write an interesting and funny article about the biology of a banana peel. In your response, the word disappointed should appear at least 2 times, and at least six section should be highlighted with markdown,  i.e *banana peel*. 
	  } 
	   } 
	   } 
	 & \\ 
 

    \theutterance \stepcounter{utterance}  

    & \multicolumn{4}{p{0.6\linewidth}}{\cellcolor[rgb]{0.8,1,0.9}{%
	\makecell[{{p{\linewidth}}}]{% 
	  \tt {\tiny [A$\rangle$GM]}  
	 **The Slippery Truth About Banana Peels**\\ \tt \\ \tt As we peel back the layers (pun intended) of the humble banana, we often overlook the fascinating biology behind its discarded skin. But fear not, dear readers, for we're about to dive into the juicy world of *banana peel* biology and uncover the secrets that lie beneath.\\ \tt \\ \tt ### **The Evolutionary Advantage of a Slippery Skin**\\ \tt \\ \tt Banana peels are notorious for their slipperiness, and it's not just because they're a favorite among monkeys. In reality, the peel's slick surface is a clever adaptation that helps the banana tree protect itself from pests and diseases. The waxy coating on the peel, known as cutin, makes it difficult for insects and fungi to get a grip, effectively keeping them at bay. It's like the banana tree is saying, "Hey, bugs, you can't handle the peel!"\\ \tt \\ \tt ### **The Disappointed Fruit of Our Labor**\\ \tt \\ \tt But what happens when we, the humans, get a little too enthusiastic with our banana-eating habits? We peel, we eat, and then we're left with a *banana peel* that's often carelessly discarded on the floor. It's a sad fate, really. The peel, once a vital part of the banana's defense system, is now nothing more than a slippery nuisance. We're disappointed, but the banana tree doesn't care – it's just happy to have rid itself of the pesky skin.\\ \tt \\ \tt ### **The Amazing World of Cellulose**\\ \tt \\ \tt So, what makes the *banana peel* so slippery, anyway? It all comes down to the structure of its cells. Banana peels are made up of a type of cellulose called microfibrils, which are long, thin strands of cellulose that give the peel its strength and flexibility. But it's not just the microfibrils themselves that make the peel slippery – it's the way they're arranged. The microfibrils are aligned in a specific pattern, creating a surface that's perfect for slipping and sliding. It's like the banana peel is saying, "Come on, human, take a tumble!"\\ \tt \\ \tt ### **The Disappointed Scientist**\\ \tt \\ \tt As scientists, we're often disappointed when our experiments don't yield the results we expect. But when it comes to the *banana peel*, we're in for a surprise. Despite its reputation for being a nuisance, the peel is actually a treasure trove of scientific knowledge. By studying the peel's structure and composition, we can gain insights into the biology of plants and even develop new materials with similar properties. It's a win-win situation – we get to learn something new, and the banana tree gets to keep its skin.\\ \tt \\ \tt ### **The Future of Banana Peel Technology**\\ \tt \\ \tt So, what's next for the humble *banana peel*? With its unique combination of strength, flexibility, and slipperiness, it's no wonder that scientists are taking notice. Imagine a world where banana peels are used to create sustainable materials for packaging, textiles, or even biomedical applications. It's a future that's full of possibilities, and we're excited to see where the research takes us. Who knows – maybe one day, we'll be disappointed that we didn't think of it sooner!\\ \tt \\ \tt ### **The Final Slip**\\ \tt \\ \tt As we conclude our journey into the world of *banana peel* biology, we're left with a newfound appreciation for the humble fruit that's often overlooked. So next time you're peeling a banana, remember the amazing biology that's going on beneath your fingers. And who knows – maybe you'll even be inspired to create something new and innovative with the discarded peel. After all, as the saying goes, "one man's trash is another man's treasure." 
	  } 
	   } 
	   } 
	 & & \\ 
 

    \theutterance \stepcounter{utterance}  

    & & & \multicolumn{2}{p{0.3\linewidth}}{\cellcolor[rgb]{0.95,0.95,0.95}{%
	\makecell[{{p{\linewidth}}}]{% 
	  \tt {\tiny [GM$|$GM]}  
	 {'keywords:frequency': {'relation': 'at least', 'keyword': 'disappointed', 'frequency': 2}, 'detectable_format:number_highlighted_sections': {'num_highlights': 6}} 
	  } 
	   } 
	   } 
	 & & \\ 
 

    \theutterance \stepcounter{utterance}  

    & & & \multicolumn{2}{p{0.3\linewidth}}{\cellcolor[rgb]{0.95,0.95,0.95}{%
	\makecell[{{p{\linewidth}}}]{% 
	  \tt {\tiny [GM$|$GM]}  
	 game_result = WIN 
	  } 
	   } 
	   } 
	 & & \\ 
 

\end{supertabular}
}

\end{document}
