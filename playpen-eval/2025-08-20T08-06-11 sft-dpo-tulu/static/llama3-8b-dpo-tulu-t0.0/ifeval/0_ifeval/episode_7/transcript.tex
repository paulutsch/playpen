
\documentclass{article}
\usepackage{colortbl}
\usepackage{makecell}
\usepackage{multirow}
\usepackage{supertabular}

\begin{document}

\newcounter{utterance}

\twocolumn

{ \footnotesize  \setcounter{utterance}{1}
\setlength{\tabcolsep}{0pt}
\begin{supertabular}{c@{$\;$}|p{.15\linewidth}@{}p{.15\linewidth}p{.15\linewidth}p{.15\linewidth}p{.15\linewidth}p{.15\linewidth}}

    \# & $\;$A & \multicolumn{4}{c}{Game Master} & $\;\:$B\\
    \hline 

    \theutterance \stepcounter{utterance}  

    & & \multicolumn{4}{p{0.6\linewidth}}{\cellcolor[rgb]{0.9,0.9,0.9}{%
	\makecell[{{p{\linewidth}}}]{% 
	  \tt {\tiny [A$\langle$GM]}  
	 Take the text below as a starting point, and make it a complete article: "You may have to meet with a helper to work out a parenting plan. The first would be to estimate how many time you have everyday for parenting, and is that enough...."\\ \tt Avoid using the following keywords: sleep, cook, feed\\ \tt Mention the keyword 'schedule' for more than 5 times. 
	  } 
	   } 
	   } 
	 & \\ 
 

    \theutterance \stepcounter{utterance}  

    & \multicolumn{4}{p{0.6\linewidth}}{\cellcolor[rgb]{0.8,1,0.9}{%
	\makecell[{{p{\linewidth}}}]{% 
	  \tt {\tiny [A$\rangle$GM]}  
	 **Creating a Parenting Plan: Understanding Your Daily Schedule**\\ \tt \\ \tt When parents separate or divorce, creating a parenting plan is often a crucial step in ensuring the well-being and stability of their children. A parenting plan outlines how parents will share responsibilities, make decisions, and allocate time with their children. To create an effective parenting plan, it's essential to understand your daily schedule and how it will impact your ability to care for your child.\\ \tt \\ \tt You may have to meet with a helper, such as a mediator or a family law attorney, to work out a parenting plan that suits your needs. The first step in this process is to estimate how many hours you have available every day for parenting and whether that is enough to meet your child's needs. This requires creating a detailed schedule that accounts for all aspects of your daily life, including work, personal time, and other responsibilities.\\ \tt \\ \tt To start, make a list of all the activities you need to fit into your daily schedule, such as work, exercise, personal time, and time spent with your child. Then, estimate how many hours you have available each day for parenting. Consider your work schedule, including your commute, work hours, and any overtime you may be required to work. Also, think about your personal time, including time for hobbies, self-care, and relaxation.\\ \tt \\ \tt Once you have a clear picture of your daily schedule, you can begin to create a parenting schedule that works for you and your child. This may involve allocating specific days or times for parenting, such as every other weekend or one evening per week. It's essential to be realistic about your schedule and to prioritize your child's needs.\\ \tt \\ \tt A well-planned schedule can help ensure that your child receives the care and attention they need, while also allowing you to maintain a sense of balance and stability in your life. By creating a schedule that works for everyone, you can reduce conflict and stress, and create a more positive co-parenting experience.\\ \tt \\ \tt In addition to creating a parenting schedule, it's also essential to consider other aspects of your daily routine, such as transportation, childcare, and extracurricular activities. By including these details in your schedule, you can ensure that your child's needs are met and that you are able to provide a stable and loving environment.\\ \tt \\ \tt When creating a parenting plan, it's essential to be flexible and open to compromise. Your schedule may need to change over time, and it's crucial to be able to adapt to these changes. By working together with your co-parent and creating a schedule that works for everyone, you can create a positive and supportive environment for your child.\\ \tt \\ \tt In conclusion, creating a parenting plan requires a detailed understanding of your daily schedule and how it will impact your ability to care for your child. By creating a schedule that works for everyone, you can reduce conflict and stress, and create a more positive co-parenting experience. Remember to be flexible and open to compromise, and to prioritize your child's needs above all else. 
	  } 
	   } 
	   } 
	 & & \\ 
 

    \theutterance \stepcounter{utterance}  

    & & & \multicolumn{2}{p{0.3\linewidth}}{\cellcolor[rgb]{0.95,0.95,0.95}{%
	\makecell[{{p{\linewidth}}}]{% 
	  \tt {\tiny [GM$|$GM]}  
	 {'keywords:forbidden_words': {'forbidden_words': ['sleep', 'cook', 'feed']}, 'keywords:frequency': {'relation': 'at least', 'keyword': 'schedule', 'frequency': 6}} 
	  } 
	   } 
	   } 
	 & & \\ 
 

    \theutterance \stepcounter{utterance}  

    & & & \multicolumn{2}{p{0.3\linewidth}}{\cellcolor[rgb]{0.95,0.95,0.95}{%
	\makecell[{{p{\linewidth}}}]{% 
	  \tt {\tiny [GM$|$GM]}  
	 game_result = WIN 
	  } 
	   } 
	   } 
	 & & \\ 
 

\end{supertabular}
}

\end{document}
