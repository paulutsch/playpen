
\documentclass{article}
\usepackage{colortbl}
\usepackage{makecell}
\usepackage{multirow}
\usepackage{supertabular}

\begin{document}

\newcounter{utterance}

\twocolumn

{ \footnotesize  \setcounter{utterance}{1}
\setlength{\tabcolsep}{0pt}
\begin{supertabular}{c@{$\;$}|p{.15\linewidth}@{}p{.15\linewidth}p{.15\linewidth}p{.15\linewidth}p{.15\linewidth}p{.15\linewidth}}

    \# & $\;$A & \multicolumn{4}{c}{Game Master} & $\;\:$B\\
    \hline 

    \theutterance \stepcounter{utterance}  

    & & \multicolumn{4}{p{0.6\linewidth}}{\cellcolor[rgb]{0.9,0.9,0.9}{%
	\makecell[{{p{\linewidth}}}]{% 
	  \tt {\tiny [A$\langle$GM]}  
	 Translate the following sentence into German and then criticize it: Werner was a good friend of mine, but not very smart.\\ \tt Avoid the word "schlau" throughout your response. 
	  } 
	   } 
	   } 
	 & \\ 
 

    \theutterance \stepcounter{utterance}  

    & \multicolumn{4}{p{0.6\linewidth}}{\cellcolor[rgb]{0.8,1,0.9}{%
	\makecell[{{p{\linewidth}}}]{% 
	  \tt {\tiny [A$\rangle$GM]}  
	 The translation of the sentence into German is:\\ \tt \\ \tt Werner war ein guter Freund von mir, aber nicht sehr intelligent.\\ \tt \\ \tt However, this translation might sound a bit formal or even slightly negative, as the word "intelligent" can have a somewhat formal or even condescending tone.\\ \tt \\ \tt A more casual and friendly translation could be:\\ \tt \\ \tt Werner war ein guter Freund von mir, aber nicht sehr clever.\\ \tt \\ \tt This translation uses the word "clever", which is a more informal and colloquial way to express the idea that someone is not very intelligent or wise. It's a more natural and friendly way to express this sentiment in everyday conversation.\\ \tt \\ \tt Alternatively, you could also use the phrase "aber nicht so begabt" (but not so gifted) or "aber nicht so geistig" (but not so mentally gifted), which are more neutral and don't have the same negative connotation as "intelligent". 
	  } 
	   } 
	   } 
	 & & \\ 
 

    \theutterance \stepcounter{utterance}  

    & & & \multicolumn{2}{p{0.3\linewidth}}{\cellcolor[rgb]{0.95,0.95,0.95}{%
	\makecell[{{p{\linewidth}}}]{% 
	  \tt {\tiny [GM$|$GM]}  
	 {'keywords:forbidden_words': {'forbidden_words': ['schlau']}} 
	  } 
	   } 
	   } 
	 & & \\ 
 

    \theutterance \stepcounter{utterance}  

    & & & \multicolumn{2}{p{0.3\linewidth}}{\cellcolor[rgb]{0.95,0.95,0.95}{%
	\makecell[{{p{\linewidth}}}]{% 
	  \tt {\tiny [GM$|$GM]}  
	 game_result = WIN 
	  } 
	   } 
	   } 
	 & & \\ 
 

\end{supertabular}
}

\end{document}
