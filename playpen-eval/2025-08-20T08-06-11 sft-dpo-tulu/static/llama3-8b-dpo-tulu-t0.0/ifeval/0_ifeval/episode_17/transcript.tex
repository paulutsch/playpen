
\documentclass{article}
\usepackage{colortbl}
\usepackage{makecell}
\usepackage{multirow}
\usepackage{supertabular}

\begin{document}

\newcounter{utterance}

\twocolumn

{ \footnotesize  \setcounter{utterance}{1}
\setlength{\tabcolsep}{0pt}
\begin{supertabular}{c@{$\;$}|p{.15\linewidth}@{}p{.15\linewidth}p{.15\linewidth}p{.15\linewidth}p{.15\linewidth}p{.15\linewidth}}

    \# & $\;$A & \multicolumn{4}{c}{Game Master} & $\;\:$B\\
    \hline 

    \theutterance \stepcounter{utterance}  

    & & \multicolumn{4}{p{0.6\linewidth}}{\cellcolor[rgb]{0.9,0.9,0.9}{%
	\makecell[{{p{\linewidth}}}]{% 
	  \tt {\tiny [A$\langle$GM]}  
	 The Jimenez family has 20 members. Each member has at least one pet. If there are 32 pets in total, what is the maximum number of members that can possibly have 3 pets? Solve this step by step, and wrap your entire response (reasoning process and the final answer) with double quotation marks. 
	  } 
	   } 
	   } 
	 & \\ 
 

    \theutterance \stepcounter{utterance}  

    & \multicolumn{4}{p{0.6\linewidth}}{\cellcolor[rgb]{0.8,1,0.9}{%
	\makecell[{{p{\linewidth}}}]{% 
	  \tt {\tiny [A$\rangle$GM]}  
	 "Let's break down the problem step by step:\\ \tt \\ \tt 1. The Jimenez family has 20 members, and each member has at least one pet. This means the total number of pets is at least 20.\\ \tt \\ \tt 2. However, we are given that there are 32 pets in total. This means that the number of pets is 32, which is more than the minimum of 20.\\ \tt \\ \tt 3. Let's assume that 'x' members have 3 pets each. Then, the remaining (20 - x) members will have 1 pet each.\\ \tt \\ \tt 4. The total number of pets can be represented as: 3x + (20 - x) = 32\\ \tt \\ \tt 5. Simplifying the equation, we get: 2x + 20 = 32\\ \tt \\ \tt 6. Subtracting 20 from both sides, we get: 2x = 12\\ \tt \\ \tt 7. Dividing both sides by 2, we get: x = 6\\ \tt \\ \tt 8. This means that the maximum number of members that can possibly have 3 pets is 6.\\ \tt \\ \tt Therefore, the maximum number of members that can possibly have 3 pets is 6." 
	  } 
	   } 
	   } 
	 & & \\ 
 

    \theutterance \stepcounter{utterance}  

    & & & \multicolumn{2}{p{0.3\linewidth}}{\cellcolor[rgb]{0.95,0.95,0.95}{%
	\makecell[{{p{\linewidth}}}]{% 
	  \tt {\tiny [GM$|$GM]}  
	 {'startend:quotation': {}} 
	  } 
	   } 
	   } 
	 & & \\ 
 

    \theutterance \stepcounter{utterance}  

    & & & \multicolumn{2}{p{0.3\linewidth}}{\cellcolor[rgb]{0.95,0.95,0.95}{%
	\makecell[{{p{\linewidth}}}]{% 
	  \tt {\tiny [GM$|$GM]}  
	 game_result = WIN 
	  } 
	   } 
	   } 
	 & & \\ 
 

\end{supertabular}
}

\end{document}
