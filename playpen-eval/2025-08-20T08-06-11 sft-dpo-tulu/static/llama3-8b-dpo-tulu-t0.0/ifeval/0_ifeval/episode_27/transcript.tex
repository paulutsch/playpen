
\documentclass{article}
\usepackage{colortbl}
\usepackage{makecell}
\usepackage{multirow}
\usepackage{supertabular}

\begin{document}

\newcounter{utterance}

\twocolumn

{ \footnotesize  \setcounter{utterance}{1}
\setlength{\tabcolsep}{0pt}
\begin{supertabular}{c@{$\;$}|p{.15\linewidth}@{}p{.15\linewidth}p{.15\linewidth}p{.15\linewidth}p{.15\linewidth}p{.15\linewidth}}

    \# & $\;$A & \multicolumn{4}{c}{Game Master} & $\;\:$B\\
    \hline 

    \theutterance \stepcounter{utterance}  

    & & \multicolumn{4}{p{0.6\linewidth}}{\cellcolor[rgb]{0.9,0.9,0.9}{%
	\makecell[{{p{\linewidth}}}]{% 
	  \tt {\tiny [A$\langle$GM]}  
	 Write a casual blog post about how the outer solar system is different from the inner solar system, and what that means for the possibility of life. Wrap your entire response with double quotation marks. Your response should contain 17 or more sentences. 
	  } 
	   } 
	   } 
	 & \\ 
 

    \theutterance \stepcounter{utterance}  

    & \multicolumn{4}{p{0.6\linewidth}}{\cellcolor[rgb]{0.8,1,0.9}{%
	\makecell[{{p{\linewidth}}}]{% 
	  \tt {\tiny [A$\rangle$GM]}  
	 "Exploring the vastness of our solar system, we often find ourselves comparing the inner and outer regions. The inner solar system, comprising the four rocky planets - Mercury, Venus, Earth, and Mars - is a vastly different place from the outer solar system, which includes the gas giants and their moons. The primary distinction lies in the distance from the sun, with the inner planets receiving a significant amount of solar radiation, while the outer planets are much farther away and receive much less.\\ \tt \\ \tt This difference in solar radiation has a profound impact on the formation and evolution of the planets. The inner planets are rocky and relatively small, with temperatures that can fluctuate wildly between day and night. In contrast, the outer planets are gas giants, composed primarily of hydrogen and helium, and are much larger in size. Their atmospheres are also much more stable, with temperatures that remain relatively constant.\\ \tt \\ \tt The reduced solar radiation in the outer solar system also means that the conditions for life as we know it are much more favorable. Liquid water, a crucial ingredient for life, is more likely to exist on the moons of the gas giants, such as Europa and Enceladus, which have subsurface oceans. These moons are thought to have the necessary energy and nutrients to support life, making them prime targets in the search for extraterrestrial life.\\ \tt \\ \tt The outer solar system is also home to a diverse range of moons, each with its unique characteristics and potential for life. Some, like Titan, have thick atmospheres and lakes of liquid methane, while others, like Ganymede, have subsurface oceans and potential for hydrothermal activity. The variety of environments in the outer solar system presents a wide range of possibilities for life to exist, making it an exciting and promising area of research.\\ \tt \\ \tt In contrast, the inner solar system is a much more hostile environment for life. The rocky planets are subject to extreme temperatures, radiation, and atmospheric loss, making it difficult for life to thrive. While Mars, in particular, has been a focus of interest for potential life, the harsh conditions and lack of liquid water make it a less likely candidate.\\ \tt \\ \tt The discovery of exoplanets, particularly those in the habitable zones of their stars, has also led to a reevaluation of the possibility of life in the outer solar system. While the conditions on these planets may be similar to those on Earth, the presence of a star and the resulting solar radiation can have a significant impact on the planet's habitability. The outer solar system, with its reduced solar radiation, may offer a more stable and long-term environment for life to exist.\\ \tt \\ \tt In conclusion, the outer solar system is a vastly different place from the inner solar system, with conditions that are more favorable for life. The presence of liquid water, stable atmospheres, and diverse environments on the moons of the gas giants make it an exciting area of research. While the inner solar system is a challenging environment for life, the outer solar system offers a promising and potentially life-supporting region of our solar system." 
	  } 
	   } 
	   } 
	 & & \\ 
 

    \theutterance \stepcounter{utterance}  

    & & & \multicolumn{2}{p{0.3\linewidth}}{\cellcolor[rgb]{0.95,0.95,0.95}{%
	\makecell[{{p{\linewidth}}}]{% 
	  \tt {\tiny [GM$|$GM]}  
	 {'startend:quotation': {}, 'length_constraints:number_sentences': {'relation': 'at least', 'num_sentences': 17}} 
	  } 
	   } 
	   } 
	 & & \\ 
 

    \theutterance \stepcounter{utterance}  

    & & & \multicolumn{2}{p{0.3\linewidth}}{\cellcolor[rgb]{0.95,0.95,0.95}{%
	\makecell[{{p{\linewidth}}}]{% 
	  \tt {\tiny [GM$|$GM]}  
	 game_result = WIN 
	  } 
	   } 
	   } 
	 & & \\ 
 

\end{supertabular}
}

\end{document}
