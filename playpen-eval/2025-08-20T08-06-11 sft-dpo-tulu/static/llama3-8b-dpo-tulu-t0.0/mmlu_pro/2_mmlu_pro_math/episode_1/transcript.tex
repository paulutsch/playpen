
\documentclass{article}
\usepackage{colortbl}
\usepackage{makecell}
\usepackage{multirow}
\usepackage{supertabular}

\begin{document}

\newcounter{utterance}

\twocolumn

{ \footnotesize  \setcounter{utterance}{1}
\setlength{\tabcolsep}{0pt}
\begin{supertabular}{c@{$\;$}|p{.15\linewidth}@{}p{.15\linewidth}p{.15\linewidth}p{.15\linewidth}p{.15\linewidth}p{.15\linewidth}}

    \# & $\;$A & \multicolumn{4}{c}{Game Master} & $\;\:$B\\
    \hline 

    \theutterance \stepcounter{utterance}  

    & & \multicolumn{4}{p{0.6\linewidth}}{\cellcolor[rgb]{0.9,0.9,0.9}{%
	\makecell[{{p{\linewidth}}}]{% 
	  \tt {\tiny [A$\langle$GM]}  
	 The following are multiple choice questions (with answers) about math. Answer only with "The answer is (X)" where X is the correct letter choice.\\ \tt \\ \tt Question:\\ \tt The symmetric group $S_n$ has $\\ \tt \factorial{n}$ elements, hence it is not true that $S_{10}$ has 10 elements.\\ \tt Find the characteristic of the ring 2Z.\\ \tt Options:\\ \tt A. 0\\ \tt B. 30\\ \tt C. 3\\ \tt D. 10\\ \tt E. 12\\ \tt F. 50\\ \tt G. 2\\ \tt H. 100\\ \tt I. 20\\ \tt J. 5\\ \tt The answer is (A)\\ \tt \\ \tt Question:\\ \tt Let V be the set of all real polynomials p(x). Let transformations T, S be defined on V by T:p(x) -> xp(x) and S:p(x) -> p'(x) = d/dx p(x), and interpret (ST)(p(x)) as S(T(p(x))). Which of the following is true?\\ \tt Options:\\ \tt A. ST + TS is the identity map of V onto itself.\\ \tt B. TS = 0\\ \tt C. ST = 1\\ \tt D. ST - TS = 0\\ \tt E. ST = T\\ \tt F. ST = 0\\ \tt G. ST = TS\\ \tt H. ST - TS is the identity map of V onto itself.\\ \tt I. TS = T\\ \tt J. ST = S\\ \tt The answer is (H)\\ \tt \\ \tt Question:\\ \tt Let A be the set of all ordered pairs of integers (m, n) such that 7m + 12n = 22. What is the greatest negative number in the set B = {m + n : (m, n) \in A}?\\ \tt Options:\\ \tt A. -5\\ \tt B. 0\\ \tt C. -3\\ \tt D. -7\\ \tt E. -4\\ \tt F. -6\\ \tt G. -1\\ \tt H. -2\\ \tt I. -9\\ \tt J. N/A\\ \tt The answer is (E)\\ \tt \\ \tt Question:\\ \tt A tank initially contains a salt solution of 3 grams of salt dissolved in 100 liters of water. A salt solution containing 0.02 grams of salt per liter of water is sprayed into the tank at a rate of 4 liters per minute. The sprayed solution is continually mixed with the salt solution in the tank, and the mixture flows out of the tank at a rate of 4 liters per minute. If the mixing is instantaneous, how many grams of salt are in the tank after 100 minutes have elapsed?\\ \tt Options:\\ \tt A. 3 + e^-2\\ \tt B. 2 - e^-4\\ \tt C. 2 - e^-2\\ \tt D. 3 + e^-4\\ \tt E. 2 + e^-3\\ \tt F. 2 - e^-3\\ \tt G. 3 - e^-2\\ \tt H. 2 + e^-2\\ \tt I. 2 + e^-4\\ \tt J. 2\\ \tt The answer is (I)\\ \tt \\ \tt Question:\\ \tt A total of 30 players will play basketball at a park. There will be exactly 5 players on each team. Which statement correctly explains how to find the number of teams needed?\\ \tt Options:\\ \tt A. Multiply 5 by 5 to find 25 teams.\\ \tt B. Divide 30 by 5 to find 6 teams.\\ \tt C. Add 5 to 30 to find 35 teams.\\ \tt D. Subtract 30 from 5 to find -25 teams.\\ \tt E. Divide 5 by 30 to find 0.1667 teams.\\ \tt F. Add 5 to 30 then divide by 2 to find 17.5 teams.\\ \tt G. N/A\\ \tt H. N/A\\ \tt I. N/A\\ \tt J. N/A\\ \tt The answer is (B)\\ \tt \\ \tt Question:\\ \tt The relation between the selling price of a car (in $1,000) and its age (in years) is estimated from a random sample of cars of a specific model. The relation is given by the following formula: Selling price = 15.9 – 0.983 (age) Which of the following can we conclude from this equation?\\ \tt Options:\\ \tt A. For every year the car gets older, the selling price drops by approximately $983.\\ \tt B. A new car costs on the average $9,830.\\ \tt C. A new car costs on the average $15,900.\\ \tt D. For every year the car gets older, the selling price goes down by approximately 9.83 percent.\\ \tt E. For every year the car gets older, the selling price goes up by approximately $983.\\ \tt F. For every year the car gets older, the selling price drops by approximately $159.\\ \tt G. For every year the car gets older, the selling price goes up by approximately $1,590.\\ \tt H. A new car costs on the average $983.\\ \tt I. For every year the car gets older, the selling price remains the same.\\ \tt J. For every year the car gets older, the selling price drops by approximately $1,590.\\ \tt  
	  } 
	   } 
	   } 
	 & \\ 
 

    \theutterance \stepcounter{utterance}  

    & \multicolumn{4}{p{0.6\linewidth}}{\cellcolor[rgb]{0.8,1,0.9}{%
	\makecell[{{p{\linewidth}}}]{% 
	  \tt {\tiny [A$\rangle$GM]}  
	 The answer is (J). 
	  } 
	   } 
	   } 
	 & & \\ 
 

    \theutterance \stepcounter{utterance}  

    & & & \multicolumn{2}{p{0.3\linewidth}}{\cellcolor[rgb]{0.95,0.95,0.95}{%
	\makecell[{{p{\linewidth}}}]{% 
	  \tt {\tiny [GM$|$GM]}  
	 J 
	  } 
	   } 
	   } 
	 & & \\ 
 

    \theutterance \stepcounter{utterance}  

    & & & \multicolumn{2}{p{0.3\linewidth}}{\cellcolor[rgb]{0.95,0.95,0.95}{%
	\makecell[{{p{\linewidth}}}]{% 
	  \tt {\tiny [GM$|$GM]}  
	 A 
	  } 
	   } 
	   } 
	 & & \\ 
 

    \theutterance \stepcounter{utterance}  

    & & & \multicolumn{2}{p{0.3\linewidth}}{\cellcolor[rgb]{0.95,0.95,0.95}{%
	\makecell[{{p{\linewidth}}}]{% 
	  \tt {\tiny [GM$|$GM]}  
	 game_result = LOSE 
	  } 
	   } 
	   } 
	 & & \\ 
 

\end{supertabular}
}

\end{document}
