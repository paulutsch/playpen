
\documentclass{article}
\usepackage{colortbl}
\usepackage{makecell}
\usepackage{multirow}
\usepackage{supertabular}

\begin{document}

\newcounter{utterance}

\twocolumn

{ \footnotesize  \setcounter{utterance}{1}
\setlength{\tabcolsep}{0pt}
\begin{supertabular}{c@{$\;$}|p{.15\linewidth}@{}p{.15\linewidth}p{.15\linewidth}p{.15\linewidth}p{.15\linewidth}p{.15\linewidth}}

    \# & $\;$A & \multicolumn{4}{c}{Game Master} & $\;\:$B\\
    \hline 

    \theutterance \stepcounter{utterance}  

    & & \multicolumn{4}{p{0.6\linewidth}}{\cellcolor[rgb]{0.9,0.9,0.9}{%
	\makecell[{{p{\linewidth}}}]{% 
	  \tt {\tiny [A$\langle$GM]}  
	 The following are multiple choice questions (with answers) about biology. Answer only with "The answer is (X)" where X is the correct letter choice.\\ \tt \\ \tt Question:\\ \tt Which of the following represents an accurate statement concerning arthropods?\\ \tt Options:\\ \tt A. They possess an exoskeleton composed primarily of peptidoglycan.\\ \tt B. They possess an open circulatory system with a dorsal heart.\\ \tt C. They are members of a biologically unsuccessful phylum incapable of exploiting diverse habitats and nutrition sources.\\ \tt D. They lack paired, jointed appendages.\\ \tt E. N/A\\ \tt F. N/A\\ \tt G. N/A\\ \tt H. N/A\\ \tt I. N/A\\ \tt J. N/A\\ \tt The answer is (B)\\ \tt \\ \tt Question:\\ \tt In a given population, 1 out of every 400 people has a cancer caused by a completely recessive allele, b. Assuming the population is in Hardy-Weinberg equilibrium, which of the following is the expected proportion of individuals who carry the b allele but are not expected to develop the cancer?\\ \tt Options:\\ \tt A. 19/400\\ \tt B. 1/400\\ \tt C. 40/400\\ \tt D. 38/400\\ \tt E. 2/400\\ \tt F. 1/200\\ \tt G. 20/400\\ \tt H. 50/400\\ \tt I. N/A\\ \tt J. N/A\\ \tt The answer is (D)\\ \tt \\ \tt Question:\\ \tt A mutation in a bacterial enzyme changed a previously polar amino acid into a nonpolar amino acid. This amino acid was located at a site distant from the enzyme’s active site. How might this mutation alter the enzyme’s substrate specificity?\\ \tt Options:\\ \tt A. By changing the enzyme’s pH optimum\\ \tt B. By changing the enzyme's molecular weight\\ \tt C. An amino acid change away from the active site increases the enzyme's substrate specificity.\\ \tt D. By changing the shape of the protein\\ \tt E. By changing the enzyme's temperature optimum\\ \tt F. By altering the enzyme's ability to be denatured\\ \tt G. By changing the enzyme’s location in the cell\\ \tt H. By changing the enzyme's color\\ \tt I. An amino acid change away from the active site cannot alter the enzyme’s substrate specificity.\\ \tt J. By altering the enzyme's rate of reaction\\ \tt The answer is (D)\\ \tt \\ \tt Question:\\ \tt Which of the following is not a way to form recombinant DNA?\\ \tt Options:\\ \tt A. Translation\\ \tt B. Conjugation\\ \tt C. Specialized transduction\\ \tt D. Transformation\\ \tt E. N/A\\ \tt F. N/A\\ \tt G. N/A\\ \tt H. N/A\\ \tt I. N/A\\ \tt J. N/A\\ \tt The answer is (A)\\ \tt \\ \tt Question:\\ \tt Which of the following is not known to be involved in the control of cell division?\\ \tt Options:\\ \tt A. Microtubules\\ \tt B. Checkpoints\\ \tt C. DNA polymerase\\ \tt D. Centrosomes\\ \tt E. Cyclins\\ \tt F. Mitochondria\\ \tt G. Protein kinases\\ \tt H. Fibroblast cells\\ \tt I. N/A\\ \tt J. N/A\\ \tt The answer is (H)\\ \tt \\ \tt Question:\\ \tt Distinguish between the two types of proteins produced by ribosomes.\\ \tt Options:\\ \tt A. Structural proteins provide temporary cellular functions, while functional proteins provide permanent structure to cells\\ \tt B. Structural proteins control cell activity while functional proteins are part of the cell protoplasm\\ \tt C. Structural proteins catalyze chemical reactions while functional proteins are involved in the structure of the cell\\ \tt D. Structural proteins are soluble in the cell while functional proteins are generally insoluble\\ \tt E. Structural proteins are found only outside the cell, while functional proteins are exclusively found inside the cell\\ \tt F. Structural proteins are only used for muscle contraction, while functional proteins are responsible for energy production\\ \tt G. Structural proteins are part of the cell protoplasm and contribute to cell and body growth, while functional proteins control cell activity such as hormone production or nutrient metabolism.\\ \tt H. Structural proteins are unable to catalyze any reactions, while functional proteins form the structural components of ribosomes\\ \tt I. Structural proteins are synthesized from amino acids in the cytoplasm, while functional proteins are synthesized in the nucleus\\ \tt J. Structural proteins are produced by free ribosomes, while functional proteins are produced by membrane-bound ribosomes\\ \tt  
	  } 
	   } 
	   } 
	 & \\ 
 

    \theutterance \stepcounter{utterance}  

    & \multicolumn{4}{p{0.6\linewidth}}{\cellcolor[rgb]{0.8,1,0.9}{%
	\makecell[{{p{\linewidth}}}]{% 
	  \tt {\tiny [A$\rangle$GM]}  
	 The answer is (B)\\ \tt \\ \tt The answer is (C)\\ \tt \\ \tt The answer is (D)\\ \tt \\ \tt The answer is (A) 
	  } 
	   } 
	   } 
	 & & \\ 
 

    \theutterance \stepcounter{utterance}  

    & & & \multicolumn{2}{p{0.3\linewidth}}{\cellcolor[rgb]{0.95,0.95,0.95}{%
	\makecell[{{p{\linewidth}}}]{% 
	  \tt {\tiny [GM$|$GM]}  
	 ParseError: Response contains more than a one answer, but the expected format is to only give one. 
	  } 
	   } 
	   } 
	 & & \\ 
 

    \theutterance \stepcounter{utterance}  

    & & & \multicolumn{2}{p{0.3\linewidth}}{\cellcolor[rgb]{0.95,0.95,0.95}{%
	\makecell[{{p{\linewidth}}}]{% 
	  \tt {\tiny [GM$|$GM]}  
	 game_result = ABORT 
	  } 
	   } 
	   } 
	 & & \\ 
 

\end{supertabular}
}

\end{document}
